\section[An Application to Electrical Networks]{An Application to Electrical Networks\footnotemark}
\label{sec:1_5}\index{electrical networks}

\footnotetext{This section is independent of Section 1.4}

In an electrical network it is often necessary to find the current in amperes (A) flowing in various parts of the network\index{system of linear equations!electrical networks!application to}. These networks usually contain resistors that retard the current. The resistors are indicated by a symbol (\tn{here1}{} \quad \tn{here2}{})\begin{circuitikz}[remember picture, overlay,scale=0.5, transform shape]
\draw[transform canvas={yshift=0.1cm}] (here1) to [R] (here2);
\end {circuitikz}, and the resistance is measured in ohms ($\Omega$). Also, the current is increased at various points by voltage sources (for example, a battery). The voltage of these sources is measured in volts (V), and they are represented by the symbol (\tn{node3}{} \; \tn{node4}{}). We assume these voltage sources have no resistance. The flow of current is governed by the following principles.
\begin{circuitikz}[remember picture, overlay,scale=0.5, transform shape]
\draw[transform canvas={yshift=0.1cm}] (node3) to [battery1, v^= $\;$] (node4);
\end {circuitikz}

\begin{theorem*}[label=thm:001806]{Ohm's Law}

The current $I$ and the voltage drop $V$ across a resistance $R$ are related by the equation $V = RI$.\index{Ohm's Law}

\end{theorem*}

\begin{theorem*}[label=thm:001809]{Kirchhoff's Laws}

\begin{enumerate}
\item (Junction Rule) The current flow into a junction equals the current flow out of that junction.\index{junction rule}

\item (Circuit Rule) The algebraic sum\index{sum!algebraic sum} of the voltage drops (due to resistances) around any closed circuit of the network must equal the sum of the voltage increases around the circuit.\index{circuit rule}
\index{Kirchhoff's Laws}
\end{enumerate}

\end{theorem*}

\noindent When applying rule 2, select a direction (clockwise or counterclockwise) around the closed circuit and then consider all voltages and currents positive when in this direction 
and negative when in the opposite direction. This is why the term \textit{algebraic sum}\index{algebraic sum} is used in rule 2. Here is an example.

\begin{example}{}{001817}

Find the various currents in the circuit shown.

\begin{solution}
\begin{wrapfigure}[10]{l}{5cm}
  \vspace*{-1em}
	\begin{circuitikz}[scale=0.75, transform shape]
\draw[dkgreenvect, text=black] (0,3) to [battery1, v= $ 10\; V$, color=dkgreenvect] (-3,3)
      to [R, l= $ 20 \Omega $, color=dkgreenvect] (-3,0)
      to [short, i_=$I_1$, color=dkgreenvect] (-1.5,0)
      to [short, i_=$I_6$, color=dkgreenvect] (0,0)
      to [battery1, v^= $ 5\; V$, i>^=$I_2$, color=dkgreenvect] (0,3)
(0,0) to [R = $ 5 \Omega $, i_=$I_4$, color=dkgreenvect] (3,0)
      to (3.25,0)
      to [battery1, v= $ 20\; V$, color=dkgreenvect] (3.25,3)
      to [R, l_= $ 10 \Omega $, i=$I_3$, color=dkgreenvect] (0,3)
(3,0) to (3,-2.5)
	  to [battery1, v_= $ 10\; V$, i>^=$I_5$, color=dkgreenvect] (-1.5,-2.5) 
	  to [R = $ 5 \Omega $, color=dkgreenvect] (-1.5,0)
;
\fill[black] (0,0) circle (2pt) ;
\fill[black] (0,3) circle (2pt) ;
\fill[black] (3,0) circle (2pt) ;
\fill[black] (-1.5,0) circle (2pt) ;
\node[above] at (3,0) {$D$};
\node[above] at (0,3) {$A$};
\node[above] at (-1.5,0) {$B$};
\node[below] at (0,0) {$C$};
\end{circuitikz}

	%\captionof{figure}{\label{fig:001824}}
\end{wrapfigure}
  First apply the junction rule at junctions $A$, $B$, $C$, and $D$ to obtain

\begin{equation*}
\arraycolsep=1pt
\begin{array}{lrcl}
	\mbox{Junction } A & I_1 & = & I_2 + I_3 \\
	\mbox{Junction } B & I_6 & = & I_1 + I_5 \\
	\mbox{Junction } C & \quad I_2 + I_4 & = & I_6 \\
	\mbox{Junction } D & I_3 + I_5 & = & I_4 \\
\end{array}
\end{equation*}

Note that these equations are not independent (in fact, the third is an easy consequence of the other three).

Next, the circuit rule insists that the sum of the voltage increases (due to the sources) around a closed circuit must equal the sum of the voltage drops (due to resistances). By Ohm's law, the voltage loss across a resistance $R$ (in the direction of the current $I$) is $RI$. Going counterclockwise around three closed circuits yields

\begin{equation*}
\arraycolsep=1pt
\begin{array}{lrrrcl}
	\mbox{Upper left} &  10 & + & 5 & = & 20I_1 \\
	\mbox{Upper right} & \quad  -5 & +& 20 & = & 10I_3 + 5I_4 \\
	\mbox{Lower} & & &-10 & = & -5I_5 - 5I_4 \\
\end{array}
\end{equation*}

Hence, disregarding the redundant equation obtained at junction $C$, we have six equations in the six unknowns $I_1, \dots, I_6$. The solution is

\begin{equation*}
\def\arraystretch{1.5}
\begin{array}{ll}
	I_1 = \frac{15}{20} & I_4 = \frac{28}{20} \\
	I_2 = \frac{-1}{20} & I_5 = \frac{12}{20} \\
	I_3 = \frac{16}{20} & I_6 = \frac{27}{20}
\end{array}
\end{equation*}

The fact that $I_2$ is negative means, of course, that this current is in the opposite direction, with a magnitude of $\frac{1}{20}$ amperes.

\end{solution}

\end{example}

