\section{An Application to Quadratic Forms}
\label{sec:8_8}\index{orthogonality!quadratic forms}\index{quadratic form}

An expression like $x_{1}^2 + x_{2}^2 + x_{3}^2 - 2x_{1}x_{3} + x_{2}x_{3}$ is called a quadratic form in the variables $x_{1}$, $x_{2}$, and $x_{3}$. In this section we show that new variables $y_{1}$, $y_{2}$, and $y_{3}$ can always be found so that the quadratic form, when expressed in terms of the new variables, has no cross terms $y_{1}y_{2}$, $y_{1}y_{3}$, or $y_{2}y_{3}$. Moreover, we do this for forms involving any finite number of variables using orthogonal diagonalization. This has far-reaching applications; quadratic forms arise in such diverse areas as statistics, physics, the theory of functions of several variables, number theory, and geometry.


\begin{definition}{Quadratic Form}{026966}
A \textbf{quadratic form}\index{quadratic form}\index{diagonalization!quadratic form} $q$ in the $n$ variables $x_1, x_{2}, \dots, x_{n}$ is a linear combination of terms $x_{1}^2, x_{2}^2, \dots, x_{n}^2$, and cross terms $x_{1}x_{2}, x_{1}x_{3}, x_{2}x_{3}, \dots$ .
\end{definition}

If $n = 3$, $q$ has the form
\begin{equation*}
q = a_{11}x_{1}^2 + a_{22}x_{2}^2 + a_{33}x_{3}^2 + a_{12}x_{1}x_{2} + a_{21}x_2x_1+ a_{13}x_{1}x_{3} + a_{31}x_{3}x_{1} + a_{23}x_{2}x_{3} + a_{32}x_{3}x_{2}
\end{equation*}
In general
\begin{equation*}
q = a_{11}x_{1}^2 + a_{22}x_{2}^2 + \cdots + a_{nn}x_{n}^2 + a_{12}x_{1}x_{2} + a_{13}x_{1}x_{3} + \cdots
\end{equation*}
This sum can be written compactly as a matrix product
\begin{equation*}
q = q(\vect{x}) = \vect{x}^TA\vect{x}
\end{equation*}
where $\vect{x} = (x_{1}, x_{2}, \dots, x_{n})$ is thought of as a column, and $A = \leftB a_{ij} \rightB$ is a real $n \times n$ matrix. Note that if $i \neq j$, two separate terms $a_{ij}x_{i}x_{j}$ and $a_{ji}x_{j}x_{i}$ are listed, each of which involves $x_{i}x_{j}$, and they can (rather cleverly) be replaced by
\begin{equation*}
\frac{1}{2}(a_{ij} + a_{ji})x_{i}x_{j} \quad \mbox{ and } \quad \frac{1}{2}(a_{ij} + a_{ji})x_{j}x_{i}
\end{equation*}
respectively, \textit{without altering the quadratic form}. Hence there is no loss of generality in assuming that $x_{i}x_{j}$ and $x_{j}x_{i}$ have the same coefficient in the sum for $q$. In other words, \textbf{we may assume that} $A$ \textbf{is symmetric}\index{symmetric form}.


\begin{example}{}{027002}
Write $q = x_1^2 + 3x_3^2+2x_1x_2 - x_1x_3$ in the form $q(\vect{x}) = \vect{x}^{T}A\vect{x}$, where $A$ is a symmetric $3 \times 3$ matrix.


\begin{solution}
  The cross terms are $2x_{1}x_{2} = x_{1}x_{2} + x_{2}x_{1}$ and $-x_{1}x_{3} = -\frac{1}{2}x_{1}x_{3} - \frac{1}{2}x_{3}x_{1}$.


Of course, $x_{2}x_{3}$ and $x_{3}x_{2}$ both have coefficient zero, as does $x^{2}_{2}$. Hence
\begin{equation*}
q(\vect{x}) = \leftB \begin{array}{rrr}
x_{1} & x_{2} & x_{3} 
\end{array}\rightB \leftB \begin{array}{rrr}
1 & 1 & -\frac{1}{2} \\
1 & 0 & 0 \\
-\frac{1}{2} & 0 & 3
\end{array}\rightB \leftB \begin{array}{c}
x_{1} \\
x_{2} \\
x_{3}
\end{array}\rightB
\end{equation*}
is the required form (verify).
\end{solution}
\end{example}

We shall assume from now on that all quadratic forms are given by
\begin{equation*}
q(\vect{x}) = \vect{x}^TA\vect{x}
\end{equation*}
where $A$ is symmetric. Given such a form, the problem is to find new variables $y_{1}, y_{2}, \dots, y_{n}$, related to $x_{1}, x_{2}, \dots, x_{n}$, with the property that when $q$ is expressed in terms of $y_{1}, y_{2}, \dots, y_{n}$, there are no cross terms. If we write
\begin{equation*}
\vect{y} = (y_{1}, y_{2}, \dots, y_{n})^T
\end{equation*}
this amounts to asking that $q = \vect{y}^{T}D\vect{y}$ where $D$ is diagonal. It turns out that this can always be accomplished and, not surprisingly, that $D$ is the matrix obtained when the symmetric matrix $A$ is orthogonally diagonalized. In fact, as Theorem~\ref{thm:024303} shows, a matrix $P$ can be found that is orthogonal (that is, $P^{-1} = P^{T}$) and diagonalizes $A$:
\begin{equation*}
P^TAP = D = \leftB \begin{array}{cccc}
\lambda_{1} & 0 & \cdots & 0 \\
0 & \lambda_{2} & \cdots & 0 \\
\vdots & \vdots && \vdots \\
0 & 0 & \cdots & \lambda_{n}
\end{array}\rightB
\end{equation*}
The diagonal entries $\lambda_{1}, \lambda_{2}, \dots, \lambda_{n}$ are the (not necessarily distinct) eigenvalues of $A$, repeated according to their multiplicities in $c_{A}(x)$, and the columns of $P$ are corresponding (orthonormal) eigenvectors of $A$. As $A$ is symmetric, the $\lambda_{i}$ are real by Theorem~\ref{thm:016397}.


Now define new variables $\vect{y}$ by the equations
\begin{equation*}
\vect{x} = P\vect{y} \quad \mbox{ equivalently } \quad \vect{y} = P^T\vect{x}
\end{equation*}
Then substitution in $q(\vect{x}) = \vect{x}^{T}A\vect{x}$ gives
\begin{equation*}
q = (P\vect{y})^TA(P\vect{y}) = \vect{y}^T(P^TAP)\vect{y} = \vect{y}^TD\vect{y} = \lambda_{1}y_{1}^2 + \lambda_{2}y_{2}^2 + \cdots + \lambda_{n}y_{n}^2
\end{equation*}
Hence this change of variables produces the desired simplification in $q$.


\begin{theorem}{Diagonalization Theorem}{027060}
Let $q = \vect{x}^{T}A\vect{x}$ be a quadratic form in the variables $x_{1}, x_{2}, \dots, x_{n}$, where $\vect{x} = (x_{1}, x_{2}, \dots, x_{n})^{T}$ and $A$ is a symmetric $n \times n$ matrix. Let $P$ be an orthogonal matrix such that $P^{T}AP$ is diagonal, and define new variables $\vect{y} = (y_{1}, y_{2}, \dots, y_{n})^{T}$ by
\begin{equation*}
\vect{x} = P\vect{y} \quad \mbox{ equivalently } \quad \vect{y} = P^T\vect{x}
\end{equation*}
If $q$ is expressed in terms of these new variables $y_{1}, y_{2}, \dots, y_{n}$, the result is
\begin{equation*}
q = \lambda_{1}y_{1}^2 + \lambda_{2}y_{2}^2 + \cdots + \lambda_{n}y_{n}^2
\end{equation*}
where $\lambda_{1}, \lambda_{2}, \dots, \lambda_{n}$ are the eigenvalues of $A$ repeated according to their multiplicities.
\end{theorem}\index{diagonalization theorem}

Let $q = \vect{x}^{T}A\vect{x}$ be a quadratic form where $A$ is a symmetric matrix and let $\lambda_{1}, \dots, \lambda_{n}$ be the (real) eigenvalues of $A$ repeated according to their multiplicities. A corresponding set $\{\vect{f}_{1}, \dots, \vect{f}_{n}\}$ of orthonormal eigenvectors for $A$ is called a set of \textbf{principal axes}\index{principal axes} for the quadratic form $q$. (The reason for the name will become clear later.) The orthogonal matrix $P$ in Theorem~\ref{thm:027060} is given as $P = \leftB \begin{array}{ccc} \vect{f}_{1} & \cdots & \vect{f}_{n} \end{array} \rightB$, so the variables $X$ and $Y$ are related by
\begin{equation*}
\vect{x} = P\vect{y} = \leftB \begin{array}{cccc}
\vect{f}_{1} & \vect{f}_{2}& \cdots & \vect{f}_{n} 
\end{array}\rightB \leftB \begin{array}{c}
y_{1} \\
y_{2} \\
\vdots \\
y_{n}
\end{array}\rightB = y_{1}\vect{f}_{1} + y_{2}\vect{f}_{2} + \cdots + y_{n}\vect{f}_{n}
\end{equation*}
Thus the new variables $y_{i}$ are the coefficients when $\vect{x}$ is expanded in terms of the orthonormal basis $\{\vect{f}_{1}, \dots, \vect{f}_{n}\}$ of $\RR^n$. In particular, the coefficients $y_{i}$ are given by $y_{i} = \vect{x} \cdot \vect{f}_{i}$ by the expansion theorem (Theorem~\ref{thm:015082}). Hence $q$ itself is easily computed from the eigenvalues $\lambda_{i}$ and the principal axes $\vect{f}_{i}$:
\begin{equation*}
q = q(\vect{x}) = \lambda_{1}(\vect{x} \dotprod \vect{f}_{1})^2 + \cdots + \lambda_{n}(\vect{x} \dotprod \vect{f}_{n})^2
\end{equation*}
\begin{example}{}{027107}
Find new variables $y_{1}$, $y_{2}$, $y_{3}$, and $y_{4}$ such that
\begin{equation*}
q = 3(x_{1}^2 + x_{2}^2 + x_{3}^2 +x_{4}^2) + 2x_{1}x_{2} - 10x_{1}x_{3} + 10x_{1}x_{4} + 10x_{2}x_{3} - 10x_{2}x_{4} + 2x_{3}x_{4}
\end{equation*}
has diagonal form, and find the corresponding principal axes.


\begin{solution}
  The form can be written as $q = \vect{x}^{T}A\vect{x}$, where
\begin{equation*}
\vect{x} = \leftB \begin{array}{c}
x_{1} \\
x_{2} \\
x_{3} \\
x_{4}
\end{array}\rightB \quad \mbox{ and } \quad 
A = \leftB \begin{array}{rrrr}
3 & 1 & -5 & 5 \\
1 & 3 & 5 & -5 \\
-5 & 5 & 3 & 1 \\
5 & -5 & 1 & 3
\end{array}\rightB
\end{equation*}
A routine calculation yields
\begin{equation*}
c_{A}(x) = \func{det}(xI - A) = (x - 12)(x + 8)(x - 4)^2
\end{equation*}
so the eigenvalues are $\lambda_{1} = 12$, $\lambda_{2} = -8$, and $\lambda_{3} = \lambda_{4} = 4$. Corresponding orthonormal eigenvectors are the principal axes:
\begin{equation*}
\vect{f}_{1} = \frac{1}{2} \leftB \begin{array}{r}
1 \\
-1 \\
-1 \\
1
\end{array}\rightB \quad
\vect{f}_{2} = \frac{1}{2} \leftB \begin{array}{r}
1 \\
-1 \\
1 \\
-1
\end{array}\rightB \quad
\vect{f}_{3} = \frac{1}{2} \leftB \begin{array}{r}
1 \\
1 \\
1 \\
1
\end{array}\rightB \quad
\vect{f}_{4} = \frac{1}{2} \leftB \begin{array}{r}
1 \\
1 \\
-1 \\
-1
\end{array}\rightB 
\end{equation*}
The matrix
\begin{equation*}
P = \leftB \begin{array}{cccc}
\vect{f}_{1} & \vect{f}_{2} & \vect{f}_{3} & \vect{f}_{4}
\end{array}\rightB  = \frac{1}{2} \leftB \begin{array}{rrrr}
1 & 1 & 1 & 1 \\
-1 & -1 & 1 & 1 \\
-1 & 1 & 1 & -1 \\
1 & -1 & 1 & -1
\end{array}\rightB
\end{equation*}
is thus orthogonal, and $P^{-1}AP = P^{T}AP$ is diagonal. Hence the new variables $\vect{y}$ and the old variables $\vect{x}$ are related by $\vect{y} = P^{T}\vect{x}$ and $\vect{x} = P\vect{y}$. Explicitly,


\begin{align*}
y_{1} = \frac{1}{2}(x_{1} - x_{2} - x_{3} + x_{4}) & &x_{1} &= \frac{1}{2}(y_{1} + y_{2} + y_{3} + y_{4}) \\
y_{2} = \frac{1}{2}(x_{1} - x_{2} + x_{3} - x_{4}) & &x_{2} &= \frac{1}{2}(-y_{1} - y_{2} + y_{3} + y_{4}) \\
y_{3} = \frac{1}{2}(x_{1} + x_{2} + x_{3} + x_{4}) & &x_{3} &= \frac{1}{2}(-y_{1} + y_{2} + y_{3} - y_{4}) \\
y_{4} = \frac{1}{2}(x_{1} + x_{2} - x_{3} - x_{4}) & &x_{4} &= \frac{1}{2}(y_{1} - y_{2} + y_{3} - y_{4}) 
\end{align*}

If these $x_{i}$ are substituted in the original expression for $q$, the result is
\begin{equation*}
q = 12y_{1}^2 - 8y_{2}^2 + 4y_{3}^2 + 4y_{4}^2
\end{equation*}
This is the required diagonal form.
\end{solution}
\end{example}

It is instructive to look at the case of quadratic forms in two variables $x_{1}$ and $x_{2}$. Then the principal axes can always be found by rotating the $x_{1}$ and $x_{2}$ axes counterclockwise about the origin through an angle $\theta$. This rotation is a linear transformation $R_{\theta} : \RR^2 \to \RR^2$, and it is shown in Theorem~\ref{thm:006021} that $R_{\theta}$ has matrix $P = \leftB \begin{array}{rr}
\cos\theta & -\sin\theta \\
\sin\theta & \cos\theta
\end{array}\rightB$. If $\{\vect{e}_{1}, \vect{e}_{2}\}$ denotes the standard basis of $\RR^2$, the rotation produces a new basis $\{\vect{f}_{1}, \vect{f}_{2}\}$ given by
\begin{equation} \label{rotationEq}
\vect{f}_{1} = R_{\theta}(\vect{e}_{1}) = \leftB \begin{array}{r}
\cos\theta \\
\sin\theta 
\end{array}\rightB \quad \mbox{ and } \quad
\vect{f}_{2} = R_{\theta}(\vect{e}_{2}) = \leftB \begin{array}{r}
-\sin\theta \\
\cos\theta 
\end{array}\rightB
\end{equation}

\begin{wrapfigure}[7]{l}{5cm} 
\vspace*{-2em}
\centering
\begin{tikzpicture}[scale=0.9]
%set up of axis environment
\begin{axis}[disabledatascaling, 
    width=5cm, 
    height=5cm, 
    xlabel={$x_1$}, 
    ylabel={$x_2$}, 
    axis lines=middle, 
    xtick={\empty}, 
    ytick={\empty}, 
    xticklabels=\empty, 
    yticklabels=\empty, 
    every axis x label/.style={
      at={(ticklabel* cs:1.05)},
      anchor=west,
    },
    every axis y label/.style={
      at={(ticklabel* cs:1.05)},
      anchor=south,
    },
	clip mode = individual,
    domain=-5:5, 
    samples=100, 
    xmin=-1, 
    xmax=3, 
    ymin=-1, 
    ymax=3]
    
\def \angleTheta{30}
\coordinate (ptY1) at (cos \angleTheta * 2, sin \angleTheta * 2);
\coordinate (ptY2) at ({cos (\angleTheta + 90)}, {sin (\angleTheta + 90)});
\coordinate (ptX1) at ({cos \angleTheta * 2 - sin \angleTheta * 1}, 0); %cos theta (y_1) - sin theta (y_2)
\coordinate (ptX2) at (0, sin \angleTheta * 2 + cos \angleTheta * 1); %sin theta (y_1) + cos theta (y_2)
\coordinate (ptP) at (ptX1 |- ptX2);

%green lines
\draw [dkgreenvect, thick, ->, rotate=\angleTheta, name path=lineY1] (-1, 0)--(3, 0) node[text=black, right] {\footnotesize $y_1$}; %y_1
\draw [dkgreenvect, thick, ->, rotate=\angleTheta] (0, -1)--(0, 3) node[text=black, left] {\footnotesize $y_2$}; %y_2

%dashed lines
\draw [dkbluevect, thick, dashed] (ptX1)--(ptP)--(ptX2);
\draw [dkbluevect, thick, dashed] (ptY1)--(ptP)--(ptY2);

%points
\fill (ptX1) circle (2pt);
\fill (ptX2) circle (2pt);
\fill (ptP) circle (2pt);
\fill (ptY1) circle (2pt);
\fill (ptY2) circle (2pt);

%labels
\draw [dkbluevect, thick, ->] (0.5, 0) arc [start angle=0, end angle={\angleTheta - 0}, radius=0.5];
\node at (0.8, 0.2) {\footnotesize $\theta$};
\node at (-0.2, -0.4) {\footnotesize $O$};
\node [below] at (ptX1) {\footnotesize $x_1$};
\node [left] at (ptX2) {\footnotesize $x_2$};
\node [above right] at (ptP) {\footnotesize $\vect{p}$};
\node [below] at (ptY1) {\footnotesize $y_1$};
\node [left] at (ptY2) {\footnotesize $y_2$};

\end{axis}
\end{tikzpicture}

%\caption{\label{fig:027161}}
\end{wrapfigure}

Given a point $\vect{p} = \leftB \begin{array}{c}
x_{1} \\
x_{2}
\end{array}\rightB = x_{1}\vect{e}_{1}  + x_{2}\vect{e}_{2}$
 in the original system, let $y_{1}$ and $y_{2}$ be the coordinates of $\vect{p}$ in the new system (see the diagram). That is,
\begin{equation}\label{rotationEq2}
\leftB \begin{array}{c}
x_{1} \\
x_{2}
\end{array}\rightB = \vect{p} = y_{1}\vect{f}_{1} + y_{2}\vect{f}_{2} = \leftB \begin{array}{rr}
\cos\theta & -\sin\theta \\
\sin\theta & \cos\theta
\end{array}\rightB \leftB \begin{array}{c}
y_{1} \\
y_{2}
\end{array}\rightB
\end{equation}
Writing $\vect{x} = \leftB \begin{array}{c}
x_{1} \\
x_{2}
\end{array}\rightB$
and $\vect{y} = \leftB \begin{array}{c}
y_{1} \\
y_{2}
\end{array}\rightB$, this reads $\vect{x} = P\vect{y}$ so, since $P$ is orthogonal, this is the change of variables formula for the rotation as in Theorem~\ref{thm:027060}.

If $r \neq 0 \neq s$, the graph of the equation $rx_{1}^2 + sx_{2}^2 = 1$ is called an \textbf{ellipse}\index{ellipse}\index{graphs!ellipse} if $rs > 0$ and a \textbf{hyperbola}\index{hyperbola}\index{graphs!hyperbola} if $rs < 0$. More generally, given a quadratic form
\begin{equation*}
q = ax_{1}^2 + bx_{1}x_{2} + cx_{2}^2 \quad \mbox{ where not all of } a, b, \mbox{ and } c \mbox{ are zero}
\end{equation*}
the graph of the equation $q = 1$ is called a \textbf{conic}\index{conic graph}. We can now completely describe this graph. There are two special cases which we leave to the reader.


\begin{enumerate}
\item \textit{If exactly one of} $a$ \textit{and} $c$ \textit{is zero, then the graph of} $q = 1$ \textit{is a} \textbf{parabola}\index{parabola}.

\end{enumerate}

\noindent So we assume that $a \neq 0$ and $c \neq 0$. In this case, the description depends on the quantity $b^{2} - 4ac$, called the \textbf{discriminant}\index{discriminant} of the quadratic form $q$.


\begin{enumerate}
\setcounter{enumi}{1}
\item \textit{If} $b^{2} - 4ac = 0$, \textit{then either both} $a \geq 0$ \textit{and} $c \geq 0$, \textit{or both} $a \leq 0$ \textit{and} $c \leq 0$. \\ \textit{Hence} $q = (\sqrt{a}x_{1} + \sqrt{c}x_{2})^2$ \textit{or} $q = (\sqrt{-a}x_{1} + \sqrt{-c}x_{2})^2$, \textit{so the graph of} $q = 1$ \textit{is a} \textbf{pair of straight lines}\index{line!straight, pair of} \textit{in either case}.
\end{enumerate}

\noindent So we also assume that $b^{2} - 4ac \neq 0$. But then the next theorem asserts that there exists a rotation of the plane about the origin which transforms the equation $ax_{1}^2 + bx_{1}x_{2} + cx_{2}^2 = 1$ into either an ellipse or a hyperbola, and the theorem also provides a simple way to decide which conic it is.


\begin{theorem}{}{027179}
Consider the quadratic form $q = ax_{1}^2 + bx_{1}x_{2} + cx_{2}^2$ where $a$, $c$, and $b^{2} - 4ac$ are all nonzero.


\begin{enumerate}
\item There is a counterclockwise rotation of the coordinate axes about the origin such that, in the new coordinate system, $q$ has no cross term.

\item The graph of the equation
\begin{equation*}
ax_{1}^2 + bx_{1}x_{2} + cx_{2}^2 = 1
\end{equation*}
is an ellipse if $b^{2} - 4ac < 0$ and an hyperbola if $b^{2} - 4ac > 0$.

\end{enumerate}
\end{theorem}

\begin{proof}
If $b = 0$, $q$ \textit{already} has no cross term and (1) and (2) are clear. So assume $b \neq 0$.
The matrix $A = \leftB \begin{array}{cc}
a & \frac{1}{2}b \\
\frac{1}{2}b & c
\end{array}\rightB$ of $q$ has characteristic polynomial $c_A(x) = x^2 - (a + c)x - \frac{1}{4}(b^2 - 4ac)$. If we write $d = \sqrt{b^2 + (a - c)^2}$ for convenience; then the quadratic formula gives the eigenvalues
\begin{equation*}
\lambda_{1} = \frac{1}{2}[a + c - d] \quad \mbox{ and } \quad \lambda_{2} = \frac{1}{2}[a + c + d]
\end{equation*}
with corresponding principal axes
\begin{align*}
\vect{f}_{1} &= \frac{1}{\sqrt{b^2 + (a - c - d)^2}}\leftB \begin{array}{c}
	a - c - d \\
	b
\end{array}\rightB \quad \mbox{ and } \\
\vect{f}_{2} &= \frac{1}{\sqrt{b^2 + (a - c - d)^2}}\leftB \begin{array}{c}
	-b \\
	a - c - d 
	\end{array}\rightB
\end{align*}
as the reader can verify. These agree with equation (\ref{rotationEq}) above if $\theta$ is an angle such that
\begin{equation*}
\cos\theta = \frac{a - c - d}{\sqrt{b^2 + (a - c - d)^2}} \quad \mbox{ and } \quad \sin\theta = \frac{b}{\sqrt{b^2 + (a - c - d)^2}}
\end{equation*}
Then $P = \leftB \begin{array}{cc}
\vect{f}_{1} & \vect{f}_{2}
\end{array}\rightB = \leftB \begin{array}{rr}
\cos\theta & -\sin\theta \\
\sin\theta & \cos\theta
\end{array}\rightB$ diagonalizes $A$ and equation (\ref{rotationEq2}) becomes the formula $\vect{x} = P\vect{y}$ in Theorem~\ref{thm:027060}. This proves (1).


Finally, $A$ is similar to $\leftB \begin{array}{cc}
\lambda_{1} & 0 \\
0 & \lambda_{2}
\end{array}\rightB$
 so $\lambda_{1}\lambda_{2} = \func{det}A = \frac{1}{4}(4ac - b^2)$. Hence the graph of $\lambda_{1}y_{1}^2 + \lambda_{2}y_{2}^2 = 1$ is an ellipse if $b^{2} < 4ac$ and an hyperbola if $b^{2} > 4ac$. This proves (2).
\end{proof}

\begin{example}{}{027215}
Consider the equation $x^{2} + xy + y^{2} = 1$. Find a rotation so that the equation has no cross term.

\begin{solution}
\begin{wrapfigure}[9]{l}{5cm}
\centering
\begin{tikzpicture}
%set up of axis environment
\begin{axis}[disabledatascaling, 
width=5cm, 
height=5cm, 
xlabel={$x_1$}, 
ylabel={$x_2$}, 
axis lines=middle, 
xtick={\empty}, 
ytick={\empty}, 
xticklabels=\empty, 
yticklabels=\empty, 
every axis x label/.style={
	at={(ticklabel* cs:1.05)},
	anchor=west,
},
every axis y label/.style={
	at={(ticklabel* cs:1.05)},
	anchor=south,
},
clip mode = individual,
domain=-5:5, 
samples=100, 
xmin=-2, 
xmax=2, 
ymin=-2, 
ymax=2]

\def \widthA {0.9}
\draw [dkbluevect, thick, rotate=45] (0, 0) ellipse [x radius = 2 * \widthA, y radius = 1.4 * \widthA];
\draw [dkgreenvect, thick, ->] (1.5, -1.5)--(-1.5, 1.5) node[left, text=black] {\small $y_1$};
\draw [dkgreenvect, thick, ->] (1.5, 1.5)--(-1.5, -1.5) node[left, text=black] {\small $y_2$};
\draw [dkbluevect, thick, -latex] (0.4, 0) arc [start angle=0, end angle=3/4 * 180, radius=0.4] node[above, pos=0.35] {\scriptsize $\frac{3 \pi}{4}$};

\end{axis}
\end{tikzpicture}

%\captionof{figure}{\label{fig:027233}}
\end{wrapfigure}

\setlength{\rightskip}{0pt plus 200pt}
Here $a = b = c = 1$ in the notation of Theorem~\ref{thm:027179}, so $\cos\theta = \frac{-1}{\sqrt{2}}$ and $\sin\theta = \frac{1}{\sqrt{2}}$. Hence $\theta = \frac{3\pi}{4}$ will do it. The new variables are $y_{1} = \frac{1}{\sqrt{2}}(x_{2} - x_{1})$ and $y_{2} = \frac{-1}{\sqrt{2}}(x_{2} + x_{1})$ by (\ref{rotationEq2}), and the equation becomes $y_{1}^2 + 3y_{2}^2 = 2$. The angle $\theta$ has been chosen such that the new $y_{1}$ and $y_{2}$ axes are the axes of symmetry of the ellipse (see the diagram). The eigenvectors $\vect{f}_{1} = \frac{1}{\sqrt{2}} \leftB \begin{array}{r}
 -1 \\
 1
 \end{array}\rightB$
 and $\vect{f}_{2} = \frac{1}{\sqrt{2}} \leftB \begin{array}{r}
 -1 \\
 -1
 \end{array}\rightB$
 point along these axes of symmetry, and this is the reason for the name \textit{principal axes}.
\end{solution}
\end{example}

The determinant of any orthogonal matrix $P$ is either $1$ or $-1$ (because $PP^{T} = I$). The orthogonal matrices $\leftB \begin{array}{rr}
\cos\theta & -\sin\theta \\
\sin\theta & \cos\theta
\end{array}\rightB$ arising from rotations all have determinant $1$. More generally, given any quadratic form $q = \vect{x}^{T}A\vect{x}$, the orthogonal matrix $P$ such that $P^{T}AP$ is diagonal can always be chosen so that $\func{det }P = 1$ by interchanging two eigenvalues (and hence the corresponding columns of $P$). It is shown in Theorem~\ref{thm:032199} that orthogonal $2 \times 2$ matrices with determinant $1$ correspond to rotations. Similarly, it can be shown that orthogonal $3 \times 3$ matrices with determinant $1$ correspond to rotations about a line through the origin. This extends Theorem~\ref{thm:027179}: Every quadratic form in two or three variables can be diagonalized by a rotation of the coordinate system.


\subsection*{Congruence}
\index{congruence}\index{congruent matrices}\index{matrix!congruent matrices}\index{symmetric matrix!congruence}

We return to the study of quadratic forms in general.


\begin{theorem}{}{027243}
If $q(\vect{x}) = \vect{x}^{T}A\vect{x}$ is a quadratic form given by a symmetric matrix $A$, then $A$ is uniquely determined by $q$.
\end{theorem}

\begin{proof}
Let $q(\vect{x}) = \vect{x}^{T}B\vect{x}$ for all $\vect{x}$ where $B^{T} = B$. If $C = A - B$, then $C^{T} = C$ and $\vect{x}^{T}C\vect{x} = 0$ for all $\vect{x}$. We must show that $C = 0$. Given $\vect{y}$ in $\RR^n$,
\begin{align*}
0 = (\vect{x} + \vect{y})^TC(\vect{x} + \vect{y}) &= \vect{x}^TC\vect{x} + \vect{x}^TC\vect{y} + \vect{y}^TC\vect{x} + \vect{y}^TC\vect{y} \\
&= \vect{x}^TC\vect{y} + \vect{y}^TC\vect{x}
\end{align*}

But $\vect{y}^{T}C\vect{x} = (\vect{x}^{T}C\vect{y})^{T} = \vect{x}^{T}C\vect{y}$ (it is $1 \times 1$). Hence $\vect{x}^{T}C\vect{y} = 0$ for all $\vect{x}$ and $\vect{y}$ in $\RR^n$. If $\vect{e}_{j}$ is column $j$ of $I_{n}$, then the $(i, j)$-entry of $C$ is $\vect{e}_{i}^{T}C\vect{e}_{j} = 0$. Thus $C = 0$.
\end{proof}

\noindent Hence we can speak of \textit{the} symmetric matrix of a quadratic form.


On the other hand, a quadratic form $q$ in variables $x_{i}$ can be written in several ways as a linear combination of squares of new variables, even if the new variables are required to be linear combinations of the $x_{i}$. For example, if $q = 2x_{1}^2 - 4x_{1}x_{2} + x_{2}^2$ then
\begin{equation*}
q = 2(x_{1} - x_{2})^2 - x_{2}^2 \quad \mbox{ and } \quad q = -2x_{1}^2 + (2x_{1} - x_{2})^2
\end{equation*}
The question arises: How are these changes of variables related, and what properties do they share? To investigate this, we need a new concept.


Let a quadratic form $q = q(\vect{x}) = \vect{x}^{T}A\vect{x}$ be given in terms of variables $\vect{x} = (x_{1}, x_{2}, \dots, x_{n})^{T}$. If the new variables $\vect{y} = (y_{1}, y_{2}, \dots, y_{n})^{T}$ are to be linear combinations of the $x_{i}$, then $\vect{y} = A\vect{x}$ for some $n \times n$ matrix $A$. Moreover, since we want to be able to solve for the $x_{i}$ in terms of the $y_{i}$, we ask that the matrix $A$ be invertible. Hence suppose $U$ is an invertible matrix and that the new variables $\vect{y}$ are given by
\begin{equation*}
\vect{y} = U^{-1}\vect{x}, \quad \mbox{ equivalently } \vect{x} = U\vect{y}
\end{equation*}
In terms of these new variables, $q$ takes the form
\begin{equation*}
q = q(\vect{x}) = (U\vect{y})^TA(U\vect{y}) = \vect{y}^T(U^TAU)\vect{y}
\end{equation*}
That is, $q$ has matrix $U^{T}AU$ with respect to the new variables $\vect{y}$. Hence, to study changes of variables in quadratic forms, we study the following relationship on matrices: Two $n \times n$ matrices $A$ and $B$ are called \textbf{congruent}, written $A \stackrel{c}{\sim} B$, if $B = U^{T}AU$ for some invertible matrix $U$. Here are some properties of congruence:
\begin{enumerate}
	\item $A \stackrel{c}{\sim} A$ \textit{for all} $A$.
	
	\item \textit{If} $A \stackrel{c}{\sim} B$, \textit{then} $B \stackrel{c}{\sim} A$.
	
	\item \textit{If} $A \stackrel{c}{\sim} B$ \textit{and} $B \stackrel{c}{\sim} C$, \textit{then} $A \stackrel{c}{\sim} C$.
	
	\item \textit{If} $A \stackrel{c}{\sim} B$, \textit{then} $A$ \textit{is symmetric if and only if} $B$ \textit{is symmetric}.
	
	\item \textit{If} $A \stackrel{c}{\sim} B$, \textit{then} $\func{rank}A = \func{rank}B$.
\end{enumerate}

\noindent The converse to (5) can fail even for symmetric matrices.


\begin{example}{}{027315}
The symmetric matrices $A = \leftB \begin{array}{rr}
1 & 0 \\
0 & 1
\end{array}\rightB$
and $B = \leftB \begin{array}{rr}
1 & 0 \\
0 & -1
\end{array}\rightB$
 have the same $\func{rank}$ but are not congruent. Indeed, if $A \stackrel{c}{\sim} B$, an invertible matrix $U$ exists such that $B= U^{T}AU = U^{T}U$. But then $-1 = \func{det }B = (\func{det }U)^{2}$, a contradiction.
\end{example}


The key distinction between $A$ and $B$ in Example~\ref{exa:027315} is that $A$ has two positive eigenvalues (counting multiplicities) whereas $B$ has only one.


\begin{theorem}{Sylvester's Law of Inertia}{027325}
If $A \stackrel{c}{\sim} B$, then $A$ and $B$ have the same number of positive eigenvalues, counting multiplicities.\index{Sylvester's Law of Inertia}
\end{theorem}

\noindent The proof is given at the end of this section.


The \textbf{index}\index{index}\index{symmetric matrix!index} of a symmetric matrix $A$ is the number of positive eigenvalues of $A$. If $q = q(\vect{x}) = \vect{x}^{T}A\vect{x}$ is a quadratic form, the \textbf{index} and \textbf{rank}\index{symmetric matrix!rank and index}\index{rank!quadratic form} of $q$ are defined to be, respectively, the index and $\func{rank}$ of the matrix $A$. As we saw before, if the variables expressing a quadratic form $q$ are changed, the new matrix is congruent to the old one. Hence the index and $\func{rank}$ depend only on $q$ and not on the way it is expressed.


Now let $q = q(\vect{x}) = \vect{x}^{T}A\vect{x}$ be any quadratic form in $n$ variables, of index $k$ and $\func{rank }r$, where $A$ is symmetric\index{rank!symmetric matrix}. We claim that new variables $\vect{z}$ can be found so that $q$ is \textbf{completely diagonalized}\index{completely diagonalized}\index{diagonalization!completely diagonalized}---that is,
\begin{equation*}
q(\vect{z}) = z_{1}^2 + \cdots + z_{k}^2 - z_{k+1}^2 - \cdots - z_{r}^2
\end{equation*}
If $k \leq r \leq n$, let $D_{n}(k, r)$ denote the $n \times n$ diagonal matrix whose main diagonal consists of $k$ ones, followed by $r - k$ minus ones, followed by $n - r$ zeros. Then we seek new variables $\vect{z}$ such that
\begin{equation*}
q(\vect{z}) = \vect{z}^TD_{n}(k, r)\vect{z}
\end{equation*}
To determine $\vect{z}$, first diagonalize $A$ as follows: Find an orthogonal matrix $P_{0}$ such that
\begin{equation*}
P_{0}^TAP_{0} = D = \func{diag}(\lambda_{1}, \lambda_{2}, \dots, \lambda_{r}, 0, \dots, 0)
\end{equation*}
is diagonal with the nonzero eigenvalues $\lambda_{1}, \lambda_{2}, \dots, \lambda_{r}$ of $A$ on the main diagonal (followed by $n - r$ zeros). By reordering the columns of $P_{0}$, if necessary, we may assume that $\lambda_{1}, \dots, \lambda_{k}$ are positive and $\lambda_{k + 1}, \dots, \lambda_{r}$ are negative. This being the case, let $D_{0}$ be the $n \times n$ diagonal matrix
\begin{equation*}
D_{0} = \func{diag} \left( \frac{1}{\sqrt{\lambda_{1}}}, \dots, \frac{1}{\sqrt{\lambda_{k}}}, \frac{1}{\sqrt{-\lambda_{k+1}}}, \dots, \frac{1}{\sqrt{-\lambda_{r}}}, 1, \dots, 1 \right)
\end{equation*}
Then $D_{0}^{T}DD_{0} = D_{n}(k, r)$, so if new variables $\vect{z}$ are given by $\vect{x} = (P_{0}D_{0})\vect{z}$, we obtain
\begin{equation*}
q(\vect{z}) = \vect{z}^TD_{n}(k, r)\vect{z} = z_{1}^2 + \cdots + z_{k}^2 - z_{k+1}^2 - \cdots - z_{r}^2
\end{equation*}
as required. Note that the change-of-variables matrix $P_{0}D_{0}$ from $\vect{z}$ to $\vect{x}$ has orthogonal columns (in fact, scalar multiples of the columns of $P_{0}$).


\begin{example}{}{027363}
Completely diagonalize the quadratic form $q$ in Example~\ref{exa:027107} and find the index and $\func{rank}$.


\begin{solution}
  In the notation of Example~\ref{exa:027107}, the eigenvalues of the matrix $A$ of $q$ are $12$, $-8$, $4$, $4$; so the index is $3$ and the $\func{rank}$ is $4$. Moreover, the corresponding orthogonal eigenvectors are $\vect{f}_{1}$, $\vect{f}_{2}$, $\vect{f}_{3}$ (see Example~\ref{exa:027107}), and $\vect{f}_{4}$. Hence $P_{0} = \leftB \begin{array}{cccc} \vect{f}_{1} & \vect{f}_{3} & \vect{f}_{4} & \vect{f}_{2} \end{array} \rightB$ is orthogonal and
\begin{equation*}
P_{0}^TAP_{0} = \func{diag}(12, 4, 4, -8)
\end{equation*}
As before, take $D_{0} = \func{diag}(\frac{1}{\sqrt{12}}, \frac{1}{2}, \frac{1}{2}, \frac{1}{\sqrt{8}})$ and define the new variables $\vect{z}$ by $\vect{x} = (P_{0}D_{0})\vect{z}$. Hence the new variables are given by $\vect{z} = D_{0}^{-1}P_{0}^T\vect{x}$. The result is
\begin{align*}
z_{1} &= \sqrt{3}(x_{1} - x_{2} - x_{3} + x_{4}) \\
z_{2} &= x_{1} + x_{2} + x_{3} + x_{4} \\
z_{3} &= x_{1} + x_{2} - x_{3} - x_{4} \\
z_{4} &= \sqrt{2}(x_{1} - x_{2} + x_{3} - x_{4}) 
\end{align*}
\end{solution}
\end{example}

This discussion gives the following information about symmetric matrices.


\begin{theorem}{}{027386}
Let $A$ and $B$ be symmetric $n \times n$ matrices, and let $0 \leq k \leq r \leq n$.


\begin{enumerate}
\item $A$ has index $k$ and $\func{rank }r$ if and only if $A \stackrel{c}{\sim} D_{n}(k, r)$.

\item $A \stackrel{c}{\sim} B$ if and only if they have the same $\func{rank}$ and index.

\end{enumerate}
\end{theorem}

\begin{proof}
\begin{enumerate}
\item If $A$ has index $k$ and $\func{rank }r$, take $U = P_{0}D_{0}$ where $P_{0}$ and $D_{0}$ are as described prior to Example~\ref{exa:027363}. Then $U^{T}AU = D_{n}(k, r)$. The converse is true because $D_{n}(k, r)$ has index $k$ and $\func{rank }r$ (using Theorem~\ref{thm:027325}).

\item If $A$ and $B$ both have index $k$ and $\func{rank }r$, then $A \stackrel{c}{\sim} D_{n}(k, r) \stackrel{c}{\sim} B$ by (1). The converse was given earlier.

\end{enumerate}
\end{proof}

\newpage
\begin{proof}[Proof of Theorem \ref{thm:027325}]

\noindent By Theorem~\ref{thm:027060}, $A \stackrel{c}{\sim} D_{1}$ and $B \stackrel{c}{\sim} D_{2}$ where $D_{1}$ and $D_{2}$ are diagonal and have the same eigenvalues as $A$ and $B$, respectively. We have $D_{1} \stackrel{c}{\sim} D_{2}$, (because $A \stackrel{c}{\sim} B$), so we may assume that $A$ and $B$ are both diagonal. Consider the quadratic form $q(\vect{x}) = \vect{x}^{T}A\vect{x}$. If $A$ has $k$ positive eigenvalues, $q$ has the form
\begin{equation*}
q(\vect{x}) = a_{1}x_{1}^2 + \cdots + a_{k}x_{k}^2 - a_{k+1}x_{k+1}^2 - \cdots - a_{r}x_{r}^2, \ \ a_{i} > 0
\end{equation*}
where $r = \func{rank }A = \func{rank }B$. The subspace $W_{1} = \{\vect{x} \mid x_{k + 1} = \cdots = x_{r} = 0\}$ of $\RR^n$ has dimension $n - r + k$ and satisfies $q(\vect{x}) > 0$ for all $\vect{x} \neq \vect{0}$ in $W_{1}$.

On the other hand, if $B = U^{T}AU$, define new variables $\vect{y}$ by $\vect{x} = U\vect{y}$. If $B$ has $k^\prime$ positive eigenvalues, $q$ has the form
\begin{equation*}
q(\vect{x}) = b_{1}y_{1}^2 + \cdots + b_{k^\prime}y_{k^\prime}^2 - b_{k^\prime+1}y_{k^\prime+1}^2 - \cdots - b_{r}y_{r}^2, \ \ b_{i} > 0
\end{equation*}
Let $\vect{f}_{1}, \dots, \vect{f}_{n}$ denote the columns of $U$. They are a basis of $\RR^n$ and
\begin{equation*}
\vect{x} = U\vect{y} = \leftB \begin{array}{ccc}
\vect{f}_{1} & \cdots & \vect{f}_{n}
\end{array}\rightB \leftB \begin{array}{c}
y_{1} \\
\vdots \\
y_{n}
\end{array}\rightB = y_{1}\vect{f}_{1} + \cdots + y_{n}\vect{f}_{n}
\end{equation*}
Hence the subspace $W_{2} = \func{span}\{\vect{f}_{k^{\prime}+1}, \dots, \vect{f}_{r}\}$ satisfies $q(\vect{x}) < 0$ for all $\vect{x} \neq \vect{0}$ in $W_{2}$. Note $\func{dim }W_{2} = r - k^{\prime}$. It follows that $W_{1}$ and $W_{2}$ have only the zero vector in common. Hence, if $B_{1}$ and $B_{2}$ are bases of $W_{1}$ and $W_{2}$, respectively, then (Exercise~\ref{ex:ex6_3_33}) $B_{1} \cup B_{2}$ is an independent set of $(n - r + k) + (r - k^{\prime}) = n + k - k^{\prime}$ vectors in $\RR^n$. This implies that $k \leq k^{\prime}$, and a similar argument shows $k^{\prime} \leq k$.
\end{proof}
