\section{Orthogonal Diagonalization}
\label{sec:8_2}\index{diagonalization!orthogonal diagonalization}\index{orthogonal diagonalization}\index{orthogonality!orthogonal diagonalization}

Recall (Theorem~\ref{thm:016068}) that an $n \times n$ matrix $A$ is diagonalizable if and only if it has $n$ linearly independent eigenvectors. Moreover, the matrix $P$ with these eigenvectors as columns is a diagonalizing matrix for $A$, that is
\begin{equation*}
P^{-1}AP \mbox{ is diagonal.}
\end{equation*}
As we have seen, the really nice bases of $\RR^n$ are the orthogonal ones, so a natural question is: which $n \times n$ matrices have an \textit{orthogonal} basis of eigenvectors? These turn out to be precisely the symmetric matrices, and this is the main result of this section.\index{eigenvector!orthogonal basis}


Before proceeding, recall that an orthogonal set of vectors is called \textit{orthonormal} if $\vectlength\vect{v}\vectlength = 1$ for each vector $\vect{v}$ in the set, and that any orthogonal set $\{\vect{v}_{1}, \vect{v}_{2}, \dots, \vect{v}_{k}\}$ can be ``\textit{normalized}'', that is converted into an orthonormal set $\lbrace \frac{1}{\vectlength \vect{v}_{1} \vectlength}\vect{v}_{1}, \frac{1}{\vectlength \vect{v}_{2} \vectlength}\vect{v}_{2}, \dots, \frac{1}{\vectlength \vect{v}_{k} \vectlength}\vect{v}_{k} \rbrace$. In particular, if a matrix $A$ has $n$ orthogonal eigenvectors, they can (by normalizing) be taken to be orthonormal. The corresponding diagonalizing matrix $P$ has orthonormal columns, and such matrices are very easy to invert.


\begin{theorem}{}{024227}
The following conditions are equivalent for an $n \times n$ matrix $P$.


\begin{enumerate}
\item $P$ is invertible and $P^{-1} = P^{T}$.

\item The rows of $P$ are orthonormal.

\item The columns of $P$ are orthonormal.

\end{enumerate}
\end{theorem}

\begin{proof}
First recall that condition (1) is equivalent to $PP^{T} = I$ by Corollary~\ref{cor:004612} of Theorem~\ref{thm:004553}. Let $\vect{x}_{1}, \vect{x}_{2}, \dots, \vect{x}_{n}$ denote the rows of $P$. Then $\vect{x}_{j}^{T}$ is the $j$th column of $P^{T}$, so the $(i, j)$-entry of $PP^{T}$ is $\vect{x}_{i} \dotprod \vect{x}_{j}$. Thus $PP^{T} = I$ means that $\vect{x}_{i} \dotprod \vect{x}_{j} = 0$ if $i \neq j$ and $\vect{x}_{i} \dotprod \vect{x}_{j} = 1$ if $i = j$. Hence condition (1) is equivalent to (2). The proof of the equivalence of (1) and (3) is similar.
\end{proof}

\begin{definition}{Orthogonal Matrices}{024256}
An $n \times n$ matrix $P$ is called an \textbf{orthogonal matrix}\index{orthogonal matrix}\index{matrix!orthogonal matrix}\footnotemark if it satisfies one (and hence all) of the conditions in Theorem~\ref{thm:024227}.
\end{definition}
\footnotetext{In view of (2) and (3) of Theorem~\ref{thm:024227}, \textit{orthonormal matrix} might be a better name. But \textit{orthogonal matrix} is standard.\index{matrix!orthogonal matrix}\index{matrix!orthonormal matrix}\index{orthogonal matrix}\index{orthonormal matrix}}

\begin{example}{}{024259}
The rotation matrix 
$\leftB \begin{array}{rr}
\cos\theta & -\sin\theta \\
\sin\theta & \cos\theta
\end{array}\rightB$ is orthogonal for any angle $\theta$.
\end{example}

These orthogonal matrices have the 
virtue that they are easy to invert---simply take the transpose. But they 
have many other important properties as well. If $T : \RR^n \to \RR^n$ is a linear operator, we will prove (Theorem~\ref{thm:032147}) that $T$ is distance preserving if and only if its matrix is orthogonal. In particular, the matrices of rotations and reflections about the origin in $\RR^2$ and $\RR^3$ are all orthogonal (see Example~\ref{exa:024259}).


It is not enough that the rows of a matrix $A$ are merely orthogonal for $A$ to be an orthogonal matrix. Here is an example.


\begin{example}{}{024269}
The matrix $\leftB \begin{array}{rrr}
2 & 1 & 1 \\
-1 & 1 & 1 \\
0 & -1 & 1 
\end{array}\rightB$
 has orthogonal rows but the columns are not orthogonal.
However, if the rows are normalized, the resulting matrix $\def\arraystretch{1.5}\leftB \begin{array}{rrr}
\frac{2}{\sqrt{6}} & \frac{1}{\sqrt{6}} & \frac{1}{\sqrt{6}} \\
\frac{-1}{\sqrt{3}} & \frac{1}{\sqrt{3}} & \frac{1}{\sqrt{3}} \\
0 & \frac{-1}{\sqrt{2}} & \frac{1}{\sqrt{2}} 
\end{array}\rightB$
 is orthogonal (so the columns are now orthonormal as the reader can verify).
\end{example}

\begin{example}{}{024275}
If $P$ and $Q$ are orthogonal matrices, then $PQ$ is also orthogonal, as is $P^{-1} = P^{T}$.


\begin{solution}
$P$ and $Q$ are invertible, so $PQ$ is also invertible and 
\begin{equation*}
(PQ)^{-1} = Q^{-1}P^{-1} = Q^{T}P^{T} = (PQ)^{T}
\end{equation*}
Hence $PQ$ is orthogonal. Similarly, 
\begin{equation*}
(P^{-1})^{-1} = P = (P^{T})^{T} = (P^{-1})^{T}
\end{equation*}
shows that $P^{-1}$ is orthogonal.
\end{solution}
\end{example}

\begin{definition}{Orthogonally Diagonalizable Matrices}{024297}
An $n \times n$ matrix $A$ is said to be \textbf{orthogonally diagonalizable}\index{orthogonally diagonalizable} when an orthogonal matrix $P$ can be found such that  $P^{-1}AP = P^{T}AP$ is diagonal.
\end{definition}

This condition turns out to characterize the symmetric matrices.

\begin{theorem}{Principal Axes Theorem}{024303}
The following conditions are equivalent for an $n \times n$ matrix $A$.
\index{orthogonality!principal axes theorem}\index{principal axes theorem}\index{symmetric matrix!orthogonal eigenvectors}

\begin{enumerate}
\item $A$ has an orthonormal set of $n$ eigenvectors.

\item $A$ is orthogonally diagonalizable.

\item $A$ is symmetric.

\end{enumerate}
\end{theorem}

\begin{proof}
(1) $\Leftrightarrow$ (2). Given (1), let $\vect{x}_{1}, \vect{x}_{2}, \dots, \vect{x}_{n}$ be orthonormal eigenvectors of $A$. Then $P = \leftB \begin{array}{cccc} 
\vect{x}_{1} & \vect{x}_{2} & \dots & \vect{x}_{n}
\end{array}\rightB$ is orthogonal, and $P^{-1}AP$ is diagonal by Theorem~\ref{thm:009214}. This proves (2). Conversely, given (2) let $P^{-1}AP$ be diagonal where $P$ is orthogonal. If $\vect{x}_{1}, \vect{x}_{2}, \dots, \vect{x}_{n}$ are the columns of $P$ then $\{\vect{x}_{1}, \vect{x}_{2}, \dots, \vect{x}_{n}\}$ is an orthonormal basis of $\RR^n$ that consists of eigenvectors of $A$ by Theorem~\ref{thm:009214}. This proves (1).

(2) $\Rightarrow$ (3). If $P^{T}AP = D$ is diagonal, where $P^{-1} = P^{T}$, then $A = PDP^{T}$. But $D^{T} = D$, so this gives $A^{T} = P^{TT}D^{T}P^{T} = PDP^{T} = A$.

(3) $\Rightarrow$ (2). If $A$ is an $n \times n$ symmetric matrix, we proceed by induction on $n$. If $n = 1$, $A$ is already diagonal. If $n > 1$, assume that (3) $\Rightarrow$ (2) for $(n - 1) \times (n - 1)$ symmetric matrices. By Theorem~\ref{thm:016397} let $\lambda_{1}$ be a (real) eigenvalue of $A$, and let $A\vect{x}_{1} = \lambda_{1}\vect{x}_{1}$, where $\vectlength\vect{x}_{1}\vectlength = 1$. Use the Gram-Schmidt algorithm\index{Gram-Schmidt orthogonalization algorithm}\index{orthogonality!Gram-Schmidt orthogonalization algorithm} to find an orthonormal basis $\{\vect{x}_{1}, \vect{x}_{2}, \dots, \vect{x}_{n}\}$ for $\RR^n$. Let $P_{1} = \leftB \begin{array}{cccc} 
\vect{x}_{1} & \vect{x}_{2} & \dots & \vect{x}_{n}
\end{array}\rightB$, so $P_{1}$ is an orthogonal matrix and $P_{1}^TAP_{1} = \leftB \begin{array}{cc}
\lambda_{1} & B \\
0 & A_{1}
\end{array}\rightB$
 in block form by Lemma~\ref{lem:016161}. But $P_{1}^TAP_{1}$ is symmetric ($A$ is), so it follows that $B = 0$ and $A_{1}$ is symmetric. Then, by induction, there exists an $(n - 1) \times (n - 1)$ orthogonal matrix $Q$ such that $Q^{T}A_{1}Q = D_{1}$ is diagonal. Observe that $P_{2} = \leftB \begin{array}{cc}
 1 & 0\\
 0 & Q
 \end{array}\rightB$
 is orthogonal, and compute:
\begin{align*}
(P_{1}P_{2})^TA(P_{1}P_{2}) &= P_{2}^T(P_{1}^TAP_{1})P_{2} \\
&= \leftB \begin{array}{cc}
1 & 0 \\
0 & Q^T
\end{array}\rightB \leftB \begin{array}{cc}
\lambda_{1} & 0 \\
0 & A_{1}
\end{array}\rightB\leftB \begin{array}{cc}
1 & 0 \\
0 & Q
\end{array}\rightB\\
&= \leftB \begin{array}{cc}
\lambda_{1} & 0 \\
0 & D_{1}
\end{array}\rightB
\end{align*}
is diagonal. Because $P_{1}P_{2}$ is orthogonal, this proves (2).
\end{proof}

\noindent A set of orthonormal eigenvectors of a symmetric matrix $A$ is called a set of \textbf{principal axes}\index{principal axes} for $A$. The name comes from geometry, and this is discussed in Section~\ref{sec:8_8}. Because the eigenvalues of a (real) symmetric matrix are real, Theorem~\ref{thm:024303} is also called the \textbf{real spectral theorem}\index{orthogonality!real spectral theorem}\index{real spectral theorem}, and the set of distinct eigenvalues is called the \textbf{spectrum}\index{spectrum}\index{matrix!spectrum}\index{eigenvalues!spectrum of the matrix} of the matrix. In full generality, the spectral theorem is a similar result for matrices with complex entries (Theorem~\ref{thm:025860}).


\begin{example}{}{024374}
Find an orthogonal matrix $P$ such that $P^{-1}AP$ is diagonal, where $A = \leftB \begin{array}{rrr}
1 & 0 & -1 \\
0 & 1 & 2 \\
-1 & 2 & 5 
\end{array}\rightB$.

\begin{solution}
 The characteristic polynomial of $A$ is (adding twice row 1 to row 2):
\begin{equation*}
c_{A}(x) = \func{det}\leftB \begin{array}{ccc}
x - 1 & 0 & 1 \\
0 & x - 1 & -2 \\
1 & -2 & x - 5 
\end{array}\rightB = x(x - 1)(x - 6)
\end{equation*}
Thus the eigenvalues are $\lambda = 0$, $1$, and $6$, and corresponding eigenvectors are
\begin{equation*}
\vect{x}_{1} = \leftB \begin{array}{r}
1 \\
-2 \\
1
\end{array}\rightB \;
\vect{x}_{2} = \leftB \begin{array}{r}
2 \\
1 \\
0
\end{array}\rightB \;
\vect{x}_{3} = \leftB \begin{array}{r}
-1 \\
2 \\
5
\end{array}\rightB
\end{equation*}
respectively. Moreover, by what appears to be remarkably good luck, these eigenvectors are \textit{orthogonal}. We have $\vectlength\vect{x}_{1}\vectlength^{2} = 6$, $\vectlength\vect{x}_{2}\vectlength^{2} = 5$, and $\vectlength\vect{x}_{3}\vectlength^{2} = 30$, so
\begin{equation*}
P = \leftB \begin{array}{ccc}
\frac{1}{\sqrt{6}}\vect{x}_{1} & \frac{1}{\sqrt{5}}\vect{x}_{2} & \frac{1}{\sqrt{30}}\vect{x}_{3}
\end{array}\rightB = \frac{1}{\sqrt{30}} \leftB \begin{array}{ccc}
\sqrt{5} & 2\sqrt{6} & -1 \\
-2\sqrt{5} & \sqrt{6} & 2 \\
\sqrt{5} & 0 & 5 
\end{array}\rightB
\end{equation*}
is an orthogonal matrix. Thus $P^{-1} = P^{T}$ and
\begin{equation*}
P^TAP = \leftB \begin{array}{ccc}
0 & 0 & 0 \\
0 & 1 & 0 \\
0 & 0 & 6
\end{array}\rightB
\end{equation*}
by the diagonalization algorithm.
\end{solution}
\end{example}

Actually, the fact that the eigenvectors in Example~\ref{exa:024374} are orthogonal is no coincidence. Theorem~\ref{thm:016090} guarantees they are linearly independent (they correspond to distinct eigenvalues); the fact that the matrix is \textit{symmetric} implies that they are orthogonal. To prove this we need the following useful fact about symmetric matrices.\index{eigenvector!orthogonal eigenvectors}


\begin{theorem}{}{024396}
If A is an $n \times n$ symmetric matrix, then
\begin{equation*}
(A\vect{x}) \dotprod \vect{y} = \vect{x} \dotprod (A\vect{y})
\end{equation*}
for all columns $\vect{x}$ and $\vect{y}$ in $\RR^n$.\footnotemark
\end{theorem}
\footnotetext{The converse also holds (Exercise \ref{ex:8_2_15}).}

\begin{proof}
Recall that $\vect{x} \dotprod \vect{y} = \vect{x}^{T} \vect{y}$ for all columns $\vect{x}$ and $\vect{y}$. Because $A^{T} = A$, we get
\begin{equation*}
(A\vect{x}) \dotprod \vect{y} = (A\vect{x})^T\vect{y} = \vect{x}^TA^T\vect{y} =  \vect{x}^TA\vect{y} = \vect{x} \dotprod (A\vect{y})
\end{equation*}
\end{proof}

\begin{theorem}{}{024407}
If $A$ is a symmetric matrix, then eigenvectors of $A$ corresponding to distinct eigenvalues are orthogonal.
\end{theorem}

\begin{proof}
Let $A\vect{x} = \lambda \vect{x}$ and $A\vect{y} = \mu \vect{y}$, where $\lambda \neq \mu$. Using Theorem~\ref{thm:024396}, we compute
\begin{equation*}
\lambda(\vect{x} \dotprod \vect{y}) = (\lambda\vect{x}) \dotprod \vect{y} = (A\vect{x}) \dotprod \vect{y} = \vect{x} \dotprod (A\vect{y}) = \vect{x} \dotprod (\mu\vect{y}) = \mu(\vect{x} \dotprod \vect{y})
\end{equation*}
Hence $(\lambda - \mu)(\vect{x} \dotprod \vect{y}) = 0$, and so $\vect{x} \dotprod \vect{y} = 0$ because $\lambda \neq \mu$.
\end{proof}

Now the procedure for diagonalizing a symmetric $n \times n$ matrix is clear. Find the distinct eigenvalues (all real by Theorem~\ref{thm:016397})
 and find orthonormal bases for each eigenspace (the Gram-Schmidt 
algorithm\index{Gram-Schmidt orthogonalization algorithm}\index{orthogonality!Gram-Schmidt orthogonalization algorithm} may be needed). Then the set of all these basis vectors is 
orthonormal (by Theorem~\ref{thm:024407}) and contains $n$ vectors. Here is an example.


\begin{example}{}{024416}
Orthogonally diagonalize the symmetric matrix $A = \leftB \begin{array}{rrr}
8 & -2 & 2 \\
-2 & 5 & 4 \\
2 & 4 & 5
\end{array}\rightB$.


\begin{solution}
  The characteristic polynomial is
\begin{equation*}
c_{A}(x) = \func{det} \leftB \begin{array}{ccc}
x-8 & 2 & -2 \\
2 & x-5 & -4 \\
-2 & -4 & x-5
\end{array}\rightB = x(x-9)^2
\end{equation*}
Hence the distinct eigenvalues are $0$ and $9$ of multiplicities $1$ and $2$, respectively, so $\func{dim}(E_{0}) = 1$ and $\func{dim}(E_{9}) = 2$ by Theorem~\ref{thm:016250} ($A$ is diagonalizable, being symmetric). Gaussian elimination gives
\begin{equation*}
E_{0}(A) = \func{span}\{\vect{x}_{1}\}, \enskip \vect{x}_{1} = \leftB \begin{array}{r}
1 \\
2 \\
-2
\end{array}\rightB, \quad \mbox{ and } \quad E_{9}(A) = \func{span} \left\lbrace \leftB \begin{array}{r}
	-2 \\
	1 \\
	0
	\end{array}\rightB, \leftB \begin{array}{r}
2 \\
0 \\
1
\end{array}\rightB \right\rbrace
\end{equation*}
The eigenvectors in $E_{9}$ are both orthogonal to $\vect{x}_{1}$ as Theorem~\ref{thm:024407} guarantees, but not to each other. However, the Gram-Schmidt process yields an orthogonal basis
\begin{equation*}
\{\vect{x}_{2}, \vect{x}_{3}\} \mbox{ of } E_{9}(A) \quad \mbox{ where } \quad \vect{x}_{2} = \leftB \begin{array}{r}
-2 \\
1 \\
0
\end{array}\rightB \mbox{ and }  \vect{x}_{3} = \leftB \begin{array}{r}
2 \\
4 \\
5
\end{array}\rightB
\end{equation*}
Normalizing gives orthonormal vectors $\{\frac{1}{3}\vect{x}_{1}, \frac{1}{\sqrt{5}}\vect{x}_{2}, \frac{1}{3\sqrt{5}}\vect{x}_{3}\}$, so
\begin{equation*}
P = \leftB \begin{array}{rrr}
\frac{1}{3}\vect{x}_{1} & \frac{1}{\sqrt{5}}\vect{x}_{2} & \frac{1}{3\sqrt{5}}\vect{x}_{3}
\end{array}\rightB = \frac{1}{3\sqrt{5}}\leftB \begin{array}{rrr}
\sqrt{5} & -6 & 2 \\
2\sqrt{5} & 3 & 4 \\
-2\sqrt{5} & 0 & 5
\end{array}\rightB
\end{equation*}
is an orthogonal matrix such that $P^{-1}AP$ is diagonal.


It is worth noting that other, more convenient, diagonalizing matrices $P$ exist. For example, $\vect{y}_{2} = \leftB \begin{array}{r}
2 \\
1 \\
2
\end{array}\rightB$ and $\vect{y}_{3} = \leftB \begin{array}{r}
-2 \\
2 \\
1
\end{array}\rightB$
 lie in $E_{9}(A)$ and they are orthogonal. Moreover, they both have norm $3$ (as does $\vect{x}_{1}$), so
\begin{equation*}
Q = \leftB \begin{array}{ccc}
\frac{1}{3}\vect{x}_{1} & \frac{1}{3}\vect{y}_{2} & \frac{1}{3}\vect{y}_{3}
\end{array}\rightB = \frac{1}{3}\leftB \begin{array}{rrr}
1 & 2 & -2 \\
2 & 1 & 2 \\
-2 & 2 & 1
\end{array}\rightB
\end{equation*}
is a nicer orthogonal matrix with the property that $Q^{-1}AQ$ is diagonal.
\end{solution}
\end{example}

\begin{wrapfigure}[12]{l}{5cm} 
\vspace*{-2em}
\centering
\begin{tikzpicture}[scale=0.8]
%set up of axis environment
\begin{axis}[disabledatascaling, 
    width=5cm, 
    height=5cm, 
    xlabel={$x_1$}, 
    ylabel={$x_2$}, 
    axis lines=middle, 
    xtick={\empty}, 
    ytick={\empty}, 
    xticklabels=\empty, 
    yticklabels=\empty, 
    every axis x label/.style={
      at={(ticklabel* cs:1.05)},
      anchor=west,
    },
    every axis y label/.style={
      at={(ticklabel* cs:1.05)},
      anchor=south,
    },
    domain=1/2.5:2.5,
    samples=100, 
    xmin=-3.1, 
    xmax=3.1, 
    ymin=-3.1, 
    ymax=3.1]

\addplot [dkgreenvect, thick] {1/x};
\addplot [dkgreenvect, thick, domain= -2.5: -0.4] {1/x};

\node [below left] at (0, 0) {\footnotesize $O$};
\node at (1.5, -1) {\footnotesize $x_1x_2 = 1$};
\end{axis}
\end{tikzpicture}

\smallskip

\begin{tikzpicture}[scale=0.8]
%set up of axis environment
\begin{axis}[disabledatascaling, 
width=5cm, 
height=5cm, 
axis lines=middle, 
xtick={\empty}, 
ytick={\empty}, 
xticklabels=\empty, 
yticklabels=\empty, 
every axis x label/.style={
	at={(ticklabel* cs:1.05)},
	anchor=west,
},
every axis y label/.style={
	at={(ticklabel* cs:1.05)},
	anchor=south,
},
clip mode=individual,
domain=1/2.5:2.5,
samples=100, 
xmin=-3.1, 
xmax=3.1, 
ymin=-3.1, 
ymax=3.1]

\addplot [dkgreenvect, thick] {1/x};
\addplot [dkgreenvect, thick, domain= -2.5: -0.4] {1/x};
\draw [dkgreenvect, thick, ->] (-2.5, -2.5) -- (2.5, 2.5);
\draw [dkgreenvect, thick, ->] (2.5, -2.5) -- (-2.5, 2.5);

\node at (-0.3, -0.7) {\footnotesize $O$};
\node at (3, -0.8) {\footnotesize $y_{1}^2 - y_{2}^2 = 1$}; 
\node at (3, 2.5) {\footnotesize $y_1$};
\node at (-3, 2.5) {\footnotesize $y_2$};
\end{axis}
\end{tikzpicture}

%\caption{\label{fig:024448}}
\end{wrapfigure}

If $A$ is symmetric and a set of orthogonal eigenvectors of $A$ is given, the eigenvectors are called principal axes\index{eigenvector!principal axes} of $A$. The name comes from geometry. An expression $q = ax_{1}^2 + bx_{1}x_{2} + cx_{2}^2$ is called a \textbf{quadratic form}\index{orthogonality!quadratic forms}\index{quadratic form} in the variables $x_{1}$ and $x_{2}$, and the graph of the equation $q = 1$ is called a \textbf{conic} in these variables. For example, if $q = x_{1}x_{2}$, the graph of $q = 1$ is given in the first diagram.

But if we introduce new variables $y_{1}$ and $y_{2}$ by setting $x_{1} = y_{1} + y_{2}$ and $x_{2} = y_{1} - y_{2}$, then $q$ becomes $q = y_{1}^2 - y_{2}^2$, a diagonal form with no cross term $y_{1}y_{2}$ (see the second diagram). Because of this, the $y_{1}$ and $y_{2}$ axes are called the principal axes for the conic (hence the name). Orthogonal diagonalization provides a systematic method for finding principal axes. Here is an illustration.

\vspace*{2em}
\begin{example}{}{024463}
Find principal axes for the quadratic form $q = x_{1}^2 -4x_{1}x_{2} + x_{2}^2$.


\begin{solution}
  In order to utilize diagonalization, we first express $q$ in matrix form. Observe that
\begin{equation*}
q = \leftB \begin{array}{cc}
x_{1} & x_{2} 
\end{array}\rightB \leftB \begin{array}{rr}
1 & -4 \\
0 & 1 
\end{array}\rightB \leftB \begin{array}{c}
x_{1} \\
x_{2} 
\end{array}\rightB
\end{equation*}
The matrix here is not symmetric, but we can remedy that by writing
\begin{equation*}
q = x_{1}^2 -2x_{1}x_{2} - 2x_{2}x_{1} + x_{2}^2
\end{equation*}
Then we have
\begin{equation*}
q = \leftB \begin{array}{cc}
x_{1} & x_{2} 
\end{array}\rightB \leftB \begin{array}{rr}
1 & -2 \\
-2 & 1 
\end{array}\rightB \leftB \begin{array}{c}
x_{1} \\
x_{2} 
\end{array}\rightB = \vect{x}^TA\vect{x}
\end{equation*}
where $\vect{x} = \leftB \begin{array}{c}
x_{1} \\
x_{2} 
\end{array}\rightB$ and $A = \leftB \begin{array}{rr}
1 & -2 \\
-2 & 1
\end{array}\rightB$ is symmetric. The eigenvalues of $A$ are $\lambda_{1} = 3$ and $\lambda_{2} = -1$, with corresponding (orthogonal) eigenvectors $\vect{x}_{1} = \leftB \begin{array}{r}
1 \\
-1
\end{array}\rightB$
and $\vect{x}_{2} = \leftB \begin{array}{c}
1 \\
1
\end{array}\rightB$. Since $\vectlength \vect{x}_{1} \vectlength = \vectlength \vect{x}_{2} \vectlength = \sqrt{2}$, so
\begin{equation*}
P = \frac{1}{\sqrt{2}}\leftB \begin{array}{rr}
1 & 1 \\
-1 & 1
\end{array}\rightB \mbox{ is orthogonal and } P^TAP = D = \leftB \begin{array}{rr}
3 & 0 \\
0 & -1
\end{array}\rightB
\end{equation*}
Now define new variables $\leftB \begin{array}{c}
y_{1} \\
y_{2} 
\end{array}\rightB = \vect{y}$ by $\vect{y} = P^{T}\vect{x}$, equivalently $\vect{x} = P\vect{y}$ (since $P^{-1} = P^{T}$). Hence
\begin{equation*}
y_{1} = \frac{1}{\sqrt{2}}(x_{1} - x_{2}) \quad \mbox{ and } \quad y_{2} = \frac{1}{\sqrt{2}}(x_{1} + x_{2})
\end{equation*}
In terms of $y_{1}$ and $y_{2}$, $q$ takes the form
\begin{equation*}
q = \vect{x}^TA\vect{x} = (P\vect{y})^TA(P\vect{y}) = \vect{y}^T(P^TAP)\vect{y} = \vect{y}^TD\vect{y} = 3y_{1}^2 - y_{2}^2
\end{equation*}
Note that $\vect{y} = P^{T}\vect{x}$ is obtained from $\vect{x}$ by a counterclockwise rotation of $\frac{\pi}{4}$
 (see Theorem~\ref{thm:004693}).
\end{solution}
\end{example}

Observe that the quadratic form $q$ in Example~\ref{exa:024463} can be diagonalized in other ways. For example
\begin{equation*}
q = x_{1}^2 - 4x_{1}x_2 + x_{2}^2 = z_{1}^2 - \frac{1}{3}z_{2}^2
\end{equation*}
where $z_{1} = x_{1} -2x_{2}$ and $z_{2} = 3x_{2}$. We examine this more carefully in Section~\ref{sec:8_8}.


If we are willing to replace ``diagonal'' by ``upper triangular'' in the principal axes theorem, we can weaken the requirement that $A$ is symmetric to insisting only that $A$ has real eigenvalues.

\begin{theorem}{Triangulation Theorem}{024503}
If $A$ is an $n \times n$ matrix with $n$ real eigenvalues, an orthogonal matrix $P$ exists such that $P^{T}AP$ is upper triangular.\footnotemark \index{triangulation theorem}\index{orthogonality!triangulation theorem}
\end{theorem}
\footnotetext{There is also a lower triangular version.}

\begin{proof}
We modify the proof of Theorem~\ref{thm:024303}. If $A\vect{x}_{1} = \lambda_{1}\vect{x}_{1}$ where $\vectlength\vect{x}_{1}\vectlength = 1$, let $\{\vect{x}_{1}, \vect{x}_{2}, \dots, \vect{x}_{n}\}$ be an orthonormal basis of $\RR^n$, and let $P_{1} = \leftB \begin{array}{cccc}
\vect{x}_{1} & \vect{x}_{2} & \cdots &  \vect{x}_{n}
\end{array}\rightB$. Then $P_{1}$ is orthogonal and $P_{1}^TAP_{1} = \leftB \begin{array}{cc}
\lambda_{1} & B \\
0 & A_{1}
\end{array}\rightB$ in block form. By induction, let $Q^{T}A_{1}Q = T_{1}$ be upper triangular where $Q$ is of size $(n-1)\times(n-1)$ and orthogonal. Then $P_{2} = \leftB \begin{array}{cc}
1 & 0 \\
0 & Q
\end{array}\rightB$ is orthogonal, so $P = P_{1}P_{2}$ is also orthogonal and $P^TAP = \leftB \begin{array}{cc}
\lambda_{1} & BQ \\
0 & T_{1}
\end{array}\rightB$
 is upper triangular.
\end{proof}

\noindent The proof of Theorem~\ref{thm:024503} gives no way to construct the matrix $P$. However, an algorithm will be given in Section~\ref{sec:11_1} where an improved version of Theorem~\ref{thm:024503} is presented. In a different direction, a version of Theorem~\ref{thm:024503} holds for an arbitrary matrix with complex entries (Schur's theorem in Section~\ref{sec:8_6}).


As for a diagonal matrix, the eigenvalues of an upper triangular matrix are displayed along the main diagonal. Because $A$ and $P^{T}AP$ have the same determinant and trace whenever $P$ is orthogonal, Theorem~\ref{thm:024503} gives:


\begin{corollary}{}{024536}
If $A$ is an $n \times n$ matrix with real eigenvalues $\lambda_{1}, \lambda_{2}, \dots, \lambda_{n}$ (possibly not all distinct), then $\func{det}A = \lambda_{1}\lambda_{2} \dots \lambda_{n}$ and $\func{tr}A = \lambda_{1} + \lambda_{2} + \dots  + \lambda_{n}$.
\end{corollary}

\noindent This corollary remains true even if the eigenvalues are not real (using Schur's theorem).

