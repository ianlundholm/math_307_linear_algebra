\section{Determinants and Matrix Inverses}
\label{sec:3_2}

\index{inverses!determinants}
In this section, several theorems about determinants are derived. One consequence of these theorems is that a square matrix $A$ is invertible if and only if $\func{det } A \neq 0$. Moreover, determinants are used to give a formula for $A^{-1}$
 which, in turn, yields a formula (called Cramer's rule)\index{determinants!Cramer's Rule}\index{Cramer's Rule}\index{inverses!Cramer's Rule} for the 
solution of any system of linear equations with an invertible 
coefficient matrix.

We begin with a remarkable theorem (due
 to Cauchy in 1812)\index{Cauchy, Augustin Louis} about the determinant of a product of matrices. The 
proof is given at the end of this section.

\begin{theorem}{Product Theorem}{008196}
If $A$ and $B$ are $n \times n$ matrices, then $\func{det}(AB) = \func{det } A \func{det } B$.\index{determinants!product of matrices (product theorem)}\index{product!determinant of product of matrices}\index{product!theorem}
\end{theorem}

The complexity of matrix multiplication
 makes the product theorem quite unexpected. Here is an example where it
 reveals an important numerical identity.


\begin{example}{}{008202}
If $A = \leftB \begin{array}{rr} 
a & b \\
-b & a 
\end{array} \rightB$ and $B = \leftB \begin{array}{rr} 
c & d \\
-d & c 
\end{array} \rightB$
then $AB = \leftB \begin{array}{cc} 
ac-bd & ad+bc \\
-(ad+bc) & ac-bd 
\end{array} \rightB$.

\medskip

Hence $\func{det } A \func{det } B = \func{det}(AB)$ gives the identity
\begin{equation*}
(a^2 + b^2)(c^2+d^2) = (ac-bd)^2 + (ad+bc)^2
\end{equation*}
\end{example}

Theorem~\ref{thm:008196} extends easily to $\func{det}(ABC) = \func{det } A \func{det } B \func{det } C$. In fact, induction gives
\begin{equation*}
\func{det}(A_1A_2 \cdots A_{k-1}A_k) = \func{det } A_1 \func{det } A_2 \cdots \func{det } A_{k-1} \func{det } A_k
\end{equation*}
for any square matrices $A_1, \dots, A_k$ of the same size. In particular, if each $A_i = A$, we obtain
\begin{equation*}
\func{det} (A^k) = (det A)^k, \mbox{ for any } k\geq 1
\end{equation*}
We can now give the invertibility condition.\index{inverses!square matrices!application to}\index{invertibility condition}


\begin{theorem}{}{008217}
An $n \times n$ matrix $A$ is invertible if and only if $\func{det } A \neq 0$. When this is the case,
$ \func{det} (A^{-1}) = \frac{1}{\func{det } A}$ 
\end{theorem}

\begin{proof}
If $A$ is invertible, then $AA^{-1}=I$; so the product theorem gives
\begin{equation*}
1 = \func{det } I = \func{det} (AA^{-1}) = \func{det } A \func{det } A^{-1}
\end{equation*}
Hence, $\func{det } A \neq 0$ and also $\func{det } A^{-1} = \frac{1}{\func{det } A}$.

Conversely, if $\func{det } A \neq 0$, we show that $A$ can be carried to $I$ by elementary row operations (and invoke Theorem \ref{thm:004553}). Certainly, $A$ can be carried to its reduced row-echelon form $R$, so $R = E_k \cdots E_2E_1A$ where the $E_i$ are elementary matrices (Theorem \ref{thm:005294}). Hence the product theorem gives
\begin{equation*}
\func{det } R = \func{det } E_k \cdots \func{det } E_2 \func{det } E_1 \func{det } A
\end{equation*}
Since $\func{det } E \neq 0$ for all elementary matrices $E$, this shows $\func{det } R \neq 0$. In particular, $R$ has no row of zeros, so $R = I$ because $R$ is square and reduced row-echelon. This is what we wanted.
\end{proof}

\begin{example}{}{008234}
For which values of $c$ does $A = \leftB \begin{array}{rcr}
1 & 0 & -c \\
-1 & 3 & 1 \\
0 & 2c & -4
\end{array} \rightB$ 
 have an inverse?

\begin{solution}
Compute $\func{det } A$ by first adding $c$ times column 1 to column 3 and then expanding along row 1.
\begin{equation*}
\func{det } A = \func{det} \leftB \begin{array}{rcr}
1 & 0 & -c \\
-1 & 3 & 1 \\
0 & 2c & -4
\end{array} \rightB  = \func{det} \leftB \begin{array}{rcc}
1 & 0 & 0 \\
-1 & 3 & 1-c \\
0 & 2c & -4
\end{array} \rightB = 2(c+2)(c-3) 
\end{equation*}
Hence, $\func{det } A = 0$ if $c = -2$ or $c = 3$, and $A$ has an inverse if $c \neq -2$ and $c \neq 3$.
\end{solution}
\end{example}

\begin{example}{}{008243}
If a product $A_1A_2\cdots A_k$ of square matrices is invertible, show that each $A_i$ is invertible.

\begin{solution}
 We have $\func{det }A_1\func{det }A_2\cdots \func{det }A_k = \func{det}(A_1A_2\cdots A_k)$ by the product theorem, and $\func{det}(A_1A_2\cdots A_k) \neq 0$ by Theorem~\ref{thm:008217} because $A_1A_2\cdots A_k$ is invertible. Hence
\begin{equation*}
\func{det } A_1 \func{det } A_2 \cdots \func{det } A_k \neq 0
\end{equation*}
so $\func{det } A_i \neq 0$ for each $i$. This shows that each $A_i$ is invertible, again by Theorem~\ref{thm:008217}.
\end{solution}
\end{example}

\begin{theorem}{}{008268}
If $A$ is any square matrix, $\func{det } A^T = \func{det } A$.\index{determinants!square matrices}\index{square matrix ($n \times n$ matrix)!determinants}
\end{theorem}

\begin{proof}
Consider first the case of an elementary matrix $E$. If $E$ is of type I or II, then $E^T = E$; so certainly $\func{det } E^T = \func{det } E$. If $E$ is of type III, then $E^T$ is also of type III; so $\func{det } E^T = 1 = \func{det } E$ by Theorem \ref{thm:007779}. Hence, $\func{det } E^T = \func{det } E$ for every elementary matrix $E$.

Now let $A$ be any square matrix. If $A$ is not invertible, then neither is $A^T$; so $\func{det } A^T = 0 = \func{det } A$ by Theorem~\ref{thm:008217}. On the other hand, if $A$ is invertible, then $A = E_k \cdots E_2E_1$, where the $E_i$ are elementary matrices (Theorem \ref{thm:005336}). Hence, $A^T = E^T_1E^T_2 \cdots E^T_k$ so the product theorem gives


\begin{align*}
\func{det } A^T = \func{det } E^T_1 \func{det } E^T_2 \cdots \func{det } E^T_k & = \func{det } E_1 \func{det } E_2 \cdots \func{det } E_k \\
&= \func{det } E_k \cdots \func{det } E_2 \func{det } E_1 \\
&= \func{det } A
\end{align*}
This completes the proof.
\end{proof}

\begin{example}{}{008291 }
If $\func{det } A = 2$ and $\func{det } B = 5$, calculate $\func{det}(A^3 B^{-1}A^TB^2)$.
\begin{solution}
 We use several of the facts just derived.
\begin{align*}
\func{det}(A^3 B^{-1}A^TB^2) &= \func{det} (A^3) \func{det}(B^{-1}) \func{det}(A^T) \func{det} (B^2)\\
&= (\func{det } A)^3 \frac{1}{\func{det } B} \func{det } A (\func{det } B)^2 \\
&= 2^3 \cdot \frac{1}{5} \cdot 2 \cdot 5^2 \\
&= 80
\end{align*}
\end{solution}
\end{example}

\begin{example}{}{008302}
A square matrix is called \textbf{orthogonal}\index{matrix!orthogonal matrix}\index{orthogonal matrix}\index{square matrix ($n \times n$ matrix)!orthogonal matrix} if $A^{-1} = A^T$. What are the possible values of $\func{det } A$ if $A$ is orthogonal?


\begin{solution}
  If $A$ is orthogonal, we have $I = AA^T$. Take determinants to obtain 
\begin{equation*}
1 = \func{det } I = \func{det}(AA^T) = \func{det } A \func{det } A^T = (\func{det } A)^2
\end{equation*}
Since $\func{det } A$ is a number, this means $\func{det } A = \pm 1$.
\end{solution}
\end{example}

Hence Theorems \ref{thm:006021} and \ref{thm:006096} imply that rotation about the origin and reflection about a line through the origin in $\RR^2$ have orthogonal matrices with determinants $1$ and $-1$ respectively. In fact they are the \textit{only} such transformations of  $\RR^2$. We have more to say about this in Section~\ref{sec:8_2}.\index{reflections!about a line through the origin}\index{rotations!about the origin!and orthogonal matrices}


\subsection*{Adjugates}

In Section~\ref{sec:2_4} we defined the adjugate of a 2 $\times$ 2 matrix $A = \leftB \begin{array}{rr}
a & b \\
c & d 
\end{array} \rightB$ 
 to be $\func{adj} (A) = \leftB \begin{array}{rr}
d & -b \\
-c & a 
\end{array} \rightB$.
 Then we verified that $A (\func{adj } A) = (\func{det } A)I = (\func{adj } A)A$ and hence that, if $\func{det} A \neq 0$, $A^{-1} = \frac{1}{\func{det } A} \func{adj } A$. We are now able to define the adjugate of an arbitrary square matrix 
and to show that this formula for the inverse remains valid (when the 
inverse exists).\index{determinants!adjugate}\index{adjugate}\index{inverses!adjugate}


Recall that the $(i, j)$-cofactor $c_{ij}(A)$ of a square matrix $A$ is a number defined for each position $(i, j)$ in the matrix. If $A$ is a square matrix, the \textbf{cofactor matrix of} $A$ is defined to be the matrix $\leftB c_{ij}(A)\rightB$ whose $(i, j)$-entry is the $(i, j)$-cofactor of $A$. \index{cofactor matrix}\index{square matrix ($n \times n$ matrix)!cofactor matrix}


\begin{definition}{Adjugate of a Matrix}{008326}
The \textbf{adjugate}\footnotemark
of $A$, denoted $\func{adj }(A)$, is the transpose of this cofactor matrix; in symbols,
\begin{equation*}
\func{adj}(A) = \leftB c_{ij}(A) \rightB^T
\end{equation*}
\end{definition}
\footnotetext{This is also called the classical adjoint of $A$, but the term ``adjoint'' has another meaning.\index{classical adjoint}}

This agrees with the earlier definition for a $2 \times 2$ matrix $A$ as the reader can verify.


\begin{example}{}{008331}
Compute the adjugate of $A = \leftB \begin{array}{rrr}
1 & 3 & -2 \\
0 & 1 & 5 \\
-2 & -6 & 7 
\end{array}\rightB$
 and calculate $A (\func{adj } A)$ and $(\func{adj } A)A$.


\begin{solution}
 We first find the cofactor matrix.

\begin{align*}
\leftB \begin{array}{rrr}
c_{11}(A) & c_{12}(A) & c_{13}(A) \\
c_{21}(A) & c_{22}(A) & c_{23}(A) \\
c_{31}(A) & c_{32}(A) & c_{33}(A) 
\end{array}\rightB
&= \leftB \begin{array}{ccc}
\left| \begin{array}{rr}
1 & 5 \\
-6 & 7 
\end{array}\right| & -\left| \begin{array}{rr}
0 & 5 \\
-2 & 7 
\end{array}\right| &
\left| \begin{array}{rr}
0 & 1 \\
-2 & -6 
\end{array}\right| \\
& & \\
-\left| \begin{array}{rr}
3 & -2 \\
-6 & 7 
\end{array}\right| & \left| \begin{array}{rr}
1 & -2 \\
-2 & 7 
\end{array}\right| &
-\left| \begin{array}{rr}
1 & 3 \\
-2 & -6 
\end{array}\right| \\
& & \\
\left| \begin{array}{rr}
3 & -2 \\
1 & 5 
\end{array}\right| & -\left| \begin{array}{rr}
1 & -2 \\
0 & 5 
\end{array}\right| &
\left| \begin{array}{rr}
1 & 3 \\
0 & 1 
\end{array}\right| 
\end{array}\rightB \\
&= \leftB \begin{array}{rrr}
37 & -10 & 2 \\
-9 & 3 & 0 \\
17 & -5 & 1 
\end{array}\rightB
\end{align*}

Then the adjugate of $A$ is the transpose of this cofactor matrix.
\begin{equation*}
\func{adj } A =  \leftB \begin{array}{rrr}
37 & -10 & 2 \\
-9 & 3 & 0 \\
17 & -5 & 1 
\end{array}\rightB ^T = \leftB \begin{array}{rrr}
37 & -9 & 17 \\
-10 & 3 & -5 \\
2 & 0 & 1 
\end{array}\rightB
\end{equation*}
The computation of $A (\func{adj} A)$ gives
\begin{equation*}
A(\func{adj } A) = \leftB \begin{array}{rrr}
1 & 3 & -2 \\
0 & 1 & 5 \\
-2 & -6 & 7
\end{array}\rightB \leftB \begin{array}{rrr}
37 & -9 & 17 \\
-10 & 3 & -5 \\
2 & 0 & 1 
\end{array}\rightB = \leftB \begin{array}{rrr}
3 & 0 & 0 \\
0 & 3 & 0 \\
0 & 0 & 3 
\end{array}\rightB = 3I
\end{equation*}
and the reader can verify that also $(\func{adj } A)A = 3I$. Hence, analogy with the $2 \times 2$ case would indicate that $\func{det } A = 3$; this is, in fact, the case.
\end{solution}
\end{example}

The relationship $A(\func{adj } A) = (\func{det }A)I$ holds for any square matrix $A$. To see why this is so, consider the general $3 \times 3$ case. Writing $c_{ij}(A) = c_{ij}$ for short, we have
\begin{equation*}
\func{adj } A = \leftB \begin{array}{rrr}
c_{11} & c_{12} & c_{13} \\
c_{21} & c_{22} & c_{23} \\
c_{31} & c_{32} & c_{33} 
\end{array}\rightB^T = \leftB \begin{array}{rrr}
c_{11} & c_{21} & c_{31} \\
c_{12} & c_{22} & c_{32} \\
c_{13} & c_{23} & c_{33} 
\end{array}\rightB
\end{equation*}
If $A = \leftB a_{ij} \rightB$ in the usual notation, we are to verify that $A(\func{adj }A) = (\func{det }A)I$. That is,
\begin{equation*}
A(\func{adj } A) =\leftB \begin{array}{rrr}
a_{11} & a_{12} & a_{13} \\
a_{21} & a_{22} & a_{23} \\
a_{31} & a_{32} & a_{33} 
\end{array}\rightB \leftB \begin{array}{rrr}
c_{11} & c_{21} & c_{31} \\
c_{12} & c_{22} & c_{32} \\
c_{13} & c_{23} & c_{33}
\end{array}\rightB
 = \leftB \begin{array}{ccc}
\func{det } A & 0 & 0 \\
0 & \func{det } A & 0 \\
0 & 0 & \func{det } A
\end{array}\rightB
\end{equation*}
Consider the $(1, 1)$-entry in the product. It is given by $a_{11}c_{11} + a_{12}c_{12} + a_{13}c_{13}$, and this is just the cofactor expansion of $\func{det }A$ along the first row of $A$. Similarly, the $(2, 2)$-entry and the $(3, 3)$-entry are the cofactor expansions of $\func{det }A$ along rows $2$ and $3$, respectively.


So it remains to be seen why the off-diagonal elements in the matrix product $A(\func{adj }A)$ are all zero. Consider the $(1, 2)$-entry of the product. It is given by $a_{11}c_{21} + a_{12}c_{22} + a_{13}c_{23}$. This \textit{looks} like the cofactor expansion of the determinant of \textit{some} matrix. To see which, observe that $c_{21}$, $c_{22}$, and $c_{23}$ are all computed by \textit{deleting} row $2$ of $A$ (and one of the columns), so they remain the same if row $2$ of $A$ is changed. In particular, if row $2$ of $A$ is replaced by row $1$, we obtain
\begin{equation*}
a_{11}c_{21}+a_{12}c_{22}+a_{13}c_{23} = \func{det} \leftB \begin{array}{rrr}
a_{11} & a_{12} & a_{13} \\
a_{11} & a_{12} & a_{13} \\
a_{31} & a_{32} & a_{33} 
\end{array}\rightB = 0
\end{equation*}
where the expansion is along row $2$ and where the determinant is zero 
because two rows are identical. A similar argument shows that the other 
off-diagonal entries are zero.


This argument works in general and yields the first part of Theorem~\ref{thm:008370}. The second assertion follows from the first by multiplying through by the scalar $\frac{1}{\func{det } A}$.

\begin{theorem}{Adjugate Formula}{008370}
If A is any square matrix, then
\begin{equation*}
A(\func{adj } A) = (\func{det } A)I = (\func{adj } A)A
\end{equation*}
In particular, if det A $\neq$ 0, the inverse of A is given by
\begin{equation*}
A^{-1} =\frac{1}{\func{det } A}\func{adj } A
\end{equation*}\index{adjugate formula}
\end{theorem}

\noindent It is important to note that this theorem is \textit{not} an efficient way to find the inverse of the matrix $A$. For example, if $A$ were $10 \times 10$, the calculation of $\func{adj } A$ would require computing $10^{2} = 100$ determinants of $9 \times 9$ matrices! On the other hand, the matrix inversion algorithm would find $A^{-1}$ with about the same effort as finding $\func{det }A$. Clearly, Theorem~\ref{thm:008370} is not a \textit{practical} result: its virtue is that it gives a formula for $A^{-1}$ that is useful for \textit{theoretical} purposes.


\begin{example}{}{008382}
Find the $(2, 3)$-entry of $A^{-1}$ if  $A = \leftB \begin{array}{rrr}
2 & 1 & 3 \\
5 & -7 & 1 \\
3 & 0 & -6 
\end{array}\rightB$.

\begin{solution}
  First compute 
\begin{equation*}
\func{det } A = \left| \begin{array}{rrr}
2 & 1 & 3 \\
5 & -7 & 1 \\
3 & 0 & -6 
\end{array} \right| = \left| \begin{array}{rrr}
2 & 1 & 7 \\
5 & -7 & 11 \\
3 & 0 & 0
\end{array} \right| = 
3 \left| \begin{array}{rr}
1 & 7 \\
-7 & 11 
\end{array}\right| = 180
\end{equation*}
Since $A^{-1} = \frac{1}{\func{det } A}\func{adj } A = \frac{1}{180} \leftB c_{ij}(A)\rightB ^T$,
 the $(2, 3)$-entry of $A^{-1}$ is the $(3, 2)$-entry of the matrix $\frac{1}{180}\leftB c_{ij}(A) \rightB$; that is, it equals 
$\frac{1}{180} c_{32} (A) = \frac{1}{180} \left( - \left| \begin{array}{rr}
2 & 3 \\
5 & 1 
\end{array}\right| \right) = \frac{13}{180}.$
\end{solution}
\end{example}


\begin{example}{}{008396}
If $A$ is $n \times n$, $n \geq 2$, show that $\func{det}(\func{adj }A) = (\func{det }A)^{n-1}$.


\begin{solution}
  Write $d = \func{det } A$; we must show that $\func{det}(\func{adj }A) = d^{n-1}$. We have $A(\func{adj }A) = dI$ by Theorem~\ref{thm:008370}, so taking determinants gives $d\func{det}(\func{adj }A) = d^{n}$. Hence we are done if $d \neq 0$. Assume $d = 0$; we must show that $\func{det}(\func{adj }A) = 0$, that is, $\func{adj }A$ is not invertible. If $A \neq 0$, this follows from $A(\func{adj }A) = dI = 0$; if $A = 0$, it follows because then $\func{adj }A = 0$.
\end{solution}
\end{example}

\subsection*{Cramer's Rule}
\index{linear equation!Cramer's Rule}

Theorem~\ref{thm:008370} has a nice application to linear equations. Suppose
\begin{equation*}
A\vect{x}=\vect{b}
\end{equation*}
is a system of $n$ equations in $n$ variables $x_{1}, x_{2}, \dots , x_{n}$. Here $A$ is the $n \times n$ coefficient matrix\index{matrix!coefficient matrix}, and $\vect{x}$ and $\vect{b}$ are the columns
\begin{equation*}
\vect{x} = \leftB \begin{array}{c}
x_1 \\
x_2 \\
\vdots \\
x_n 
\end{array} \rightB \mbox{ and }
\vect{b} = \leftB \begin{array}{c}
b_1 \\
b_2 \\
\vdots \\
b_n 
\end{array} \rightB 
\end{equation*}
of variables and constants, respectively. If $\func{det }A \neq 0$, we left multiply by $A^{-1}$ to obtain the solution $\vect{x} = A^{-1}\vect{b}$. When we use the adjugate formula, this becomes
\begin{align*}
\leftB \begin{array}{c}
x_1 \\
x_2 \\
\vdots \\
x_n 
\end{array} \rightB &= \frac{1}{\func{det } A} (\func{adj } A)\vect{b} \\
&= \frac{1}{\func{det } A}
\leftB \begin{array}{cccc}
c_{11}(A) & c_{21}(A) & \cdots & c_{n1}(A) \\
c_{12}(A) & c_{22}(A) & \cdots & c_{n2}(A) \\
\vdots & \vdots & & \vdots \\
c_{1n}(A) & c_{2n}(A) & \cdots & c_{nn}(A)
\end{array}\rightB
 \leftB \begin{array}{c}
b_1 \\
b_2 \\
\vdots \\
b_n 
\end{array} \rightB 
\end{align*}
Hence, the variables $x_{1}, x_{2}, \dots, x_{n}$ are given by
\begin{align*}
x_1 &= \frac{1}{\func{det } A} \leftB b_1c_{11}(A) + b_2c_{21}(A) + \cdots + b_nc_{n1}(A)\rightB \\
x_2 &= \frac{1}{\func{det } A} \leftB b_1c_{12}(A) + b_2c_{22}(A) + \cdots + b_nc_{n2}(A)\rightB \\
& \hspace{5em} \vdots \hspace{5em} \vdots\\
x_n &= \frac{1}{\func{det } A} \leftB b_1c_{1n}(A) + b_2c_{2n}(A) + \cdots + b_nc_{nn}(A)\rightB 
\end{align*}

Now the quantity $b_{1}c_{11}(A) + b_{2}c_{21}(A) + \cdots + b_{n}c_{n1}(A)$ occurring in the formula for $x_{1}$ looks like the cofactor expansion of the determinant of a matrix. The cofactors involved are $c_{11}(A), c_{21}(A), \dots, c_{n1}(A)$, corresponding to the first column of $A$. If $A_{1}$ is obtained from $A$ by replacing the first column of $A$ by $\vect{b}$, then $c_{i1}(A_{1}) = c_{i1}(A)$ for each $i$ because column $1$ is deleted when computing them. Hence, expanding $\func{det}(A_{1})$ by the first column gives
\begin{align*}
\func{det } A_1 &= b_1c_{11}(A_1) + b_2c_{21}(A_1) + \cdots + b_nc_{n1}(A_1) \\
 	 &= b_1c_{11}(A) + b_2c_{21}(A) + \cdots + b_nc_{n1}(A)\\
	 &= (\func{det } A)x_1 
\end{align*}
Hence, $x_1 = \frac{\func{det } A_1}{\func{det } A}$
 and similar results hold for the other variables.

\begin{theorem}{Cramer's Rule\footnotemark}{008446}
If $A$ is an invertible $n \times n$ matrix, the solution to the system
\begin{equation*}
A\vect{x}=\vect{b}
\end{equation*}
of $n$ equations in the variables $x_{1}, x_{2}, \dots, x_{n}$ is given by
\begin{equation*}
x_1 = \frac{\func{det } A_1}{\func{det } A}, \; x_2 = \frac{\func{det } A_2}{\func{det } A}, \;\cdots, \; x_n = \frac{\func{det } A_n}{\func{det } A}
\end{equation*}
where, for each $k$, $A_k$ is the matrix obtained from $A$ by replacing column $k$ by $\vect{b}$. \index{inverses!Cramer's Rule}
\end{theorem}
\footnotetext{Gabriel
 Cramer (1704--1752)\index{Cramer, Gabriel} was a Swiss mathematician who wrote an introductory 
work on algebraic curves. He popularized the rule that bears his name, 
but the idea was known earlier.}

\begin{example}{}{008457}
Find $x_{1}$, given the following system of equations.
\begin{equation*}
\arraycolsep=1pt
\begin{array}{rrrrrrr}
 	 5x_1 & + & x_2 & - & x_3  & = & 4\\
	9x_1& + & x_2 & - & x_3 & = & 1 \\
	x_1& - & x_2 & + & 5x_3 & = & 2 \\
\end{array}
\end{equation*}
\vspace{0em}
\begin{solution}
  Compute the determinants of the coefficient matrix $A$ and the matrix $A_{1}$ obtained from it by replacing the first column by the column of constants.
\begin{align*}
\func{det } A &= \func{det} \leftB \begin{array}{rrr}
5 & 1 & -1 \\
9 & 1 & -1 \\
1 & -1 & 5 
\end{array}\rightB  = -16 \\
\func{det } A_1 &= \func{det} \leftB \begin{array}{rrr}
4 & 1 & -1 \\
1 & 1 & -1 \\
2 & -1 & 5 
\end{array}\rightB  = 12
\end{align*}
Hence, $x_1 = \frac{\func{det } A_1}{\func{det } A} = -\frac{3}{4}$
 by Cramer's rule.
\end{solution}
\end{example}

Cramer's rule is \textit{not} an efficient way to solve linear systems or invert matrices. True, it enabled us to calculate $x_{1}$ here without computing $x_{2}$ or $x_{3}$.
 Although this might seem an advantage, the truth of the matter is that,
 for large systems of equations, the number of computations needed to 
find \textit{all} the variables by the gaussian algorithm is comparable to the number required to find \textit{one}  of the determinants involved in Cramer's rule. Furthermore, the 
algorithm works when the matrix of the system is not invertible and even
 when the coefficient matrix is not square. Like the adjugate formula, 
then, Cramer's rule is \textit{not} a practical numerical technique; its virtue is theoretical.

\subsection*{Polynomial Interpolation}
\index{determinants!polynomial interpolation}\index{polynomials!interpolating the polynomial}

Given a set of data, it is often the case that one is interested to understand a trend so to forecast other values. One such method is that of modeling the trend with a polynomial, here is an example. 

\begin{example}{}{008475}
A forester wants to estimate the age  (in years) of a tree by measuring the diameter of the trunk (in cm). She
 obtains the following data:
 
\begin{wrapfigure}{l}{5cm}

\begin{tikzpicture}
%set up of axis environment
\begin{axis}[disabledatascaling,
    clip mode = individual,
    width=5cm, 
    height=4cm, 
    font=\footnotesize,
    clip mode=individual,
    xlabel={Diameter}, 
    ylabel={Age}, 
    axis lines=middle,
    hide obscured x ticks=false,
    enlargelimits=true,
    xtick={0,5,9.5,12,15}, 
    ytick={0,1,2,...,6}, 
    xticklabels={0,5,10,12,15}, 
    yticklabels={,,2,,4,,6}, 
    every axis x label/.style={
      at={(ticklabel* cs:1.05)},
      anchor=south,
    },
    every axis y label/.style={
      at={(ticklabel* cs:1.05)},
      anchor=west,
    },
	domain=4:17,
	samples=100,
    xmin=0, 
    xmax=15, 
    ymin=0, 
    ymax=6, 
    ]
\addplot[dkgreenvect, thick] {7/10*\x - 1/50*pow(x,2)};
\fill[black] (5,3) circle (1pt);
\fill[black] (9.5,4.845) circle (1pt);
\fill[black] (12,5.52) circle (1pt);
\fill[black] (15,6) circle (1pt);
\draw[dashed,dkbluevect](12,0)--(12,5.52);
\node[above, font=\footnotesize] at (3.9,3) {$(5,3)$};
\node[above, font=\footnotesize] at (8.2,5) {$(10,5)$};
\node[above, font=\footnotesize] at (15,6) {$(15,6)$};
\end{axis}
\end{tikzpicture}

%\captionof{figure}{\label{fig:008478}}
\end{wrapfigure} 
\begin{center}
\hspace*{0cm}
\begin{tabu} to 15cm {|l|c|c|c|}
\hline
& Tree 1 & Tree 2 & Tree 3 \\ \hline
Trunk Diameter & $5$ & $10$ & $15$ \\
Age & $3$ & $5$ & $6$ \\
\hline
\end{tabu}
\end{center}

Use this date to estimate the age of a tree with a trunk diameter of 12 cm.

\begin{solution}
The forester decides to ``fit'' a quadratic polynomial
\begin{equation*}
p(x) = r_0 + r_1x + r_2x^2
\end{equation*}
to the data, that is choose the coefficients $r_{0}$, $r_{1}$, and $r_{2}$ so that $p(5) = 3$, $p(10) = 5$, and $p(15) = 6$, and then use $p(12)$ as the estimate. These conditions give three linear equations:
\begin{equation*}
\arraycolsep=1pt
\begin{array}{rrrrrrr}
 	 r_0 & + & 5r_1 & + & 25r_2  & = & 3\\
 	 r_0 & + & 10r_1 & + & 100r_2  & = & 5\\
 	 r_0 & + & 15r_1 & + & 225r_2  & = & 6
\end{array}
\end{equation*}
The (unique) solution is $r_0=0, r_1 = \frac{7}{10},$ and $r_2 = -\frac{1}{50}$
, so
\begin{equation*}
p(x) = \frac{7}{10}x - \frac{1}{50}x^2 = \frac{1}{50}x (35-x)
\end{equation*}
Hence the estimate is $p(12) = 5.52$.
\end{solution}
\end{example}

As in Example~\ref{exa:008475}, it often happens that two variables $x$ and $y$ are related but the actual functional form $y = f(x)$ of the relationship is unknown. Suppose that for certain values $x_{1}, x_{2}, \dots, x_{n}$ of $x$ the corresponding values $y_{1}, y_{2}, \dots, y_{n}$ are known (say from experimental measurements). One way to estimate the value of $y$ corresponding to some other value $a$ of $x$ is to find a polynomial\footnote{A \textbf{polynomial} is an expression of the form $a_{0} + a_{1}x + a_{2}x^{2} + \dots  + a_{n}x^{n}$ where the $a_{i}$ are numbers and $x$ is a variable. If $a_{n} \neq 0$, the integer $n$ is called the degree of the polynomial, and $a_{n}$ is called the leading coefficient\index{coefficients!leading coefficient}. See Appendix~\ref{chap:appdpolynomials}.}
\begin{equation*}
p(x) = r_0 + r_1x + r_2x^2 + \cdots + r_{n-1}x^{n-1}
\end{equation*}
that ``fits'' the data, that is $p(x_{i}) = y_{i}$ holds for each $i = 1, 2, \dots , n$. Then the estimate for $y$ is $p(a)$. As we will see, such a polynomial always exists if the $x_{i}$ are distinct.


The conditions that $p(x_{i}) = y_{i}$ are
\begin{equation*}
\arraycolsep=1pt
\def\arraystretch{1.25}
\begin{array}{ccccccccccc}
r_0 & + & r_1x_1 & + & r_2x_1^2 & + & \cdots & + & r_{n-1}x_1^{n-1} & = &y_1 \\
r_0 & + & r_1x_2 & + & r_2x_2^2 & + & \cdots & + & r_{n-1}x_2^{n-1} & = &y_2 \\
 & \vdots & & \vdots & & \vdots & & & \vdots & \vdots & \\
r_0 & + & r_1x_n & + & r_2x_n^2 & + & \cdots & + & r_{n-1}x_n^{n-1} & = & y_n 
\end{array}
\end{equation*}
In matrix form, this is
\begin{equation}\label{eq:polymatrixform}
\leftB \begin{array}{ccccc}
1 & x_1 & x_1^2 & \cdots & x_1^{n-1}\\
1 & x_2 & x_2^2 & \cdots & x_2^{n-1}\\
\vdots & \vdots & \vdots & & \vdots \\
1 & x_n & x_n^2 & \cdots & x_n^{n-1}
\end{array}\rightB
\leftB \begin{array}{c}
r_0 \\
r_1 \\
\vdots \\
r_{n-1}
\end{array} \rightB
=
\leftB \begin{array}{c}
y_1 \\
y_2 \\
\vdots \\
y_n
\end{array} \rightB
\end{equation}
It can be shown (see Theorem~\ref{thm:008552}) that the determinant of the coefficient matrix\index{determinants!coefficient matrix}\index{coefficient matrix} equals the product of all terms $(x_{i} - x_{j})$ with $i > j$ and so is nonzero (because the $x_{i}$ are distinct). Hence the equations have a unique solution $r_{0}, r_{1}, \dots, r_{n-1}$. This proves


\begin{theorem}{}{008520}
Let $n$ data pairs $(x_{1}, y_{1}), (x_{2}, y_{2}), \dots, (x_{n}, y_{n})$ be given, and assume that the $x_{i}$ are distinct. Then there exists a unique polynomial
\begin{equation*}
p(x) = r_0 + r_1x + r_2x^2 + \cdots + r_{n-1}x^{n-1}
\end{equation*}
such that $p(x_{i}) = y_{i}$ for each $i = 1, 2, \dots, n$.
\end{theorem}

\noindent The polynomial in Theorem~\ref{thm:008520} is called the \textbf{interpolating polynomial}\index{interpolating polynomial} for the data.


We conclude by evaluating the determinant of the coefficient matrix in Equation \ref{eq:polymatrixform}. If $a_{1}, a_{2}, \dots, a_{n}$ are numbers, the determinant
\begin{equation*}
\func{det} \leftB \begin{array}{ccccc}
1 & a_1 & a_1^2 & \cdots & a_1^{n-1}\\
1 & a_2 & a_2^2 & \cdots & a_2^{n-1}\\
1 & a_3 & a_3^2 & \cdots & a_3^{n-1}\\
\vdots & \vdots & \vdots & & \vdots \\
1 & a_n & a_n^2 & \cdots & a_n^{n-1}\\
\end{array}\rightB 
\end{equation*}
is called a \textbf{Vandermonde determinant}\index{determinants!Vandermonde determinant}\index{Vandermonde determinant}.\footnote{Alexandre Th\'{e}ophile Vandermonde (1735--1796) was a French mathematician who made contributions to the theory of equations.}
 There is a simple formula for this determinant. If $n = 2$, it equals $(a_{2} - a_{1})$; if $n = 3$, it is $(a_{3} - a_{2})(a_{3} - a_{1})(a_{2} - a_{1})$ by Example \ref{exa:007851}. The general result is the product
\begin{equation*}
\prod_{ 1 \le j < i \le n} (a_i - a_j)
\end{equation*}
of all factors $(a_{i} - a_{j})$ where $1 \leq j < i \leq n$. For example, if $n = 4$, it is
\begin{equation*}
(a_4-a_3)(a_4-a_2)(a_4-a_1)(a_3-a_2)(a_3-a_1)(a_2-a_1)
\end{equation*}
\begin{theorem}{}{008552}
Let $a_{1}, a_{2}, \dots , a_{n}$ be numbers where $n \geq 2$. Then the corresponding Vandermonde determinant is given by
\begin{equation*}
\func{det} \leftB \begin{array}{ccccc}
1 & a_1 & a_1^2 & \cdots & a_1^{n-1}\\
1 & a_2 & a_2^2 & \cdots & a_2^{n-1}\\
1 & a_3 & a_3^2 & \cdots & a_3^{n-1}\\
\vdots & \vdots & \vdots & & \vdots \\
1 & a_n & a_n^2 & \cdots & a_n^{n-1}\\
\end{array}\rightB = \prod_{1 \le j < i \le n}(a_i - a_j)
\end{equation*}
\end{theorem}

\begin{proof}
We may assume that the $a_{i}$ are distinct; otherwise both sides are zero. We proceed by induction on $n \geq 2$; we have it for $n = 2, 3$. So assume it holds for $n - 1$. The trick is to replace $a_{n}$ by a variable $x$, and consider the determinant
\begin{equation*}
p(x) = \func{det} \leftB \begin{array}{ccccc}
1 & a_1 & a_1^2 & \cdots & a_1^{n-1} \\
1 & a_2 & a_2^2 & \cdots & a_2^{n-1} \\
\vdots & \vdots & \vdots & & \vdots \\
1 & a_{n-1} & a_{n-1}^2 & \cdots & a_{n-1}^{n-1} \\
1 & x & x^2 & \cdots & x^{n-1} 
\end{array} \rightB
\end{equation*}
Then $p(x)$ is a polynomial of degree at most $n - 1$ (expand along the last row), and $p(a_{i}) = 0$ for each $i = 1, 2, \dots, n - 1$ because in each case there are two identical rows in the determinant. In particular, $p(a_{1}) = 0$, so we have $p(x) = (x - a_{1})p_{1}(x)$ by the factor theorem (see Appendix~\ref{chap:appdpolynomials}). Since $a_{2} \neq a_{1}$, we obtain $p_{1}(a_{2}) = 0$, and so $p_{1}(x) = (x - a_{2})p_{2}(x)$. Thus $p(x) = (x - a_{1})(x - a_{2})p_{2}(x)$. As the $a_{i}$ are distinct, this process continues to obtain
\begin{equation}\label{eq:polynomial}
p(x) = (x-a_1)(x-a_2) \cdots (x-a_{n-1})d 
\end{equation}
where $d$ is the coefficient of $x^{n-1}$ in $p(x)$. By the cofactor expansion of $p(x)$ along the last row we get
\begin{equation*}
d = (-1)^{n+n}\func{det} \leftB \begin{array}{ccccc}
1 & a_1 & a_1^2 & \cdots & a_1^{n-2} \\
1 & a_2 & a_2^2 & \cdots & a_2^{n-2} \\
\vdots & \vdots & \vdots & & \vdots \\
1 & a_{n-1} & a_{n-1}^2 & \cdots & a_{n-1}^{n-2} 
\end{array}\rightB
\end{equation*}
Because $(-1)^{n+n}=1$ the induction hypothesis shows that $d$ is the product of all factors $(a_{i} - a_{j})$ where $1 \leq j < i \leq n - 1$. The result now follows from Equation \ref{eq:polynomial} by substituting $a_{n}$ for $x$ in $p(x)$.
\end{proof}

\begin{proof}[Proof of Theorem \ref{thm:008196}]
\index{determinants!product of matrices (product theorem)}\index{product!theorem}
If $A$ and $B$ are $n \times n$ matrices we must show that
\begin{equation}\label{eq:polyproof1}
\func{det} (AB) = \func{det } A \func{det } B
\end{equation}
Recall that if $E$ is an elementary matrix obtained by doing one row operation to $I_{n}$, then doing that operation to a matrix $C$ (Lemma \ref{lem:005213}) results in $EC$. By looking at the three types of elementary matrices separately, Theorem \ref{thm:007779} shows that
\begin{equation}\label{eq:polyproof2}
\func{det} (EC) = \func{det } E \func{det } C \quad \mbox{ for any matrix } C
\end{equation}
Thus if $E_{1}, E_{2}, \dots, E_{k}$ are all elementary matrices, it follows by induction that
\begin{equation}\label{eq:polyproof3}
\func{det} (E_k \cdots E_2E_1C) = \func{det } E_k \cdots \func{det } E_2 \func{det } E_1 \func{det } C \mbox{ for any matrix } C
\end{equation}
\textit{Lemma.} If $A$ has no inverse, then $\func{det } A = 0$.


\textit{Proof}. Let $A \to R$ where $R$ is reduced row-echelon, say $E_{n} \cdots E_{2}E_{1}A = R$. Then $R$ has a row of zeros by Part (4) of Theorem \ref{thm:004553},  and hence $\func{det } R = 0$. But then Equation \ref{eq:polyproof3} gives $\func{det }A = 0$ because $\func{det } E \neq 0$ for any elementary matrix $E$. This proves the Lemma.


Now we can prove Equation \ref{eq:polyproof1} by considering two cases.


\noindent \textit{Case 1}. $A$ \textit{has no inverse}. Then $AB$ also has no inverse (otherwise $A[B(AB)^{-1}] = I$ so $A$ is invertible by Corollary \ref{cor:004612} to Theorem \ref{thm:004553}). Hence the above Lemma (twice) gives
\begin{equation*}
\func{det} (AB) = 0 = 0 \func{det } B = \func{det } A \func{det } B
\end{equation*}
proving Equation \ref{eq:polyproof1} in this case.


\noindent \textit{Case 2}. $A$ \textit{has an inverse.} Then $A$ is a product of elementary matrices by Theorem \ref{thm:005336}, say $A = E_{1}E_{2}\cdots E_{k}$. Then Equation \ref{eq:polyproof3} with $C = I$ gives
\begin{equation*}
\func{det } A = \func{det} (E_1E_2\cdots E_k)= \func{det } E_1 \func{det } E_2 \cdots \func{det } E_k
\end{equation*}
But then Equation \ref{eq:polyproof3} with $C = B$ gives
\begin{equation*}
\func{det}(AB) = \func{det} \left[(E_1E_2\cdots E_k)B \right]= \func{det } E_1 \func{det } E_2 \cdots \func{det } E_k \func{det } B = \func{det } A \func{det } B
\end{equation*}
and Equation \ref{eq:polyproof1} holds in this case too.
\end{proof}


