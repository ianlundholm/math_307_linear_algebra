\noindent With each square matrix we can calculate a number, called the determinant of the matrix, which tells us whether or not the matrix is invertible. In fact, determinants can be used to give a formula for the inverse of a matrix. They also arise in calculating certain numbers (called eigenvalues) associated with the matrix. These eigenvalues are essential to a technique called diagonalization that is used in many applications where it is desired to predict the future behaviour of a system. For example, we use it to predict whether a species will become extinct. \index{determinants!and eigenvalues}\index{diagonalization!eigenvalues}\index{eigenvalues!and determinants}

Determinants were first studied by Leibnitz\index{Leibnitz} in 1696, and the term ``determinant'' was first used in 1801 by Gauss is his \textit{Disquisitiones Arithmeticae}. Determinants are much older than matrices (which were introduced by Cayley\index{Cayley, Arthur} in 1878) and were used extensively in the eighteenth and nineteenth centuries, primarily because of their significance in geometry (see Section \ref{sec:4_4}). Although they are somewhat less important today, determinants still play a role in the theory and application of matrix algebra. 
