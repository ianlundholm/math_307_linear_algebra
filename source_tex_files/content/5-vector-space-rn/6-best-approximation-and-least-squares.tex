\section{Best Approximation and Least Squares}
\label{sec:5_6}

Often an exact solution to a problem in applied mathematics is difficult to obtain. However, it is usually just as useful to find arbitrarily close approximations to a solution. In particular, finding ``linear approximations'' is a potent technique in applied mathematics. One basic case is the situation where a system of linear equations has no solution, and it is desirable to find a ``best approximation'' to a solution to the system. In this section best approximations are defined and a method for finding them is described. The result is then applied to ``least squares'' approximation of data.\index{best approximation}\index{solution!best approximation to}

Suppose $A$ is an $m \times n$ matrix and $\vect{b}$ is a column in $\RR^m$, and consider the system
\begin{equation*}
A\vect{x} = \vect{b}
\end{equation*}
of $m$ linear equations in $n$ variables. This need not have a solution. However, given any column $\vect{z} \in \RR^n$, the distance $\vectlength\vect{b} - A\vect{z}\vectlength$ is a measure of how far $A\vect{z}$ is from $\vect{b}$. Hence it is natural to ask whether there is a column $\vect{z}$ in $\RR^n$ that is as close as possible to a solution in the sense that
\begin{equation*}
\vectlength \vect{b} - A\vect{z} \vectlength
\end{equation*}
is the minimum value of $\vectlength\vect{b} - A\vect{x}\vectlength$ as $\vect{x}$ ranges over all columns in $\RR^n$.

The answer is ``yes'', and to describe it define
\begin{equation*}
U = \{A\vect{x} \mid \vect{x} \mbox{ lies in } \RR^n \}
\end{equation*}
\begin{wrapfigure}[10]{l}{5cm} 
\centering
\begin{tikzpicture}
\coordinate (pt0) at (0, 0);
\coordinate (ptAz) at (1, 0);
\coordinate (ptB) at (1, 2);

\filldraw[draw=dkbluevect, fill=ltgreenvect!50!white, thick] (-1, 1)-- ++(3, 0) -- ++(0.5, -2) -- ++(-4, 0)--cycle;

%green lines
\draw [dkgreenvect, -latex, thick] (pt0)--(ptAz) node[text=black, below, midway] {\footnotesize $A\vect{z}$};
\draw [dkgreenvect, -latex, thick] (pt0)--(ptB) node[text=black, left, pos=0.8] {\footnotesize $\vect{b}$};
\draw [dkgreenvect, -latex, thick] (ptAz)--(ptB) node[text=black, right, pos=0.8] {\footnotesize $\vect{b} - A\vect{z}$};

%dashed lines
\draw [dkbluevect, dashed, thick] (ptAz) -- ++(45:0.75);
\draw [dkbluevect, dashed, thick] (ptAz) -- ++(-45:0.75);

%right angles
\def \heightA {0.2} %height/width of right angle
\draw [dkbluevect, thick] ($ (ptAz) + (0, \heightA) $)-- ++(45:\heightA) -- ++(0, -\heightA);
\draw [dkbluevect, thick] ($ (ptAz) + (0, \heightA) $)-- ++(-45:\heightA) -- ++(0, -\heightA);

%labels
\node [below] at (pt0) {\footnotesize $\vect{0}$};
\node at (2.5, 0) {\footnotesize $U$};
\end{tikzpicture}
%\caption{\label{fig:016714}}
\end{wrapfigure}
This is a subspace of $\RR^m$ (verify) and we want a vector $A\vect{z}$ in $U$ as close as possible to $\vect{b}$. That there is such a vector is clear geometrically if $m = 3$ by the diagram. In general such a vector $A\vect{z}$ exists by a general result called the \textit{projection theorem}\index{orthogonality!projection theorem}\index{projection theorem} that will be proved in Chapter~\ref{chap:8} (Theorem~\ref{thm:023885}). Moreover, the projection theorem gives a simple way to compute $\vect{z}$ because it also shows that the vector $\vect{b} - A\vect{z}$ is \textit{orthogonal} to every vector $A\vect{x}$ in $U$. Thus, for all $\vect{x}$ in $\RR^n$,
\begin{align*}
0 = (A\vect{x}) \dotprod (\vect{b} - A\vect{z}) = (A\vect{x})^T(\vect{b} - A\vect{z}) &= \vect{x}^TA^T(\vect{b} - A\vect{z}) \\
&= \vect{x} \dotprod [A^T(\vect{b} - A\vect{z})]
\end{align*}
In other words, the vector $A^{T}(\vect{b} - A\vect{z})$ in $\RR^n$ is orthogonal to \textit{every} vector in $\RR^n$ and so must be zero (being orthogonal to itself). Hence $\vect{z}$ satisfies
\begin{equation*}
(A^TA)\vect{z} = A^T\vect{b}
\end{equation*}

\begin{definition}{Normal Equations}{016724} %5.14 
This is a system of linear equations called the \textbf{normal equations}\index{normal equations} for $\vect{z}$.\index{gaussian elimination!normal equations}\index{system of linear equations!normal equations}
\end{definition}
\noindent Note that this system can have more than one solution (see Exercise~\ref{ex:5_6_5}). However, the $n \times n$ matrix $A^{T}A$ is invertible if (and only if) the columns of $A$ are linearly independent (Theorem~\ref{thm:015672}); so, in this case, $\vect{z}$ is uniquely determined and is given explicitly by $\vect{z} = (A^{T}A)^{-1}A^{T}\vect{b}$. However, the most efficient way to find $\vect{z}$ is to apply gaussian elimination to the normal equations.

This discussion is summarized in the following theorem.

\begin{theorem}{Best Approximation Theorem}{016733} %theorem1
Let $A$ be an $m \times n$ matrix, let $\vect{b}$ be any column in $\RR^m$, and consider the system
\begin{equation*}
A\vect{x} = \vect{b}
\end{equation*}
of $m$ equations in $n$ variables.\index{best approximation theorem}

\begin{enumerate}
\item Any solution $\vect{z}$ to the normal equations
\begin{equation*}
(A^TA)\vect{z} = A^T\vect{b}
\end{equation*}
is a best approximation to a solution to $A\vect{x} = \vect{b}$ in the sense that $\vectlength\vect{b} - A\vect{z}\vectlength$ is the minimum value of $\vectlength\vect{b} - A\vect{x}\vectlength$ as $\vect{x}$ ranges over all columns in $\RR^n$.

\item If the columns of $A$ are linearly independent, then $A^{T}A$ is invertible and $\vect{z}$ is given uniquely by $\vect{z} = (A^{T}A)^{-1}A^{T}\vect{b}$.

\end{enumerate}
\end{theorem}

\noindent We note in passing that if $A$ is $n \times n$ and invertible, then
\begin{equation*}
\vect{z} = (A^TA)^{-1} A^T\vect{b} = A^{-1}\vect{b}
\end{equation*}
is the solution to the system of equations, and $\vectlength\vect{b} - A\vect{z}\vectlength = 0$. Hence if $A$ has independent columns, then $(A^{T}A)^{-1}A^{T}$ is playing the role of the inverse of the nonsquare matrix $A$. The matrix $A^{T}(AA^{T})^{-1}$ plays a similar role when the rows of $A$ are linearly independent. These are both special cases of the \textbf{generalized inverse}\index{generalized inverse}\index{inverses!generalized inverse} of a matrix $A$ (see Exercise~\ref{ex:5_6_14}). However, we shall not pursue this topic here.

\begin{example}{}{016759}
The system of linear equations
\begin{equation*}
\arraycolsep=1pt
\begin{array}{rlrcr}
	3x & - &  y & = & 4 \\
 	 x & + & 2y & = & 0 \\
 	2x & + &  y & = & 1
\end{array}
\end{equation*}
has no solution. Find the vector 
$ \vect{z} = 
\leftB \begin{array}{r}
x_0 \\
y_0
\end{array} \rightB$ that best approximates a solution.

\begin{solution}
In this case,
\begin{equation*}
A = \leftB \begin{array}{rr}
3 & -1 \\
1 & 2 \\
2 & 1
\end{array} \rightB
\mbox{, so }
A^TA = \leftB \begin{array}{rrr}
3 & 1 & 2 \\
-1 & 2 & 1
\end{array} \rightB
\leftB \begin{array}{rr}
3 & -1 \\
1 & 2 \\
2 & 1
\end{array} \rightB
= \leftB \begin{array}{rr}
14 & 1 \\
 1 & 6
\end{array} \rightB
\end{equation*}
is invertible. The normal equations $(A^{T}A)\vect{z} = A^{T}\vect{b}$ are
\begin{equation*}
\leftB \begin{array}{rr}
14 & 1 \\
1 & 6
\end{array} \rightB
\vect{z} = 
\leftB \begin{array}{r}
14 \\
-3
\end{array} \rightB
\mbox{, so }
\vect{z} = \frac{1}{83}
\leftB \begin{array}{r}
87 \\
-56
\end{array} \rightB
\end{equation*}
Thus $x_0 = \frac{87}{83}$ and $y_0 = \frac{-56}{83}$. With these values of $x$ and $y$, the left sides of the equations are, approximately,
\begin{equation*}
\arraycolsep=1pt
\def\arraystretch{1.5}
\begin{array}{rlrcrcr}
3x_0 & - &  y_0 & = & \frac{317}{83} & = & 3.82 \\
 x_0 & + & 2y_0 & = & \frac{-25}{83} & = & -0.30 \\
2x_0 & + &  y_0 & = & \frac{118}{83} & = & 1.42
\end{array}
\end{equation*}
This is as close as possible to a solution.
\end{solution}
\end{example}

\begin{example}{}{016776}
The average number $g$ of goals per game scored by a hockey player seems to be related linearly to two factors: the number $x_{1}$ of years of experience and the number $x_{2}$ of goals in the preceding 10 games. The data on the following page were collected on four players. Find the linear function $g = a_{0} + a_{1}x_{1} + a_{2}x_{2}$ that best fits these data.
\begin{equation*}
\begin{array}{|c|c|c|}
\hline
	g & x_1 & x_2 \\ \hline
	0.8 & 5 & 3 \\
	0.8 & 3 & 4 \\
	0.6 & 1 & 5 \\
	0.4 & 2 & 1 \\ \hline
\end{array}
\end{equation*}
\begin{solution}
If the relationship is given by $g = r_{0} + r_{1}x_{1} + r_{2}x_{2}$, then the data can be described as follows:
\begin{equation*}
\leftB \begin{array}{rrr}
1 & 5 & 3 \\
1 & 3 & 4 \\
1 & 1 & 5 \\
1 & 2 & 1
\end{array} \rightB
\leftB \begin{array}{r}
r_0 \\
r_1 \\
r_2
\end{array} \rightB
= \leftB \begin{array}{r}
0.8 \\
0.8 \\
0.6 \\
0.4
\end{array} \rightB
\end{equation*}
Using the notation in Theorem~\ref{thm:016733}, we get
\begin{align*}
\vect{z} &= (A^TA)^{-1}A^T\vect{b} \\
&= \frac{1}{42}
\leftB \begin{array}{rrr}
119 & -17 & -19 \\
-17 &  5  &   1 \\
-19 &  1  &   5
\end{array} \rightB
\leftB \begin{array}{rrrr}
1 & 1 & 1 & 1 \\
5 & 3 & 1 & 2 \\
3 & 4 & 5 & 1 
\end{array} \rightB
\leftB \begin{array}{r}
0.8 \\
0.8 \\
0.6 \\
0.4
\end{array} \rightB
= \leftB \begin{array}{r}
0.14 \\
0.09 \\
0.08
\end{array} \rightB
\end{align*}
Hence the best-fitting function is $g = 0.14 + 0.09x_{1} + 0.08x_{2}$. The amount of computation would have been reduced if the normal equations had been constructed and then solved by gaussian elimination.
\end{solution}
\end{example}

\subsection*{Least Squares Approximation}
\index{least squares approximation}

In many scientific investigations, data are collected that relate two variables. For example, if $x$ is the number of dollars spent on advertising by a manufacturer and $y$ is the value of sales in the region in question, the manufacturer could generate data by spending $x_{1}, x_{2}, \dots, x_{n}$ dollars at different times and measuring the corresponding sales values $y_{1}, y_{2}, \dots, y_{n}$.

\begin{wrapfigure}{l}{5cm} 
\centering
\begin{tikzpicture}[scale=0.9]
%set up of axis environment
\begin{axis}[disabledatascaling, 
width=5cm, 
height=5cm, 
xlabel={$x$}, 
ylabel={$y$}, 
axis lines=middle, 
xtick={\empty}, 
ytick={\empty}, 
xticklabels=\empty, 
yticklabels=\empty, 
every axis x label/.style={
	at={(ticklabel* cs:1.05)},
	anchor=west,
},
every axis y label/.style={
	at={(ticklabel* cs:1.05)},
	anchor=south,
},
clip mode=individual,
domain=0.5:6.5, 
samples=100, 
xmin=-0.5, 
xmax=6, 
ymin=-0.5, 
ymax=6]

\coordinate (x1) at (1, 1.9);
\coordinate (x2) at (2, 2);
\coordinate (x3) at (3, 3.5);
\coordinate (x4) at (4, 4.5);
\coordinate (x5) at (5, 5.5);

\addplot[dkgreenvect, thick] {\x + 0.5}; %line 1
\addplot[dkgreenvect, thick, domain=1:6.5] {1.5 * \x - 1}; %line 2
\fill[black] (x1) circle (2pt); 
\fill[black] (x2) circle (2pt); 
\fill[black] (x3) circle (2pt); 
\fill[black] (x4) circle (2pt); 
\fill[black] (x5) circle (2pt); 

\node [above] at (x1) {\scriptsize $(x_1, y_1)$};
\node [right] at (x2) {\scriptsize $(x_2, y_2)$};
\node [right] at (x3) {\scriptsize $(x_3, y_3)$};
\node [right] at (x4) {\scriptsize $(x_4, y_4)$};
\node [right] at (x5) {\scriptsize $(x_5, y_5)$};
\node [above] at (6, 6) {\scriptsize Line $1$};
\node [above] at (4, 6) {\scriptsize Line $2$};
\node [below left] at (0, 0) {\scriptsize $\vect{0}$};

\end{axis}
\end{tikzpicture}
%\caption{\label{fig:016810}}
\end{wrapfigure}

Suppose it is known that a linear relationship exists between the variables $x$ and $y$---in other words, that $y = a + bx$ for some constants $a$ and $b$. If the data are plotted, the points $(x_{1}, y_{1}), (x_{2}, y_{2}), \dots, (x_{n}, y_{n})$ may appear to lie on a straight line and estimating $a$ and $b$ requires finding the ``best-fitting'' line through these data points. For example, if five data points occur as shown in the diagram, line 1 is clearly a better fit than line 2. In general, the problem is to find the values of the constants $a$ and $b$ such that the line $y = a + bx$ best approximates the data in question. Note that an exact fit would be obtained if $a$ and $b$ were such that $y_{i} = a + bx_{i}$ were true for each data point $(x_{i}, y_{i})$. But this is too much to expect. Experimental errors in measurement are bound to occur, so the choice of $a$ and $b$ should be made in such a way that the errors between the observed values $y_{i}$ and the corresponding fitted values $a + bx_{i}$ are in some sense minimized. Least squares approximation is a way to do this.

The first thing we must do is explain exactly what we mean by the \textit{best fit} of a line $y = a + bx$ to an observed set of data points $(x_{1}, y_{1}), (x_{2}, y_{2}), \dots, (x_{n}, y_{n})$. For convenience, write the linear function $r_{0} + r_{1}x$ as
\begin{equation*}
f(x) = r_0 +r_1x
\end{equation*}
so that the fitted points (on the line) have coordinates $(x_{1}, f(x_{1})), \dots, (x_{n}, f(x_{n}))$.

\begin{wrapfigure}{l}{5cm} 
\centering
\begin{tikzpicture}[scale=0.9]
%set up of axis environment
\begin{axis}[disabledatascaling, 
width=5cm, 
height=5cm, 
xlabel={$x$}, 
ylabel={$y$}, 
axis lines=middle, 
xtick={\empty}, 
ytick={\empty}, 
xticklabels=\empty, 
yticklabels=\empty, 
every axis x label/.style={
	at={(ticklabel* cs:1.05)},
	anchor=west,
},
every axis y label/.style={
	at={(ticklabel* cs:1.05)},
	anchor=south,
},
clip mode=individual,
domain=-0.25:6.5, 
samples=100, 
xmin=-0.5, 
xmax=6, 
ymin=-0.5, 
ymax=6]

\coordinate (x1f) at (1, 2);
\coordinate (x1y) at ($(x1f) + (0, -0.7)$);
\coordinate (xif) at (3, 4);
\coordinate (xiy) at ($(xif) + (0, 0.7)$);
\coordinate (xnf) at (4, 5);
\coordinate (xny) at ($(xnf) + (0, -0.5)$);

%dashed lines
\draw [dkbluevect, thick, dashed] (x1y)--(x1y |- 0, 0)
	(xif)--(xif |- 0, 0)
	(xny)--(xny |- 0, 0);

%point labels
\node[right, rectangle, fill=white] at (x1f) {\scriptsize $(x_1, f(x_1))$};
\node[right] at ($(x1y) + (0, -0.2)$) {\scriptsize $(x_1, y_1)$};
\node[right, rectangle, fill=white] at ($(xif) + (0, -0.3)$) {\scriptsize $(x_i, f(x_i))$};
\node at ($(xiy) + (-0.4, 0.4)$) {\scriptsize $(x_i, y_i)$};
\node[right] at (xnf) {\scriptsize $(x_n, f(x_n))$};
\node[right] at ($(xny) + (0, -0.2)$) {\scriptsize $(x_n, y_n)$};

%green lines
\addplot[dkgreenvect, thick] {\x + 1};
\draw [dkgreenvect, thick] (x1f)-- coordinate[midway](midD1) (x1y) 
	(xif)-- coordinate[midway](midDi) (xiy)
	(xnf)-- coordinate[midway](midDn) (xny);

%points
\fill[black] (x1f) circle (2pt);
\fill[black] (x1y) circle (2pt); 
\fill[black] (xif) circle (2pt); 
\fill[black] (xiy) circle (2pt); 
\fill[black] (xnf) circle (2pt); 
\fill[black] (xny) circle (2pt); 

%blue label curves
\draw[dkbluevect, thick] (midD1) arc [start angle=270, end angle=200, radius=0.5] node (labelD1) {};
\draw[dkbluevect, thick] (midDn) arc [start angle=270, end angle=200, radius=0.5] node (labelDn) {};
%line labels
\node at ($ (labelD1) + (0, 0.3) $) {\scriptsize $d_1$};
\node at ($ (midDi) + (-0.4, 0) $) {\scriptsize $d_i$};
\node at ($ (labelDn) + (0, 0.4) $) {\scriptsize $d_n$};
\node [rotate=45] at (1.75, 3.25) {\scriptsize $y = f(x)$};

\node [below left] at (0, 0) {\scriptsize $\vect{0}$};
\node at (x1y |- 0, -0.5) {\scriptsize $x_1$};
\node at (xiy |- 0, -0.5) {\scriptsize $x_i$};
\node at (xny |- 0, -0.5) {\scriptsize $x_n$};

\end{axis}
\end{tikzpicture}
%\caption{\label{fig:016834}}
\end{wrapfigure}
 
The second diagram is a sketch of what the line $y = f(x)$ might look like. For each $i$ the observed data point $(x_{i}, y_{i})$ and the fitted point $(x_{i}, f(x_{i}))$ need not be the same, and the distance $d_{i}$ between them measures how far the line misses the observed point. For this reason $d_{i}$ is often called the \textbf{error}\index{error} at $x_{i}$, and a natural measure of how close the line $y = f(x)$ is to the observed data points is the sum $d_{1} + d_{2} + \dots + d_{n}$ of all these errors. However, it turns out to be better to use the sum of squares
\begin{equation*}
S = d_1^2 + d_2^2 + \dotsb + d_n^2
\end{equation*}
as the measure of error, and the line $y = f(x)$ is to be chosen so as to make this sum as small as possible. This line is said to be the \textbf{least squares approximating line}\index{least squares approximating line}\index{line!least squares approximating line} for the data points $(x_{1}, y_{1}), (x_{2}, y_{2}), \dots, (x_{n}, y_{n})$.

The square of the error $d_i$ is given by $ d_i^2 = [y_i - f(x_i)]^2 $ for each $i$, so the quantity $S$ to be minimized is the sum:
\begin{equation*}
S = [y_1 - f(x_1)]^2 + [y_2 - f(x_2)]^2 + \dotsb + [y_n - f(x_n)]^2
\end{equation*}
Note that all the numbers $x_{i}$ and $y_{i}$ are \textit{given} here; what is required is that the \textit{function} $f$ be chosen in such a way as to minimize $S$. Because $f(x) = r_{0} + r_{1}x$, this amounts to choosing $r_{0}$ and $r_{1}$ to minimize $S$. This problem can be solved using Theorem~\ref{thm:016733}. The following notation is convenient.
\begin{equation*}
\vect{x} = 
\leftB \begin{array}{c}
x_1 \\
x_2 \\
\vdots \\
x_n
\end{array} \rightB
\ \vect{y} = 
\leftB \begin{array}{c}
y_1 \\
y_2 \\
\vdots \\
y_n
\end{array} \rightB
\quad \mbox{ and } \quad f(\vect{x}) = 
\leftB \begin{array}{c}
f(x_1) \\
f(x_2) \\
\vdots \\
f(x_n)
\end{array} \rightB
= \leftB \begin{array}{c}
r_0 + r_1x_1 \\
r_0 + r_1x_2 \\
\vdots \\
r_0 + r_1x_n \\
\end{array} \rightB
\end{equation*}
Then the problem takes the following form: Choose $r_{0}$ and $r_{1}$ such that
\begin{equation*}
S = [y_1 - f(x_1)]^2 + [y_2 - f(x_2)]^2 + \dotsb + [y_n - f(x_n)]^2 = \vectlength \vect{y} - f(\vect{x}) \vectlength ^2
\end{equation*}
is as small as possible. Now write
\begin{equation*}
M = 
\leftB \begin{array}{cc}
1 & x_1 \\
1 & x_2 \\
\vdots & \vdots \\
1 & x_n
\end{array} \rightB
\quad \mbox{ and } \quad \vect{r} = 
\leftB \begin{array}{r}
r_0 \\
r_1
\end{array} \rightB
\end{equation*}
Then $M\vect{r} = f(\vect{x})$, so we are looking for a column $\vect{r} = 
\leftB \begin{array}{r}
r_0 \\
r_1
\end{array} \rightB$ such that $\vectlength\vect{y} - M\vect{r}\vectlength^{2}$ is as small as possible. In other words, we are looking for a best approximation $\vect{z}$ to the system $M\vect{r} = \vect{y}$. Hence Theorem~\ref{thm:016733} applies directly, and we have

\begin{theorem}{}{016878} %theorem2
Suppose that n data points $(x_{1}, y_{1}), (x_{2}, y_{2}), \dots, (x_{n}, y_{n})$ are given, where at least two of $x_{1}, x_{2}, \dots, x_{n}$ are distinct. Put
\begin{equation*}
\vect{y} = 
\leftB \begin{array}{c}
y_1 \\
y_2 \\
\vdots \\
y_n
\end{array} \rightB
\ M =
\leftB \begin{array}{cc}
1 & x_1 \\
1 & x_2 \\
\vdots & \vdots \\
1 & x_n
\end{array} \rightB
\end{equation*}
Then the least squares approximating line for these data points has equation
\begin{equation*}
y = z_0 +z_1x
\end{equation*}
where
$ \vect{z} = 
\leftB \begin{array}{r}
z_0 \\
z_1
\end{array} \rightB$ is found by gaussian elimination from the normal equations
\begin{equation*}
(M^TM)\vect{z} = M^T\vect{y}
\end{equation*}
The condition that at least two of $x_{1}, x_{2}, \dots, x_{n}$  are distinct ensures that $M^{T}M$ is an invertible matrix, so $\vect{z}$ is unique:
\begin{equation*}
\vect{z} = (M^TM)^{-1}M^T\vect{y}
\end{equation*}
\end{theorem}

\begin{example}{}{016901}
Let data points $(x_{1}, y_{1}), (x_{2}, y_{2}), \dots, (x_{5}, y_{5})$ be given as in the accompanying table. Find the least squares approximating line for these data.
\begin{equation*}
\begin{array}{|c|c|}
\hline
x & y \\ \hline
1 & 1 \\
3 & 2 \\
4 & 3 \\
6 & 4 \\
7 & 5 \\
\hline
\end{array}
\end{equation*}
\begin{solution}
  In this case we have
\begin{align*}
M^TM &= 
\leftB \begin{array}{cccc}
	1 & 1 & \cdots & 1 \\
	x_1 & x_2 & \cdots & x_5
\end{array} \rightB
\leftB \begin{array}{cc}
	1 & x_1 \\
	1 & x_2 \\
	\vdots & \vdots \\
	1 & x_5 
\end{array} \rightB \\
&= \leftB \begin{array}{cc}
	5 & x_1 + \dotsb + x_5 \\
	x_1 + \dotsb + x_5 & x_1^2 + \dotsb + x_5^2 \\
\end{array} \rightB
= \leftB \begin{array}{rr}
	 5 & 21 \\
	21 & 111
\end{array} \rightB \\
\mbox{and } M^T\vect{y} &=
\leftB \begin{array}{cccc}
	1 & 1 & \cdots & 1 \\
	x_1 & x_2 & \cdots & x_5
\end{array} \rightB
\leftB \begin{array}{c}
	y_1 \\
	y_2 \\
	\vdots \\
	y_5 
\end{array} \rightB \\
&= \leftB \begin{array}{c}
	y_1 + y_2 + \dotsb + y_5 \\
	x_1y_1 + x_2y_2 + \dotsb + x_5y_5
\end{array} \rightB
= \leftB \begin{array}{r}
	15 \\
	78
\end{array} \rightB
\end{align*}
so the normal equations $(M^{T}M)\vect{z} = M^{T}\vect{y}$ for 
$\vect{z} = 
\leftB \begin{array}{r}
z_0 \\
z_1
\end{array} \rightB$ become
\begin{equation*}
\leftB \begin{array}{rr}
 5 & 21 \\
21 & 111
\end{array} \rightB
\leftB \begin{array}{r}
z_0 \\
z_1
\end{array} \rightB
= \leftB \begin{array}{r}
15 \\
78
\end{array} \rightB
\end{equation*}
The solution (using gaussian elimination) is $\vect{z} = 
\leftB \begin{array}{r}
z_0 \\
z_1
\end{array} \rightB
=\leftB \begin{array}{r}
0.24 \\
0.66
\end{array} \rightB$ to two decimal places, so the least squares approximating line for these data is $y = 0.24 + 0.66x$. Note that $M^{T}M$ is indeed invertible here (the determinant is $114$), and the exact solution is
\begin{equation*}
\vect{z} = (M^TM)^{-1}M^T\vect{y} = \frac{1}{114}
\leftB \begin{array}{rr}
111 & -21 \\
-21 &   5
\end{array} \rightB
\leftB \begin{array}{rr}
15 \\
78
\end{array} \rightB
= \frac{1}{114}
\leftB \begin{array}{r}
27 \\
75
\end{array} \rightB
= \frac{1}{38}
\leftB \begin{array}{r}
9 \\
25
\end{array} \rightB
\end{equation*}
\end{solution}
\end{example}

\subsection*{Least Squares Approximating Polynomials}
Suppose now that, rather than a straight line, we want to find a polynomial
\begin{equation*}
y = f(x) = r_0 + r_1x + r_2x^2 + \dots + r_mx^m
\end{equation*}
of degree $m$ that best approximates the data pairs $(x_{1}, y_{1}), (x_{2}, y_{2}), \dots, (x_{n}, y_{n})$. As before, write
\begin{equation*}
\vect{x} = \leftB \begin{array}{c}
x_1 \\
x_2 \\
\vdots \\
x_n
\end{array} \rightB \ \vect{y} = \leftB \begin{array}{c}
y_1 \\
y_2 \\
\vdots \\
y_n
\end{array} \rightB
\quad \mbox{ and } \quad f(\vect{x}) = 
\leftB \begin{array}{c}
f(x_1) \\
f(x_2) \\
\vdots \\
f(x_n)
\end{array} \rightB
\end{equation*}
For each $x_{i}$ we have two values of the variable $y$, the observed value $y_{i}$, and the computed value $f(x_{i})$. The problem is to choose $f(x)$---that is, choose $r_{0}, r_{1}, \dots, r_{m}$ ---such that the $f(x_{i})$ are as close as possible to the $y_{i}$. Again we define ``as close as possible'' by the least squares condition: We choose the $r_{i}$ such that
\begin{equation*}
\vectlength \vect{y} - f(\vect{x}) \vectlength ^2 = [y_1 - f(x_1)]^2 + [y_2 - f(x_2)]^2 + \dotsb + [y_n - f(x_n)]^2 
\end{equation*}
is as small as possible.

\begin{definition}{Least Squares Approximation}{016945} %5.15 
A polynomial $f(x)$ satisfying this condition is called a \textbf{least squares approximating polynomial}\index{least squares approximating polynomial}\index{polynomials!least squares approximating polynomial} of degree $m$ for the given data pairs.
\end{definition}

\noindent If we write
\begin{equation*}
M = \def\arraystretch{1.3}
\leftB \begin{array}{ccccc}
1 & x_1 & x_1^2 & \cdots & x_1^m \\
1 & x_2 & x_2^2 & \cdots & x_2^m \\
\vdots & \vdots & \vdots & & \vdots \\
1 & x_n & x_n^2 & \cdots & x_n^m \\
\end{array} \rightB
\quad \mbox{ and } \quad \vect{r} = 
\leftB \begin{array}{c}
r_0 \\
r_1 \\
\vdots \\
r_m
\end{array} \rightB
\end{equation*}
we see that $f(\vect{x}) = M\vect{r}$. Hence we want to find $\vect{r}$ such that $\vectlength\vect{y} - M\vect{r}\vectlength^{2}$ is as small as possible; that is, we want a best approximation $\vect{z}$ to the system $M\vect{r} = \vect{y}$. Theorem~\ref{thm:016733} gives the first part of Theorem~\ref{thm:016951}.

\begin{theorem}{}{016951} %theorem3
Let $n$ data pairs $(x_{1}, y_{1}), (x_{2}, y_{2}), \dots, (x_{n}, y_{n})$ be given, and write
\begin{equation*}
\vect{y} = 
\leftB \begin{array}{c}
y_1 \\
y_2 \\
\vdots \\
y_n
\end{array} \rightB
\ M = 
\def\arraystretch{1.3}
\leftB \begin{array}{ccccc}
1 & x_1 & x_1^2 & \cdots & x_1^m \\
1 & x_2 & x_2^2 & \cdots & x_2^m \\
\vdots & \vdots & \vdots & & \vdots \\
1 & x_n & x_n^2 & \cdots & x_n^m \\
\end{array} \rightB
\ \vect{z} = 
\leftB \begin{array}{c}
z_0 \\
z_1 \\
\vdots \\
z_m
\end{array} \rightB
\end{equation*}
\begin{enumerate}
\item If $\vect{z}$ is any solution to the normal equations
\begin{equation*}
(M^TM)\vect{z} = M^T\vect{y}
\end{equation*}
then the polynomial
\begin{equation*}
z_0 + z_1x + z_2x^2 + \dots + z_mx^m
\end{equation*}
is a least squares approximating polynomial of degree m for the given data pairs.

\item If at least $m + 1$ of the numbers $x_{1}, x_{2}, \dots, x_{n}$ are distinct (so $n \geq m + 1$), the matrix $M^{T}M$ is invertible and $\vect{z}$ is uniquely determined by
\begin{equation*}
\vect{z} = (M^TM)^{-1}M^T\vect{y}
\end{equation*}
\end{enumerate}
\end{theorem}

\begin{proof}
It remains to prove (2), and for that we show that the columns of $M$ are linearly independent (Theorem~\ref{thm:015672}). Suppose a linear combination of the columns vanishes:
\begin{equation*}
r_0 \leftB \begin{array}{c}
	1 \\
	1 \\
	\vdots \\
	1
\end{array} \rightB
+ r_1 \leftB \begin{array}{c}
	x_1 \\
	x_2 \\
	\vdots \\
	x_n
\end{array} \rightB
+ \dots + r_m \leftB \begin{array}{c}
	x_1^m \\
	x_2^m \\
	\vdots \\
	x_n^m
\end{array} \rightB
= \leftB \begin{array}{c}
	0 \\
	0 \\
	\vdots \\
	0 \\
\end{array} \rightB
\end{equation*}
If we write $q(x) = r_{0} + r_{1}x + \dots + r_{m}x^{m}$, equating coefficients shows that 
\begin{equation*}
q(x_{1}) = q(x_{2}) = \dots = q(x_{n}) = 0
\end{equation*}
Hence $q(x)$ is a polynomial of degree $m$ with at least $m + 1$ distinct roots, so $q(x)$ must be the zero polynomial (see Appendix \ref{chap:appdpolynomials} or Theorem~\ref{thm:020203}). Thus $r_{0} = r_{1} = \dots = r_{m} = 0$ as required.
\end{proof}

\begin{example}{}{016988}
Find the least squares approximating quadratic $y = z_{0} + z_{1}x + z_{2}x^{2}$ for the following data points.
\begin{equation*}
(-3, 3), (-1, 1), (0, 1), (1, 2), (3, 4)
\end{equation*}
\begin{solution}
This is an instance of Theorem~\ref{thm:016951} with $m = 2$. Here
\begin{equation*}
\vect{y} =
\leftB \begin{array}{r}
3\\
1\\
1\\
2\\
4
\end{array} \rightB
\ M = 
\leftB \begin{array}{rrr}
1 & -3 &  9 \\
1 & -1 &  1 \\
1 &  0 &  0 \\
1 &  1 &  1 \\
1 &  3 &  9
\end{array} \rightB
\end{equation*}
Hence,
\begin{equation*}
M^TM =
\leftB \begin{array}{rrrrr}
 1 &  1 & 1 & 1 & 1 \\
-3 & -1 & 0 & 1 & 3 \\
 9 &  1 & 0 & 1 & 9 
\end{array} \rightB
\leftB \begin{array}{rrr}
1 & -3 &  9 \\
1 & -1 &  1 \\
1 &  0 &  0 \\
1 &  1 &  1 \\
1 &  3 &  9
\end{array} \rightB
= \leftB \begin{array}{rrr}
 5 &  0 & 20 \\
 0 & 20 &  0 \\
20 &  0 & 164 
\end{array} \rightB
\end{equation*}
\begin{equation*}
M^T\vect{y} = 
\leftB \begin{array}{rrrrr}
 1 &  1 & 1 & 1 & 1 \\
-3 & -1 & 0 & 1 & 3 \\
 9 &  1 & 0 & 1 & 9 
\end{array} \rightB
\leftB \begin{array}{r}
3 \\
1 \\
1 \\
2 \\
4 
\end{array} \rightB
= \leftB \begin{array}{r}
11 \\
4 \\
66
\end{array} \rightB
\end{equation*}
The normal equations for $\vect{z}$ are
\begin{equation*}
\leftB \begin{array}{rrr}
 5 &  0 & 20 \\
 0 & 20 &  0 \\
20 &  0 & 164 
\end{array} \rightB
\vect{z} = 
\leftB \begin{array}{r}
11 \\
4 \\
66
\end{array} \rightB \quad
\mbox{ whence } \vect{z} = 
\leftB \begin{array}{r}
1.15 \\
0.20 \\
0.26
\end{array} \rightB
\end{equation*}
This means that the least squares approximating quadratic for these data is $y = 1.15 + 0.20x + 0.26x^{2}$.
\end{solution}
\end{example}

\subsection*{Other Functions}
There is an extension of Theorem~\ref{thm:016951} that should be mentioned. Given data pairs $(x_{1}, y_{1}), (x_{2}, y_{2}), \\ \dots, (x_{n}, y_{n})$, that theorem shows how to find a polynomial
\begin{equation*}
f(x) = r_0 + r_1x + \dots + r_mx^m 
\end{equation*}
such that $\vectlength\vect{y} - f(\vect{x})\vectlength^{2}$ is as small as possible, where $\vect{x}$ and $f(\vect{x})$ are as before. Choosing the appropriate polynomial $f(x)$ amounts to choosing the coefficients $r_{0}, r_{1}, \dots, r_{m}$, and Theorem~\ref{thm:016951} gives a formula for the optimal choices. Here $f(x)$ is a linear combination of the functions $1, x, x^{2}, \dots, x^{m}$ where the $r_{i}$ are the coefficients\index{coefficients!of the polynomial}\index{polynomials!coefficients}, and this suggests applying the method to other functions. If $f_{0}(x), f_{1}(x), \dots, f_{m}(x)$ are given functions, write
\begin{equation*}
f(x) = r_0f_0(x) + r_1f_1(x) + \dots + r_mf_m(x)
\end{equation*}
where the $r_{i}$ are real numbers. Then the more general question is whether $r_{0}, r_{1}, \dots, r_{m}$ can be found such that $\vectlength\vect{y} - f(\vect{x})\vectlength^{2}$ is as small as possible where
\begin{equation*}
f(\vect{x}) =
\leftB \begin{array}{c}
f(x_1) \\
f(x_2) \\
\vdots \\
f(x_m)
\end{array} \rightB
\end{equation*}
Such a function $f(\vect{x})$ is called a \textbf{least squares best approximation}\index{least squares best approximation} for these data pairs of the form $r_{0}f_{0}(x) + r_{1}f_{1}(x) + \dots + r_{m}f_{m}(x)$, $r_{i}$ in $\RR$. The proof of Theorem~\ref{thm:016951} goes through to prove

\begin{theorem}{}{017041} %theorem4
Let $n$ data pairs $(x_{1}, y_{1}), (x_{2}, y_{2}), \dots, (x_{n}, y_{n})$ be given, and suppose that $m + 1$ functions $f_{0}(x), f_{1}(x), \dots, f_{m}(x)$ are specified. Write
\begin{equation*}
\vect{y} = \leftB \begin{array}{c}
y_1 \\
y_2 \\
\vdots \\
y_n
\end{array} \rightB
\quad M = \leftB \begin{array}{cccc}
	f_0(x_1) & f_1(x_1) & \cdots & f_m(x_1) \\
	f_0(x_2) & f_1(x_2) & \cdots & f_m(x_2) \\
	\vdots & \vdots & & \vdots \\
	f_0(x_n) & f_1(x_n) & \cdots & f_m(x_n) \\
\end{array} \rightB
\quad
\vect{z} = \leftB \begin{array}{c}
z_1 \\
z_2 \\
\vdots \\
z_m
\end{array} \rightB
\end{equation*}
\begin{enumerate}
\item If $\vect{z}$ is any solution to the normal equations
\begin{equation*}
(M^TM)\vect{z} = M^T\vect{y}
\end{equation*}
then the function
\begin{equation*}
z_0f_0(x) + z_1f_1(x) + \dots + z_mf_m(x)
\end{equation*}
is the best approximation for these data among all functions of the form $r_{0}f_{0}(x) + r_{1}f_{1}(x) + \dots + r_{m}f_{m}(x)$ where the $r_{i}$ are in $\RR$.

\item If $M^{T}M$ is invertible (that is, if $\func{rank}(M) = m + 1$), then $\vect{z}$ is uniquely determined; in fact, $\vect{z} = (M^{T}M)^{-1}(M^{T}\vect{y})$.

\end{enumerate}
\end{theorem}

\noindent Clearly Theorem~\ref{thm:017041} contains Theorem~\ref{thm:016951} as a special case, but there is no simple test in general for whether $M^{T}M$ is invertible. Conditions for this to hold depend on the choice of the functions $f_{0}(x), f_{1}(x), \dots, f_{m}(x)$.

\begin{example}{}{017077}
Given the data pairs $(-1, 0)$, $(0, 1)$, and $(1, 4)$, find the least squares approximating function of the form $r_{0}x + r_{1}2^{x}$.

\begin{solution}
The functions are $f_{0}(x) = x$ and $f_{1}(x) = 2^{x}$, so the matrix $M$ is
\begin{equation*}
M = \leftB \begin{array}{cc}
f_0(x_1) & f_1(x_1) \\
f_0(x_2) & f_1(x_2) \\
f_0(x_3) & f_1(x_3)
\end{array} \rightB
= \leftB \begin{array}{rl}
-1 & 2^{-1} \\
0 & 2^0 \\
1 & 2^1
\end{array} \rightB
=\frac{1}{2} \leftB \begin{array}{rr}
-2 & 1 \\
0 & 2 \\
2 & 4
\end{array} \rightB
\end{equation*}
In this case $M^TM = \frac{1}{4}
\leftB \begin{array}{rr}
8 & 6 \\
6 & 21
\end{array} \rightB$ is invertible, so the normal equations
\begin{equation*}
\frac{1}{4}
\leftB \begin{array}{rr}
8 & 6 \\
6 & 21
\end{array} \rightB
\vect{z} = 
\leftB \begin{array}{r}
4 \\
9 
\end{array} \rightB
\end{equation*}
have a unique solution $\vect{z} = \frac{1}{11}
\leftB \begin{array}{r}
10 \\
16
\end{array} \rightB.$ Hence the best-fitting function of the form $r_0x + r_12^x$ is $\overline{f}(x) = \frac{10}{11}x + \frac{16}{11}2^x$. Note that $\overline{f}(\vect{x}) =
\leftB \begin{array}{c}
	\overline{f}(-1) \\
	\overline{f}(0) \\
	\overline{f}(1) \\
\end{array} \rightB
= \def\arraystretch{1.5}
\leftB \begin{array}{r}
\frac{-2}{11} \\
\frac{16}{11} \\
\frac{42}{11}
\end{array} \rightB$, compared with $\vect{y} = 
\leftB \begin{array}{r}
0 \\
1 \\
4
\end{array} \rightB$
\end{solution}
\end{example}
