\section{Subspaces and Spanning}
\label{sec:5_1}

In Section~\ref{sec:2_2} we introduced the set $\RR^n$ of all $n$-tuples\index{$n$-tuples} (called \textit{vectors}), and began our investigation of the matrix transformations $\RR^n \to \RR^m$ given by matrix multiplication by an $m \times n$ matrix. Particular attention was paid to the euclidean plane $\RR^2$ where certain simple geometric transformations were seen to be matrix transformations. Then in Section~\ref{sec:2_6} we introduced linear transformations, showed that they are all matrix 
transformations, and found the matrices of rotations and reflections in $\RR^2$. We returned to this in Section~\ref{sec:4_4} where we showed that projections, reflections, and rotations of $\RR^2$ and $\RR^3$ were all linear, and where we related areas and volumes to determinants.

In this chapter we investigate $\RR^n$ in full generality, and introduce some of the most important concepts and methods in linear algebra. The $n$-tuples in $\RR^n$ will continue to be denoted $\vect{x}$, $\vect{y}$, and so on, and will be written as rows or columns depending on the context.

\subsection*{Subspaces of $\RR^n$ }

\begin{definition}{Subspace of $\RR^n$}{013442}
A set\footnotemark $U$ of vectors in $\RR^n$ is called a \textbf{subspace}\index{subspaces!defined} of $\RR^n$ if it satisfies the following properties:\index{set of all ordered $n$-tuples ($\RR^n$)!subspaces}

\begin{itemize}
\item[S1.] The zero vector $\vect{0} \in U$.\index{vectors!zero vector}\index{zero vector}

\item[S2.] If $\vect{x} \in U$ and $\vect{y} \in U$, then $\vect{x} + \vect{y} \in U$.

\item[S3.] If $\vect{x} \in U$, then $a\vect{x} \in U$ for every real number $a$.

\end{itemize}
\end{definition}
\footnotetext{We use the language of sets. Informally, a \textbf{set}\index{set} $X$ is a collection of objects, called the \textbf{elements}\index{elements of the set} of the set. The fact that $x$ is an element of $X$ is denoted $x \in X$. Two sets $X$ and $Y$ are called equal (written $X = Y$)\index{equal!sets} if they have the same elements. If every element of $X$ is in the set $Y$, we say that $X$ is a \textbf{subset}\index{subset} of $Y$, and write $X \subseteq Y$. Hence $X \subseteq Y$ and $Y \subseteq X$ both hold if and only if $X = Y$.}

\noindent We say that the subset $U$ is \textbf{closed under addition}\index{closed under addition}\index{addition!closed under addition}\index{subspaces!closed under addition} if S2 holds, and that $U$ is \textbf{closed under scalar multiplication}\index{closed under scalar multiplication}\index{scalar multiplication!closed under scalar multiplication}\index{subspaces!closed under scalar multiplication} if S3 holds.

Clearly $\RR^n$ is a subspace of itself, and this chapter is about these subspaces and their properties. The set $U = \{\vect{0}\}$,
consisting of only the zero vector, is also a subspace because
$\vect{0} + \vect{0} = \vect{0}$ and $a\vect{0} = \vect{0}$ for each
$a$ in $\RR$; it is called the \textbf{zero subspace}\index{zero subspace}\index{subspaces!zero subspace}. Any subspace of $\RR^n$ other than \{$\vect{0}$\} or
$\RR^n$ is called a \textbf{proper}\index{proper subspace}\index{subspaces!proper subspace} subspace.

\begin{wrapfigure}[4]{l}{5cm} 
	\centering
	\vspace*{-2em}
	\begin{tikzpicture}[scale=0.8]
\coordinate (origin) at (0, 0);

\draw[thick, -latex] (origin)--(2.5, 0) node [right] {\small $y$};
\draw[thick, -latex] (origin)--(0, 2) node [above] {\small $z$};
\draw[thick, -latex] (origin)--(-1, -1) node [below left] {\small $x$};

%plane
\draw[color=dkbluevect, fill=ltbluevect, thick] (-2, 1)-- ++(2.5, 0)-- ++(2, -2) -- ++(-2.5, 0)--cycle;

%dashed lines
\draw[thick, dashed] (origin)--(1.5, 0);
\draw[thick] (origin)--(0, 1);
\draw[thick, dashed] (origin)--(-0.5, -0.5);

%green line
\draw[dkgreenvect, -latex, thick] (origin)--(0.25, 1.5) node [right, text=black] {\small $\vect{n}$};

%right angles
\draw[thick, dkbluevect] (0, 0.2) -- ++(45:0.2)-- ++(0, -0.2);
\draw[thick, dkbluevect] (0, 0.2) -- ++(-45:0.2)-- ++(0, -0.2);

\node[below] at (1, -1) {\small $M$};
\end{tikzpicture}
	%\caption{\label{fig:013465}}
\end{wrapfigure}

We saw in Section~\ref{sec:4_2} that every plane $M$ through the origin in $\RR^3$
 has equation $ax + by + cz = 0$ where $a$, $b$, and $c$ are not all zero. Here  $\vect{n} = 
 \leftB \begin{array}{r}
 a \\
 b \\
 c
 \end{array} \rightB $ 
 is a normal for the plane and
\begin{equation*}
	M = \{ \vect{v} \mbox{ in } \RR^3 \mid \vect{n} \dotprod \vect{v} = 0 \}
\end{equation*}
where 
$\vect{v} = 
\leftB \begin{array}{r}
x \\
y \\
z
\end{array} \rightB$
 and $\vect{n} \dotprod \vect{v}$ denotes the dot product introduced in Section~\ref{sec:2_2} (see the diagram).\footnote{We are using set notation here\index{set notation}. In general \{$q \mid p$\} means the set of all objects $q$ with property $p$.}
 Then $M$ is a subspace of $\RR^3$. Indeed we show that $M$ satisfies S1, S2, and S3 as follows:

\begin{itemize}
\item[\textit{S1.}]  $\vect{0} \in M$ \textit{because} $\vect{n} \dotprod \vect{0} = 0$\textit{;}

\item[\textit{S2.}] \textit{If} $\vect{v} \in M$ \textit{and} $\vect{v}_{1} \in M$ \textit{, then} $\vect{n} \dotprod (\vect{v} + \vect{v}_{1}) = \vect{n} \dotprod \vect{v} + \vect{n} \dotprod \vect{v}_{1} = 0 + 0 = 0$ \textit{, so} $\vect{v} + \vect{v}_{1} \in M$\textit{;}

\item[\textit{S3.}] \textit{If} $\vect{v} \in M$ \textit{, then} $\vect{n} \dotprod (a\vect{v}) = a(\vect{n} \dotprod \vect{v}) = a(0) = 0$ \textit{, so } $a\vect{v} \in M$\textit{.}

\end{itemize}

\noindent This proves the first part of

\begin{example}{}{013478}
\begin{wrapfigure}[5]{l}{5cm}
\centering
\begin{tikzpicture}[scale=0.9]
\coordinate (origin) at (0, 0);

\draw[thick, -latex] (origin)--(2, 0) node [right] {\small $y$};
\draw[thick, -latex] (origin)--(0, 2) node [above] {\small $z$};
\draw[thick, -latex] (origin)--(-1, -1) node [below left] {\small $x$};

\def \angleD {60}
\draw[thick, dkbluevect] (origin)--(\angleD:2.5)
	(origin)--(180 + \angleD :2);
\draw[thick, -latex, dkgreenvect] (origin)--(\angleD:1.25) node [right, text=black] {\small $\vect{d}$};

\node[right] at (\angleD:2.0) {\small $L$};
\end{tikzpicture}
%\caption{\label{fig:013486}}
\end{wrapfigure}

\setlength{\rightskip}{0pt plus 200pt}
Planes\index{planes} and lines through the origin in $\RR^3$ are all subspaces of $\RR^3$.\index{subspaces!planes and lines through the origin}

\begin{solution}
We dealt with planes above. If $L$ is a line through the origin with direction vector $\vect{d}$, then $L = \{t\vect{d} \mid t \in \RR\}$ (see the diagram). We leave it as an exercise to verify that $L$ satisfies S1, S2, and S3.\index{line!through the origin}
\vspace{4em}
\end{solution}
\end{example}

\noindent Example~\ref{exa:013478} shows that lines through the origin in $\RR^2$ are subspaces; in fact, they are the \textit{only} proper subspaces of $\RR^2$ (Exercise \ref{ex:5_1_24}). Indeed, we shall see in Example~\ref{exa:014470} that lines and planes through the origin in $\RR^3$ are the only proper subspaces of $\RR^3$. Thus the geometry of lines and planes through the origin is captured by the subspace concept. (Note that \textit{every} line or plane is just a translation of one of these.)

Subspaces can also be used to describe important features of an $m \times n$ matrix $A$. The \textbf{null space}\index{null space} of $A$, denoted $\func{null} A$, and the \textbf{image space}\index{image space} of $A$, denoted $\func{im} A$, are defined by\index{subspaces!$m \times n$ matrix}
\begin{equation*}
\func{null } A = \{ \vect{x} \in \RR^n \mid A\vect{x} = \vect{0}\} \quad \mbox{ and } \quad \func{im } A = \{A\vect{x} \mid \vect{x} \in \RR^n\}
\end{equation*}
In the language of Chapter~\ref{chap:2}, $\func{null} A$ consists of all solutions $\vect{x}$ in $\RR^n$ of the homogeneous system $A\vect{x} = \vect{0}$, and $\func{im} A$ is the set of all vectors $\vect{y}$ in $\RR^m$ such that $A\vect{x} = \vect{y}$ \textit{has} a solution $\vect{x}$. Note that $\vect{x}$ is in $\func{null} A$ if it satisfies the \textit{condition} $A\vect{x} = \vect{0}$, while $\func{im} A$ consists of vectors of the \textit{form} $A\vect{x}$ for some $\vect{x}$ in $\RR^n$. These two ways to describe subsets occur frequently.

\begin{example}{}{013498}
If $A$ is an $m \times n$ matrix, then:\index{$m \times n$ matrix!subspaces}

\begin{enumerate}
\item $\func{null} A$ is a subspace of $\RR^n$.

\item $\func{im} A$ is a subspace of $\RR^m$.

\end{enumerate}

\begin{solution}
\begin{enumerate}
\item The zero vector $\vect{0} \in \RR^n$ lies in $\func{null} A$ because $A\vect{0} = \vect{0}$.\footnotemark \;  If $\vect{x}$ and $\vect{x}_{1}$ are in $\func{null} A$, then $\vect{x} + \vect{x}_{1}$ and $a\vect{x}$ are in $\func{null} A$ because they satisfy the required condition:
\begin{equation*}
A(\vect{x} + \vect{x}_1) = A\vect{x} + A\vect{x}_1 = \vect{0} + \vect{0} = \vect{0} \quad \mbox{and} \quad A(a\vect{x}) = a(A\vect{x}) = a\vect{0} = \vect{0}
\end{equation*}
Hence $\func{null} A$ satisfies S1, S2, and S3, and so is a subspace of $\RR^n$.

\item The zero vector $\vect{0} \in \RR^m$ lies in $\func{im} A$ because $\vect{0} = A\vect{0}$. Suppose that $\vect{y}$ and $\vect{y}_{1}$ are in $\func{im} A$, say $\vect{y} = A\vect{x}$ and $\vect{y}_{1} = A\vect{x}_{1}$ where $\vect{x}$ and $\vect{x}_{1}$ are in $\RR^n$. Then
\begin{equation*}
\vect{y} + \vect{y}_1 = A\vect{x} + A\vect{x}_1 = A(\vect{x} + \vect{x}_1) \quad \mbox{and} \quad a\vect{y} = a(A\vect{x}) = A(a\vect{x})
\end{equation*}
show that $\vect{y} + \vect{y}_{1}$ and $a\vect{y}$ are both in $\func{im} A$ (they have the required form). Hence $\func{im} A$ is a subspace of $\RR^m$.

\end{enumerate}
\end{solution}
\end{example}
\footnotetext{We are using $\vect{0}$ to represent the zero vector in both $\RR^m$ and $\RR^n$. This abuse of notation is common and causes no confusion once everybody knows what is going on.}

There are other important subspaces associated with a matrix $A$ that clarify basic properties of $A$. If $A$ is an $n \times n$ matrix and $\lambda$ is any number, let
\begin{equation*}
E_\lambda(A) = \{\vect{x} \in \RR^n \mid A\vect{x} = \lambda\vect{x} \}
\end{equation*}
A vector $\vect{x}$ is in $E_\lambda(A)$ if and only if $(\lambda I - A)\vect{x} = \vect{0}$, so Example~\ref{exa:013498} gives:

\begin{example}{}{013534}
$E_\lambda(A) = \func{null}(\lambda I- A)$ is a subspace of $\RR^n$ for each $n \times n$ matrix $A$ and number $\lambda$.
\end{example}

\noindent $E_\lambda(A)$ is called the \textbf{eigenspace}\index{eigenspace} of $A$ corresponding to $\lambda$. The reason for the name is that, in the terminology of Section~\ref{sec:3_3}, $\lambda$ is an \textbf{eigenvalue} of $A$ if $E_\lambda(A) \neq \{\vect{0}\}$. In this case the nonzero vectors in  $E_\lambda(A)$ are called the \textbf{eigenvectors}\index{eigenvector!nonzero vectors} of $A$ corresponding to $\lambda$.\index{eigenvalues!and eigenspace}

The reader should not get the impression that \textit{every} subset of $\RR^n$ is a subspace. For example:
\begin{align*}
U_1 &= \left \{
\leftB \begin{array}{l}
x \\
y
\end{array} \rightB \,
\middle| \, x \geq 0 \right\} 
\mbox{ satisfies S1 and S2, but not S3;} \\
U_2 &= \left \{
\leftB \begin{array}{l}
x \\
y
\end{array} \rightB  \,
\middle| \, x^2 = y^2 \right\} 
\mbox{ satisfies S1 and S3, but not S2;}
\end{align*}
Hence neither $U_{1}$ nor $U_{2}$ is a subspace of $\RR^2$. (However, see Exercise \ref{ex:5_1_20}.)

\subsection*{Spanning Sets}

Let $\vect{v}$ and $\vect{w}$ be two nonzero, nonparallel vectors in $\RR^3$ with their tails at the origin. The plane $M$ through the origin containing these vectors is described in Section~\ref{sec:4_2} by saying that $\vect{n} = \vect{v} \times \vect{w}$ is a \textit{normal} for $M$, and that $M$ consists of all vectors $\vect{p}$ such that $\vect{n} \dotprod \vect{p} = 0$.\footnote{The vector $\vect{n} = \vect{v} \times \vect{w}$ is nonzero because $\vect{v}$ and $\vect{w}$ are not parallel.}
 While this is a very useful way to look at planes, there is another approach that is at least as useful in $\RR^3$ and, more importantly, works for all subspaces of $\RR^n$ for any $n \geq 1$.\index{set of all ordered $n$-tuples ($\RR^n$)!spanning sets}\index{spanning sets}\index{subspaces!spanning sets}

\begin{wrapfigure}{l}{5cm} 
\centering
\begin{tikzpicture}[scale=0.9]
\coordinate (origin) at (0, 0);
\coordinate (pt0) at (1, 0);
\coordinate (ptV) at (1.5, 0.5);
\coordinate (ptAV) at (2.5, 1.5);
\coordinate (ptW) at (2, -1/3);
\coordinate (ptBW) at (2.5, -0.5);
\coordinate (ptP) at ($(ptAV) +(ptBW) - (pt0)$);

\draw[dkbluevect, thick] (origin)-- ++(2, 2) -- ++(3, -1) -- ++(-2, -2)--cycle;

%green lines
\draw[dkgreenvect, thick, -latex] (pt0)--(ptV);
\draw[dkgreenvect, thick, -latex] (pt0)--(ptAV);
\draw[dkgreenvect, thick, -latex] (pt0)--(ptW);
\draw[dkgreenvect, thick, -latex] (pt0)--(ptBW);
\draw[dkgreenvect, thick, -latex] (pt0)--(ptP);

%dashed lines
\draw[dkbluevect, thick, dashed] (ptAV)--(ptP)--(ptBW);

\node [left] at (pt0) {\footnotesize $\vect{0}$};
\node [left] at (ptV) {\footnotesize $\vect{v}$};
\node [left] at (ptAV) {\footnotesize $a\vect{v}$};
\node at ($ (ptW) + (-0.2, -0.15) $) {\footnotesize $\vect{w}$};
\node [right] at (ptBW) {\footnotesize $b\vect{w}$};
\node [right] at (ptP) {\footnotesize $\vect{p}$};
\draw[dkbluevect, thick] (3.8, 0.1) to [bend left] ++(0.5, 0);
\node [right] at (4.2, 0) {\footnotesize $M$};
\end{tikzpicture}
%\caption{\label{fig:013558}}
\end{wrapfigure}

The idea is as follows: Observe that, by the diagram, a vector $\vect{p}$ is in $M$ if and only if it has the form
\begin{equation*}
\vect{p} = a\vect{v} + b\vect{w}
\end{equation*}
for certain real numbers $a$ and $b$ (we say that $\vect{p}$ is a \textit{linear combination}\index{linear combinations!spanning sets} of $\vect{v}$ and $\vect{w}$).
Hence we can describe $M$ as
\begin{equation*}
M = \{a\vect{v} + b\vect{w} \mid  a, b \in \RR \}.\footnote{In particular, this implies that any vector $\vect{p}$ orthogonal to $\vect{v} \times \vect{w}$ must be a linear combination $\vect{p} = a\vect{v} + b\vect{w}$ of $\vect{v}$ and $\vect{w}$ for some $a$ and $b$. Can you prove this directly?}
\end{equation*}
and we say that $\{\vect{v}, \vect{w}\}$ is a \textit{spanning set} for $M$. It is this notion of a spanning set that provides a way to describe all subspaces of $\RR^n$.

As in Section~\ref{sec:1_3}, given vectors $\vect{x}_{1}, \vect{x}_{2}, \dots, \vect{x}_{k}$ in $\RR^n$, a vector of the form
\begin{equation*}
t_1\vect{x}_1 + t_2\vect{x}_2 + \dots + t_k\vect{x}_k \quad \mbox{where the } t_i \mbox{ are scalars}
\end{equation*}
is called a \textbf{linear combination} of the $\vect{x}_{i}$, and $t_{i}$ is called the \textbf{coefficient}\index{coefficients!linear combination} of $\vect{x}_{i}$ in the linear combination.

\begin{definition}{Linear Combinations and Span in $\RR^n$}{013571}
The set of all such linear combinations is called the \textbf{span}\index{span} of the $\vect{x}_{i}$ and is denoted
\begin{equation*}
\func{span}\{\vect{x}_1, \vect{x}_2, \dots, \vect{x}_k\} = \{t_1\vect{x}_1 + t_2\vect{x}_2 + \dots + t_k\vect{x}_k \mid t_i \mbox{ in } \RR\}
\end{equation*}
If $V = \func{span}\{\vect{x}_{1}, \vect{x}_{2}, \dots, \vect{x}_{k}\}$, we say that $V$ is \textbf{spanned} by the vectors $\vect{x}_{1}, \vect{x}_{2}, \dots, \vect{x}_{k}$, and that the vectors $\vect{x}_{1}, \vect{x}_{2}, \dots, \vect{x}_{k}$ \textbf{span} the space $V$.
\end{definition}

\noindent Here are two examples:
\begin{equation*}
\func{span}\{ \vect{x} \} = \{t\vect{x} \mid t \in \RR\}
\end{equation*}
which we write as $\func{span}\{\vect{x}\} = \RR \vect{x}$ for simplicity.
\begin{equation*}
\func{span}\{\vect{x}, \vect{y}\} = \{r\vect{x} + s\vect{y} \mid  r, s \in \RR \}
\end{equation*}
In particular, the above discussion shows that, if $\vect{v}$ and $\vect{w}$ are two nonzero, nonparallel vectors in $\RR^3$, then
\begin{equation*}
M = \func{span}\{\vect{v}, \vect{w} \}
\end{equation*}
is the plane in $\RR^3$ containing $\vect{v}$ and $\vect{w}$. Moreover, if $\vect{d}$ is any nonzero vector in $\RR^3$ (or $\RR^2$), then
\begin{equation*}
L = \func{span} \{\vect{v}\} = \{t\vect{d} \mid t \in \RR\} = \RR\vect{d}
\end{equation*}
is the line with direction vector $\vect{d}$. Hence lines and planes can both be described in terms of spanning sets.

\begin{example}{}{013596}
Let $\vect{x} = (2, -1, 2, 1)$ and $\vect{y} = (3, 4, -1, 1)$ in $\RR^4$. Determine whether $\vect{p} = (0, -11, 8, 1)$ or $\vect{q} = (2, 3, 1, 2)$ are in $U = \func{span}\{\vect{x}, \vect{y}\}$.

\begin{solution}
The vector $\vect{p}$ is in $U$ if and only if $\vect{p} = s\vect{x} + t\vect{y}$ for scalars $s$ and $t$. Equating components gives equations
\begin{equation*}
2s + 3t = 0, \quad -s + 4t = -11, \quad 2s - t = 8, \quad \mbox{and} \quad s + t = 1
\end{equation*}
This linear system has solution $s = 3$ and $t = -2$, so $\vect{p}$ is in $U$. On the other hand, asking that $\vect{q} = s\vect{x} + t\vect{y}$ leads to equations
\begin{equation*}
2s + 3t = 2, \quad -s +4t = 3, \quad 2s -t = 1, \quad \mbox{and} \quad s + t = 2
\end{equation*}
and this system has \textit{no} solution. So $\vect{q}$ does \textit{not} lie in $U$.
\end{solution}
\end{example}

\vspace*{-1em}
\begin{theorem}{Span Theorem}{013606}
Let $U = \func{span}\{\vect{x}_{1}, \vect{x}_{2}, \dots, \vect{x}_{k}\}$ in $\RR^n$. Then:

\begin{enumerate}
\item $U$ is a subspace of $\RR^n$ containing each $\vect{x}_{i}$.

\item If $W$ is a subspace of $\RR^n$ and each $\vect{x}_{i} \in W$, then $U \subseteq W$.

\end{enumerate}
\end{theorem}

\begin{proof}
\vspace*{-0.5em}
\begin{enumerate}
\item The zero vector $\vect{0}$ is in $U$ because $\vect{0} = 0\vect{x}_{1} + 0\vect{x}_{2} + \dots + 0\vect{x}_{k}$ is a linear combination of the $\vect{x}_{i}$. If $\vect{x} = t_{1}\vect{x}_{1} + t_{2}\vect{x}_{2} + \dots + t_{k}\vect{x}_{k}$ and $\vect{y} = s_{1}\vect{x}_{1} + s_{2}\vect{x}_{2} + \dots + s_{k}\vect{x}_{k}$ are in $U$, then $\vect{x} + \vect{y}$ and $a\vect{x}$ are in $U$ because
\vspace*{-0.5em}
\begin{eqnarray*}
\vect{x} + \vect{y} &=& (t_1 + s_1)\vect{x}_1 + (t_2 + s_2)\vect{x}_2 + \dots + (t_k + s_k)\vect{x}_k, \mbox{ and} \\
a\vect{x} &=& (at_1)\vect{x}_1 + (at_2)\vect{x}_2 + \dots + (at_k)\vect{x}_k
\end{eqnarray*}
Finally each $\vect{x}_i$ is in $U$ (for example, $\vect{x}_2 = 0\vect{x}_1 + 1\vect{x}_2 + \dots + 0\vect{x}_k$) so S1, S2, and S3 are satisfied for $U$, proving (1).

\item Let $\vect{x} = t_{1}\vect{x}_{1} + t_{2}\vect{x}_{2} + \dots + t_{k}\vect{x}_{k}$ where the $t_{i}$ are scalars and each $\vect{x}_{i} \in W$. Then each $t_{i}\vect{x}_{i} \in W$ because $W$ satisfies S3. But then $\vect{x} \in W$ because $W$ satisfies S2 (verify). This proves (2).
\end{enumerate}
\vspace*{-2em}\end{proof}

Condition (2) in Theorem~\ref{thm:013606} can be expressed by saying that $\func{span}\{\vect{x}_{1}, \vect{x}_{2}, \dots, \vect{x}_{k}\}$ is the \textit{smallest} subspace of $\RR^n$ that contains each $\vect{x}_{i}$. This is useful for showing that two subspaces $U$ and $W$ are equal, since this amounts to showing that both $U \subseteq W$ and $W \subseteq U$. Here is an example of how it is used.

\begin{example}{}{013667}
If $\vect{x}$ and $\vect{y}$ are in $\RR^n$, show that $\func{span}\{\vect{x}, \vect{y}\} = \func{span}\{\vect{x} + \vect{y}, \vect{x} - \vect{y}\}$.

\begin{solution}
Since both $\vect{x} + \vect{y}$ and $\vect{x} - \vect{y}$ are in $\func{span}\{\vect{x}, \vect{y}\}$, Theorem~\ref{thm:013606} gives
\begin{equation*}
\func{span}\{\vect{x} + \vect{y}, \vect{x} - \vect{y} \} \subseteq \func{span}\{\vect{x}, \vect{y}\}
\end{equation*}
But $\vect{x} = \frac{1}{2}(\vect{x} + \vect{y}) + \frac{1}{2}(\vect{x} - \vect{y})$ and $\vect{y} = \frac{1}{2}(\vect{x} + \vect{y}) - \frac{1}{2}(\vect{x} - \vect{y})$ are both in $\func{span}\{\vect{x} + \vect{y}, \vect{x} - \vect{y}\}$, so
\begin{equation*}
\func{span}\{\vect{x}, \vect{y}\} \subseteq \func{span}\{\vect{x} + \vect{y}, \vect{x} - \vect{y} \}
\end{equation*}
again by Theorem~\ref{thm:013606}. Thus $\func{span}\{\vect{x}, \vect{y}\} = \func{span}\{\vect{x} + \vect{y}, \vect{x} - \vect{y}\}$, as desired.
\end{solution}
\end{example}

It turns out that many important subspaces are best described by giving a spanning set. Here are three examples, beginning with an important spanning set for $\RR^n$ itself. 

Recall from Definition \ref{def:standardbasisRn} the standard basis \index{standard basis}  $\{\vect{e}_{1}, \vect{e}_{2}, \dots, \vect{e}_{n}\}$ of  $\RR^n$ as the set of columns of the $n \times n$ identity matrix. 
If $ \vect{x} = 
	\leftB \begin{array}{r}
	x_1 \\
	x_2 \\
	\vdots \\
	x_n
	\end{array} \rightB $
 is any vector in $\RR^n$, then $\vect{x} = x_{1}\vect{e}_{1} + x_{2}\vect{e}_{2} + \dots + x_{n}\vect{e}_{n}$, as the reader can verify. This proves:

\begin{example}{}{013694}
$\RR^n = \func{span}\{\vect{e}_{1}, \vect{e}_{2}, \dots, \vect{e}_{n}\}$ where $\vect{e}_{1}, \vect{e}_{2}, \dots, \vect{e}_{n}$ are the columns of $I_{n}$.
\end{example}

If $A$ is an $m \times n$ matrix $A$, the next two examples show that it is a routine matter to find spanning sets for $\func{null} A$ and $\func{im} A$.

\begin{example}{}{013706}
Given an $m \times n$ matrix $A$, let $\vect{x}_{1}, \vect{x}_{2}, \dots, \vect{x}_{k}$ denote the basic solutions to the system $A\vect{x} = \vect{0}$ given by the gaussian algorithm. Then
\begin{equation*}
\func{null } A = \func{span}\{\vect{x}_1, \vect{x}_2, \dots, \vect{x}_k\}
\end{equation*}
\begin{solution}
If $\vect{x} \in \func{null} A$, then $A\vect{x} = \vect{0}$ so Theorem~\ref{thm:001586} shows that $\vect{x}$ is a linear combination of the basic solutions; that is, $\func{null} A \subseteq \func{span}\{\vect{x}_{1}, \vect{x}_{2}, \dots, \vect{x}_{k}\}$. On the other hand, if $\vect{x}$ is in $\func{span}\{\vect{x}_{1}, \vect{x}_{2}, \dots, \vect{x}_{k}\}$, then $\vect{x} = t_{1}\vect{x}_{1} + t_{2}\vect{x}_{2} + \dots + t_{k}\vect{x}_{k}$ for scalars $t_{i}$, so
\begin{equation*}
A\vect{x} = t_1A\vect{x}_1 + t_2A\vect{x}_2 + \dots +  t_kA\vect{x}_k = t_1\vect{0} + t_2\vect{0} + \dots + t_k\vect{0} = \vect{0}
\end{equation*}
This shows that $\vect{x} \in \func{null} A$, and hence that $\func{span}\{\vect{x}_{1}, \vect{x}_{2}, \dots, \vect{x}_{k}\} \subseteq \func{null} A$. Thus we have equality.
\end{solution}
\end{example}

\begin{example}{}{013734}
Let $\vect{c}_{1}, \vect{c}_{2}, \dots, \vect{c}_{n}$ denote the columns of the $m \times n$ matrix $A$. Then
\begin{equation*}
\func{im } A = \func{span}\{\vect{c}_1, \vect{c}_2, \dots, \vect{c}_n\}
\end{equation*}
\begin{solution}
If $\{\vect{e}_{1}, \vect{e}_{2}, \dots, \vect{e}_{n}\}$ is the standard basis of $\RR^n$, observe that
\begin{equation*}
\leftB \begin{array}{cccc}
A\vect{e}_1 & A\vect{e}_2 & \cdots & A\vect{e}_n
\end{array}\rightB = A \leftB \begin{array}{cccc}
\vect{e}_1 & \vect{e}_2 & \cdots & \vect{e}_n \end{array}\rightB = AI_n = A = \leftB \begin{array}{cccc}
\vect{c}_1 & \vect{c}_2 & \cdots  \vect{c}_n \end{array}\rightB.
\end{equation*}
Hence $\vect{c}_{i} = A\vect{e}_{i}$ is in $\func{im} A$ for each $i$, so $\func{span}\{\vect{c}_{1}, \vect{c}_{2}, \dots, \vect{c}_{n}\} \subseteq \func{ im} A$.

Conversely, let $\vect{y}$ be in $\func{im} A$, say $\vect{y} = A\vect{x}$ for some $\vect{x}$ in $\RR^n$. If
$\vect{x} = 
\leftB \begin{array}{c}
x_1 \\
x_2 \\
\vdots \\
x_n
\end{array} \rightB
$, then Definition~\ref{def:002668} gives
\begin{equation*}
\vect{y} = A\vect{x} = x_1\vect{c}_1 + x_2\vect{c}_2 + \dots + x_n\vect{c}_n \mbox{ is in } \func{span} \{\vect{c}_1, \vect{c}_2, \ \dots, \vect{c}_n\}
\end{equation*}
This shows that $\func{im} A \subseteq \func{span}\{\vect{c}_{1}, \vect{c}_{2}, \dots, \vect{c}_{n}\}$, and the result follows.
\end{solution}
\end{example}
