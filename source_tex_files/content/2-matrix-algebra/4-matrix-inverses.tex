\section{Matrix Inverses}
\label{sec:2_4}

Three basic operations on matrices, addition, multiplication, and subtraction, are analogs for matrices of the same operations for numbers. In this section we introduce the matrix analog of numerical division.\index{matrix algebra!numerical division}


To begin, consider how a numerical equation $ax = b$ is solved when $a$ and $b$ are known numbers. If $a = 0$, there is no solution (unless $b = 0$). But if $a \neq 0$, we can multiply both sides by the inverse $a^{-1} = \frac{1}{a}$ to obtain the solution $x = a^{-1}b$. Of course multiplying by $a^{-1}$ is just dividing by $a$, and the property of $a^{-1}$ that makes this work is that $a^{-1}a = 1$. Moreover, we saw in Section~\ref{sec:2_2} that the role that $1$ plays in arithmetic is played in matrix algebra by the identity matrix $I$. This suggests the following definition. 


\begin{definition}{Matrix Inverses}{004202}
If $A$ is a square matrix, a matrix $B$ is called an \textbf{inverse}\index{inverses!defined}\index{matrix algebra!inverses} of $A$ if and only if
\begin{equation*}
AB = I \quad \mbox{ and } \quad BA = I
\end{equation*}
A matrix $A$ that has an inverse is called an \textbf{invertible matrix}\index{invertible matrix!defined}\index{matrix!invertible matrix}.\footnotemark
\end{definition}
\footnotetext{\label{fn:inversematrices}Only square matrices have inverses. Even though it is plausible that nonsquare matrices $A$ and $B$ could exist such that $AB = I_{m}$ and $BA = I_{n}$, where $A$ is $m \times n$ and $B$ is $n \times m$, we claim that this forces $n = m$. Indeed, if $m < n$ there exists a nonzero column $\vect{x}$ such that $A\vect{x} = \vect{0}$ (by Theorem~\ref{thm:001473}), so $\vect{x} = I_{n}\vect{x} = (BA)\vect{x} = B(A\vect{x}) = B(\vect{0}) = \vect{0}$, a contradiction. Hence $m \geq n$. Similarly, the condition $AB = I_{m}$ implies that $n \geq m$. Hence $m = n$ so $A$ is square.}

\hspace*{0em}\vspace*{-3em}
\begin{example}{}{004207}
Show that $B = \leftB \begin{array}{rr}
-1 & 1 \\
1 & 0
\end{array} \rightB$
 is an inverse of $A = \leftB \begin{array}{rr}
 0 & 1 \\
 1 & 1
 \end{array} \rightB$.

\begin{solution}
  Compute $AB$ and $BA$.
\begin{equation*}
AB = \leftB \begin{array}{rr}
0 & 1 \\
1 & 1
\end{array} \rightB \leftB \begin{array}{rr}
-1 & 1 \\
1 & 0
\end{array} \rightB = \leftB \begin{array}{rr}
1 & 0 \\
0 & 1
\end{array} \rightB \quad
BA = \leftB \begin{array}{rr}
-1 & 1 \\
1 & 0
\end{array} \rightB \leftB \begin{array}{rr}
0 & 1 \\
1 & 1
\end{array} \rightB = \leftB \begin{array}{rr}
1 & 0 \\
0 & 1
\end{array} \rightB
\end{equation*}
Hence $AB = I = BA$, so $B$ is indeed an inverse of $A$.
\end{solution}
\end{example}

\hspace*{0em}\vspace*{-3em}
\begin{example}{}{004217}
Show that $A = \leftB \begin{array}{rr}
0 & 0 \\
1 & 3
\end{array} \rightB$
 has no inverse.


\begin{solution}
  Let $B = \leftB \begin{array}{rr}
  a & b \\
  c & d
  \end{array} \rightB$
 denote an arbitrary $2 \times 2$ matrix. Then
\begin{equation*}
AB = \leftB \begin{array}{rr}
0 & 0 \\
1 & 3
\end{array} \rightB \leftB \begin{array}{rr}
a & b \\
c & d
\end{array} \rightB = \leftB \begin{array}{cc}
0 & 0 \\
a + 3c & b + 3d
\end{array} \rightB
\end{equation*}
so $AB$ has a row of zeros. Hence $AB$ cannot equal $I$ for any $B$.
\end{solution}
\end{example}

\vspace*{-1em}
The argument in Example~\ref{exa:004217} shows that no zero matrix has an inverse. But Example~\ref{exa:004217} also shows that, unlike arithmetic, \textit{it is possible for a nonzero matrix to have no inverse}. However, if a matrix \textit{does} have an inverse, it has only one.\index{inverses!nonzero matrix}\index{inverses!and zero matrices}\index{zero matrix!no inverse}

\vspace*{-1em}
\begin{theorem}{}{004227}
If $B$ and $C$ are both inverses of $A$, then $B = C$.
\end{theorem}

\begin{proof}
Since $B$ and $C$ are both inverses of $A$, we have $CA = I = AB$. Hence 
\begin{equation*}
B = IB = (CA)B = C(AB) = CI = C
\end{equation*}
\end{proof}

If $A$ is an invertible matrix, the (unique) inverse of $A$ is denoted $A^{-1}$. Hence $A^{-1}$ (when it exists) is a square matrix of the same size as $A$ with the property that
\begin{equation*}
AA^{-1} = I \quad \mbox{ and } \quad A^{-1}A = I
\end{equation*}
These equations characterize $A^{-1}$ in the following sense: 

\begin{quotation}
\noindent\textbf{Inverse Criterion:} {\slshape If somehow a matrix $B$ can be found such that $AB = I$ and $BA = I$, then $A$ is invertible and $B$ is the inverse of $A$; in symbols, $B = A^{-1}$.}
\end{quotation}

\noindent This is a way to verify that the inverse of a matrix exists. Example~\ref{exa:004241} and Example~\ref{exa:004261} offer illustrations.


\begin{example}{}{004241}
If $A = \leftB \begin{array}{rr}
0 & -1 \\
1 & -1
\end{array} \rightB$, show that $A^{3} = I$ and so find $A^{-1}$.


\begin{solution}
  We have $A^{2} = \leftB \begin{array}{rr}
  0 & -1 \\
  1 & -1
  \end{array} \rightB \leftB \begin{array}{rr}
  0 & -1 \\
  1 & -1
  \end{array} \rightB = \leftB \begin{array}{rr}
  -1 & 1 \\
  -1 & 0
  \end{array} \rightB$, and so
\begin{equation*}
A^{3} = A^{2}A = \leftB \begin{array}{rr}
-1 & 1 \\
-1 & 0
\end{array} \rightB \leftB \begin{array}{rr}
0 & -1 \\
1 & -1
\end{array} \rightB = \leftB \begin{array}{rr}
1 & 0 \\
0 & 1
\end{array} \rightB = I
\end{equation*}
Hence $A^{3} = I$, as asserted. This can be written as $A^{2}A = I = AA^{2}$, so it shows that $A^{2}$ is the inverse of $A$. That is, $A^{-1} = A^{2} = \leftB \begin{array}{rr}
-1 & 1 \\
-1 & 0
\end{array} \rightB$.
\end{solution}
\end{example}

The next example presents a useful formula for the inverse\index{determinants!and inverses} of a $2 \times 2$ matrix $A = \leftB \begin{array}{cc}
a & b \\
c & d
\end{array} \rightB$ when it exists. To state it, we define the \textbf{determinant}\index{determinants!defined}\index{square matrix ($n \times n$ matrix)!determinants} $\func{det }A$ and the \textbf{adjugate}\index{adjugate}\index{determinants!adjugate} $\func{adj }A$ of the matrix $A$ as follows:
\begin{equation*}
\func{det}\leftB \begin{array}{cc}
a & b \\
c & d
\end{array} \rightB = ad - bc, \quad \mbox{ and } \quad \func{adj} \leftB \begin{array}{cc}
a & b \\
c & d
\end{array} \rightB = \leftB \begin{array}{rr}
d & -b \\
-c & a
\end{array} \rightB
\end{equation*}
\begin{example}{}{004261}
If $A = \leftB \begin{array}{cc}
a & b \\
c & d
\end{array} \rightB$, show that $A$ has an inverse if and only if $\func{det } A \neq 0$, and in this case
\begin{equation*}
A^{-1} = \frac{1}{\func{det } A} \func{adj } A
\end{equation*}
\begin{solution}
  For convenience, write $e = \func{det } A = ad - bc$ and 
  $B = \func{adj } A = \leftB \begin{array}{rr}
  d & -b \\
  -c & a
  \end{array} \rightB$. Then $AB = eI = BA$ as the reader can verify. So if $e \neq 0$, scalar multiplication by $\frac{1}{e}$ gives 
\begin{equation*}
A(\frac{1}{e}B) = I = (\frac{1}{e}B)A
\end{equation*}
Hence $A$ is invertible and $A^{-1} = \frac{1}{e}B$. Thus it remains only to show that if $A^{-1}$ exists, then $e \neq 0$.

We prove this by showing that assuming $e = 0$ leads to a contradiction. In fact, if $e = 0$, then $AB = eI = 0$, so left multiplication by $A^{-1}$ gives $A^{-1}AB = A^{-1}0$; that is, $IB = 0$, so $B = 0$. But this implies that $a$, $b$, $c$, and $d$ are \textit{all} zero, so $A = 0$, contrary to the assumption that $A^{-1}$ exists.
\end{solution}
\end{example}

\noindent As an illustration, if $A = \leftB \begin{array}{rr}
2 & 4 \\
-3 & 8
\end{array} \rightB$
 then $\func{det } A = 2 \cdot 8 - 4 \cdot (-3) = 28 \neq 0$. Hence $A$ is invertible and $A^{-1} = \frac{1}{\func{det } A} \func{adj } A = \frac{1}{28} \leftB \begin{array}{rr}
 8 &-4 \\
 3 & 2
 \end{array} \rightB$, as the reader is invited to verify.


The determinant and adjugate will be defined in Chapter~\ref{chap:3} for any square matrix, and the conclusions in Example~\ref{exa:004261} will be proved in full generality.


\subsection*{Inverses and Linear Systems}


Matrix inverses can be used to solve certain systems of linear equations. Recall that a \textit{system} of linear equations can be written as a \textit{single} matrix equation\index{inverses!and linear systems}\index{system of linear equations!inverses and}
\begin{equation*}
A\vect{x} = \vect{b}
\end{equation*}
where $A$ and $\vect{b}$ are known and $\vect{x}$ is to be determined. If $A$ is invertible, we multiply each side of the equation on the left by $A^{-1}$ to get
\begin{align*}
A^{-1}A\vect{x} &= A^{-1}\vect{b} \\
I\vect{x} &= A^{-1}\vect{b} \\
\vect{x} &= A^{-1}\vect{b}
\end{align*}
This gives the solution to the system of equations (the reader should verify that $\vect{x} = A^{-1}\vect{b}$ really does satisfy $A\vect{x} = \vect{b}$). Furthermore, the argument shows that if $\vect{x}$ is \textit{any} solution, then necessarily $\vect{x} = A^{-1}\vect{b}$, so the solution is unique. Of course the technique works only when the coefficient matrix $A$ has an inverse. This proves Theorem~\ref{thm:004292}.


\begin{theorem}{}{004292}
Suppose a system of $n$ equations in $n$ variables is written in matrix form as
\begin{equation*}
A\vect{x} = \vect{b}
\end{equation*}
If the $n \times n$ coefficient matrix $A$ is invertible, the system has the unique solution
\begin{equation*}
\vect{x} = A^{-1}\vect{b}
\end{equation*}
\end{theorem}

\begin{example}{}{004298}
Use Example~\ref{exa:004261} to solve the system $\left\lbrace \arraycolsep=1pt \begin{array}{rrrrr}
5x_{1} & - & 3x_{2} & = & -4 \\
7x_{1} & + & 4x_{2} & = & 8
\end{array} \right.$.

\begin{solution}
  In matrix form this is $A\vect{x} = \vect{b}$ where $A = \leftB \begin{array}{rr}
  5 & -3 \\
  7 & 4
  \end{array} \rightB$, $\vect{x} = \leftB \begin{array}{c}
  x_{1} \\
  x_{2}
  \end{array} \rightB$, and $\vect{b} = \leftB \begin{array}{r}
 -4 \\
 8
 \end{array} \rightB$. Then $\func{det } A = 5 \cdot 4 - (-3) \cdot 7 = 41$, so $A$ is invertible and $A^{-1} = \frac{1}{41} \leftB \begin{array}{rr}
 4 & 3 \\
 -7 & 5
 \end{array} \rightB$
 by Example~\ref{exa:004261}. Thus Theorem~\ref{thm:004292} gives
\begin{equation*}
\vect{x} = A^{-1}\vect{b} = \frac{1}{41} \leftB \begin{array}{rr}
4 & 3 \\
-7 & 5
\end{array} \rightB \leftB \begin{array}{r}
-4 \\
8
\end{array} \rightB = \frac{1}{41} \leftB \begin{array}{r}
8 \\
68
\end{array} \rightB
\end{equation*}
so the solution is $x_{1} = \frac{8}{41}$ and $x_{2} = \frac{68}{41}$.
\end{solution}
\end{example}

\subsection*{An Inversion Method}

If a matrix $A$ is $n \times n$ and invertible\index{inverses!square matrices!application to}, it is desirable to have an efficient technique for finding the inverse. The following procedure will be justified in Section \ref{sec:2_5}.

\begin{theorem*}[label=thm:004348]{Matrix Inversion Algorithm}
If $A$ is an invertible (square) matrix, there exists a sequence of elementary row operations that carry $A$ to the identity matrix $I$ of the same size, written $A \to I$. This same series of row operations carries $I$ to $A^{-1}$; that is, $I \to A^{-1}$. The algorithm can be summarized as follows:
\begin{equation*}
\leftB \begin{array}{cc}
A & I
\end{array} \rightB \rightarrow 
\leftB \begin{array}{cc}
I & A^{-1}
\end{array} \rightB
\end{equation*}
where the row operations on $A$ and $I$ are carried out simultaneously.\index{inverses!inversion algorithm}\index{inversion algorithm}\index{matrix inversion algorithm}
\end{theorem*}

\begin{example}{}{004354}
Use the inversion algorithm to find the inverse of the matrix
\begin{equation*}
A = \leftB \begin{array}{rrr}
2 & 7 & 1 \\
1 & 4 & -1 \\
1 & 3 & 0
\end{array} \rightB
\end{equation*}
\begin{solution}
  Apply elementary row operations to the double matrix
\begin{equation*}
\leftB \begin{array}{rrr}
A & I
\end{array} \rightB = \leftB \begin{array}{rrr|rrr}
2 & 7 & 1 & 1 & 0 & 0 \\
1 & 4 & -1 & 0 & 1 & 0 \\
1 & 3 & 0 & 0 & 0 & 1
\end{array} \rightB
\end{equation*}
so as to carry $A$ to $I$. First interchange rows 1 and 2.
\begin{equation*}
\leftB \begin{array}{rrr|rrr}
1 & 4 & -1 & 0 & 1 & 0 \\
2 & 7 & 1 & 1 & 0 & 0 \\
1 & 3 & 0 & 0 & 0 & 1
\end{array} \rightB
\end{equation*}
Next subtract $2$ times row 1 from row 2, and subtract row 1 from row 3.
\begin{equation*}
\leftB \begin{array}{rrr|rrr}
1 & 4 & -1 & 0 & 1 & 0 \\
0 & -1 & 3 & 1 & -2 & 0 \\
0 & -1 & 1 & 0 & -1 & 1
\end{array} \rightB
\end{equation*}
Continue to reduced row-echelon form.
\begin{equation*}
\leftB \begin{array}{rrr|rrr}
1 & 0 & 11 & 4 & -7 & 0 \\
0 & 1 & -3 & -1 & 2 & 0 \\
0 & 0 & -2 & -1 & 1 & 1
\end{array} \rightB
\end{equation*}
\begin{equation*}
\leftB \def\arraystretch{1.5} \begin{array}{rrr|rrr}
1 & 0 & 0 & \frac{-3}{2} & \frac{-3}{2} & \frac{11}{2} \\
0 & 1 & 0 & \frac{1}{2} & \frac{1}{2} & \frac{-3}{2} \\
0 & 0 & 1 & \frac{1}{2} & \frac{-1}{2} & \frac{-1}{2}
\end{array} \rightB
\end{equation*}
Hence $A^{-1} = \frac{1}{2} \leftB \begin{array}{rrr}
-3 & -3 & 11 \\
1 & 1 & -3 \\
1 & -1 & -1
\end{array} \rightB$, as is readily verified.
\end{solution}
\end{example}

Given any $n \times n$ matrix $A$, Theorem~\ref{thm:001017} shows that $A$ can be carried by elementary row operations to a matrix $R$ in reduced row-echelon form. If $R = I$, the matrix $A$ is invertible (this will be proved in the next section), so the algorithm produces $A^{-1}$. If $R \neq I$, then $R$ has a row of zeros (it is square), so no system of linear equations $A\vect{x} = \vect{b}$ can have a unique solution. But then $A$ is not invertible by Theorem~\ref{thm:004292}. Hence, the algorithm is effective in the sense conveyed in Theorem~\ref{thm:004371}.


\begin{theorem}{}{004371}
If $A$ is an $n \times n$ matrix, either $A$ 
can be reduced to $I$ by elementary row operations or it cannot. In the 
first case, the algorithm produces $A^{-1}$; in the second case, $A^{-1}$ does not exist.
\end{theorem}

\subsection*{Properties of Inverses}


The following properties of an invertible matrix are used everywhere.\index{inverses!properties of inverses}


\begin{example}{Cancellation Laws}{004379}
 Let $A$ be an invertible matrix. Show that:\index{cancellation laws}\index{inverses!cancellation laws}


\begin{enumerate}
\item If $AB = AC$, then $B = C$.

\item If $BA = CA$, then $B = C$.

\end{enumerate}

\begin{solution}
  Given the equation $AB = AC$, left multiply both sides by $A^{-1}$ to obtain $A^{-1}AB = A^{-1}AC$. Thus $IB = IC$, that is $B = C$. This proves (1) and the proof of (2) is left to the reader.
\end{solution}
\end{example}

\noindent Properties (1) and (2) in Example~\ref{exa:004379} are described by saying that an invertible matrix can be ``left cancelled'' and ``right cancelled'', respectively\index{invertible matrix!left cancelled}\index{invertible matrix!right cancelled}\index{left cancelled invertible matrix}\index{right cancelled invertible matrix}. Note however that ``mixed'' cancellation does not hold in general: If $A$ is invertible and $AB = CA$, then $B$ and $C$ may \textit{not} be equal, even if both are $2 \times 2$. Here is a specific example:\index{invertible matrix!``mixed'' cancellation}\index{``mixed'' cancellation}
\begin{equation*}
A = \leftB \begin{array}{rr}
1 & 1 \\
0 & 1
\end{array} \rightB,\ B = \leftB \begin{array}{rr}
0 & 0 \\
1 & 2
\end{array} \rightB, C = \leftB \begin{array}{rr}
1 & 1 \\
1 & 1
\end{array} \rightB
\end{equation*}
Sometimes the inverse of a matrix is given by a formula. Example~\ref{exa:004261} is one illustration; Example~\ref{exa:004397} and Example~\ref{exa:004423} provide two more. The idea is the \textit{Inverse Criterion}: If a matrix $B$ can be found such that $AB = I = BA$, then $A$ is invertible and $A^{-1} = B$.


\begin{example}{}{004397}
If $A$ is an invertible matrix, show that the transpose $A^{T}$ is also invertible. Show further that the inverse of $A^{T}$ is just the transpose of $A^{-1}$; in symbols, $(A^{T})^{-1} = (A^{-1})^{T}$.


\begin{solution}
  $A^{-1}$ exists (by assumption). Its transpose $(A^{-1})^{T}$ is the candidate proposed for the inverse of $A^{T}$. Using the inverse criterion, we test it as follows:
\begin{equation*}
\arraycolsep=1pt
\begin{array}{lllllll}
A^{T}(A^{-1})^{T} & = & (A^{-1}A)^{T} & = & I^{T} & = & I \\
(A^{-1})^{T}A^{T} & = & (AA^{-1})^{T} & = & I^{T} & = & I
\end{array}
\end{equation*}
Hence $(A^{-1})^{T}$ is indeed the inverse of $A^{T}$; that is, $(A^{T})^{-1} = (A^{-1})^{T}$.
\end{solution}
\end{example}


\begin{example}{}{004423}
If $A$ and $B$ are invertible $n \times n$ matrices, show that their product $AB$ is also invertible and $(AB)^{-1} = B^{-1}A^{-1}$.

\begin{solution}
  We are given a candidate for the inverse of $AB$, namely $B^{-1}A^{-1}$. We test it as follows:
\begin{align*}
(B^{-1}A^{-1})(AB) &= B^{-1}(A^{-1}A)B = B^{-1}IB = B^{-1}B = I \\
(AB)(B^{-1}A^{-1}) &= A(BB^{-1})A^{-1} = AIA^{-1} = AA^{-1} = I
\end{align*}
Hence $B^{-1}A^{-1}$ is the inverse of $AB$; in symbols, $(AB)^{-1} = B^{-1}A^{-1}$.
\end{solution}
\end{example}

We now collect several basic properties of matrix inverses for reference.

\begin{theorem}{}{004442}
All the following matrices are square matrices of the same size.

\begin{enumerate}
\item $I$ is invertible and $I^{-1} = I$.

\item If $A$ is invertible, so is $A^{-1}$, and $(A^{-1})^{-1} = A$.

\item If $A$ and $B$ are invertible, so is $AB$, and $(AB)^{-1} = B^{-1}A^{-1}$.

\item If $A_{1}, A_{2}, \dots, A_{k}$ are all invertible, so is their product $A_{1}A_{2} \cdots A_{k}$, and 
\begin{equation*}
(A_{1}A_{2} \cdots A_{k})^{-1} = A_{k}^{-1} \cdots A_{2}^{-1}A_{1}^{-1}.
\end{equation*}

\item If $A$ is invertible, so is $A^k$ for any $k \geq 1$, and $(A^{k})^{-1} = (A^{-1})^{k}$.

\item If $A$ is invertible and $a \neq 0$ is a number, then $aA$ is invertible and $(aA)^{-1} = \frac{1}{a}A^{-1}$.

\item If $A$ is invertible, so is its transpose $A^{T}$, and $(A^{T})^{-1} = (A^{-1})^{T}$.

\end{enumerate}
\end{theorem}

\begin{proof}
\vspace*{-0.75em}
\begin{enumerate}
\item This is an immediate consequence of the fact that $I^{2} = I$.

\item The equations $AA^{-1} = I = A^{-1}A$ show that $A$ is the inverse of $A^{-1}$; in symbols, $(A^{-1})^{-1} = A$.

\item This is Example~\ref{exa:004423}.

\item Use induction on $k$. If $k = 1$, there is nothing to prove, and if $k = 2$, the result is property 3. If $k > 2$, assume inductively that $(A_1A_2 \cdots A_{k-1})^{-1} = A_{k-1}^{-1} \cdots A_2^{-1}A_1^{-1}$. We apply this fact together with property 3 as follows:
\begin{align*}
\leftB A_{1}A_{2} \cdots A_{k-1}A_{k} \rightB^{-1}
&= \leftB \left(A_{1}A_{2} \cdots A_{k-1}\right)A_{k} \rightB^{-1} \\
&= A_{k}^{-1}\left(A_{1}A_{2} \cdots A_{k-1}\right)^{-1} \\
&= A_{k}^{-1}\left(A_{k-1}^{-1} \cdots A_{2}^{-1}A_{1}^{-1}\right)
\end{align*}
So the proof by induction is complete.

\item This is property 4 with $A_{1} = A_{2} = \cdots  = A_{k} = A$.

\item This is left as Exercise~\ref{ex:ex2_4_29}.

\item This is Example~\ref{exa:004397}.

\end{enumerate}
\vspace*{-2em}\end{proof}

The reversal of the order of the inverses in properties 3 and 4 of Theorem~\ref{thm:004442} is a consequence of the fact that matrix multiplication is not 
commutative. Another manifestation of this comes when matrix equations are dealt with. If a matrix equation $B = C$ is given, it can be \textit{left-multiplied}\index{matrix multiplication!left-multiplication} by a matrix $A$ to yield $AB = AC$. Similarly, \textit{right-multiplication}\index{matrix multiplication!right-multiplication} gives $BA = CA$. However, we cannot mix the two\index{matrix multiplication!non-commutative}: If $B = C$, it need \textit{not} be the case that $AB = CA$ even if $A$ is invertible, for example, $A = \leftB \begin{array}{rr}
1 & 1 \\
0 & 1
\end{array} \rightB$, $B = \leftB \begin{array}{rr}
0 & 0 \\
1 & 0
\end{array} \rightB = C$.

Part 7 of Theorem~\ref{thm:004442} together with the fact that $(A^{T})^{T} = A$ gives


\begin{corollary}{}{004537}
A square matrix $A$ is invertible if and only if $A^{T}$ is invertible.
\end{corollary}

\begin{example}{}{004541}
Find $A$ if $(A^{T} - 2I)^{-1} = \leftB \begin{array}{rr}
2 & 1 \\
-1 & 0
\end{array} \rightB$.

\begin{solution}
  By Theorem~\ref{thm:004442}(2) and Example~\ref{exa:004261}, we have
\begin{equation*}
(A^{T} - 2I) = \leftB \left(A^{T} - 2I\right)^{-1} \rightB^{-1} = \leftB \begin{array}{rr}
2 & 1 \\
-1 & 0
\end{array} \rightB^{-1} = \leftB \begin{array}{rr}
0 & -1 \\
1 & 2
\end{array} \rightB
\end{equation*}
Hence $A^{T} = 2I + \leftB \begin{array}{rr}
0 & -1 \\
1 & 2
\end{array} \rightB = \leftB \begin{array}{rr}
2 & -1 \\
1 & 4
\end{array} \rightB$, so $A = \leftB \begin{array}{rr}
 2 & 1 \\
 -1 & 4
 \end{array} \rightB$
 by Theorem~\ref{thm:004442}(7).
\end{solution}
\end{example}

The following important theorem collects a number of conditions all equivalent\footnote{If $p$ and $q$ are statements, we say that $p$ \textbf{implies}\index{implies} $q$ (written $p \Rightarrow q$) if $q$ is true whenever $p$ is true. The statements are called \textbf{equivalent}\index{equivalent!statements} if both $p \Rightarrow q$ and $q \Rightarrow p$ (written $p \Leftrightarrow q$, spoken ``$p$ if and only if $q$''). See Appendix~\ref{chap:appbproofs}.} to invertibility. It will be referred to frequently below.


\begin{theorem}{Inverse Theorem}{004553}
The following conditions are equivalent for an $n \times n$ matrix $A$:\index{inverse theorem}\index{inverses!inverse theorem}

\begin{enumerate}
\item $A$ is invertible.

\item The homogeneous system $A\vect{x} = \vect{0}$ has only the trivial solution $\vect{x} = \vect{0}$.

\item $A$ can be carried to the identity matrix $I_{n}$ by elementary row operations.

\item The system $A\vect{x} = \vect{b}$ has at least one solution $\vect{x}$ for every choice of column $\vect{b}$.

\item There exists an $n \times n$ matrix $C$ such that $AC = I_{n}$.

\end{enumerate}
\end{theorem}

\begin{proof}
We show that each of these conditions implies the next, and that (5) implies (1).

(1) $\Rightarrow$ (2). If $A^{-1}$ exists, then $A\vect{x} = \vect{0}$ gives $\vect{x} = I_{n}\vect{x} = A^{-1}A\vect{x} = A^{-1}\vect{0} = \vect{0}$.


(2) $\Rightarrow$ (3). Assume that (2) is true. Certainly $A \to R$ by row operations where $R$ is a reduced, row-echelon matrix. It suffices to show that $R = I_{n}$. Suppose that this is not the case. Then $R$ has a row of zeros (being square). Now consider the augmented matrix $\leftB \begin{array}{c|c}
A & \vect{0}
\end{array} \rightB$ of the system $A\vect{x} = \vect{0}$. Then $\leftB \begin{array}{c|c}
A & \vect{0}
\end{array} \rightB \to \leftB \begin{array}{c|c}
	R & \vect{0}
\end{array} \rightB$ is the reduced form, and $\leftB \begin{array}{c|c}
R & \vect{0}
\end{array} \rightB$ also has a row of zeros. Since $R$ is square there must be at least one nonleading variable, and hence at least one parameter. Hence the system $A\vect{x} = \vect{0}$ has infinitely many solutions, contrary to (2). So $R = I_{n}$ after all.


(3) $\Rightarrow$ (4). Consider the augmented matrix $\leftB \begin{array}{c|c}
A & \vect{b}
\end{array} \rightB$ of the system $A\vect{x} = \vect{b}$. Using (3), let $A \to I_{n}$ by a sequence of row operations. Then these same operations carry $\leftB \begin{array}{c|c}
A & \vect{b}
\end{array} \rightB \to \leftB \begin{array}{c|c}
I_{n} & \vect{c}
\end{array} \rightB$ for some column $\vect{c}$. Hence the system $A\vect{x} = \vect{b}$ has a solution (in fact unique) by gaussian elimination. This proves (4).


(4) $\Rightarrow$ (5). Write $I_{n} = \leftB \begin{array}{cccc}
\vect{e}_{1} & \vect{e}_{2} & \cdots & \vect{e}_{n}
\end{array} \rightB$ where $\vect{e}_{1}, \vect{e}_{2}, \dots, \vect{e}_{n}$ are the columns of $I_{n}$. For each \newline $j = 1, 2, \dots, n$, the system $A\vect{x} = \vect{e}_{j}$ has a solution $\vect{c}_{j}$ by (4), so $A\vect{c}_{j} = \vect{e}_{j}$. Now let $C = \leftB \begin{array}{cccc}
\vect{c}_{1} & \vect{c}_{2} & \cdots & \vect{c}_{n}
\end{array} \rightB$ be the $n \times n$ matrix with these matrices $\vect{c}_{j}$ as its columns. Then Definition~\ref{def:003447} gives (5):
\begin{equation*}
AC = A \leftB \begin{array}{cccc}
\vect{c}_{1} & \vect{c}_{2} & \cdots & \vect{c}_{n}
\end{array} \rightB = \leftB \begin{array}{cccc}
A\vect{c}_{1} & A\vect{c}_{2} & \cdots & A\vect{c}_{n}
\end{array} \rightB = \leftB \begin{array}{cccc}
\vect{e}_{1} & \vect{e}_{2} & \cdots & \vect{e}_{n}
\end{array} \rightB = I_{n}
\end{equation*}
(5) $\Rightarrow$ (1). Assume that (5) is true so that $AC = I_{n}$ for some matrix $C$. Then $C\vect{x} = 0$ implies $\vect{x} = \vect{0}$ (because $\vect{x} = I_{n}\vect{x} = AC\vect{x} = A\vect{0} = \vect{0}$). Thus condition (2) holds for the matrix $C$ rather than $A$. Hence the argument above that (2) $\Rightarrow$ (3) $\Rightarrow$ (4) $\Rightarrow$ (5) (with $A$ replaced by $C$) shows that a matrix $C^\prime$ exists such that $CC^\prime = I_{n}$. But then
\begin{equation*}
A = AI_{n} = A(CC^\prime) = (AC)C^\prime = I_{n}C^\prime = C^\prime
\end{equation*}
Thus $CA = CC^\prime = I_{n}$ which, together with $AC = I_{n}$, shows that $C$ is the inverse of $A$. This proves (1).
\end{proof}

The proof of (5) $\Rightarrow$ (1) in Theorem~\ref{thm:004553} shows that if $AC = I$ for square matrices, then necessarily $CA = I$, and hence that $C$ and $A$ are inverses of each other. We record this important fact for reference.


%\setcounter{CorollaryCounter}{0} 
%\begin{corollary}{}{004612}
\begin{corollary}[code={\setcounter{\tcbcounter}{0}}]{}{004612}
If $A$ and $C$ are square matrices such that $AC = I$, then also $CA = I$. In particular, both $A$ and $C$ are invertible, $C = A^{-1}$, and $A = C^{-1}$.
\end{corollary}

Here is a quick way to remember Corollary \ref{cor:004612}. If $A$ is a square matrix, then 
\begin{enumerate}
\item If $AC=I$ then $C=A^{-1}$.
\item If $CA=I$ then $C=A^{-1}$.
\end{enumerate}

\noindent Observe that Corollary~\ref{cor:004612} is false if $A$ and $C$ are not square matrices. For example, we have
\begin{equation*}
\leftB \begin{array}{rrr}
1 & 2 & 1 \\
1 & 1 & 1
\end{array} \rightB \leftB \begin{array}{rr}
-1 & 1 \\
1 & -1 \\
0 & 1
\end{array} \rightB = I_{2} \quad \mbox{ but } 
\leftB \begin{array}{rr}
-1 & 1 \\
1 & -1 \\
0 & 1
\end{array} \rightB \leftB \begin{array}{rrr}
1 & 2 & 1 \\
1 & 1 & 1
\end{array} \rightB \neq I_{3}
\end{equation*}
In fact, it is verified in the footnote on page~\pageref{fn:inversematrices} that if $AB = I_{m}$ and $BA = I_{n}$, where $A$ is $m \times n$ and $B$ is $n \times m$, then $m = n$ and $A$ and $B$ are (square) inverses of each other.


An $n \times n$ matrix $A$ has $\func{rank }n$ if and only if (3) of Theorem~\ref{thm:004553} holds. Hence


\begin{corollary}{}{004623}
An $n \times n$ matrix $A$ is invertible if and only if $\func{rank }A = n$.
\end{corollary}

Here is a useful fact about inverses of block matrices.


\begin{example}{}{004627}
Let $P = \leftB \begin{array}{cc}
A & X \\
0 & B
\end{array} \rightB$
 and $Q = \leftB \begin{array}{cc}
 A & 0 \\
 Y & B
 \end{array} \rightB$
 be block matrices where $A$ is $m \times m$ and $B$ is $n \times n$ (possibly $m \neq n$).


\begin{enumerate}[label={\alph*.}]
\item Show that $P$ is invertible if and only if $A$ and $B$ are both invertible. In this case, show that 
\begin{equation*}
P^{-1} = \leftB \begin{array}{cc}
A^{-1} & -A^{-1}XB^{-1} \\
0 & B^{-1}
\end{array} \rightB
\end{equation*}


\item Show that $Q$ is invertible if and only if $A$ and $B$ are both invertible. In this case, show that 
\begin{equation*}
Q^{-1} = \leftB \begin{array}{cc}
A^{-1} & 0 \\
-B^{-1}YA^{-1} & B^{-1}
\end{array} \rightB
\end{equation*}


\end{enumerate}

\begin{solution}
  We do (a.) and leave (b.) for the reader.


\begin{enumerate}[label={\alph*.}]
\item If $A^{-1}$ and $B^{-1}$ both exist, write $R = \leftB \begin{array}{cc}
A^{-1} & -A^{-1}XB^{-1} \\
0 & B^{-1}
\end{array} \rightB$. Using block multiplication, one verifies that $PR = I_{m+n} = RP$, so $P$ is invertible, and $P^{-1} = R$. Conversely, suppose that $P$ is invertible, and write $P^{-1} = \leftB \begin{array}{cc}
 C & V \\
 W & D
 \end{array} \rightB$
 in block form, where $C$ is $m \times m$ and $D$ is $n \times n$.


Then the equation $PP^{-1} = I_{n+m}$ becomes
\begin{equation*}
\leftB \begin{array}{cc}
A & X \\
0 & B
\end{array} \rightB \leftB \begin{array}{cc}
C & V \\
W & D
\end{array} \rightB = \leftB \begin{array}{cc}
AC + XW & AV + XD \\
BW & BD
\end{array} \rightB = I_{m + n} = \leftB \begin{array}{cc}
I_{m} & 0 \\
0 & I_{n}
\end{array} \rightB
\end{equation*}
using block notation. Equating corresponding blocks, we find
\begin{equation*}
AC + XW = I_{m}, \quad BW = 0, \quad \mbox{ and } BD = I_{n}
\end{equation*}
Hence $B$ is invertible because $BD = I_{n}$ (by Corollary~\ref{cor:004537}), then $W = 0$ because $BW = 0$, and finally, $AC = I_{m}$ (so $A$ is invertible, again by Corollary~\ref{cor:004537}).

\end{enumerate}
\end{solution}
\end{example}


\subsection*{Inverses of Matrix Transformations}


Let $T = T_{A} : \RR^{n} \to \RR^{n}$ denote the matrix transformation induced by the $n \times n$ matrix $A$. Since $A$ is square, it may very well be invertible, and this leads to the question:\index{inverses!linear transformation}\index{matrix transformations}\index{linear transformations!inverses}


\begin{center}
What does it mean geometrically for $T$ that $A$ is invertible?
\end{center}
To answer this, let $T^\prime = T_{A^{-1}} : \RR^{n} \to \RR^{n}$ denote the transformation induced by $A^{-1}$. Then
\begin{equation}\label{eq:inverse1}
\begin{array}{lll}
T^\prime \leftB T(\vect{x}) \rightB = A^{-1} \leftB A\vect{x} \rightB = I\vect{x} = \vect{x} & & \\
& & \mbox{for all } \vect{x} \mbox{ in } \RR^{n} \\
T \leftB T^\prime(\vect{x}) \rightB = A \leftB A^{-1}\vect{x} \rightB = I\vect{x} = \vect{x} & &
\end{array}
\end{equation} 
The first of these equations asserts that, if $T$ carries $\vect{x}$ to a vector $T(\vect{x})$, then $T^\prime$ carries $T(\vect{x})$ right back to $\vect{x}$; that is $T^\prime$ ``reverses'' the action of $T$. Similarly $T$ ``reverses'' the action of $T^\prime$. Conditions (\ref{eq:inverse1}) can be stated compactly in terms of composition:
\begin{equation}\label{eq:inverse2}
T^\prime \circ T = 1_{\RR^{n}} \quad \mbox{ and } \quad T \circ T^\prime = 1_{\RR^{n}}
\end{equation}
When these conditions hold, we say that the matrix transformation $T^\prime$ is an \textbf{inverse}\index{inverses!matrix transformations} of $T$, and we have shown that if the matrix $A$ of $T$ is invertible, then $T$ has an inverse (induced by $A^{-1}$).

The converse is also true: If $T$ has an inverse, then its matrix $A$ must be invertible. Indeed, suppose $S : \RR^{n} \to \RR^{n}$ is any inverse of $T$, so that $S \circ T = 1_{\RR_{n}}$ and $T \circ S = 1_{\RR_{n}}$. It can be shown that $S$ is also a matrix transformation. If $B$ is the matrix of $S$, we have
\begin{equation*}
BA\vect{x} = S \leftB T(\vect{x}) \rightB = (S \circ T)(\vect{x}) = 1_{\RR^{n}}(\vect{x}) = \vect{x} = I_{n}\vect{x} \quad \mbox{ for all } \vect{x} \mbox{ in } \RR^{n}
\end{equation*}
It follows by Theorem~\ref{thm:002985} that $BA = I_{n}$, and a similar argument shows that $AB = I_{n}$. Hence $A$ is invertible with $A^{-1} = B$. Furthermore, the inverse transformation $S$ has matrix $A^{-1}$, so $S = T^\prime$ using the earlier notation. This proves the following important theorem.


\begin{theorem}{}{004693}
Let $T : \RR^{n} \to \RR^{n}$ denote the matrix transformation induced by an $n \times n$ matrix $A$. Then

\begin{center}
$A$ is invertible if and only if $T$ has an inverse.
\end{center}

In this case, $T$ has exactly one inverse (which we denote as $T^{-1}$), and $T^{-1} : \RR^{n} \to \RR^{n}$ is the transformation induced by the matrix $A^{-1}$. In other words
\begin{equation*}
\left(T_{A}\right)^{-1} = T_{A^{-1}}
\end{equation*}
\end{theorem}

\noindent The geometrical relationship between $T$ and $T^{-1}$ is embodied in equations (\ref{eq:inverse1}) above:
\begin{equation*}
T^{-1} \leftB T(\vect{x}) \rightB = \vect{x} \quad \mbox{ and } \quad T \leftB T^{-1}(\vect{x}) \rightB = \vect{x} \quad \mbox{ for all } \vect{x} \mbox{ in } \RR^{n}
\end{equation*}
These equations are called the \textbf{fundamental identities}\index{fundamental identities} relating $T$ and $T^{-1}$. Loosely speaking, they assert that each of $T$ and $T^{-1}$ ``reverses'' or ``undoes'' the action of the other.


This geometric view of the inverse of a linear transformation provides a new way to find the inverse of a matrix $A$. More precisely, if $A$ is an invertible matrix, we proceed as follows:


\begin{enumerate}
\item \textit{Let} $T$ \textit{be the linear transformation induced by} $A$.

\item \textit{Obtain the linear transformation} $T^{-1}$ \textit{which ``reverses'' the action of} $T$.

\item \textit{Then} $A^{-1}$ \textit{is the matrix of} $T^{-1}$.

\end{enumerate}

\noindent Here is an example.


\begin{example}{}{004725}
\begin{wrapfigure}[7]{l}{5cm}
\centering

\begin{tikzpicture}
%set up of axis environment
\begin{axis}[disabledatascaling, 
    clip mode = individual,
    width=5cm, 
    height=5cm, 
    xlabel={$x$}, 
    ylabel={$y$}, 
    axis lines=middle, 
    xtick=\empty, 
    ytick=\empty, 
    xticklabels=\empty, 
    yticklabels=\empty, 
    every axis x label/.style={
      at={(ticklabel* cs:1.05)},
      anchor=north,
    },
    every axis y label/.style={
      at={(ticklabel* cs:1.05)},
      anchor=east,
    },
    domain=-0.2:1.02, 
    samples=100, 
    xmin=-0.25, 
    xmax=1.25, 
    ymin=-0.25, 
    ymax=1.25]
\draw[dkgreenvect,thick,-latex](0,0)--(1/3,1);
\draw[dkgreenvect,thick,-latex](0,0)--(1,1/3);
\addplot[dkgreenvect, thick] {\x};
\draw[dkbluevect,thick,dashed,shorten <=0.08cm](1/3,1)--(1,1/3);
\fill[black] (1/3,1) circle (2pt);
\fill[black] (1,1/3) circle (2pt);
\node[below right,font=\footnotesize,text=black] at (0,0){$\vect{0}$};
\node[font=\footnotesize,text=black] at (1.1,0.8){$y = x$};
\node[above,font=\scriptsize,text=black] at (0.7,1){$Q_{1} \leftB \begin{array}{c} x \\ y \end{array} \rightB = \leftB \begin{array}{c} y \\ x \end{array} \rightB$};
\node[right,font=\scriptsize,text=black] at (1,1/3){$\leftB \begin{array}{c} x \\ y \end{array} \rightB$};
\end{axis}
\end{tikzpicture}

\end{wrapfigure}

\setlength{\rightskip}{0pt plus 200pt}
Find the inverse of $A = \leftB \begin{array}{rr}
0 & 1 \\
1 & 0
\end{array} \rightB$
 by viewing it as a linear transformation $\RR^{2} \to \RR^{2}$.

\begin{solution}
  If $\vect{x} = \leftB \begin{array}{c}
  x \\
  y
  \end{array} \rightB$
 the vector $A\vect{x} = \leftB \begin{array}{rr}
 0 & 1 \\
 1 & 0
 \end{array} \rightB \leftB \begin{array}{c}
 x \\
 y
 \end{array} \rightB = \leftB \begin{array}{c}
 y \\
 x
 \end{array} \rightB$
 is the result of reflecting $\vect{x}$ in the line $y = x$ (see the diagram). Hence, if $Q_{1} : \RR^{2} \to \RR^{2}$ denotes reflection in the line $y = x$, then $A$ is the matrix of $Q_{1}$. Now observe that $Q_{1}$ reverses itself because reflecting a vector $\vect{x}$ twice results in $\vect{x}$. Consequently $Q_{1}^{-1} = Q_{1}$. Since $A^{-1}$ is the matrix of $Q_{1}^{-1}$ and $A$ is the matrix of $Q$, it follows that $A^{-1} = A$. Of course this conclusion is clear by simply observing directly that $A^{2} = I$, but the geometric method can often work where these other methods may be less straightforward.
\end{solution}
\end{example}
