\noindent In the study of systems of linear equations in Chapter \ref{chap:1}, we found it convenient to manipulate the augmented matrix of the system. Our aim was to reduce it to row-echelon form (using elementary row operations) and hence to write down all solutions to the system. In the present chapter we consider matrices for their own sake. While some of the motivation comes from linear equations, it turns out that matrices can be multiplied and added and so form an algebraic system somewhat analogous to the real numbers. This ``matrix algebra'' is useful in ways that are quite different from the study of linear equations. For example, the geometrical transformations obtained by rotating the euclidean plane about the origin can be viewed as multiplications by certain $2 \times 2$ matrices. These ``matrix transformations''\index{linear transformations!described} are an important tool in geometry\index{geometry} and, in turn, the geometry provides a ``picture'' of the matrices. Furthermore, matrix algebra has many other applications, some of which will be explored in this chapter. This subject is quite old and was first studied systematically in 1858 by \index{Cayley, Arthur}Arthur Cayley.\footnote{Arthur Cayley (1821-1895) showed his mathematical talent early and graduated from Cambridge in 1842 as senior wrangler. With no employment in mathematics in view, he took legal training and worked as a lawyer while continuing to do mathematics, publishing nearly 300 papers in fourteen years. Finally, in 1863, he accepted the Sadlerian professorship in Cambridge and remained there for the rest of his life, valued for his administrative and teaching skills as well as for his scholarship. His mathematical achievements were of the first rank. In addition to originating matrix theory and the theory of determinants, he did fundamental work in group theory\index{group theory}, in higher-dimensional geometry\index{higher-dimensional geometry}, and in the theory of invariants. He was one of the most prolific mathematicians of all time and produced 966 papers.\index{determinants!theory of determinants}}\index{matrix algebra!usefulness of}\index{matrix theory}
