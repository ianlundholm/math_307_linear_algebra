\section*{Exercises for \ref{sec:4_4}}

\begin{Filesave}{solutions}
\solsection{Section~\ref{sec:4_4}}
\end{Filesave}

\begin{multicols}{2}
\begin{ex}
In each case show that that $T$ is either projection on a line, reflection in a line, or rotation through an angle, and find the line or angle.

\begin{enumerate}[label={\alph*.}]
\item $T\leftB
\begin{array}{c}
x\\
y 
\end{array}
\rightB
= \frac{1}{5}
\leftB
\begin{array}{c}
x + 2y\\
2x + 4y 
\end{array}
\rightB$
\item $T\leftB
\begin{array}{c}
x\\
y 
\end{array}
\rightB
= \frac{1}{2}
\leftB
\begin{array}{c}
x - y\\
y - x
\end{array}
\rightB$
\item $T\leftB
\begin{array}{c}
x\\
y 
\end{array}
\rightB
= \frac{1}{\sqrt{2}}
\leftB
\begin{array}{c}
-x - y\\
x - y 
\end{array}
\rightB$
\item $T\leftB
\begin{array}{c}
x\\
y 
\end{array}
\rightB
= \frac{1}{5}
\leftB
\begin{array}{c}
-3x + 4y\\
4x + 3y 
\end{array}
\rightB$
\item $T\leftB
\begin{array}{c}
x\\
y 
\end{array}
\rightB =
\leftB
\begin{array}{c}
-y\\
-x 
\end{array}
\rightB$
\item $T\leftB
\begin{array}{c}
x\\
y 
\end{array}
\rightB
= \frac{1}{2}
\leftB
\begin{array}{c}
x - \sqrt{3}y\\
\sqrt{3}x + y 
\end{array}
\rightB$
\end{enumerate}
\begin{sol}
\begin{enumerate}[label={\alph*.}]
\setcounter{enumi}{1}
\item  $A = \leftB
\begin{array}{rr}
1 & -1\\
-1 & 1\\
\end{array}
\rightB$, projection on $y = -x$.

\setcounter{enumi}{3}
\item $A = \frac{1}{5}\leftB
\begin{array}{rr}
-3 & 4\\
4 & 3\\
\end{array}
\rightB$, reflection in $y = 2x$.

\setcounter{enumi}{5}
\item  $A = \frac{1}{2}\leftB
\begin{array}{rr}
1 & -\sqrt{3}\\
\sqrt{3} & 1\\
\end{array}
\rightB$,
 rotation through $\frac{\pi}{3}$.

\end{enumerate}
\end{sol}
\end{ex}

\begin{ex}
Determine the effect of the following transformations.


\begin{enumerate}[label={\alph*.}]
\item Rotation through $\frac{\pi}{2}$, followed by projection on the $y$ axis, followed by reflection in the line $y = x$.

\item Projection on the line $y = x$ followed by projection on the line $y = -x$.

\item Projection on the $x$ axis followed by reflection in the line $y = x$.

\end{enumerate}
\begin{sol}
\begin{enumerate}[label={\alph*.}]
\setcounter{enumi}{1}
\item  The zero transformation.

\end{enumerate}
\end{sol}
\end{ex}

\begin{ex}
In each case solve the problem by finding the matrix of the operator.


\begin{enumerate}[label={\alph*.}]
\item Find the projection of 
$\vect{v} = \leftB
\begin{array}{r}
1\\
-2\\
3 
\end{array}
\rightB$
 on the plane with equation $3x - 5y + 2z = 0$.

\item Find the projection of 
$\vect{v} = \leftB
\begin{array}{r}
0\\
1\\
-3 
\end{array}
\rightB$
 on the plane with equation $2x - y + 4z = 0$.

\item Find the reflection of 
$\vect{v} = \leftB
\begin{array}{r}
1\\
-2\\
3 
\end{array}
\rightB$
 in the plane with equation $x - y + 3z = 0$.

\item Find the reflection of 
$\vect{v} = \leftB
\begin{array}{r}
0\\
1\\
-3 
\end{array}
\rightB$
 in the plane with equation $2x + y -5z = 0$.

\item Find the reflection of 
$\vect{v} = \leftB
\begin{array}{r}
2\\
5\\
-1 
\end{array}
\rightB$
in the line with equation 
$\leftB
\begin{array}{r}
x\\
y\\
z 
\end{array}
\rightB
= t
\leftB
\begin{array}{r}
1\\
1\\
-2 
\end{array}
\rightB$.

\item Find the projection of 
$\vect{v} = \leftB
\begin{array}{r}
1\\
-1\\
7 
\end{array}
\rightB$
 on the line with equation $\leftB
 \begin{array}{r}
 x\\
 y\\
 z 
 \end{array}
 \rightB
 = t
 \leftB
 \begin{array}{r}
 3\\
 0\\
 4 
 \end{array}
 \rightB$.

\item Find the projection of 
$\vect{v} = \leftB
\begin{array}{r}
1\\
1\\
-3
\end{array}
\rightB$
 on the line with equation 
 $\leftB
 \begin{array}{r}
 x\\
 y\\
 z 
 \end{array}
 \rightB
 = t
 \leftB
 \begin{array}{r}
 2\\
 0\\
 -3 
 \end{array}
 \rightB$.

\item Find the reflection of 
$\vect{v} = \leftB
\begin{array}{r}
2\\
-5\\
0 
\end{array}
\rightB$
 in the line with equation 
 $\leftB
 \begin{array}{r}
 x\\
 y\\
 z 
 \end{array}
 \rightB
 = t
 \leftB
 \begin{array}{r}
 1\\
 1\\
 -3
 \end{array}
 \rightB$.

\end{enumerate}
\begin{sol}
\begin{enumerate}[label={\alph*.}]
\setcounter{enumi}{1}
\item 
$\frac{1}{21}\leftB
\begin{array}{rrr}
	17 & 2 & -8\\
	2 & 20 & 4\\
	-8 & 4 & 5
\end{array}
\rightB
\leftB
\begin{array}{r}
0\\
1\\
-3
\end{array}
\rightB$

\setcounter{enumi}{3}
\item  
$\frac{1}{30}\leftB
\begin{array}{rrr}
22 & -4 & 20\\
-4 & 28 & 10\\
20 & 10 & -20
\end{array}
\rightB
\leftB
\begin{array}{r}
0\\
1\\
-3
\end{array}
\rightB$

\setcounter{enumi}{5}
\item  
$\frac{1}{25}\leftB
\begin{array}{rrr}
9 & 0 & 12\\
0 & 0 & 0\\
12 & 0 & 16
\end{array}
\rightB
\leftB
\begin{array}{r}
1\\
-1\\
7
\end{array}
\rightB$

\setcounter{enumi}{7}
\item 
 $\frac{1}{11}\leftB
\begin{array}{rrr}
-9 & 2 & -6\\
2 & -9 & -6\\
-6 & -6 & 7
\end{array}
\rightB
\leftB
\begin{array}{r}
2\\
-5\\
0
\end{array}
\rightB$

\end{enumerate}
\end{sol}
\end{ex}

\begin{ex}
\begin{enumerate}[label={\alph*.}]
\item Find the rotation of 
$\vect{v} = \leftB
\begin{array}{r}
2\\
3\\
-1 
\end{array}
\rightB$
 about the $z$ axis through $\theta = \frac{\pi}{4}$.

\item Find the rotation of 
$\vect{v} = \leftB
\begin{array}{r}
1\\
0\\
3 
\end{array}
\rightB$
 about the $z$ axis through $\theta = \frac{\pi}{6}$.

\end{enumerate}
\begin{sol}
\begin{enumerate}[label={\alph*.}]
\setcounter{enumi}{1}
\item  
$\frac{1}{2}\leftB
\begin{array}{rrr}
\sqrt{3} & -1 & 0\\
1 & \sqrt{3} & 0\\
0 & 0 & 1
\end{array}
\rightB
\leftB
\begin{array}{r}
1\\
0\\
3
\end{array}
\rightB$

\end{enumerate}
\end{sol}
\end{ex}

\begin{ex}
Find the matrix of the rotation in $\RR^3$ about the $x$ axis through the angle $\theta$ (from the positive $y$ axis to the positive $z$ axis).
\end{ex}

\begin{ex}
Find the matrix of the rotation about the $y$ axis through the angle $\theta$ (from the positive $x$ axis to the positive $z$ axis).

\begin{sol}
$\leftB
\begin{array}{ccc}
\cos\theta & 0 & -\sin\theta\\
0 & 1 & 0\\
\sin\theta & 0 & \cos\theta
\end{array}
\rightB$
\end{sol}
\end{ex}

\begin{ex}
If $A$ is $3 \times 3$, show that the image of the line in $\RR^3$ through $\vect{p}_{0}$ with direction vector $\vect{d}$ is the line through $A\vect{p}_{0}$ with direction vector $A\vect{d}$, assuming that $A\vect{d} \neq \vect{0}$. What happens if $A\vect{d} = \vect{0}$?
\end{ex}

\begin{ex}
If $A$ is $3 \times 3$ and invertible, show that the image of the plane through the origin with normal $\vect{n}$ is the plane through the origin with normal $\vect{n}_{1} = B\vect{n}$ where $B = (A^{-1})^{T}$. [\textit{Hint}: Use the fact that $\vect{v} \dotprod  \vect{w} = \vect{v}^{T}\vect{w}$ to show that $\vect{n}_{1} \dotprod (A\vect{p}) = \vect{n} \dotprod \vect{p}$ for each $\vect{p}$ in $\RR^3$.]
\end{ex}


\begin{ex}
Let $L$ be the line through the origin in $\RR^2$ with direction vector $\vect{d} = \leftB
\begin{array}{r}
a\\
b\\
\end{array}
\rightB \neq 0$.

\begin{enumerate}[label={\alph*.}]
\item If $P_{L}$ denotes projection on $L$, show that $P_{L}$ has matrix $\frac{1}{a^2 + b^2}\leftB
\begin{array}{cc}
a^2 & ab\\
ab & b^2\\
\end{array}\rightB$.

\item If $Q_{L}$ denotes reflection in $L$, show that $Q_{L}$ has matrix $\frac{1}{a^2 + b^2}\leftB
\begin{array}{cc}
a^2 - b^2 & 2ab\\
2ab & b^2 - a^2\\
\end{array}\rightB$.

\end{enumerate}
\begin{sol}
\begin{enumerate}[label={\alph*.}]
\item  Write $\vect{v} = \leftB
\begin{array}{r}
x\\
y
\end{array}
\rightB$.

\begin{align*}
P_{L}(\vect{v}) = \left(\frac{\vect{v} \dotprod \vect{d}}{\vectlength \vect{d} \vectlength^2}\right)\vect{d} & = \frac{ax + by}{a^2 + b^2}\leftB
\begin{array}{r}
a\\
b
\end{array}
\rightB  \\
&  = \frac{1}{a^2 + b^2}\leftB
\begin{array}{c}
a^2x + aby\\
abx + b^2y
\end{array}
\rightB \\
& = \frac{1}{a^2 + b^2}\leftB
\begin{array}{c}
a^2 + ab\\
ab + b^2
\end{array}
\rightB \leftB
\begin{array}{r}
x\\
y
\end{array}
\rightB
\end{align*}

\end{enumerate}
\end{sol}
\end{ex}

\begin{ex}
Let $\vect{n}$ be a nonzero vector in $\RR^3$, let $L$ be the line through the origin with direction vector $\vect{n}$, and let $M$ be the plane through the origin with normal $\vect{n}$. Show that $P_{L}(\vect{v}) = Q_{L}(\vect{v}) + P_{M}(\vect{v})$ for all $\vect{v}$ in $\RR^3$. [In this case, we say that $P_{L} = Q_{L} + P_{M}$.]
\end{ex}

\begin{ex}
If $M$ is the plane through the origin in $\RR^3$ with normal $\vect{n} = \leftB
\begin{array}{r}
a\\
b\\
c 
\end{array}
\rightB$, show that $Q_{M}$ has matrix
\begin{equation*}{\small
\frac{1}{a^2 + b^2 + c^2}}{\footnotesize \leftB
\begin{array}{ccc}
b^2 + c^2 - a^2 & -2ab & -2ac \\
-2ab & a^2 + c^2 - b^2 & -2bc \\
-2ac & -2bc & a^2 + b^2 - c^2
\end{array}
\rightB}
\end{equation*}
\end{ex}
\end{multicols}
