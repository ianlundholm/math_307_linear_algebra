\section*{Exercises for \ref{sec:4_3}}

\begin{Filesave}{solutions}
\solsection{Section~\ref{sec:4_3}}
\end{Filesave}

\begin{multicols}{2}
\begin{ex}
If $\vect{i}$, $\vect{j}$, and $\vect{k}$ are the coordinate vectors, verify that $\vect{i} \times \vect{j} = \vect{k}$, $\vect{j} \times \vect{k} = \vect{i}$, and $\vect{k} \times \vect{i} = \vect{j}$.
\end{ex}

\begin{ex}
Show that $\vect{u} \times (\vect{v} \times \vect{w})$ need not equal $(\vect{u} \times \vect{v}) \times \vect{w}$ by calculating both when
\begin{equation*}
\vect{u} = \leftB
\begin{array}{r}
1\\
1\\
1
\end{array}
\rightB,
\vect{v} = \leftB
\begin{array}{r}
1\\
1\\
0
\end{array}
\rightB, \mbox{ and }
\vect{w} = \leftB
\begin{array}{r}
0\\
0\\
1
\end{array}
\rightB
\end{equation*}
\end{ex}

\begin{ex}
Find two unit vectors orthogonal to both $\vect{u}$ and $\vect{v}$ if:


\begin{enumerate}[label={\alph*.}]
\item 
$\vect{u} = \leftB
\begin{array}{r}
1\\
2\\
2
\end{array}
\rightB$, 
$\vect{v} = \leftB
\begin{array}{r}
2\\
-1\\
2
\end{array}
\rightB$

\item 
$\vect{u} = \leftB
\begin{array}{r}
1\\
2\\
-1
\end{array}
\rightB$, 
$\vect{v} = \leftB
\begin{array}{r}
3\\
1\\
2
\end{array}
\rightB$


\end{enumerate}
\begin{sol}
\begin{enumerate}[label={\alph*.}]
\setcounter{enumi}{1}
\item  $\pm \frac{\sqrt{3}}{3} \leftB
\begin{array}{r}
1\\
-1\\
-1
\end{array}
\rightB$.

\end{enumerate}
\end{sol}
\end{ex}

\begin{ex}
Find the area of the triangle with the following vertices.


\begin{enumerate}[label={\alph*.}]
\item $A(3, -1, 2)$, $B(1, 1, 0)$, and $C(1, 2, -1)$

\item $A(3, 0, 1)$, $B(5, 1, 0)$, and $C(7, 2, -1)$

\item $A(1, 1, -1)$, $B(2, 0, 1)$, and $C(1, -1, 3)$

\item $A(3, -1, 1)$, $B(4, 1, 0)$, and $C(2, -3, 0)$

\end{enumerate}
\begin{sol}
\begin{enumerate}[label={\alph*.}]
\setcounter{enumi}{1}
\item $0$

\setcounter{enumi}{3}
\item $\sqrt{5}$

\end{enumerate}
\end{sol}
\end{ex}

\begin{ex}
Find the volume of the parallelepiped determined by $\vect{w}$, $\vect{u}$, and $\vect{v}$ when:


\begin{enumerate}[label={\alph*.}]
\item 
$\vect{w} = \leftB
\begin{array}{r}
2\\
1\\
1
\end{array}
\rightB$, 
$\vect{v} = \leftB
\begin{array}{r}
1\\
0\\
2
\end{array}
\rightB$, and 
$\vect{u} = \leftB
\begin{array}{r}
2\\
1\\
-1
\end{array}
\rightB$

\item 
$\vect{w} = \leftB
\begin{array}{r}
1\\
0\\
3
\end{array}
\rightB$, 
$\vect{v} = \leftB
\begin{array}{r}
2\\
1\\
-3
\end{array}
\rightB$, and 
$\vect{u} = \leftB
\begin{array}{r}
1\\
1\\
1
\end{array}
\rightB$

\end{enumerate}
\begin{sol}
\begin{enumerate}[label={\alph*.}]
\setcounter{enumi}{1}
\item $7$

\end{enumerate}
\end{sol}
\end{ex}

\begin{ex}
Let $P_{0}$ be a point with vector $\vect{p}_{0}$, and let $ax + by + cz = d$ be the equation of a plane with normal 
$\vect{n} = \leftB
\begin{array}{r}
a\\
b\\
c
\end{array}
\rightB$.

\begin{enumerate}[label={\alph*.}]
\item Show that the point on the plane closest to $P_{0}$ has vector $\vect{p}$ given by
\begin{equation*}
\vect{p} = \vect{p}_{0} + \frac{d - (\vect{p}_{0} \dotprod \vect{n})}{\vectlength \vect{n} \vectlength^2}\vect{n}.
\end{equation*}
[\textit{Hint}: $\vect{p} = \vect{p}_{0} + t\vect{n}$ for some $t$, and $\vect{p} \dotprod \vect{n} = d$.]

\item Show that the shortest distance from $P_{0}$ to the plane is $\frac{|d - (\vect{p}_{0} \dotprod \vect{n})|}{\vectlength \vect{n} \vectlength}$.

\item Let $P_{0}^\prime$ denote the reflection of $P_{0}$ in the plane---that is, the point on the opposite side of the plane such that the line through $P_{0}$ and $P_{0}^\prime$ is perpendicular to the plane.


Show that $\vect{p}_{0} + 2\frac{d - (\vect{p}_{0} \dotprod \vect{n})}{\vectlength \vect{n} \vectlength^2}\vect{n}$ is the vector of $P_{0}^\prime$.

\end{enumerate}
\begin{sol}
\begin{enumerate}[label={\alph*.}]
\setcounter{enumi}{1}
\item  The distance is $\vectlength\vect{p} - \vect{p}_{0}\vectlength$; use part (a.).

\end{enumerate}
\end{sol}
\end{ex}

\begin{ex}
Simplify $(a\vect{u} + b\vect{v}) \times (c\vect{u} + d\vect{v})$.
\end{ex}

\begin{ex}
Show that the shortest distance from a point $P$ to the line through $P_{0}$ with direction vector $\vect{d}$ is $\frac{\vectlength \longvect{P_{0}P} \times \vect{d} \vectlength}{\vectlength \vect{d} \vectlength}$.
\end{ex}

\begin{ex}
Let $\vect{u}$ and $\vect{v}$ be nonzero, nonorthogonal vectors. If $\theta$ is the angle between them, show that $\tan\theta = \frac{\vectlength \vect{u} \times \vect{v} \vectlength}{\vect{u} \dotprod \vect{v}}.$
\end{ex}

\begin{ex}
Show that points $A$, $B$, and $C$ are all on one line if and only if $\longvect{AB} \times \longvect{AC} = 0$


\begin{sol}
$\vectlength \longvect{AB} \times \longvect{AC} \vectlength$ is the area of the parallelogram determined by $A$, $B$, and $C$.
\end{sol}
\end{ex}

\begin{ex}
Show that points $A$, $B$, $C$, and $D$ are all on one plane if and only if  $\longvect{AB} \dotprod (\longvect{AB} \times \longvect{AC}) = 0$
\end{ex}

\begin{ex}
Use Theorem~\ref{thm:012765} to confirm that, if $\vect{u}$, $\vect{v}$, and $\vect{w}$ are mutually perpendicular, the (rectangular) parallelepiped they determine has volume $\vectlength\vect{u}\vectlength\vectlength\vect{v}\vectlength\vectlength\vect{w}\vectlength$.

\begin{sol}
Because $\vect{u}$ and $\vect{v} \times \vect{w}$ are parallel, the angle $\theta$ between them is $0$ or $\pi$. Hence $\cos(\theta) = \pm 1$, so the volume is $|\vect{u} \dotprod (\vect{v} \times \vect{w})| = \vectlength\vect{u}\vectlength\vectlength\vect{v} \times \vect{w}\vectlength \cos(\theta) = \vectlength\vect{u}\vectlength\vectlength(\vect{v} \times \vect{w})\vectlength$. But the angle between $\vect{v}$ and $\vect{w}$ is $\frac{\pi}{2}$ so $\vectlength\vect{v} \times \vect{w}\vectlength = \vectlength\vect{v} \vectlength \vectlength\vect{w}\vectlength \cos(\frac{\pi}{2}) = \vectlength\vect{v}\vectlength\vectlength\vect{w}\vectlength$. The result follows.
\end{sol}
\end{ex}

\begin{ex}
Show that the volume of the parallelepiped determined by $\vect{u}$, $\vect{v}$, and $\vect{u} \times \vect{v}$ is $\vectlength\vect{u} \times \vect{v}\vectlength^{2}$.
\end{ex}

\begin{ex} \label{ex:ch4_3_ex14}
Complete the proof of Theorem~\ref{thm:012715}.
\end{ex}

\begin{ex} \label{ex:ch4_3_ex15}
Prove the following properties in Theorem~\ref{thm:012690}.

\begin{exenumerate}
\exitem Property 6
\exitem Property 7
\exitem Property 8
\end{exenumerate}

\begin{sol}
\begin{enumerate}[label={\alph*.}]
\setcounter{enumi}{1} 
\item If
$\vect{u} = \leftB
\begin{array}{r}
u_{1}\\
u_{2}\\
u_{3}
\end{array}
\rightB$,
$\vect{v} = \leftB
\begin{array}{r}
v_{1}\\
v_{2}\\
v_{3}
\end{array}
\rightB$ and
$\vect{w} = \leftB
\begin{array}{r}
w_{1}\\
w_{2}\\
w_{3}
\end{array}
\rightB$, then
$\vect{u} \times (\vect{v} + \vect{w}) = \func{det}\leftB
\begin{array}{rrr}
\vect{i} & u_{1} & v_{1} + w_{1}\\
\vect{j} & u_{2} & v_{2} + w_{2}\\
\vect{k} & u_{3} & v_{3} + w_{3}
\end{array}
\rightB$ \\
${} = 
\func{det}\leftB
\begin{array}{rrr}
\vect{i} & u_{1} & v_{1}\\
\vect{j} & u_{2} & v_{2}\\
\vect{k} & u_{3} & v_{3}
\end{array}
\rightB
+
\func{det}\leftB
\begin{array}{rrr}
\vect{i} & u_{1} & w_{1}\\
\vect{j} & u_{2} & w_{2}\\
\vect{k} & u_{3} & w_{3}
\end{array}
\rightB$ \\ ${} = (\vect{u} \times \vect{v}) + (\vect{u} \times \vect{w})$
where we used Exercise \ref{ex:ch4_3_ex21}. 
\end{enumerate}
\end{sol}
\end{ex}

\begin{ex} \label{ex:ch4_3_ex16}
\begin{enumerate}[label={\alph*.}]
\item Show that $\vect{w} \dotprod (\vect{u} \times \vect{v}) = \vect{u} \dotprod (\vect{v} \times \vect{w}) = \vect{v} \times (\vect{w} \times \vect{u})$ holds for all vectors $\vect{w}$, $\vect{u}$, and $\vect{v}$.

\item Show that $\vect{v} - \vect{w}$ and $(\vect{u} \times \vect{v}) + (\vect{v} \times \vect{w}) + (\vect{w} \times \vect{u})$ are orthogonal.

\end{enumerate}
\begin{sol}
\begin{enumerate}[label={\alph*.}]
\setcounter{enumi}{1}
\item  $(\vect{v} - \vect{w}) \dotprod [(\vect{u} \times \vect{v}) + (\vect{v} \times \vect{w}) + (\vect{w} \times \vect{u})] = (\vect{v} - \vect{w}) \dotprod (\vect{u} \times \vect{v}) + (\vect{v} - \vect{w}) \dotprod (\vect{v} \times \vect{w}) + (\vect{v} - \vect{w}) \dotprod (\vect{w} \times \vect{u}) = -\vect{w} \dotprod (\vect{u} \times \vect{v}) + 0 + \vect{v} \cdot (\vect{w} \times \vect{u}) = 0$.

\end{enumerate}
\end{sol}
\end{ex}

\begin{ex} \label{ex:ch4_3_ex17}
Show $\vect{u} \times (\vect{v} \times \vect{w}) = (\vect{u} \dotprod \vect{w}) \vect{v} - (\vect{u} \times \vect{v})\vect{w}$. [\textit{Hint}: First do it for $\vect{u} = \vect{i}$, $\vect{j}$, and $\vect{k}$; then write $\vect{u} = x\vect{i} + y\vect{j} + z\vect{k}$ and use Theorem~\ref{thm:012690}.]
\end{ex}

\begin{ex}
Prove the \textbf{Jacobi identity}\index{Jacobi identity}: 
\begin{equation*}
\vect{u} \times (\vect{v} \times \vect{w}) + \vect{v} \times (\vect{w} \times \vect{u}) + \vect{w} \times (\vect{u} \times \vect{v}) = \vect{0}
\end{equation*}
 [\textit{Hint}: The preceding exercise.]
\end{ex}

\begin{ex}
Show that 
\begin{equation*}(\vect{u} \times \vect{v}) \dotprod (\vect{w} \times \vect{z})  = \func{det}\leftB
\begin{array}{cc}
\vect{u} \dotprod \vect{w} & \vect{u} \dotprod \vect{z}\\
\vect{v} \dotprod \vect{w} & \vect{v} \dotprod \vect{z}
\end{array}
\rightB
\end{equation*}

[\textit{Hint}: Exercises~\ref{ex:ch4_3_ex16} and~\ref{ex:ch4_3_ex17}.] 
\end{ex}

\begin{ex}
Let $P$, $Q$, $R$, and $S$ be four points, not all on one plane, as in the diagram. Show that the volume of the pyramid they determine is
\begin{equation*}
\frac{1}{6}|\longvect{PQ} \dotprod (\longvect{PR} \times \longvect{PS})|.
\end{equation*}
[\textit{Hint}: The volume of a cone with base area $A$ and height $h$ as in the diagram below right is $\frac{1}{3}Ah$.]


\begin{figure}[H]
\centering
\begin{tikzpicture}[scale=0.8]
\coordinate (ptP) at (0, 0);
\coordinate (ptR) at (1.75, -0.75);
\coordinate (ptS) at (2.5, 0);

\draw[dkgreenvect, thick] (ptP)--(45:2) node (ptQ) {};
\fill[ltbluevect] (ptP)--(ptS)--(ptR)--cycle;
\draw[dkbluevect, thick, dashed] (ptP)--(ptS);
\draw[dkbluevect, thick] (ptP)--(ptR)--(ptS);
\draw[dkgreenvect, thick] (ptS)--(ptQ.center) 
	(ptQ.center)--(ptR);

\node[left] at (ptP) {\footnotesize $P$};
\node[above] at (ptQ) {\footnotesize $Q$};
\node[below] at (ptR) {\footnotesize $R$};
\node[right] at (ptS) {\footnotesize $S$};

%right image
\coordinate (ptLeft1) at (3.5, 0.6);
\coordinate[below=0.4 of ptLeft1] (ptLeft2);
\coordinate[below=0.4 of ptLeft2.center] (ptLeft3);
\coordinate[below=0.4 of ptLeft3.center] (ptLeft4);
\coordinate (ptRight) at (6, 0);
\coordinate (apex) at (4.75, 1.6);

\fill[ltbluevect] (ptLeft1) .. controls (3.3, 0.5) and (3.3, 0.3) .. (ptLeft2)
 .. controls (3.8, 0.0) and (3.8, -0.2) .. (ptLeft3)
 .. controls (3.3, -0.5) and (3.3, -0.7) .. (ptLeft4)
 .. controls (4.0, -1.2) and (5.5, -1.0) .. (ptRight)
 .. controls (5.5, 0.8) and (4.0, 1.0) .. (ptLeft1);
 
\draw[dkbluevect, thick] (ptLeft1) .. controls (3.3, 0.5) and (3.3, 0.3) .. (ptLeft2);
\draw[dkbluevect, thick, dashed] (ptLeft2) .. controls (3.8, 0) and (3.8, -0.2) .. (ptLeft3);
\draw[dkbluevect, thick] (ptLeft3) .. controls (3.3, -0.5) and (3.3, -0.7) .. (ptLeft4);
\draw[dkbluevect, thick] (ptLeft4) .. controls (4.0, -1.2) and (5.5, -1.0) .. (ptRight);
\draw[dkbluevect, thick, dashed] (ptLeft1) .. controls (4.0, 1.0) and (5.5, 0.8) .. (ptRight);

\draw[dkgreenvect, thick] (ptLeft1)--(apex)
 (ptLeft3)--(apex)
 (ptRight)--(apex) 
 (apex |- 0, 0) -- (apex) node[right, text=black, pos=0.6] {\footnotesize $h$}; %apex.x-coord and 0
\end{tikzpicture}
%\caption{\label{fig:012901}}
\end{figure}
\end{ex}

\begin{ex} \label{ex:ch4_3_ex21}
Consider a triangle with vertices $A$, $B$, and $C$, as in the diagram below. Let $\alpha$, $\beta$, and $\gamma$ denote the angles at $A$, $B$, and $C$, respectively, and let $a$, $b$, and $c$ denote the lengths of the sides opposite $A$, $B$, and $C$, respectively. Write $\vect{u} = \longvect{AB}$, $\vect{v} = \longvect{BC}$, and $\vect{w} = \longvect{CA}$.


\begin{figure}[H]
\centering
\begin{tikzpicture}
\coordinate (ptA) at (0, 0);
\coordinate (ptC) at (4.5, 0);
\path (ptA)--(60:2.937cm) node (ptB) {}; %alpha=60 deg, beta=80 deg, gamma = 40 deg. c = sin gamma / sin beta * b = 2.937

\draw[dkgreenvect, thick] (ptA)--(ptB.center) node[left, text=black, midway] {\small $c$}
	--(ptC) node[right, text=black, midway] {\small $a$} 
	--cycle node[above, text=black, midway] {\small $b$};

\draw[dkbluevect, thick] (ptA)+(0.5, 0) arc [start angle=0, end angle=60,radius=0.5] node[right, text=black, midway] {\small $\alpha$};
\draw[dkbluevect, thick] (ptB) ++(-40:0.5) arc [start angle=-40, end angle=-120, radius=0.5] node[below, text=black, midway] {\small $\beta$};
\draw[dkbluevect, thick] (ptC)+(-0.5, 0) arc [start angle=180, end angle=140,radius=0.5] node[left, text=black, pos=0.7] {\small $\gamma$};

\node[left] at (ptA) {\small $A$};
\node[above] at (ptB) {\small $B$};
\node[right] at (ptC) {\small $C$};
\end{tikzpicture}
%\caption{\label{fig:012904}}
\end{figure}

\begin{enumerate}[label={\alph*.}]
\item Deduce that $\vect{u} + \vect{v} + \vect{w} = \vect{0}$.

\item Show that $\vect{u} \times \vect{v} = \vect{w} \times \vect{u} = \vect{v} \times \vect{w}$. [\textit{Hint}: Compute $\vect{u} \times (\vect{u} + \vect{v} + \vect{w})$ and $\vect{v} \times (\vect{u} + \vect{v} + \vect{w})$.]

\item Deduce the \textbf{law of sines}\index{law of sines}:
\begin{equation*}
\frac{\sin\alpha}{a} = \frac{\sin\beta}{b} = \frac{\sin\gamma}{c}
\end{equation*}
\end{enumerate}
\end{ex}

\begin{ex}
Show that the (shortest) distance between two planes $\vect{n} \dotprod \vect{p} = d_{1}$ and $\vect{n} \dotprod \vect{p} = d_{2}$ with $\vect{n}$ as normal is $\frac{|d_{2} - d_{1}|}{\vectlength \vect{n} \vectlength}$.

\begin{sol}
Let $\vect{p}_{1}$ and $\vect{p}_{2}$ be vectors of points in the planes, so $\vect{p}_{1} \dotprod \vect{n} = d_{1}$ and $\vect{p}_{2} \dotprod \vect{n} = d_{2}$. The distance is the length of the projection of $\vect{p}_{2} - \vect{p}_{1}$ along $\vect{n}$; that is $\frac{|(\vect{p}_{2} - \vect{p}_{1}) \dotprod \vect{n}|}{\vectlength \vect{n} \vectlength} = \frac{|d_{1} - d_{2}|}{\vectlength \vect{n} \vectlength}$.
\end{sol}
\end{ex}


\begin{ex}
Let $A$ and $B$ be points other than the origin, and let $\vect{a}$ and $\vect{b}$ be their vectors. If $\vect{a}$ and $\vect{b}$ are not parallel, show that the plane through $A$, $B$, and the origin is given by
\begin{equation*}
\{P(x, y, z) \mid
\leftB
\begin{array}{r}
x\\
y\\
z
\end{array}
\rightB = s\vect{a} + t\vect{b} \mbox{ for some } s \mbox{ and } t \} 
\end{equation*}
\end{ex} 

\begin{ex}
Let $A$ be a $2 \times 3$ matrix of rank 2 with rows $\vect{r}_{1}$ and $\vect{r}_{2}$. Show that 
\begin{equation*}
P = \{XA \mid X = [x y]; x, y \mbox{ arbitrary}\}
\end{equation*}
 is the plane through the origin with normal $\vect{r}_{1} \times \vect{r}_{2}$. 
\end{ex}

\begin{ex}
Given the cube with vertices $P(x, y, z)$, where each of $x$, $y$, and $z$ is either $0$ or $2$, consider the plane perpendicular to the diagonal through $P(0, 0, 0)$ and $P(2, 2, 2)$ and bisecting it.


\begin{enumerate}[label={\alph*.}]
\item Show that the plane meets six of the edges of the cube and bisects them.

\item Show that the six points in (a) are the vertices of a regular hexagon.

\end{enumerate}
\end{ex}
\end{multicols}
