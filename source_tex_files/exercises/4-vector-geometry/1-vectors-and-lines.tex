\section*{Exercises for \ref{sec:4_1}}

\begin{Filesave}{solutions}
\solsection{Section~\ref{sec:4_1}}
\end{Filesave}

\begin{multicols}{2}
\begin{ex}
Compute $\vectlength\vect{v}\vectlength$ if $\vect{v}$ equals:

\begin{exenumerate}[column-sep=-5em]
\exitem $\leftB
\begin{array}{r}
2 \\
-1 \\
2
\end{array}
\rightB$ 
\exitem $\leftB
\begin{array}{r}
1 \\
-1 \\
2
\end{array}
\rightB$
\exitem $\leftB
\begin{array}{r}
1 \\
0 \\
-1
\end{array}
\rightB$
\exitem $\leftB
\begin{array}{r}
-1 \\
0 \\
2
\end{array}
\rightB$
\exitem $2\leftB
\begin{array}{r}
1 \\
-1 \\
2
\end{array}
\rightB$
\exitem $-3\leftB
\begin{array}{r}
1 \\
1 \\
2
\end{array}
\rightB$
\end{exenumerate}
\begin{sol}
\begin{enumerate}[label={\alph*.}]
\setcounter{enumi}{1}
\item $\sqrt{6}$ 

\setcounter{enumi}{3}
\item $\sqrt{5}$

\setcounter{enumi}{5}
\item $3\sqrt{6}$


\end{enumerate}
\end{sol}
\end{ex}


\begin{ex}
Find a unit vector in the direction of:

\begin{exenumerate}[column-sep=-5em]
\exitem $\leftB
\begin{array}{r}
7 \\
-1 \\
5
\end{array}
\rightB$
\exitem $\leftB
\begin{array}{r}
-2 \\
-1 \\
2
\end{array}
\rightB$
\end{exenumerate}
\begin{sol}
\begin{enumerate}[label={\alph*.}]
\setcounter{enumi}{1}
\item 
$\frac{1}{3}\leftB
\begin{array}{r}
-2 \\
-1 \\
2
\end{array}
\rightB$

\end{enumerate}
\end{sol}
\end{ex}

\begin{ex}
\begin{enumerate}[label={\alph*.}]
\item Find a unit vector in the direction from \\
$\leftB
\begin{array}{r}
3 \\
-1 \\
4
\end{array}
\rightB$
to
$\leftB
\begin{array}{r}
1\\
3 \\
5
\end{array}
\rightB$.

\item If $\vect{u} \neq \vect{0}$, for which values of $a$ is $a\vect{u}$ a unit vector?

\end{enumerate}
\end{ex}

\begin{ex}
Find the distance between the following pairs of points.

\begin{exenumerate}[column-sep=-15pt]
\exitem $\leftB
\begin{array}{r}
3 \\
-1 \\
0
\end{array}
\rightB$
and
$\leftB
\begin{array}{r}
2\\
-1 \\
1
\end{array}
\rightB$
\exitem $\leftB
\begin{array}{r}
2 \\
-1 \\
2
\end{array}
\rightB$
and
$\leftB
\begin{array}{r}
2\\
0 \\
1
\end{array}
\rightB$
\exitem $\leftB
\begin{array}{r}
-3 \\
5 \\
2
\end{array}
\rightB$
and
$\leftB
\begin{array}{r}
1\\
3 \\
3
\end{array}
\rightB$
\exitem $\leftB
\begin{array}{r}
4 \\
0 \\
-2
\end{array}
\rightB$
and
$\leftB
\begin{array}{r}
3\\
2 \\
0
\end{array}
\rightB$
\end{exenumerate}
\begin{sol}
\begin{enumerate}[label={\alph*.}]
\setcounter{enumi}{1}
\item 
$\sqrt{2}$

\setcounter{enumi}{3}
\item  $3$

\end{enumerate}
\end{sol}
\end{ex}

\begin{ex}
Use
 vectors to show that the line joining the midpoints of two sides of a 
triangle is parallel to the third side and half as long.
\end{ex}

\begin{ex}
Let $A$, $B$, and $C$ denote the three vertices of a triangle.


\begin{enumerate}[label={\alph*.}]
\item If $E$ is the midpoint of side $BC$, show that
\begin{equation*}
\longvect{AE} = \frac{1}{2}(\longvect{AB} + \longvect{AC})
\end{equation*}
\item If $F$ is the midpoint of side $AC$, show that
\begin{equation*}
\longvect{FE} = \frac{1}{2}\longvect{AB}
\end{equation*}
\end{enumerate}
\begin{sol}
\begin{enumerate}[label={\alph*.}]
\setcounter{enumi}{1}
\item 
$\longvect{FE} = \longvect{FC} + \longvect{CE} = \frac{1}{2}\longvect{AC} + \frac{1}{2}\longvect{CB} = \frac{1}{2}(\longvect{AC} + \longvect{CB}) = \frac{1}{2}\longvect{AB}$ 

\end{enumerate}
\end{sol}
\end{ex}

\begin{ex}
Determine whether $\vect{u}$ and $\vect{v}$ are parallel in each of the following cases.

\begin{enumerate}[label={\alph*.}]
\item 
$\vect{u} = \leftB
\begin{array}{r}
-3\\
-6\\
3
\end{array}
\rightB$;  
$\vect{v} = \leftB
\begin{array}{r}
5\\
10 \\
-5
\end{array}
\rightB$

\item 
$\vect{u} = \leftB
\begin{array}{r}
3\\
-6\\
3
\end{array}
\rightB$;
$\vect{v} = \leftB
\begin{array}{r}
-1\\
2 \\
-1
\end{array}
\rightB$


\item 
$\vect{u} = \leftB
\begin{array}{r}
1\\
0\\
1
\end{array}
\rightB$;
$\vect{v} = \leftB
\begin{array}{r}
-1\\
0 \\
1
\end{array}
\rightB$


\item 
$\vect{u} = \leftB
\begin{array}{r}
2\\
0\\
-1
\end{array}
\rightB$;
$\vect{v} = \leftB
\begin{array}{r}
-8\\
0 \\
4
\end{array}
\rightB$

\end{enumerate}
\begin{sol}
\begin{enumerate}[label={\alph*.}]
\setcounter{enumi}{1}
\item  Yes

\setcounter{enumi}{3}
\item  Yes

\end{enumerate}
\end{sol}
\end{ex}


\begin{ex}
Let $\vect{p}$ and $\vect{q}$ be the vectors of points $P$ and $Q$, respectively, and let $R$ be the point whose vector is $\vect{p} + \vect{q}$. Express the following in terms of $\vect{p}$ and $\vect{q}$.

\begin{exenumerate}[column-sep=-50pt] 
\exitem $\longvect{QP}$
\exitem $\longvect{QR}$
\exitem $\longvect{RP}$
\exitem $\longvect{RO}$ where $O$ is the origin
\end{exenumerate}
\begin{sol}
\begin{enumerate}[label={\alph*.}]
\setcounter{enumi}{1}
\item $\vect{p}$

\setcounter{enumi}{3}
\item $-(\vect{p} + \vect{q})$.
\end{enumerate}
\end{sol}
\end{ex}

\begin{ex}
In each case, find $\longvect{PQ}$ and $\vectlength \longvect{PQ} \vectlength$.

\begin{enumerate}[label={\alph*.}]
\item $P(1, -1, 3)$, $Q(3, 1, 0)$

\item $P(2, 0, 1)$, $Q(1, -1, 6)$

\item $P(1, 0, 1)$, $Q(1, 0, -3)$

\item $P(1, -1, 2)$, $Q(1, -1, 2)$

\item $P(1, 0, -3)$, $Q(-1, 0, 3)$

\item $P(3, -1, 6)$, $Q(1, 1, 4)$

\end{enumerate}
\begin{sol}
\begin{enumerate}[label={\alph*.}]
\setcounter{enumi}{1}
\item 
$\leftB
\begin{array}{r}
-1\\
-1\\
5
\end{array}
\rightB$, 
$\sqrt{27}$

\setcounter{enumi}{3}
\item 
$\leftB
\begin{array}{r}
0\\
0\\
0
\end{array}
\rightB$,
$0$

\setcounter{enumi}{5}
\item 
$\leftB
\begin{array}{r}
-2\\
2\\
2
\end{array}
\rightB$, 
$\sqrt{12}$


\end{enumerate}
\end{sol}
\end{ex}

\begin{ex}
In each case, find a point $Q$ such that $\longvect{PQ}$ has (i) the same direction as $\vect{v}$; (ii) the opposite direction to $\vect{v}$.


\begin{enumerate}[label={\alph*.}]
\item
$P(-1,2,2)$, $\vect{v} = \leftB
\begin{array}{r}
1\\
3\\
1
\end{array}
\rightB$

\item 
$P(3,0,-1)$, $\vect{v} = \leftB
\begin{array}{r}
2\\
-1\\
3
\end{array}
\rightB$


\end{enumerate}
\begin{sol}
\begin{enumerate}[label={\alph*.}]
\setcounter{enumi}{1}
\item  \textbf{(i)} $Q(5, -1, 2)$~\textbf{(ii)} $Q(1, 1, -4)$.

\end{enumerate}
\end{sol}
\end{ex}

\begin{ex}
Let 
$\vect{u} = \leftB
\begin{array}{r}
	3\\
	-1\\
	0
\end{array}
\rightB$,
$\vect{v} = \leftB
\begin{array}{r}
4\\
0\\
1
\end{array}
\rightB$, and 
$\vect{w} = \leftB
\begin{array}{r}
-1\\
1\\
5
\end{array}
\rightB$. In each case, find $\vect{x}$ such that:


\begin{enumerate}[label={\alph*.}]
\item $3(2\vect{u} + \vect{x}) + \vect{w} = 2\vect{x} - \vect{v}$

\item $2(3\vect{v} - \vect{x}) = 5\vect{w} + \vect{u} - 3\vect{x}$

\end{enumerate}
\begin{sol}
\begin{enumerate}[label={\alph*.}]
\setcounter{enumi}{1}
\item $\vect{x} = \vect{u} - 6\vect{v} + 5\vect{w} = 
\leftB
\begin{array}{r}
-26\\
4\\
19
\end{array}
\rightB$

\end{enumerate}
\end{sol}
\end{ex}

\begin{ex}
Let 
$\vect{u} = \leftB
\begin{array}{r}
1\\
1\\
2
\end{array}
\rightB$,
$\vect{v} = \leftB
\begin{array}{r}
0\\
1\\
2
\end{array}
\rightB$, and \newline
$\vect{w} = \leftB
\begin{array}{r}
1\\
0\\
-1
\end{array}
\rightB$. In each case, find numbers $a$, $b$, and $c$ such that $\vect{x} = a\vect{u} + b\vect{v} + c\vect{w}$.

\begin{exenumerate}
\exitem $\vect{x} = \leftB
\begin{array}{r}
2\\
-1\\
6
\end{array}
\rightB$
\exitem $\vect{x} = \leftB
\begin{array}{r}
1\\
3\\
0
\end{array}
\rightB$
\end{exenumerate}

\begin{sol}
\begin{enumerate}[label={\alph*.}]
\setcounter{enumi}{1}
\item 
$\leftB
\begin{array}{r}
a\\
b\\
c
\end{array}
\rightB
=
\leftB
\begin{array}{r}
-5\\
8\\
6
\end{array}
\rightB$
\end{enumerate}
\end{sol}
\end{ex}

\begin{ex}
Let 
$\vect{u} = \leftB
\begin{array}{r}
3\\
-1\\
0
\end{array}
\rightB$,
$\vect{v} = \leftB
\begin{array}{r}
4\\
0\\
1
\end{array}
\rightB$, and 
$\vect{z} = \leftB
\begin{array}{r}
1\\
1\\
1
\end{array}
\rightB$. In each case, show that there are no numbers $a$, $b$, and $c$ such that:


\begin{enumerate}[label={\alph*.}]
\item 
$a\vect{u} + b\vect{v} + c\vect{z} = \leftB
\begin{array}{r}
1\\
2\\
1
\end{array}
\rightB$


\item
$a\vect{u} + b\vect{v} + c\vect{z} = \leftB
\begin{array}{r}
5\\
6\\
-1
\end{array}
\rightB$


\end{enumerate}
\begin{sol}
\begin{enumerate}[label={\alph*.}]
\setcounter{enumi}{1}
\item If it holds then 
$\leftB
\begin{array}{c}
3a + 4b + c\\
-a + c\\
b + c
\end{array}
\rightB
=
\leftB
\begin{array}{c}
x_{1}\\
x_{2}\\
x_{3}
\end{array}
\rightB$. \\ \hspace*{-2em}$\leftB
\begin{array}{rrrr}
3 & 4 & 1 & x_{1}\\
-1 & 0 & 1 & x_{2}\\
0 & 1 & 1 & x_{3}
\end{array}
\rightB \to \leftB
\begin{array}{rrrc}
0 & 4 & 4 & x_{1} + 3x_{2}\\
-1 & 0 & 1 & x_{2}\\
0 & 1 & 1 & x_{3}
\end{array}
\rightB$

If there is to be a solution then $x_{1} + 3x_{2} = 4x_{3}$ must hold. This is not satisfied.
\end{enumerate}
\end{sol}
\end{ex}

\begin{ex}
Given $P_{1}(2, 1, -2)$ and $P_{2}(1, -2, 0)$. Find the coordinates of the point $P$:
\begin{enumerate}[label={\alph*.}]
\item $\frac{1}{5}$ the way from $P_{1}$ to $P_{2}$

\item $\frac{1}{4}$ the way from $P_{2}$ to $P_{1}$

\end{enumerate}
\begin{sol}
\begin{enumerate}[label={\alph*.}]
\setcounter{enumi}{1}
\item  
$\frac{1}{4}\leftB
\begin{array}{c}
5\\
-5\\
-2
\end{array}
\rightB
$

\end{enumerate}
\end{sol}
\end{ex}

\begin{ex}
Find the two points trisecting the segment between $P(2, 3, 5)$ and $Q(8, -6, 2)$.
\end{ex}

\begin{ex}
Let $P_{1}(x_{1}, y_{1}, z_{1})$ and $P_{2}(x_{2}, y_{2}, z_{2})$ be two points with vectors $\vect{p}_{1}$ and $\vect{p}_{2}$, respectively. If $r$ and $s$ are positive integers, show that the point $P$ lying $\frac{r}{r + s}$ the way from $P_{1}$ to $P_{2}$ has vector
\begin{equation*}
\vect{p} = \left( \frac{s}{r + s} \right)\vect{p}_{1} + \left( \frac{r}{r + s} \right)\vect{p}_{2}
\end{equation*}
\end{ex}

\begin{ex}
In each case, find the point $Q$:


\begin{enumerate}[label={\alph*.}]
\item
$\longvect{PQ} = \leftB
\begin{array}{r}
2\\
0\\
-3
\end{array}
\rightB$
and $P = P(2,-3,1)
$


\item
$\longvect{PQ} = \leftB
\begin{array}{r}
-1\\
4\\
7
\end{array}
\rightB$
and
$P = P(1,3,-4)
$


\end{enumerate}
\begin{sol}
\begin{enumerate}[label={\alph*.}]
\setcounter{enumi}{1}
\item  $Q(0, 7, 3)$.

\end{enumerate}
\end{sol}
\end{ex}


\begin{ex}
Let 
$\vect{u} = \leftB
\begin{array}{r}
2\\
0\\
-4
\end{array}
\rightB$
and 
$\vect{v} = \leftB
\begin{array}{r}
2\\
1\\
-2
\end{array}
\rightB$. In each case find $\vect{x}$:


\begin{enumerate}[label={\alph*.}]
\item $2\vect{u} - \vectlength \vect{v} \vectlength \vect{v} = \frac{3}{2}(\vect{u} - 2\vect{x})$

\item $3\vect{u} + 7\vect{v} = \vectlength\vect{u}\vectlength^{2}(2\vect{x} + \vect{v})$

\end{enumerate}
\begin{sol}
\begin{enumerate}[label={\alph*.}]
\setcounter{enumi}{1}
\item
$\vect{x} = \frac{1}{40}\leftB
\begin{array}{r}
-20\\
-13\\
14
\end{array}
\rightB$

\end{enumerate}
\end{sol}
\end{ex}

\begin{ex}
Find all vectors $\vect{u}$ that are parallel to 
$\vect{v} = \leftB
\begin{array}{r}
3\\
-2\\
1
\end{array}
\rightB$ 
and satisfy $\vectlength\vect{u}\vectlength = 3\vectlength\vect{v}\vectlength$.
\end{ex}

\begin{ex}
Let $P$, $Q$, and $R$ be the vertices of a parallelogram with adjacent sides $PQ$ and $PR$. In each case, find the other vertex $S$.

\begin{enumerate}[label={\alph*.}]
\item $P(3, -1, -1)$, $Q(1, -2, 0)$, $R(1, -1, 2)$

\item $P(2, 0, -1)$, $Q(-2, 4, 1)$, $R(3, -1, 0)$

\end{enumerate}
\begin{sol}
\begin{enumerate}[label={\alph*.}]
\setcounter{enumi}{1}
\item  $S(-1, 3, 2)$.

\end{enumerate}
\end{sol}
\end{ex}

\begin{ex}
In each case either prove the statement or give an example showing that it is false.


\begin{enumerate}[label={\alph*.}]
\item The zero vector $\vect{0}$ is the only vector of length 0.

\item If $\vectlength\vect{v} - \vect{w}\vectlength = 0$, then $\vect{v} = \vect{w}$.

\item If $\vect{v} = -\vect{v}$, then $\vect{v} = \vect{0}$.

\item If $\vectlength\vect{v}\vectlength = \vectlength\vect{w}\vectlength$, then $\vect{v} = \vect{w}$.

\item If $\vectlength\vect{v}\vectlength = \vectlength\vect{w}\vectlength$, then $\vect{v} = \pm\vect{w}$.

\item If $\vect{v} = t\vect{w}$ for some scalar $t$, then $\vect{v}$ and $\vect{w}$ have the same direction.

\item If $\vect{v}$, $\vect{w}$, and $\vect{v} + \vect{w}$ are nonzero, and $\vect{v}$ and $\vect{v} + \vect{w}$ parallel, then $\vect{v}$ and $\vect{w}$ are parallel.

\item $\vectlength-5\vect{v}\vectlength = -5\vectlength\vect{v}\vectlength$, for all $\vect{v}$.

\item If $\vectlength\vect{v}\vectlength = \vectlength 2\vect{v}\vectlength$, then $\vect{v} = \vect{0}$.

\item $\vectlength\vect{v} + \vect{w}\vectlength = \vectlength\vect{v}\vectlength + \vectlength\vect{w}\vectlength$, for all $\vect{v}$ and $\vect{w}$.

\end{enumerate}
\begin{sol}
\begin{enumerate}[label={\alph*.}]
\setcounter{enumi}{1}
\item  T. $\vectlength\vect{v} - \vect{w}\vectlength = 0$ implies that $\vect{v} - \vect{w} = \vect{0}$.

\setcounter{enumi}{3}
\item  F. $\vectlength\vect{v}\vectlength = \vectlength - \vect{v}\vectlength$ for all $\vect{v}$ but $\vect{v} = -\vect{v}$ only holds if $\vect{v} = \vect{0}$.

\setcounter{enumi}{5}
\item  F. If $t < 0$ they have the \textit{opposite} direction.

\setcounter{enumi}{7}
\item  F. $\vectlength -5\vect{v}\vectlength = 5\vectlength\vect{v}\vectlength$ for all $\vect{v}$, so it fails if $\vect{v} \neq \vect{0}$.

\setcounter{enumi}{9}
\item  F. Take $\vect{w} = -\vect{v}$ where $\vect{v} \neq \vect{0}$.

\end{enumerate}
\end{sol}
\end{ex}

\begin{ex}
Find the vector and parametric equations of the following lines.


\begin{enumerate}[label={\alph*.}]
\item The line parallel to 
$\leftB
\begin{array}{r}
2\\
-1\\
0
\end{array}
\rightB$
 and passing through $P(1, -1, 3)$.

\item The line passing through $P(3, -1, 4)$ and $Q(1, 0, -1)$.

\item The line passing through $P(3, -1, 4)$ and $Q(3, -1, 5)$.

\item The line parallel to 
$\leftB
\begin{array}{r}
1\\
1\\
1
\end{array}
\rightB$
and passing through $P(1, 1, 1)$.

\item The line passing through $P(1, 0, -3)$ and parallel to the line with parametric equations $x = -1 + 2t$, $y = 2 - t$, and $z = 3 + 3t$.

\item The line passing through $P(2, -1, 1)$ and parallel to the line with parametric equations $x = 2 - t$, $y = 1$, and $z = t$.

\item The lines through $P(1, 0, 1)$ that meet the line with vector equation 
$\vect{p} = \leftB
\begin{array}{r}
1\\
2\\
0
\end{array}
\rightB
+ t
\leftB
\begin{array}{r}
2\\
-1\\
2
\end{array}
\rightB$
 at points at distance 3 from $P_{0}(1, 2, 0)$.

\end{enumerate}
\begin{sol}
\begin{enumerate}[label={\alph*.}]
\setcounter{enumi}{1}
\item 
$\leftB
\begin{array}{r}
3\\
-1\\
4
\end{array}
\rightB
+ t
\leftB
\begin{array}{r}
2\\
-1\\
5
\end{array}
\rightB$;
$x = 3 + 2t$, $y = -1 -t$, $z = 4 + 5t$

\setcounter{enumi}{3}
\item
$\leftB
\begin{array}{r}
1\\
1\\
1
\end{array}
\rightB
+ t
\leftB
\begin{array}{r}
1\\
1\\
1
\end{array}
\rightB$;
$x = y = z = 1 + t$

\setcounter{enumi}{5}
\item 
$\leftB
\begin{array}{r}
2\\
-1\\
1
\end{array}
\rightB
+ t
\leftB
\begin{array}{r}
-1\\
0\\
1
\end{array}
\rightB$;
$x = 2 - t$, $y = -1$, $z = 1 + t$

\end{enumerate}
\end{sol}
\end{ex}

\begin{ex}
In each case, verify that the points $P$ and $Q$ lie on the line.

\begin{enumerate}[label={\alph*.}]
\item 
$\begin{array}[t]{ll}
x = 3 - 4t & P(-1,3,0), Q(11,0,3) \\
y = 2 + t & \\
z = 1 - t &
\end{array}
$

\item
$\begin{array}[t]{ll}
x = 4 - t & P(2,3,-3), Q(-1,3,-9) \\
y = 3 & \\
z = 1 - 2t &
\end{array}
$

\end{enumerate}
\begin{sol}
\begin{enumerate}[label={\alph*.}]
\setcounter{enumi}{1}
\item  $P$ corresponds to $t = 2$; $Q$ corresponds to $t = 5$.

\end{enumerate}
\end{sol}
\end{ex}

\begin{ex}
Find the point of intersection (if any) of the following pairs of lines.

\begin{enumerate}[label={\alph*.}]
\item 
$\begin{array}[t]{ll}
	x = 3 + t & x = 4 + 2s \\
	y = 1 - 2t & y = 6 + 3s \\
	z = 3 + 3t & z = 1 + s 
\end{array}$

\item
$\begin{array}{ll}
	x = 1 - t & x = 2s \\
	y = 2 + 2t & y = 1 + s \\
	z = -1 + 3t & z = 3 
\end{array}$

\item 
$\leftB
\begin{array}{c}
x\\
y\\
z
\end{array}
\rightB
=
\leftB
\begin{array}{r}
3\\
-1\\
2
\end{array}
\rightB
+ t
\leftB
\begin{array}{r}
1\\
1\\
-1
\end{array}
\rightB$

$\leftB
\begin{array}{c}
x\\
y\\
z
\end{array}
\rightB
=
\leftB
\begin{array}{r}
1\\
1\\
-2
\end{array}
\rightB
+ s
\leftB
\begin{array}{r}
2\\
0\\
3
\end{array}
\rightB$

\item 
$\leftB
\begin{array}{c}
x\\
y\\
z
\end{array}
\rightB
=
\leftB
\begin{array}{r}
4\\
-1\\
5
\end{array}
\rightB
+ t
\leftB
\begin{array}{r}
1\\
0\\
1
\end{array}
\rightB$

$\leftB
\begin{array}{c}
x\\
y\\
z
\end{array}
\rightB
=
\leftB
\begin{array}{r}
2\\
-7\\
12
\end{array}
\rightB
+ s
\leftB
\begin{array}{r}
0\\
-2\\
3
\end{array}
\rightB$

\end{enumerate}
\begin{sol}
\begin{enumerate}[label={\alph*.}]
\setcounter{enumi}{1}
\item  No intersection

\setcounter{enumi}{3}
\item  $P(2, -1, 3)$; $t = -2$, $s = -3$

\end{enumerate}
\end{sol}
\end{ex}

\begin{ex}
Show
 that if a line passes through the origin, the vectors of points on the 
line are all scalar multiples of some fixed nonzero vector.
\end{ex}

\begin{ex}
Show that every line parallel to the $z$ axis has parametric equations $x = x_{0}$, $y = y_{0}$, $z = t$ for some fixed numbers $x_{0}$ and $y_{0}$.
\end{ex}

\begin{ex}
Let
$\vect{d} = \leftB
\begin{array}{c}
a\\
b\\
c
\end{array}
\rightB$
 be a vector where $a$, $b$, and $c$ are \textit{all} nonzero. Show that the equations of the line through $P_{0}(x_{0}, y_{0}, z_{0})$ with direction vector $\vect{d}$ can be written in the form
\begin{equation*}
\frac{x - x_{0}}{a} = \frac{y - y_{0}}{b} = \frac{z -z_{0}}{c}
\end{equation*}
This is called the \textbf{symmetric form}\index{symmetric form}\index{vector geometry!symmetric form} of the equations.
\end{ex}

\begin{ex}
A parallelogram has sides $AB$, $BC$, $CD$, and $DA$. Given $A(1, -1, 2)$, $C(2, 1, 0)$, and the midpoint $M(1, 0, -3)$ of $AB$, find $\longvect{BD}$.
\end{ex}

\begin{ex}
Find all points $C$ on the line through $A(1, -1, 2)$ and $B = (2, 0, 1)$ such that $\vectlength \longvect{AC} \vectlength = 2 \vectlength \longvect{BC} \vectlength$.

\begin{sol}
$P(3, 1, 0)$ or $P(\frac{5}{3}, \frac{-1}{3}, \frac{4}{3})$
\end{sol}
\end{ex}

\begin{ex}
Let $A$, $B$, $C$, $D$, $E$, and $F$ be the vertices of a regular hexagon, taken in order. Show that $\longvect{AB} + \longvect{AC} + \longvect{AD} + \longvect{AE} + \longvect{AF} = 3\longvect{AD}$.
\end{ex}

\begin{ex}
\begin{enumerate}[label={\alph*.}]
\item Let $P_{1}$, $P_{2}$, $P_{3}$, $P_{4}$, $P_{5}$, and $P_{6}$ be six points equally spaced on a circle with centre $C$. Show that
\begin{equation*}
\longvect{CP}_{1} + \longvect{CP}_{2} + \longvect{CP}_{3} + \longvect{CP}_{4} + \longvect{CP}_{5} + \longvect{CP}_{6} = \vect{0}
\end{equation*}
\item Show that the conclusion in part (a) holds for any \textit{even} set of points evenly spaced on the circle.

\item Show that the conclusion in part (a) holds for \textit{three} points.

\item Do you think it works for \textit{any} finite set of points evenly spaced around the circle?

\end{enumerate}
\begin{sol}
\begin{enumerate}[label={\alph*.}]
\setcounter{enumi}{1}
\item  $\longvect{CP}_{k} = -\longvect{CP}_{n+k}$
 if $1 \leq k \leq n$, where there are $2n$ points.

\end{enumerate}
\end{sol}
\end{ex}


\begin{ex}
Consider a quadrilateral with vertices $A$, $B$, $C$, and $D$ in order (as shown in the diagram).


\begin{figure}[H]
\centering
\begin{tikzpicture}
[scale=0.7]
\coordinate (ptA) at (1, 2);
\coordinate (ptB) at (3, 2.25);
\coordinate (ptC) at (4, 0);
\coordinate (ptD) at (0, 0);

\draw[dkgreenvect,thick] (ptA)--(ptB)--(ptC)--(ptD)--cycle;
\node[above left] at (ptA) {\small $A$};
\node[above right] at (ptB) {\small $B$};
\node[right] at (ptC) {\small $C$};
\node[left] at (ptD) {\small $D$};
\end{tikzpicture}

%\caption{\label{fig:011729}}
\end{figure}

If the diagonals $AC$ and $BD$ bisect each other, show that the quadrilateral is a parallelogram. (This is the converse of Example~\ref{exa:011062}.) [\textit{Hint}: Let $E$ be the intersection of the diagonals. Show that $\longvect{AB} = \longvect{DC}$ by writing $\longvect{AB} = \longvect{AE} + \longvect{EB}$.]
\end{ex}

\begin{ex}
Consider the parallelogram $ABCD$ (see diagram), and let $E$ be the midpoint of side $AD$.


\begin{figure}[H]
\centering
\begin{tikzpicture}[scale=0.9]
\draw[dkgreenvect,thick,name path=lineAD] (0,0)--(3,1);
\draw[dkgreenvect,thick,name path=lineBE] (0,2)--(1.5,0.49);
\draw[dkgreenvect,thick,name path=lineAC] (0,0)--(3,3);
\draw[dkgreenvect,thick](0,0)--(0,2)--(3,3)--(3,1)--cycle;
\fill[dkgreenvect,thick,black, name intersections={of=lineAD and lineBE}] (intersection-1) circle (2pt);
\fill[dkgreenvect,thick,black, name intersections={of=lineBE and lineAC}] (intersection-1) circle (2pt);
\node[below] at (0,0){$A$};
\node[left] at (0,2){$B$};
\node[right] at (3,3){$C$};
\node[right] at (3,1){$D$};
\node[below] at (1.5,0.5){$E$};
\node[right] at (1,1){$F$};
\end{tikzpicture}

%\caption{\label{fig:011735}}
\end{figure}

Show that $BE$ and $AC$ trisect each other; that is, show that the intersection point is one-third of the way from $E$ to $B$ and from $A$ to $C$. [\textit{Hint}: If $F$ is one-third of the way from $A$ to $C$, show that $2\longvect{EF} = \longvect{FB}$ and argue as in Example~\ref{exa:011062}.]

\begin{sol} 
$\longvect{DA} = 2\longvect{EA}$ and $2\longvect{AF} = \longvect{FC}$, so $2\longvect{EF} = 2(\longvect{EF} + \longvect{AF}) = \longvect{DA} + \longvect{FC} = \longvect{CB} + \longvect{FC} = \longvect{FC} + \longvect{CB} = \longvect{FB}$. Hence $\longvect{EF} = \frac{1}{2}\longvect{FB}$. So $F$ is the trisection point of both $AC$ and $EB$.
\end{sol}
\end{ex}

\begin{ex}
The line from a vertex of a triangle to the midpoint of the opposite side is called a \textbf{median}\index{triangle!median} of the triangle. If the vertices of a triangle have vectors $\vect{u}$, $\vect{v}$, and $\vect{w}$, show that the point on each median that is $\frac{1}{3}$ the way from the midpoint to the vertex has vector $\frac{1}{3}(\vect{u} + \vect{v} + \vect{w})$. Conclude that the point $C$ with vector $\frac{1}{3}(\vect{u} + \vect{v} + \vect{w})$ lies on all three medians. This point $C$ is called the \textbf{centroid}\index{triangle!centroid} of the triangle.\index{centroid}
\end{ex}

\begin{ex}
Given four noncoplanar points in space, the figure with these points as vertices is called a \textbf{tetrahedron}\index{tetrahedron}.
 The line from a vertex through the centroid (see previous exercise) of 
the triangle formed by the remaining vertices is called a \textbf{median}\index{triangle!median} of the tetrahedron. If $\vect{u}$, $\vect{v}$, $\vect{w}$, and $\vect{x}$ are the vectors of the four vertices, show that the point on a median one-fourth the way from the centroid to the vertex has vector $\frac{1}{4}(\vect{u} + \vect{v} + \vect{w} + \vect{x})$. Conclude that the four medians are concurrent.\index{median!tetrahedron}\index{median!triangle}
\end{ex}
\end{multicols}


