\section*{Exercises for \ref{sec:4_2}}

\begin{Filesave}{solutions}
\solsection{Section~\ref{sec:4_2}}
\end{Filesave}

\begin{multicols}{2}
\begin{ex}
Compute $\vect{u} \dotprod \vect{v}$ where:

\begin{enumerate}[label={\alph*.}]
\item $\vect{u} = \leftB
\begin{array}{r}
2\\
-1\\
3
\end{array}
\rightB$, 
$\vect{v} = \leftB
\begin{array}{r}
-1\\
1\\
1
\end{array}
\rightB
$

\item $\vect{u} = \leftB
\begin{array}{r}
1\\
2\\
-1
\end{array}
\rightB$,
$\vect{v} = \vect{u}
$


\item $\vect{u} = \leftB
\begin{array}{r}
1\\
1\\
-3
\end{array}
\rightB$, 
$\vect{v} = \leftB
\begin{array}{r}
2\\
-1\\
1
\end{array}
\rightB
$

\item $\vect{u} = \leftB
\begin{array}{r}
3\\
-1\\
5
\end{array}
\rightB$, 
$\vect{v} = \leftB
\begin{array}{r}
6\\
-7\\
-5
\end{array}
\rightB
$

\item$\vect{u} = \leftB
\begin{array}{r}
x\\
y\\
z
\end{array}
\rightB$, 
$\vect{v} = \leftB
\begin{array}{r}
a\\
b\\
c
\end{array}
\rightB
$

\item $\vect{u} = \leftB
\begin{array}{r}
a\\
b\\
c
\end{array}
\rightB$, 
$\vect{v} = \vect{0} $

\end{enumerate}
\begin{sol}
\begin{enumerate}[label={\alph*.}]
\setcounter{enumi}{1}
\item  $6$

\setcounter{enumi}{3}
\item  $0$

\setcounter{enumi}{5}
\item  $0$

\end{enumerate}
\end{sol}
\end{ex}

\begin{ex}
Find the angle between the following pairs of vectors.


\begin{enumerate}[label={\alph*.}]
\item $\vect{u} = \leftB
\begin{array}{r}
1\\
0\\
3
\end{array}
\rightB$, 
$\vect{v} = \leftB
\begin{array}{r}
2\\
0\\
1
\end{array}
\rightB
$

\item $\vect{u} = \leftB
\begin{array}{r}
3\\
-1\\
0
\end{array}
\rightB$, 
$\vect{v} = \leftB
\begin{array}{r}
-6\\
2\\
0
\end{array}
\rightB
$

\item $\vect{u} = \leftB
\begin{array}{r}
7\\
-1\\
3
\end{array}
\rightB$, 
$\vect{v} = \leftB
\begin{array}{r}
1\\
4\\
-1
\end{array}
\rightB
$

\item $\vect{u} = \leftB
\begin{array}{r}
2\\
1\\
-1
\end{array}
\rightB$, 
$\vect{v} = \leftB
\begin{array}{r}
3\\
6\\
3
\end{array}
\rightB
$

\item $\vect{u} = \leftB
\begin{array}{r}
1\\
-1\\
0
\end{array}
\rightB$, 
$\vect{v} = \leftB
\begin{array}{r}
0\\
1\\
1
\end{array}
\rightB
$

\item $\vect{u} = \leftB
\begin{array}{r}
0\\
3\\
4
\end{array}
\rightB$, 
$\vect{v} = \leftB
\begin{array}{r}
5\sqrt{2}\\
-7\\
-1
\end{array}
\rightB
$


\end{enumerate}
\begin{sol}
\begin{enumerate}[label={\alph*.}]
\setcounter{enumi}{1}
\item  $\pi$ or $180^\circ$ 

\setcounter{enumi}{3}
\item  $\frac{\pi}{3}$ or $60^\circ$

\setcounter{enumi}{5}
\item  $\frac{2\pi}{3}$ or $120^\circ$


\end{enumerate}
\end{sol}
\end{ex}

\begin{ex}
Find all real numbers $x$ such that:


\begin{enumerate}[label={\alph*.}]
\item $\leftB
\begin{array}{r}
2\\
-1\\
3
\end{array}
\rightB$
and 
$\leftB
\begin{array}{r}
x\\
-2\\
1
\end{array}
\rightB$
are orthogonal.

\item $\leftB
\begin{array}{r}
2\\
-1\\
1
\end{array}
\rightB$
and 
$\leftB
\begin{array}{r}
1\\
x\\
2
\end{array}
\rightB$
are at an angle of $\frac{\pi}{3}$.

\end{enumerate}
\begin{sol}
\begin{enumerate}[label={\alph*.}]
\setcounter{enumi}{1}
\item  $1$ or $-17$

\end{enumerate}
\end{sol}
\end{ex}

\begin{ex}
Find all vectors $\vect{v} = \leftB
\begin{array}{r}
x\\
y\\
z
\end{array}
\rightB$ orthogonal to both:


\begin{enumerate}[label={\alph*.}]
\item $\vect{u}_{1} = \leftB
\begin{array}{r}
-1\\
-3\\
2
\end{array}
\rightB$,
$\vect{u}_{2} = \leftB
\begin{array}{r}
0\\
1\\
1
\end{array}
\rightB
$

\item $\vect{u}_{1} = \leftB
\begin{array}{r}
3\\
-1\\
2
\end{array}
\rightB$, 
$\vect{u}_{2} = \leftB
\begin{array}{r}
2\\
0\\
1
\end{array}
\rightB
$

\item $\vect{u}_{1} = \leftB
\begin{array}{r}
2\\
0\\
-1
\end{array}
\rightB$, 
$\vect{u}_{2} = \leftB
\begin{array}{r}
-4\\
0\\
2
\end{array}
\rightB
$

\item $\vect{u}_{1} = \leftB
\begin{array}{r}
2\\
-1\\
3
\end{array}
\rightB$, 
$\vect{u}_{2} = \leftB
\begin{array}{r}
0\\
0\\
0
\end{array}
\rightB
$


\end{enumerate}
\begin{sol}
\begin{enumerate}[label={\alph*.}]
\setcounter{enumi}{1}
\item  $t \leftB
\begin{array}{r}
-1\\
1\\
2
\end{array}
\rightB
$

\setcounter{enumi}{3}
\item $s \leftB
\begin{array}{r}
1\\
2\\
0
\end{array}
\rightB
+
t \leftB
\begin{array}{r}
0\\
3\\
1
\end{array}
\rightB
$


\end{enumerate}
\end{sol}
\end{ex}

\begin{ex}
Find two orthogonal vectors that are both orthogonal to $\vect{v} = \leftB
\begin{array}{r}
1\\
2\\
0
\end{array}
\rightB$.
\end{ex}

\begin{ex}
Consider the triangle with vertices $P(2, 0, -3)$, $Q(5, -2, 1)$, and $R(7, 5, 3)$.


\begin{enumerate}[label={\alph*.}]
\item Show that it is a right-angled triangle.

\item Find the lengths of the three sides and verify the Pythagorean theorem.

\end{enumerate}
\begin{sol}
\begin{enumerate}[label={\alph*.}]
\setcounter{enumi}{1}
\item  $29 + 57 = 86$

\end{enumerate}
\end{sol}
\end{ex}

\begin{ex}
Show that the triangle with vertices $A(4, -7, 9)$, $B(6, 4, 4)$, and $C(7, 10, -6)$ is not a right-angled triangle.
\end{ex}

\begin{ex}
Find the three internal angles of the triangle with vertices:


\begin{enumerate}[label={\alph*.}]
\item $A(3, 1, -2)$, $B(3, 0, -1)$, and $C(5, 2, -1)$

\item $A(3, 1, -2)$, $B(5, 2, -1)$, and $C(4, 3, -3)$

\end{enumerate}
\begin{sol}
\begin{enumerate}[label={\alph*.}]
\setcounter{enumi}{1}
\item $A = B = C = \frac{\pi}{3}$ or $60^\circ$ 

\end{enumerate}
\end{sol}
\end{ex}

\begin{ex}
Show that the line through $P_{0}(3, 1, 4)$ and $P_{1}(2, 1, 3)$ is perpendicular to the line through $P_{2}(1, -1, 2)$ and $P_{3}(0, 5, 3)$.
\end{ex}

\begin{ex}
In each case, compute the projection of $\vect{u}$ on $\vect{v}$.


\begin{enumerate}[label={\alph*.}]
\item $\vect{u} = \leftB
\begin{array}{r}
5\\
7\\
1
\end{array}
\rightB$, 
$\vect{v} = \leftB
\begin{array}{r}
2\\
-1\\
3
\end{array}
\rightB
$

\item $\vect{u} = \leftB
\begin{array}{r}
3\\
-2\\
1
\end{array}
\rightB$, 
$\vect{v} = \leftB
\begin{array}{r}
4\\
1\\
1
\end{array}
\rightB
$

\item $\vect{u} = \leftB
\begin{array}{r}
1\\
-1\\
2
\end{array}
\rightB$, 
$\vect{v} = \leftB
\begin{array}{r}
3\\
-1\\
1
\end{array}
\rightB
$

\item $\vect{u} = \leftB
\begin{array}{r}
3\\
-2\\
-1
\end{array}
\rightB$, 
$\vect{v} = \leftB
\begin{array}{r}
-6\\
4\\
2
\end{array}
\rightB
$

\end{enumerate}
\begin{sol}
\begin{enumerate}[label={\alph*.}]
\setcounter{enumi}{1}
\item  $\frac{11}{18}\vect{v}$

\setcounter{enumi}{3}
\item  $-\frac{1}{2}\vect{v}$

\end{enumerate}
\end{sol}
\end{ex}

\begin{ex}
In each case, write $\vect{u} = \vect{u}_{1} + \vect{u}_{2}$, where $\vect{u}_{1}$ is parallel to $\vect{v}$ and $\vect{u}_{2}$ is orthogonal to $\vect{v}$.

\begin{enumerate}[label={\alph*.}]
\item $\vect{u} = \leftB
\begin{array}{r}
2\\
-1\\
1
\end{array}
\rightB$, 
$\vect{v} = \leftB
\begin{array}{r}
1\\
-1\\
3
\end{array}
\rightB
$

\item $\vect{u} = \leftB
\begin{array}{r}
3\\
1\\
0
\end{array}
\rightB$, 
$\vect{v} = \leftB
\begin{array}{r}
-2\\
1\\
4
\end{array}
\rightB
$

\item $\vect{u} = \leftB
\begin{array}{r}
2\\
-1\\
0
\end{array}
\rightB$, 
$\vect{v} = \leftB
\begin{array}{r}
3\\
1\\
-1
\end{array}
\rightB
$

\item $\vect{u} = \leftB
\begin{array}{r}
3\\
-2\\
1
\end{array}
\rightB$, 
$\vect{v} = \leftB
\begin{array}{r}
-6\\
4\\
-1
\end{array}
\rightB
$

\end{enumerate}
\begin{sol}
\begin{enumerate}[label={\alph*.}]
\setcounter{enumi}{1}
\item  $\frac{5}{21}\leftB
\begin{array}{r}
2\\
-1\\
-4
\end{array}
\rightB
+
\frac{1}{21}\leftB
\begin{array}{r}
53\\
26\\
20
\end{array}
\rightB$

\setcounter{enumi}{3}
\item  
$\frac{27}{53}\leftB
\begin{array}{r}
6\\
-4\\
1
\end{array}
\rightB
+
\frac{1}{53}\leftB
\begin{array}{r}
-3\\
2\\
26
\end{array}
\rightB$


\end{enumerate}
\end{sol}
\end{ex}

\begin{ex}
Calculate the distance from the point $P$ to the line in each case and find the point $Q$ on the line closest to $P$.


\begin{enumerate}[label={\alph*.}]
\item $P(3,2-1) \quad $ \\ line:  $\leftB
\begin{array}{r}
x\\
y\\
z
\end{array}
\rightB
=
\leftB
\begin{array}{r}
2\\
1\\
3
\end{array}
\rightB
+t
\leftB
\begin{array}{r}
3\\
-1\\
-2
\end{array}
\rightB
$

\item $P(1,-1,3) \quad $ \\ line: $ \leftB
\begin{array}{r}
x\\
y\\
z
\end{array}
\rightB
=
\leftB
\begin{array}{r}
1\\
0\\
-1
\end{array}
\rightB
+t
\leftB
\begin{array}{r}
3\\
1\\
4
\end{array}
\rightB
$

\end{enumerate}
\begin{sol}
\begin{enumerate}[label={\alph*.}]
\setcounter{enumi}{1}
\item  $\frac{1}{26}\sqrt{5642}$, $Q(\frac{71}{26}, \frac{15}{26}, \frac{34}{26})$


\end{enumerate}
\end{sol}
\end{ex}

\begin{ex}
Compute $\vect{u} \times \vect{v}$ where:


\begin{enumerate}[label={\alph*.}]
\item $\vect{u} = \leftB
\begin{array}{r}
1\\
2\\
3
\end{array}
\rightB$, 
$\vect{v} = \leftB
\begin{array}{r}
1\\
1\\
2
\end{array}
\rightB
$

\item $\vect{u} = \leftB
\begin{array}{r}
3\\
-1\\
0
\end{array}
\rightB$, 
$\vect{v} = \leftB
\begin{array}{r}
-6\\
2\\
0
\end{array}
\rightB
$

\item $\vect{u} = \leftB
\begin{array}{r}
3\\
-2\\
1
\end{array}
\rightB$, 
$\vect{v} = \leftB
\begin{array}{r}
1\\
1\\
-1
\end{array}
\rightB
$

\item $\vect{u} = \leftB
\begin{array}{r}
2\\
0\\
-1
\end{array}
\rightB$, 
$\vect{v} = \leftB
\begin{array}{r}
1\\
4\\
7
\end{array}
\rightB
$

\end{enumerate}
\begin{sol}
\begin{enumerate}[label={\alph*.}]
\setcounter{enumi}{1}
\item  
$\leftB
\begin{array}{r}
0\\
0\\
0
\end{array}
\rightB
$


\setcounter{enumi}{1}
\item 
$\leftB
\begin{array}{r}
4\\
-15\\
8
\end{array}
\rightB
$

\end{enumerate}
\end{sol}
\end{ex}

\begin{ex}
Find an equation of each of the following planes.


\begin{enumerate}[label={\alph*.}]
\item Passing through $A(2, 1, 3)$, $B(3, -1, 5)$, and $C(1, 2, -3)$.

\item Passing through $A(1, -1, 6)$, $B(0, 0, 1)$, and $C(4, 7, -11)$.

\item Passing through $P(2, -3, 5)$ and parallel to the plane with equation $3x - 2y - z = 0$.

\item Passing through $P(3, 0, -1)$ and parallel to the plane with equation $2x - y + z = 3$.

\item Containing $P(3, 0, -1)$ and the line
\newline $\leftB
\begin{array}{r}
x\\
y\\
z
\end{array}
\rightB
=
\leftB
\begin{array}{r}
0\\
0\\
2
\end{array}
\rightB
+t
\leftB
\begin{array}{r}
1\\
0\\
1
\end{array}
\rightB.
$
\item Containing $P(2, 1, 0)$ and the line

$
\leftB
\begin{array}{r}
x\\
y\\
z
\end{array}
\rightB
=
\leftB
\begin{array}{r}
3\\
-1\\
2
\end{array}
\rightB
+t
\leftB
\begin{array}{r}
1\\
0\\
-1
\end{array}
\rightB.$
\item Containing the lines

$\leftB
\begin{array}{r}
x\\
y\\
z
\end{array}
\rightB
=
\leftB
\begin{array}{r}
1\\
-1\\
2
\end{array}
\rightB
+t
\leftB
\begin{array}{r}
1\\
1\\
1
\end{array}
\rightB 
$ and
\newline $\leftB
\begin{array}{r}
x\\
y\\
z
\end{array}
\rightB
=
\leftB
\begin{array}{r}
0\\
0\\
2
\end{array}
\rightB
+t
\leftB
\begin{array}{r}
1\\
-1\\
0
\end{array}
\rightB$.

\item Containing the lines $\leftB
\begin{array}{r}
x\\
y\\
z
\end{array}
\rightB
=
\leftB
\begin{array}{r}
3\\
1\\
0
\end{array}
\rightB
+t
\leftB
\begin{array}{r}
1\\
-1\\
3
\end{array}
\rightB 
$ and
$\leftB
\begin{array}{r}
x\\
y\\
z
\end{array}
\rightB
=
\leftB
\begin{array}{r}
0\\
-2\\
5
\end{array}
\rightB
+t
\leftB
\begin{array}{r}
2\\
1\\
-1
\end{array}
\rightB$.

\item Each point of which is equidistant from $P(2, -1, 3)$ and $Q(1, 1, -1)$.

\item Each point of which is equidistant from $P(0, 1, -1)$ and $Q(2, -1, -3)$.

\end{enumerate}
\begin{sol}
\begin{enumerate}[label={\alph*.}]
\setcounter{enumi}{1}
\item  $-23x + 32y + 11z = 11$

\setcounter{enumi}{3}
\item  $2x - y + z = 5$

\setcounter{enumi}{5}
\item  $2x + 3y + 2z = 7$

\setcounter{enumi}{7}
\item  $2x - 7y - 3z = -1$

\setcounter{enumi}{9}
\item  $x - y - z = 3$

\end{enumerate}
\end{sol}
\end{ex}

\begin{ex}
In each case, find a vector equation of the line.


\begin{enumerate}[label={\alph*.}]
\item Passing through $P(3, -1, 4)$ and perpendicular to the plane $3x - 2y - z = 0$.

\item Passing through $P(2, -1, 3)$ and perpendicular to the plane $2x + y = 1$.

\item Passing through $P(0, 0, 0)$ and perpendicular to the lines
$\leftB
\begin{array}{r}
x\\
y\\
z
\end{array}
\rightB
=
\leftB
\begin{array}{r}
1\\
1\\
0
\end{array}
\rightB
+t
\leftB
\begin{array}{r}
2\\
0\\
-1
\end{array}
\rightB 
$ and
$\leftB
\begin{array}{r}
x\\
y\\
z
\end{array}
\rightB
=
\leftB
\begin{array}{r}
2\\
1\\
-3
\end{array}
\rightB
+t
\leftB
\begin{array}{r}
1\\
-1\\
5
\end{array}
\rightB$.
\item Passing through $P(1, 1, -1)$, and perpendicular to the lines

$\leftB
\begin{array}{r}
x\\
y\\
z
\end{array}
\rightB
=
\leftB
\begin{array}{r}
2\\
0\\
1
\end{array}
\rightB
+t
\leftB
\begin{array}{r}
1\\
1\\
-2
\end{array}
\rightB 
$ and
 \\$\leftB
\begin{array}{r}
x\\
y\\
z
\end{array}
\rightB
=
\leftB
\begin{array}{r}
5\\
5\\
-2
\end{array}
\rightB
+t
\leftB
\begin{array}{r}
1\\
2\\
-3
\end{array}
\rightB$.

\item Passing through $P(2, 1, -1)$, intersecting the line 
$\leftB
\begin{array}{r}
x\\
y\\
z
\end{array}
\rightB
=
\leftB
\begin{array}{r}
1\\
2\\
-1
\end{array}
\rightB
+t
\leftB
\begin{array}{r}
3\\
0\\
1
\end{array}
\rightB$, and perpendicular to that line.

\item Passing through $P(1, 1, 2)$, intersecting the line 
$\leftB
\begin{array}{r}
x\\
y\\
z
\end{array}
\rightB
=
\leftB
\begin{array}{r}
2\\
1\\
0
\end{array}
\rightB
+t
\leftB
\begin{array}{r}
1\\
1\\
1
\end{array}
\rightB$, and perpendicular to that line.

\end{enumerate}
\begin{sol}
\begin{enumerate}[label={\alph*.}]
\setcounter{enumi}{1}
\item  
$\leftB
\begin{array}{r}
x\\
y\\
z
\end{array}
\rightB
=
\leftB
\begin{array}{r}
2\\
-1\\
3
\end{array}
\rightB
+t
\leftB
\begin{array}{r}
2\\
1\\
0
\end{array}
\rightB
$

\setcounter{enumi}{3}
\item 
$\leftB
\begin{array}{r}
x\\
y\\
z
\end{array}
\rightB
=
\leftB
\begin{array}{r}
1\\
1\\
-1
\end{array}
\rightB
+t
\leftB
\begin{array}{r}
1\\
1\\
1
\end{array}
\rightB
$

\setcounter{enumi}{5}
\item  
$\leftB
\begin{array}{r}
x\\
y\\
z
\end{array}
\rightB
=
\leftB
\begin{array}{r}
1\\
1\\
2
\end{array}
\rightB
+t
\leftB
\begin{array}{r}
4\\
1\\
-5
\end{array}
\rightB
$

\end{enumerate}
\end{sol}
\end{ex}

\begin{ex}
In each case, find the shortest distance from the point $P$ to the plane and find the point $Q$ on the plane closest to $P$.


\begin{enumerate}[label={\alph*.}]
\item $P(2, 3, 0)$; plane with equation $5x + y + z = 1$.

\item $P(3, 1, -1)$; plane with equation $2x + y - z = 6$.

\end{enumerate}
\begin{sol}
\begin{enumerate}[label={\alph*.}]
\setcounter{enumi}{1}
\item  $\frac{\sqrt{6}}{3}$, $Q(\frac{7}{3}, \frac{2}{3}, \frac{-2}{3})$

\end{enumerate}
\end{sol}
\end{ex}

\begin{ex}
\begin{enumerate}[label={\alph*.}]
\item Does the line through $P(1, 2, -3)$ with direction vector $\vect{d} = \leftB
\begin{array}{r}
1\\
2\\
-3
\end{array}
\rightB$ lie in the plane $2x - y - z = 3$? Explain.

\item Does the plane through $P(4, 0, 5)$, $Q(2, 2, 1)$, and $R(1, -1, 2)$ pass through the origin? Explain.

\end{enumerate}
\begin{sol}
\begin{enumerate}[label={\alph*.}]
\setcounter{enumi}{1}
\item  Yes. The equation is $5x -3y - 4z = 0$.

\end{enumerate}
\end{sol}
\end{ex}

\begin{ex}
Show that every plane containing $P(1, 2, -1)$ and $Q(2, 0, 1)$ must also contain $R(-1, 6, -5)$.
\end{ex}

\begin{ex}
Find the equations of the line of intersection of the following planes.


\begin{enumerate}[label={\alph*.}]
\item $2x -3y + 2z = 5$ and $x + 2y - z = 4$.

\item $3x + y -2z = 1$ and $x + y + z = 5$.

\end{enumerate}
\begin{sol}
\begin{enumerate}[label={\alph*.}]
\setcounter{enumi}{1}
\item  $(-2, 7, 0) + t(3, -5, 2)$

\end{enumerate}
\end{sol}
\end{ex}

\begin{ex}
In each case, find all points of intersection of the given plane and the line \newline$\leftB
\begin{array}{r}
x\\
y\\
z
\end{array}
\rightB
=
\leftB
\begin{array}{r}
1\\
-2\\
3
\end{array}
\rightB
+t
\leftB
\begin{array}{r}
2\\
5\\
-1
\end{array}
\rightB$.

\begin{exenumerate}
\exitem $x -3y + 2z = 4$
\exitem $2x - y - z = 5$
\exitem $3x - y + z = 8$
\exitem $-x -4y -3z = 6$
\end{exenumerate}
\begin{sol}
\begin{enumerate}[label={\alph*.}]
\setcounter{enumi}{1}
\item  None

\setcounter{enumi}{3}
\item  $P(\frac{13}{19}, \frac{-78}{19}, \frac{65}{19})$


\end{enumerate}
\end{sol}
\end{ex}

\begin{ex}
Find the equation of \textit{all} planes:


\begin{enumerate}[label={\alph*.}]
\item Perpendicular to the line  \\
$\leftB
\begin{array}{r}
x\\
y\\
z
\end{array}
\rightB
=
\leftB
\begin{array}{r}
2\\
-1\\
3
\end{array}
\rightB
+t
\leftB
\begin{array}{r}
2\\
1\\
3
\end{array}
\rightB$.

\item Perpendicular to the line  \\
$\leftB
\begin{array}{r}
x\\
y\\
z
\end{array}
\rightB
=
\leftB
\begin{array}{r}
1\\
0\\
-1
\end{array}
\rightB
+t
\leftB
\begin{array}{r}
3\\
0\\
2
\end{array}
\rightB$.

\item Containing the origin.

\item Containing $P(3, 2, -4)$.

\item Containing $P(1, 1, -1)$ and $Q(0, 1, 1)$.

\item Containing $P(2, -1, 1)$ and $Q(1, 0, 0)$.

\item Containing the line \\
$\leftB
\begin{array}{r}
x\\
y\\
z
\end{array}
\rightB
=
\leftB
\begin{array}{r}
2\\
1\\
0
\end{array}
\rightB
+t
\leftB
\begin{array}{r}
1\\
-1\\
0
\end{array}
\rightB$.

\item Containing the line \\
$\leftB
\begin{array}{r}
x\\
y\\
z
\end{array}
\rightB
=
\leftB
\begin{array}{r}
3\\
0\\
2
\end{array}
\rightB
+t
\leftB
\begin{array}{r}
1\\
-2\\
-1
\end{array}
\rightB$.

\end{enumerate}
\begin{sol}
\begin{enumerate}[label={\alph*.}]
\setcounter{enumi}{1}
\item  $3x + 2z = d$, $d$ arbitrary

\setcounter{enumi}{3}
\item  $a(x - 3) + b(y - 2) + c(z + 4) = 0$; $a$, $b$, and $c$ not all zero

\setcounter{enumi}{5}
\item  $ax + by + (b - a)z = a$; $a$ and $b$ not both zero

\setcounter{enumi}{7}
\item $ax + by + (a - 2b)z = 5a - 4b$; $a$ and $b$ not both zero

\end{enumerate}
\end{sol}
\end{ex}

\begin{ex}
If a plane contains two distinct points $P_{1}$ and $P_{2}$, show that it contains every point on the line through $P_{1}$ and $P_{2}$.
\end{ex}

\begin{ex}
Find the shortest distance between the following pairs of parallel lines.

\begin{enumerate}[label={\alph*.}]
\item $\leftB
\begin{array}{r}
x\\
y\\
z
\end{array}
\rightB
=
\leftB
\begin{array}{r}
2\\
-1\\
3
\end{array}
\rightB
+t
\leftB
\begin{array}{r}
1\\
-1\\
4
\end{array}
\rightB; $ \\ $
\leftB
\begin{array}{r}
x\\
y\\
z
\end{array}
\rightB
=
\leftB
\begin{array}{r}
1\\
0\\
1
\end{array}
\rightB
+t
\leftB
\begin{array}{r}
1\\
-1\\
4
\end{array}
\rightB$

\item 
$\leftB
\begin{array}{r}
x\\
y\\
z
\end{array}
\rightB
=
\leftB
\begin{array}{r}
3\\
0\\
2
\end{array}
\rightB
+t
\leftB
\begin{array}{r}
3\\
1\\
0
\end{array}
\rightB; $ \\ $
\leftB
\begin{array}{r}
x\\
y\\
z
\end{array}
\rightB
=
\leftB
\begin{array}{r}
-1\\
2\\
2
\end{array}
\rightB
+t
\leftB
\begin{array}{r}
3\\
1\\
0
\end{array}
\rightB$

\end{enumerate}
\begin{sol}
\begin{enumerate}[label={\alph*.}]
\setcounter{enumi}{1}
\item  $\sqrt{10}$

\end{enumerate}
\end{sol}
\end{ex}

\begin{ex}
Find the shortest distance between the following pairs of nonparallel lines and find the points on the lines that are closest together.

\begin{enumerate}[label={\alph*.}]
\item 
$\leftB
\begin{array}{r}
x\\
y\\
z
\end{array}
\rightB
=
\leftB
\begin{array}{r}
3\\
0\\
1
\end{array}
\rightB
+s
\leftB
\begin{array}{r}
2\\
1\\
-3
\end{array}
\rightB; $ \\ $
\leftB
\begin{array}{r}
x\\
y\\
z
\end{array}
\rightB
=
\leftB
\begin{array}{r}
1\\
1\\
-1
\end{array}
\rightB
+t
\leftB
\begin{array}{r}
1\\
0\\
1
\end{array}
\rightB$

\item 
$\leftB
\begin{array}{r}
x\\
y\\
z
\end{array}
\rightB
=
\leftB
\begin{array}{r}
1\\
-1\\
0
\end{array}
\rightB
+s
\leftB
\begin{array}{r}
1\\
1\\
1
\end{array}
\rightB; $ \\ $
\leftB
\begin{array}{r}
x\\
y\\
z
\end{array}
\rightB
=
\leftB
\begin{array}{r}
2\\
-1\\
3
\end{array}
\rightB
+t
\leftB
\begin{array}{r}
3\\
1\\
0
\end{array}
\rightB$

\item 
$\leftB
\begin{array}{r}
x\\
y\\
z
\end{array}
\rightB
=
\leftB
\begin{array}{r}
3\\
1\\
-1
\end{array}
\rightB
+s
\leftB
\begin{array}{r}
1\\
1\\
-1
\end{array}
\rightB; $ \\ $
\leftB
\begin{array}{r}
x\\
y\\
z
\end{array}
\rightB
=
\leftB
\begin{array}{r}
1\\
2\\
0
\end{array}
\rightB
+t
\leftB
\begin{array}{r}
1\\
0\\
2
\end{array}
\rightB$

\item 
$\leftB
\begin{array}{r}
x\\
y\\
z
\end{array}
\rightB
=
\leftB
\begin{array}{r}
1\\
2\\
3
\end{array}
\rightB
+s
\leftB
\begin{array}{r}
2\\
0\\
-1
\end{array}
\rightB; $ \\ $
\leftB
\begin{array}{r}
x\\
y\\
z
\end{array}
\rightB
=
\leftB
\begin{array}{r}
3\\
-1\\
0
\end{array}
\rightB
+t
\leftB
\begin{array}{r}
1\\
1\\
0
\end{array}
\rightB$

\end{enumerate}
\begin{sol}
\begin{enumerate}[label={\alph*.}]
\setcounter{enumi}{1}
\item  $\frac{\sqrt{14}}{2}$, $A(3, 1, 2)$, $B(\frac{7}{2}, -\frac{1}{2}, 3)$

\setcounter{enumi}{3}
\item  $\frac{\sqrt{6}}{6}$, $A(\frac{19}{3}, 2, \frac{1}{3})$, $B(\frac{37}{6}, \frac{13}{6}, 0)$

\end{enumerate}
\end{sol}
\end{ex}

\begin{ex}
Show that two lines in the plane with slopes $m_{1}$ and $m_{2}$ are perpendicular if and only if \\ $m_{1}m_{2} = -1$. [\textit{Hint}: Example~\ref{exa:011343}.]
\end{ex}

\begin{ex}
\begin{enumerate}[label={\alph*.}]
\item Show that, of the four diagonals of a cube, no pair is perpendicular.

\item Show that each diagonal is perpendicular to the face diagonals it does not meet.

\end{enumerate}
\begin{sol}
\begin{enumerate}[label={\alph*.}]
\setcounter{enumi}{1}
\item  Consider the diagonal $\vect{d} = \leftB
\begin{array}{r}
	a\\
	a\\
	a
\end{array}
\rightB$ 
The six face diagonals in question are $\pm\leftB
\begin{array}{r}
a\\
0\\
-a
\end{array}
\rightB$, 
$\pm\leftB
\begin{array}{r}
0\\
a\\
-a
\end{array}
\rightB$, $
\pm\leftB
\begin{array}{r}
a\\
-a\\
0
\end{array}
\rightB$. All of these are orthogonal to $\vect{d}$. The result works for the other diagonals by symmetry.

\end{enumerate}
\end{sol}
\end{ex}

\begin{ex}
Given a rectangular solid with sides of lengths $1$, $1$, and $\sqrt{2}$, find the angle between a diagonal and one of the longest sides.
\end{ex}

\begin{ex}
Consider a rectangular solid with sides of lengths $a$, $b$, and $c$. Show that it has two orthogonal diagonals if and only if the sum of two of $a^{2}$, $b^{2}$, and $c^{2}$ equals the third.

\begin{sol}
The four diagonals are $(a, b, c)$, $(-a, b, c)$, $(a, -b, c)$ and $(a, b, -c)$ or their negatives. The dot products are $\pm(-a^{2} + b^{2} + c^{2})$, $\pm(a^{2} - b^{2} + c^{2})$, and $\pm(a^{2} + b^{2} - c^{2})$.
\end{sol}
\end{ex}

\begin{ex}
Let $A$, $B$, and $C(2, -1, 1)$ be the vertices of a triangle where $\longvect{AB}$ is parallel to $\leftB
\begin{array}{r}
1\\
-1\\
1
\end{array}
\rightB$, $\longvect{AC}$ is parallel to $\leftB
\begin{array}{r}
2\\
0\\
-1
\end{array}
\rightB$, and angle $C = 90^\circ$ . Find the equation of the line through $B$ and $C$.
\end{ex}

\begin{ex}
If the diagonals of a parallelogram have equal length, show that the parallelogram is a rectangle.
\end{ex}

\begin{ex}
Given $\vect{v} = \leftB
\begin{array}{r}
x\\
y\\
z
\end{array}
\rightB$
 in component form, show that the projections of $\vect{v}$ on $\vect{i}$, $\vect{j}$, and $\vect{k}$ are $x\vect{i}$, $y\vect{j}$, and $z\vect{k}$, respectively.
\end{ex}

\begin{ex}
\begin{enumerate}[label={\alph*.}]
\item Can $\vect{u} \dotprod \vect{v} = -7$ if $\vectlength\vect{u}\vectlength = 3$ and $\vectlength\vect{v}\vectlength = 2$? Defend your answer.

\item Find $\vect{u} \dotprod \vect{v}$ if $\vect{u} = \leftB
\begin{array}{r}
2\\
-1\\
2
\end{array}
\rightB$, $\vectlength \vect{v} \vectlength = 6$, and the angle between $\vect{u}$ and $\vect{v}$ is $\frac{2\pi}{3}$.

\end{enumerate}
\end{ex}

\begin{ex}
Show $(\vect{u} + \vect{v}) \dotprod (\vect{u} - \vect{v}) = \vectlength\vect{u}\vectlength^{2} - \vectlength\vect{v}\vectlength^{2}$ for any vectors $\vect{u}$ and $\vect{v}$.
\end{ex}

\begin{ex}
\begin{enumerate}[label={\alph*.}]
\item Show $\vectlength\vect{u} + \vect{v}\vectlength^{2} + \vectlength\vect{u} - \vect{v}\vectlength^{2} = 2(\vectlength\vect{u}\vectlength^{2} + \vectlength\vect{v}\vectlength^{2})$ for any vectors $\vect{u}$ and $\vect{v}$.

\item What does this say about parallelograms?

\end{enumerate}
\begin{sol}
\begin{enumerate}[label={\alph*.}]
\setcounter{enumi}{1}
\item  The sum of the squares of the lengths of the diagonals equals the sum of the squares of the lengths of the four sides.

\end{enumerate}
\end{sol}
\end{ex}

\begin{ex}
Show that if the diagonals of a parallelogram are perpendicular, it is necessarily a rhombus. [\textit{Hint}: Example~\ref{exa:011899}.]
\end{ex}

\begin{ex}
Let $A$ and $B$ be the end points of a diameter of a circle (see the diagram). If $C$ is any point on the circle, show that $AC$ and $BC$ are perpendicular. [\textit{Hint}: Express $\longvect{AB} \dotprod (\longvect{AB} \times \longvect{AC}) = 0$ and $\longvect{BC}$ in terms of $\vect{u} = \longvect{OA}$ and $\vect{v} = \longvect{OC}$, where $O$ is the centre.]


\begin{figure}[H]
\centering
\begin{tikzpicture}[scale=0.7]
\coordinate (ptO) at (0, 0);
\coordinate (ptA) at (-2, 0);
\coordinate (ptB) at (2, 0);

\draw[dkgreenvect, thick] (ptO)--(60:2cm) node(ptC){}
 (ptC.center)--(ptB)
 (ptB)--(ptA)
 (ptA)--(ptC.center);
\draw[dkbluevect, thick] (ptO) circle (2);

\node[below] at (ptO) {\small $O$};
\node[left] at (ptA) {\small $A$};
\node[right] at (ptB) {\small $B$};
\node[above] at (ptC) {\small $C$};
\end{tikzpicture}
	
%\caption{\label{fig:012573}}
\end{figure}
\end{ex}

\begin{ex}
Show that $\vect{u}$ and $\vect{v}$ are orthogonal, if and only if $\vectlength\vect{u} + \vect{v}\vectlength^{2} = \vectlength\vect{u}\vectlength^{2} + \vectlength\vect{v}\vectlength^{2}$.
\end{ex}

\begin{ex}
Let $\vect{u}$, $\vect{v}$, and $\vect{w}$ be pairwise orthogonal vectors.


\begin{enumerate}[label={\alph*.}]
\item Show that $\vectlength\vect{u} + \vect{v} + \vect{w}\vectlength^{2} = \vectlength\vect{u}\vectlength^{2} + \vectlength\vect{v}\vectlength^{2} + \vectlength\vect{w}\vectlength^{2}$.

\item If $\vect{u}$, $\vect{v}$, and $\vect{w}$ are all the same length, show that they all make the same angle with $\vect{u} + \vect{v} + \vect{w}$.

\end{enumerate}
\begin{sol}
\begin{enumerate}[label={\alph*.}]
\setcounter{enumi}{1}
\item  The angle $\theta$ between $\vect{u}$ and $(\vect{u} + \vect{v} + \vect{w})$ is given by $\cos\theta = \frac{\vect{u} \dotprod (\vect{u} + \vect{v} + \vect{w})}{\vectlength \vect{u} \vectlength \vectlength \vect{u} + \vect{v} + \vect{w} \vectlength} = \frac{\vectlength \vect{u} \vectlength}{\sqrt{\vectlength \vect{u} \vectlength^2 + \vectlength \vect{v} \vectlength^2 + \vectlength \vect{w} \vectlength^2}} = \frac{1}{\sqrt{3}}$ because $\vectlength \vect{u} \vectlength = \vectlength \vect{v} \vectlength = \vectlength \vect{w} \vectlength$. Similar remarks apply to the other angles.

\end{enumerate}
\end{sol}
\end{ex}

\begin{ex}
\begin{enumerate}[label={\alph*.}]
\item Show that $\vect{n} = \leftB
\begin{array}{r}
a\\
b
\end{array}
\rightB$ is orthogonal to every vector along the line $ax + by + c = 0$.

\item Show that the shortest distance from $P_{0}(x_{0}, y_{0})$ to the line is $\frac{|ax_{0} + by_{0} + c|}{\sqrt{a^2 + b^2}}$.


[\textit{Hint}: If $P_{1}$ is on the line, project $\vect{u} = \longvect{P_{1}P}_{0}$ on $\vect{n}$.]

\end{enumerate}
\begin{sol}
\begin{enumerate}[label={\alph*.}]
\setcounter{enumi}{1}
\item  Let $\vect{p}_{0}$, $\vect{p}_{1}$ be the vectors of $P_{0}$, $P_{1}$, so $\vect{u} = \vect{p}_{0} - \vect{p}_{1}$. Then $\vect{u} \cdot \vect{n} = \vect{p}_{0} \cdot \vect{n}$ -- $\vect{p}_{1} \cdot \vect{n} = (ax_{0} + by_{0}) - (ax_{1} + by_{1}) = ax_{0} + by_{0} + c$. Hence the distance is 
\begin{equation*}
\left\vectlength \left( \frac{\vect{u} \dotprod \vect{n}}{\vectlength \vect{n} \vectlength^2}\right)\vect{n} \right\vectlength = \frac{|\vect{u} \dotprod \vect{n}|}{\vectlength \vect{n} \vectlength}
\end{equation*}
 as required.

\end{enumerate}
\end{sol}
\end{ex}

\begin{ex}
Assume $\vect{u}$ and $\vect{v}$ are nonzero vectors that are not parallel. Show that $\vect{w} = \vectlength\vect{u}\vectlength\vect{v} + \vectlength\vect{v}\vectlength\vect{u}$ is a nonzero vector that bisects the angle between $\vect{u}$ and $\vect{v}$.
\end{ex}

\begin{ex}
Let $\alpha$, $\beta$, and $\gamma$ be the angles a vector $\vect{v} \neq \vect{0}$ makes with the positive $x$, $y$, and $z$ axes, respectively. Then $\cos \alpha$, $\cos \beta$, and $\cos \gamma$ are called the \textbf{direction cosines}\index{direction cosines} of the vector $\vect{v}$.

\begin{enumerate}[label={\alph*.}]
\item If $\vect{v} = \leftB
\begin{array}{r}
a\\
b\\
c
\end{array}
\rightB$, show that $\cos\alpha = \frac{a}{\vectlength \vect{v} \vectlength}$, $\cos\beta = \frac{b}{\vectlength \vect{v} \vectlength}$, and $\cos\gamma= \frac{c}{\vectlength \vect{v} \vectlength}$.

\item Show that $\cos^{2} \alpha + \cos^2 \beta + \cos^2 \gamma = 1$.

\end{enumerate}
\begin{sol}
\begin{enumerate}[label={\alph*.}]
\setcounter{enumi}{1}
\item  This follows from \textbf{(a)} because $\vectlength\vect{v}\vectlength^{2} = a^{2} + b^{2} + c^{2}$.

\end{enumerate}
\end{sol}
\end{ex}

\begin{ex}
Let $\vect{v} \neq \vect{0}$ be any nonzero vector and suppose that a vector $\vect{u}$ can be written as $\vect{u} = \vect{p} + \vect{q}$, where $\vect{p}$ is parallel to $\vect{v}$ and $\vect{q}$ is orthogonal to $\vect{v}$. Show that $\vect{p}$ must equal the projection of $\vect{u}$ on $\vect{v}$. [\textit{Hint}: Argue as in the proof of Theorem~\ref{thm:011958}.]
\end{ex}

\begin{ex}
Let $\vect{v} \neq \vect{0}$ be a nonzero vector and let $a \neq 0$ be a scalar. If $\vect{u}$ is any vector, show that the projection of $\vect{u}$ on $\vect{v}$ equals the projection of $\vect{u}$ on $a\vect{v}$.
\end{ex}

\begin{ex}
\begin{enumerate}[label={\alph*.}]
\item Show that the \textbf{Cauchy-Schwarz inequality}\index{Cauchy-Schwarz inequality} $|\vect{u} \dotprod \vect{v}| \leq \vectlength\vect{u}\vectlength\vectlength\vect{v}\vectlength$ holds for all vectors $\vect{u}$ and $\vect{v}$. [\textit{Hint}: $|\cos \theta| \leq 1$ for all angles $\theta$.]

\item Show that $|\vect{u} \dotprod \vect{v}| = \vectlength\vect{u}\vectlength\vectlength\vect{v}\vectlength$ if and only if $\vect{u}$ and $\vect{v}$ are parallel.


[\textit{Hint}: When is $\cos \theta = \pm 1$?]

\item Show that $|x_{1}x_{2} + y_{1}y_{2} + z_{1}z_{2}| \\ \leq \sqrt{x_{1}^2 + y_{1}^2 + z_{1}^2}\sqrt{x_{2}^2 + y_{2}^2 + z_{2}^2}$

holds for all numbers $x_{1}$, $x_{2}$, $y_{1}$, $y_{2}$, $z_{1}$, and $z_{2}$.

\item Show that $|xy + yz + zx| \leq x^{2} + y^{2} + z^{2}$ for all $x$, $y$, and $z$.

\item Show that $(x + y + z)^{2} \leq 3(x^{2} + y^{2} + z^{2})$ holds for all $x$, $y$, and $z$.

\end{enumerate}
\begin{sol}
\begin{enumerate}[label={\alph*.}]
\setcounter{enumi}{3}
\item  Take $\leftB
\begin{array}{c}
x_{1}\\
y_{1}\\
z_{1}
\end{array}
\rightB
= 
\leftB
\begin{array}{c}
x\\
y\\
z
\end{array}
\rightB$ and $\leftB
\begin{array}{c}
x_{2}\\
y_{2}\\
z_{2}
\end{array}
\rightB
= 
\leftB
\begin{array}{c}
y\\
z\\
x
\end{array}
\rightB$ in (\textbf{c}).


\end{enumerate}
\end{sol}
\end{ex}

\begin{ex}
Prove that the \textbf{triangle inequality}\index{triangle!inequality}\index{triangle inequality} $\vectlength \vect{u} + \vect{v} \vectlength \leq \vectlength\vect{u}\vectlength + \vectlength\vect{v}\vectlength$ holds for all vectors $\vect{u}$ and $\vect{v}$. [\textit{Hint}: Consider the triangle with $\vect{u}$ and $\vect{v}$ as two sides.]
\end{ex}
\end{multicols}
