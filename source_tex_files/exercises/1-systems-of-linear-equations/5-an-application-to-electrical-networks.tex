\newpage

\section*{Exercises for \ref{sec:1_5}}

\begin{Filesave}{solutions}
\solsection{Section~\ref{sec:1_5}}
\end{Filesave}

In Exercises 1 to 4, find the currents in the circuits.
\begin{multicols}{2}
\begin{ex}
\begin{figure}[H]
\vspace*{-2em}
\centering
\begin{circuitikz}[scale=0.8]
\draw[dkgreenvect,text=black] (0,0) to [battery1, v_= $20\; V$, color=dkgreenvect, font=\small] (0,2)
      to [R = $ 6 \Omega $, i=$I_1$, color=dkgreenvect, font=\small] (4,2)
      to (4,0)
      to [R = $ 4 \Omega $,i>_=$I_2$, color=dkgreenvect, font=\small] (0,0)
(4,0) to [battery1, v= $10\; V$, color=dkgreenvect, font=\small] (4,-2)
      to [R = $ 2 \Omega $,i>^=$I_3$, color=dkgreenvect, font=\small] (0,-2)
      to (0,0)
;
\fill[black] (0,0) circle (2pt) ;
\fill[black] (4,0) circle (2pt) ;
\end{circuitikz}

%\caption{\label{fig:001841}}
\end{figure}
\end{ex}

\vspace*{-2em}
\begin{ex}
\begin{figure}[H]
\centering
\begin{circuitikz}
\ctikzset{label/align=straight}
\draw[dkgreenvect, text=black] (0,0) to [battery1, i=$I_1 $, v= $5\; V$, color=dkgreenvect, font=\small] (3,2)
      to [R = $ 5 \Omega $, color=dkgreenvect, font=\small] (6,0)
      to [R = $ 10 \Omega $,i_=$\;\;\; I_2$,  color=dkgreenvect, font=\small] (0,0)
	(0,0) to [R, l_=$ 5 \Omega $,i=$I_3$, color=dkgreenvect, font=\small] (3,-2)
	  to [battery1, v_= $10\; V$, color=dkgreenvect, font=\small] (6,0)
;
\fill[black] (0,0) circle (2pt) ;
\fill[black] (4,0) circle (2pt) ;
\end{circuitikz}

%\caption{\label{fig:001844}}
\end{figure}
\begin{sol}
$ I_1 = -\frac{1}{5}$, $I_2 = \frac{3}{5}$, $I_3 = \frac{4}{5} $
\end{sol}
\end{ex}

\begin{ex}
\begin{figure}[H]
\centering
\begin{circuitikz}
\draw[dkgreenvect, text=black] (0,0) to [R = $ 10 \Omega $, color=dkgreenvect, font=\small] (0,-2.5)
	   to [battery1, v_= $10\; V$, color=dkgreenvect, font=\small] (2.5,-2.5)
	   to [battery1, v_= $5\; V$, i>=$I_2$, color=dkgreenvect, font=\small] (2.5,0)
	   to [battery1, v_= $5\; V$, i>_=$I_1$, color=dkgreenvect, font=\small] (0,0)
(2.5,-2.5) to [R = $ 10 \Omega $, i>^=$I_4$, color=dkgreenvect, font=\small] (5,-2.5)
	   to [battery1, v_= $5\; V$, i<=$I_5$, color=dkgreenvect, font=\small] (5,0)
       to [R = $ 20 \Omega $, i_=$I_3$, color=dkgreenvect, font=\small] (2.5,0)
(5,-2.5) to [R = $ 20 \Omega $, color=dkgreenvect, font=\small] (7.5,-2.5)
       to [short, i=$I_6$, color=dkgreenvect, font=\small] (7.5,0)
       to [battery1, v_= $20\; V$, color=dkgreenvect, font=\small] (5,0)
;
\fill[black] (2.5,-2.5) circle (2pt) ;
\fill[black] (5,-2.5) circle (2pt) ;
\fill[black] (2.5,0) circle (2pt) ;
\fill[black] (5,0) circle (2pt) ;
\end{circuitikz}

%\caption{\label{fig:001847}}
\end{figure}
\end{ex}

\vspace*{-2em}
\begin{ex}
All resistances are $10 \Omega$.

\begin{figure}[H]
\centering
\begin{circuitikz}[scale=0.6]
\draw[dkgreenvect, text=black] (0,0) to [battery1, v_= $20\; V$, i>_=$I_1$, color=dkgreenvect, font=\small] (6,0)
      to [R, i_=$I_4$, color=dkgreenvect, font=\small] (3,5)
      to [R, i>_=$I_6$, color=dkgreenvect, font=\small] (0,0)
(3,5) to [battery1, v_= $\;$, i=$I_2$, color=dkgreenvect, font=\small] (3,2)
      to [R, i>^=$I_5$, color=dkgreenvect, font=\small] (0,0)
(6,0) to [R, i_=$I_3$, color=dkgreenvect, font=\small] (3,2)
;
\fill[black] (0,0) circle (2pt) ;
\fill[black] (3,2) circle (2pt) ;
\fill[black] (3,5) circle (2pt) ;
\fill[black] (6,0) circle (2pt) ;
\node[font=\small] at (3,3.85) {$10\; V$};
\end{circuitikz}

%\caption{\label{fig:001850}}
\end{figure}
\begin{sol}
$ I_1 = 2$, $I_2 = 1$, $I_3 = \frac{1}{2}$, $I_4 = \frac{3}{2}$, $I_5 = \frac{3}{2}$, $I_6 = \frac{1}{2}$

\end{sol}
\end{ex}
%\end{multicols}

%\setlength{\columnsep}{-20pt}
%\begin{multicols}{2}
\begin{ex}
\newline
Find the voltage $x$ such that the current $I_1 = 0$.
\begin{figure}[H]
\centering
\begin{circuitikz}[scale=0.75]
\draw[dkgreenvect, text=black] (0,0) to [battery1, v_= $x\; V$, color=dkgreenvect, font=\small] (2.5,0)
      to [short, i_=$I_3$, color=dkgreenvect, font=\small] (5,0)
(5,3) to [battery1, v^= $5\; V$, color=dkgreenvect, font=\small] (5,0)
(5,3) to [R, l_= $ 2 \Omega $, color=dkgreenvect, font=\small] (2.5,3)
      to [R = $ 1 \Omega $, color=dkgreenvect, font=\small] (2.5,1.5)
      to [battery1, v^= $2\; V$, i>_=$I_2$, color=dkgreenvect, font=\small] (2.5,0)
(2.5,3) to [short, i=$I_1$, color=dkgreenvect, font=\small] (0,3)
	  to [R = $ 1 \Omega $, color=dkgreenvect, font=\small] (0,0)
;
\fill[black] (2.5,0) circle (2pt) ;
\fill[black] (2.5,3) circle (2pt) ;
\end{circuitikz}

%\caption{\label{fig:001854}}
\end{figure}
\end{ex}
\end{multicols}
\setlength{\columnsep}{10pt}

