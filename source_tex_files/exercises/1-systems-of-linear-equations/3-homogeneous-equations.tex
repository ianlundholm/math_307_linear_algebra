\section*{Exercises for \ref{sec:1_3}}

\begin{Filesave}{solutions}
\solsection{Section~\ref{sec:1_3}}
\end{Filesave}

\begin{multicols}{2}
\begin{ex}
Consider the following statements about a system of linear equations with augmented matrix $A$. In each case either prove the statement or give an example for which it is false.

\begin{enumerate}[label={\alph*.}]
\item If the system is homogeneous, every solution is trivial.

\item If the system has a nontrivial solution, it cannot be homogeneous.

\item If there exists a trivial solution, the system is homogeneous.

\item If the system is consistent, it must be homogeneous.
\end{enumerate}

\noindent\textit{Now assume that the system is homogeneous.}

\begin{enumerate}[label={\alph*.}]
\setcounter{enumi}{4}
\item If there exists a nontrivial solution, there is no trivial solution.

\item If there exists a solution, there are infinitely many solutions.

\item If there exist nontrivial solutions, the row-echelon form of $A$ has a row of zeros.

\item If the row-echelon form of $A$ has a row of zeros, there exist nontrivial solutions.

\item If a row operation is applied to the system, the new system is also homogeneous.

\end{enumerate}
\begin{sol}
\begin{enumerate}[label={\alph*.}]
\setcounter{enumi}{1}
\item  False. 
$A = 
\leftB \begin{array}{rrr|r}
	1 & 0 & 1 & 0 \\
	0 & 1 & 1 & 0	
\end{array} \rightB$

\setcounter{enumi}{3}
\item  False. 
$A =
\leftB \begin{array}{rrr|r}
	1 & 0 & 1 & 1 \\
	0 & 1 & 1 & 0	
\end{array} \rightB$

\setcounter{enumi}{5}
\item  False. 
$A =
\leftB \begin{array}{rr|r}
	1 & 0 & 0 \\
	0 & 1 & 0	
\end{array} \rightB$

\setcounter{enumi}{7}
\item  False. 
$A =
\leftB \begin{array}{rr|r}
	1 & 0 & 0 \\
	0 & 1 & 0 \\
	0 & 0 & 0
\end{array} \rightB$

\end{enumerate}
\end{sol}
\end{ex}

\begin{ex}
In each of the following, find all values of $a$ for which the system has nontrivial solutions, and determine all solutions in each case.

\begin{exenumerate}
\exitem 
$\arraycolsep=1pt
\begin{array}[t]{rlrlrcr}
	 x & - & 2y & + &  z & = & 0 \\
	 x & + & ay & - & 3z & = & 0 \\
	-x & + & 6y & - & 5z & = & 0 \\
\end{array}$
\exitem 
$\arraycolsep=1pt
\begin{array}[t]{rlrlrcr}
	x & + & 2y & + &  z & = & 0 \\
	x & + & 3y & + & 6z & = & 0 \\
	2x & + & 3y & + & az & = & 0 \\
\end{array}$
\exitem 
$\arraycolsep=1pt
\begin{array}[t]{rlrlrcr}
	x & + &  y & - &  z & = & 0 \\
	  &   & ay & - &  z & = & 0 \\
	 x & + &  y & + & az & = & 0 \\
\end{array}$
\exitem 
$\arraycolsep=1pt
\begin{array}[t]{rlrlrcr}
	ax & + &  y & + &  z & = & 0 \\
	x & + &  y & - &  z & = & 0 \\
	 x & + &  y & + & az & = & 0 \\
\end{array}$
\end{exenumerate}
\begin{sol}
\begin{enumerate}[label={\alph*.}]
\setcounter{enumi}{1}
\item  $a = -3$, $x = 9t$, $y = -5t$, $z = t$

\setcounter{enumi}{3}
\item  $a = 1$, $x = -t$, $y = t$, $z = 0$; or $a = -1$, $x = t$, $y = 0$, $z = t$

\end{enumerate}
\end{sol}
\end{ex}

\begin{ex}
Let $\vect{x} = 
\leftB \begin{array}{r}
	2 \\
	1 \\
	-1
\end{array} \rightB$, $\vect{y} = 
\leftB \begin{array}{r}
	1 \\
	0 \\
	1
\end{array} \rightB$, and \\ $\vect{z} = 
\leftB \begin{array}{r}
	1 \\
	1 \\
	-2
\end{array} \rightB$. In each case, either write $\vect{v}$ as a linear combination of $\vect{x}$, $\vect{y}$, and $\vect{z}$, or show that it is not such a linear combination.

\begin{exenumerate}
\exitem 
$\vect{v} = 
\leftB \begin{array}{r}
	0 \\
	1 \\
	-3	
\end{array} \rightB$
\exitem 
$\vect{v} = 
\leftB \begin{array}{r}
	4 \\
	3 \\
	-4
\end{array} \rightB$
\exitem 
$\vect{v} = 
\leftB \begin{array}{r}
	3 \\
	1 \\
	0
\end{array} \rightB$
\exitem 
$\vect{v} = 
\leftB \begin{array}{r}
	3 \\
	0 \\
	3
\end{array} \rightB$
\end{exenumerate}
\begin{sol}
\begin{enumerate}[label={\alph*.}]
\setcounter{enumi}{1}
\item  Not a linear combination.

\setcounter{enumi}{3}
\item  $\vect{v} = \vect{x} + 2\vect{y} - \vect{z}$

\end{enumerate}
\end{sol}
\end{ex}

\begin{ex}
In each case, either express $\vect{y}$ as a linear combination of $\vect{a}_1$, $\vect{a}_2$, and $\vect{a}_3$, or show that it is not such a linear combination. Here:
\begin{equation*}
\vect{a}_1 =
\leftB \begin{array}{r}
	-1 \\
	3 \\
	0 \\
	1
\end{array} \rightB, \
\vect{a}_2 =
\leftB \begin{array}{r}
	3 \\
	1 \\
	2 \\
	0
\end{array} \rightB,
\mbox{ and } \vect{a}_3 =
\leftB \begin{array}{r}
	1 \\
	1 \\
	1 \\
	1
\end{array} \rightB
\end{equation*}

\begin{exenumerate}
\exitem 
$\vect{y} =
\leftB \begin{array}{r}
	1 \\
	2 \\
	4 \\
	0
\end{array} \rightB$
\exitem 
$\vect{y} =
\leftB \begin{array}{r}
	-1 \\
	9 \\
	2 \\
	6
\end{array} \rightB$
\end{exenumerate}
\begin{sol}
\begin{enumerate}[label={\alph*.}]
\setcounter{enumi}{1}
\item  $\vect{y} = 2\vect{a}_1 - \vect{a}_2 + 4\vect{a}_3$.

\end{enumerate}
\end{sol}
\end{ex}

\begin{ex}
For each of the following homogeneous systems, find a set of basic solutions and express the general solution as a linear combination of these basic solutions.

\begin{enumerate}[label={\alph*.}]
\item 
$\arraycolsep=1pt
\begin{array}[t]{rlrlrlrlrcr}
	 x_1 & + & 2x_2 & - &  x_3 & + & 2x_4 & + &  x_5 & = & 0 \\
	 x_1 & + & 2x_2 & + & 2x_3 &   &      & + &  x_5 & = & 0 \\
	2x_1 & + & 4x_2 & - & 2x_3 & + & 3x_4 & + &  x_5 & = & 0
\end{array}$

\item 
$\arraycolsep=1pt
\begin{array}[t]{rlrlrlrlrcr}
	 x_1 & + & 2x_2 & - &  x_3 & + &  x_4 & + &  x_5 & = & 0 \\
	-x_1 & - & 2x_2 & + & 2x_3 &   &      & + &  x_5 & = & 0 \\
	-x_1 & - & 2x_2 & + & 3x_3 & + &  x_4 & + & 3x_5 & = & 0
\end{array}$

\item 
$\arraycolsep=1pt
\begin{array}[t]{rlrlrlrlrcr}
	 x_1 & + &  x_2 & - &  x_3 & + & 2x_4 & + &  x_5 & = & 0 \\
	 x_1 & + & 2x_2 & - &  x_3 & + &  x_4 & + &  x_5 & = & 0 \\
	2x_1 & + & 3x_2 & - &  x_3 & + & 2x_4 & + &  x_5 & = & 0 \\
	4x_1 & + & 5x_2 & - & 2x_3 & + & 5x_4 & + & 2x_5 & = & 0
\end{array}$

\item 
$\arraycolsep=1pt
\begin{array}[t]{rlrlrlrlrcr}
	 x_1 & + &  x_2 & - & 2x_3 & - & 2x_4 & + & 2x_5 & = & 0 \\
	2x_1 & + & 2x_2 & - & 4x_3 & - & 4x_4 & + &  x_5 & = & 0 \\
	 x_1 & - &  x_2 & + & 2x_3 & + & 4x_4 & + &  x_5 & = & 0 \\
	-2x_1 & - & 4x_2 & + & 8x_3 & + &10x_4 & + &  x_5 & = & 0
\end{array}$

\end{enumerate}
\begin{sol}
\begin{enumerate}[label={\alph*.}]
\setcounter{enumi}{1}
\item  
$r \leftB \begin{array}{r}
	-2 \\
	1 \\
	0 \\
	0 \\
	0
\end{array} \rightB
+ s
\leftB \begin{array}{r}
	-2 \\
	0 \\
	-1 \\
	1 \\
	0
\end{array} \rightB
+ t
\leftB \begin{array}{r}
	-3 \\
	0 \\
	-2 \\
	0 \\
	1
\end{array} \rightB$

\setcounter{enumi}{3}
\item 
$s \leftB \begin{array}{r}
	0 \\
	2 \\
	1 \\
	0 \\
	0
\end{array} \rightB
+ t
\leftB \begin{array}{r}
	-1 \\
	3 \\
	0 \\
	1 \\
	0
\end{array} \rightB$
\end{enumerate}
\end{sol}
\end{ex}

\begin{ex} %1.3.6
\begin{enumerate}[label={\alph*.}]
\item Does Theorem~\ref{thm:001473} imply that the system 
$\left \{
\arraycolsep=1pt
\begin{array}{rcl}
	-z + 3y & = & 0 \\
	2x - 6y & = & 0
\end{array}
\right.$ has nontrivial solutions? Explain.

\item Show that the converse to Theorem~\ref{thm:001473} is not true. That is, show that the existence of nontrivial solutions does \textit{not} imply that there are more variables than equations.

\end{enumerate}
\begin{sol}
\begin{enumerate}[label={\alph*.}]
\setcounter{enumi}{1}
\item  The system in (a) has nontrivial solutions.

\end{enumerate}
\end{sol}
\end{ex}

\begin{ex} %1.3.7
In each case determine how many solutions (and how many parameters) are possible for a homogeneous system of four linear equations in six variables with augmented matrix $A$. Assume that $A$ has nonzero entries. Give all possibilities.

\begin{exenumerate}
\exitem Rank $A = 2$.
\exitem Rank $A = 1$.
\exitem* $A$ has a row of zeros.
\exitem* The row-echelon form of $A$ has a row of zeros.
\end{exenumerate}
\begin{sol}
\begin{enumerate}[label={\alph*.}]
\setcounter{enumi}{1}
\item  By Theorem~\ref{thm:001133}, there are $n - r = 6 - 1 = 5$ parameters and thus infinitely many solutions.

\setcounter{enumi}{3}
\item  If $R$ is the row-echelon form of $A$, then $R$ has a row of zeros and 4 rows in all. Hence $R$ has $r = \func{rank } A = 1$, $2$, or $3$. Thus there are $n - r = 6 - r = 5$, $4$, or $3$ parameters and thus infinitely many solutions.

\end{enumerate}
\end{sol}
\end{ex}


\begin{ex}
The graph of an equation $ax + by + cz = 0$ is a plane through the origin (provided that not all of $a$, $b$, and $c$ are zero). Use Theorem~\ref{thm:001473} to show that two planes through the origin have a point in common other than the origin $(0, 0, 0)$.
\end{ex}

\begin{ex} %1.3.9
\begin{enumerate}[label={\alph*.}]
\item Show that there is a line through any pair of points in the plane. [\textit{Hint}: Every line has equation $ax + by + c = 0$, where $a$, $b$, and $c$ are not all zero.]

\item Generalize and show that there is a plane $ax + by + cz + d = 0$ through any three points in space.

\end{enumerate}
\begin{sol}
\begin{enumerate}[label={\alph*.}]
\setcounter{enumi}{1}
\item  That the graph of $ax + by + cz = d$ contains three points leads to 3 linear equations homogeneous in variables $a$, $b$, $c$, and $d$. Apply Theorem~\ref{thm:001473}.

\end{enumerate}
\end{sol}
\end{ex}

\begin{ex} %1.3.10
The graph of 
\begin{equation*}
a(x^2 + y^2) + bx + cy + d = 0
\end{equation*}
 is a circle if $a \neq 0$. Show that there is a circle through any three points in the plane that are not all on a line.
\end{ex}

\begin{ex} %1.3.11
Consider a homogeneous system of linear equations in $n$ variables, and suppose that the augmented matrix has rank $r$. Show that the system has nontrivial solutions if and only if $n > r$.

\begin{sol}
There are $n - r$ parameters (Theorem~\ref{thm:001133}), so there are nontrivial solutions if and only if $n - r > 0$.
\end{sol}
\end{ex}

\begin{ex} %1.3.12
If a consistent (possibly nonhomogeneous) system of linear equations has more variables than equations, prove that it has more than one solution.
\end{ex}
\end{multicols}
