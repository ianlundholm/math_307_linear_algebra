\section*{Exercises for \ref{sec:3_4}}

\begin{Filesave}{solutions}
\solsection{Section~\ref{sec:3_4}}
\end{Filesave}

\begin{multicols}{2}
\begin{ex}
In each case find the characteristic polynomial, eigenvalues, eigenvectors, and (if possible) an invertible matrix $P$ such that $P^{-1}AP$ is diagonal.


\begin{exenumerate}
\exitem $A = \leftB \begin{array}{rr}
1 & 2 \\
3 & 2 
\end{array}\rightB$
\exitem $A = \leftB \begin{array}{rr}
2 & -4 \\
-1 & -1 
\end{array}\rightB$
\exitem $A = \leftB \begin{array}{rrr}
7 & 0 & -4 \\
0 & 5 & 0 \\
5 & 0 & -2 
\end{array}\rightB$
\exitem $A = \leftB \begin{array}{rrr}
1 & 1 & -3 \\
2 & 0 & 6 \\
1 & -1 & 5 
\end{array}\rightB$
\exitem $A = \leftB \begin{array}{rrr}
1 & -2 & 3 \\
2 & 6 &-6 \\
1 & 2 & -1 
\end{array}\rightB$
\exitem  $A = \leftB \begin{array}{rrr}
0 & 1 & 0 \\
3 & 0 & 1 \\
2 & 0 & 0 
\end{array}\rightB$
\exitem $A = \leftB \begin{array}{rrr}
3 & 1 & 1 \\
-4 & -2 & -5 \\
2 & 2 & 5 
\end{array}\rightB$
\exitem $A = \leftB \begin{array}{rrr}
2 & 1 & 1 \\
0 & 1 & 0 \\
1 & -1 & 2 
\end{array}\rightB$
\exitem* $A = \leftB \begin{array}{rrr}
\lambda & 0 & 0 \\
0 & \lambda & 0 \\
0 & 0 & \mu 
\end{array}\rightB$, $\lambda \neq \mu $
\end{exenumerate}
\begin{sol}
\begin{enumerate}[label={\alph*.}]
\setcounter{enumi}{1}
\item $(x-3)(x+2); 3; -2$; $\leftB \begin{array}{r}
4 \\
-1
\end{array} \rightB, \leftB \begin{array}{r}
1 \\
1
\end{array} \rightB$; \\ $P = \leftB \begin{array}{rr}
4 & 1 \\
-1 & 1 
\end{array} \rightB$; $P^{-1}AP = \leftB \begin{array}{rr}
3 & 0 \\
0 & -2 
\end{array}\rightB.$

\setcounter{enumi}{3}
\item $(x-2)^3 ; 2 ; \leftB \begin{array}{c}
1 \\
1 \\
0
\end{array}\rightB, \leftB \begin{array}{r}
-3 \\
0 \\
1
\end{array}\rightB$;  No such $P$; Not diagonalizable.

\setcounter{enumi}{5}
\item $(x+1)^2(x-2) ; -1, -2; \leftB \begin{array}{r}
-1 \\
1 \\
2
\end{array}\rightB, \leftB \begin{array}{r}
1 \\
2 \\
1
\end{array}\rightB$; No such $P$; Not diagonalizable. Note that this matrix and the matrix in Example~\ref{exa:009262} have the same characteristic polynomial, but that matrix is diagonalizable.

\setcounter{enumi}{7}
\item $(x-1)^2(x-3) ; 1, 3; \leftB \begin{array}{r}
-1 \\
0 \\
1
\end{array}\rightB, \leftB \begin{array}{r}
1 \\
0 \\
1
\end{array}\rightB$
 No such $P$; Not diagonalizable.
\end{enumerate}
\end{sol}
\end{ex}


\begin{ex}
In each case, find $P^{-1}AP$ and then compute $A^{n}$.


\begin{enumerate}[label={\alph*.}]
\item $ A = \leftB \begin{array}{rr}
6 & -5 \\
2 & -1 
\end{array}\rightB, P = \leftB \begin{array}{rr}
1 & 5 \\
1 & 2 
\end{array}\rightB$


\item $ A = \leftB \begin{array}{rr}
-7 & -12 \\
6 & -10 
\end{array}\rightB, P = \leftB \begin{array}{rr}
-3 & 4 \\
2 & -3 
\end{array}\rightB$

[\textit{Hint}: $(PDP^{-1})^{n} = PD^{n}P^{-1}$ for each $n = 1, 2, \dots$.]

\end{enumerate}
\begin{sol}
\begin{enumerate}[label={\alph*.}]
\setcounter{enumi}{1}
\item  $ P^{-1}AP = \leftB \begin{array}{rr}
1 & 0 \\
0 & 2
\end{array}\rightB$, so $A^n  = P \leftB \begin{array}{rr}
1 & 0 \\
0 & 2^n 
\end{array}\rightB P^{-1} = \leftB \begin{array}{cc}
9 - 8 \cdot 2^n & 12(1-2^n) \\
6(2^n-1) & 9\cdot 2^n - 8
\end{array}\rightB$

\end{enumerate}
\end{sol}
\end{ex}


\begin{ex}
\begin{enumerate}[label={\alph*.}]
\item If $A = \leftB \begin{array}{rr}
1 & 3 \\
0 & 2 \end{array} \rightB$
 and $B = \leftB \begin{array}{rr}
2 & 0 \\
0 & 1 
\end{array}\rightB$
 verify that $A$ and $B$ are diagonalizable, but $AB$ is not.

\item If $D = \leftB \begin{array}{rr}
1 & 0 \\
0 & -1 
\end{array}\rightB$
 find a diagonalizable matrix $A$ such that $D + A$ is not diagonalizable.

\end{enumerate}
\begin{sol}
\begin{enumerate}[label={\alph*.}]
\setcounter{enumi}{1}
\item  $A = \leftB \begin{array}{rr}
0 & 1 \\
0 & 2 
\end{array}\rightB$ 


\end{enumerate}
\end{sol}
\end{ex}

\begin{ex}
If $A$ is an $n \times n$ matrix, show that $A$ is diagonalizable if and only if $A^{T}$ is diagonalizable.
\end{ex}

\begin{ex}
If $A$ is diagonalizable, show that each of the following is also diagonalizable.


\begin{enumerate}[label={\alph*.}]
\item $A^{n}$, $n \geq 1$

\item $kA$, $k$ any scalar.

\item $p(A)$, $p(x)$ any polynomial (Theorem~\ref{thm:008997})

\item $U^{-1}AU$ for any invertible matrix $U$.

\item $kI + A$ for any scalar $k$.

\end{enumerate}
\begin{sol}
\begin{enumerate}[label={\alph*.}]
\setcounter{enumi}{1}
\item  and d. $PAP^{-1} = D$ is diagonal, then b. $P^{-1}(kA)P = kD$ is diagonal, and  d. $Q(U^{-1}AU)Q = D$ where $Q = PU$.

\end{enumerate}
\end{sol}
\end{ex}

\begin{ex}
Give an example of two diagonalizable matrices $A$ and $B$ whose sum $A + B$ is not diagonalizable.

\begin{sol}
$\leftB \begin{array}{cc}
1 & 1 \\
0 & 1 
\end{array}\rightB$
 is not diagonalizable by Example~\ref{exa:009234}. But $\leftB \begin{array}{rr}
1 & 1 \\
0 & 1 
\end{array}\rightB = \leftB \begin{array}{rr}
2 & 1 \\
0 & -1 
\end{array}\rightB + \leftB \begin{array}{rr}
-1 & 0 \\
0 & 2 
\end{array}\rightB$
 where $\leftB \begin{array}{rr} 
2 & 1 \\
0 & -1 
\end{array}\rightB$  has diagonalizing matrix $P = \leftB \begin{array}{rr}
1 & -1 \\
0 & 3 
\end{array}\rightB$
 and $\leftB \begin{array}{rr}
-1 & 0 \\
0 & 2 
\end{array}\rightB$
 is already diagonal.
\end{sol}
\end{ex}

\begin{ex}
If $A$ is diagonalizable and $1$ and $-1$ are the only eigenvalues, show that $A^{-1} = A$.
\end{ex}

\begin{ex}
If $A$ is diagonalizable and $0$ and $1$ are the only eigenvalues, show that $A^{2} = A$.

\begin{sol}
We have $\lambda^{2} = \lambda$ for every eigenvalue $\lambda$ (as $\lambda = 0, 1$) so $D^{2} = D$, and so $A^{2} = A$ as in Example~\ref{exa:009262}.
\end{sol}
\end{ex}

\begin{ex}
If $A$ is diagonalizable and $\lambda \geq 0$ for each eigenvalue of $A$, show that $A = B^{2}$ for some matrix $B$.
\end{ex}

\begin{ex}
If $P^{-1}AP$ and $P^{-1}BP$ are both diagonal, show that $AB = BA$. [\textit{Hint}: Diagonal matrices commute.]
\end{ex}

\begin{ex}
A square matrix $A$ is called \textbf{nilpotent}\index{matrix!nilpotent matrix}\index{nilpotent}\index{square matrix ($n \times n$ matrix)!nilpotent matrix} if $A^{n} = 0$ for some $n \geq 1$. Find all nilpotent diagonalizable matrices. [\textit{Hint}: Theorem~\ref{thm:008997}.]
\end{ex}




\begin{ex}
Let $A$ be diagonalizable with real eigenvalues and assume that $A^{m} = I$ for some $m \geq 1$.


\begin{enumerate}[label={\alph*.}]
\item Show that $A^{2} = I$.

\item If $m$ is odd, show that $A = I$.


[\textit{Hint}: Theorem~\ref{thm:034138}]

\end{enumerate}
\begin{sol}
\begin{enumerate}[label={\alph*.}]
\item  If $A\vect{x} = \lambda\vect{x}$, then $A^{k}\vect{x} = \lambda^{k}\vect{x}$ for each $k$. Hence $\lambda^{m}\vect{x} = A^{m}\vect{x} = \vect{x}$, so $\lambda^{m} = 1$. As $\lambda$ is real, $\lambda = \pm 1$ by the Hint. So if $P^{-1}AP = D$ is diagonal, then $D^{2} = I$ by Theorem~\ref{thm:009214}. Hence $A^{2} = PD^{2}P = I$.

\end{enumerate}
\end{sol}
\end{ex}


\begin{ex}
Let $A^{2} = I$, and assume that $A \neq I$ and $A \neq -I$.


\begin{enumerate}[label={\alph*.}]
\item Show that the only eigenvalues of $A$ are $\lambda = 1$ and $\lambda = -1$.

\item Show that $A$ is diagonalizable. [\textit{Hint}: Verify that $A(A + I) = A + I$ and $A(A - I) = -(A - I)$, and then look at nonzero columns of $A + I$ and of $A - I$.]

\item If $Q_{m} : \RR^2 \to \RR^2$  is reflection in the line $y = mx$ where $m \neq 0$, use (b) to show that the matrix of $Q_{m}$ is diagonalizable for each $m$.

\item Now prove (c) geometrically using Theorem~\ref{thm:009136}.

\end{enumerate}
\end{ex}

\begin{ex}
Let $A = \leftB \begin{array}{rrr}
2 & 3 & -3 \\
1 & 0 & -1 \\
1 & 1 & -2 
\end{array} \rightB$
 and $B = \leftB \begin{array}{rrr}
0 & 1 & 0 \\
3 & 0 & 1 \\
2 & 0 & 0  
\end{array} \rightB$.
 Show that $c_{A}(x) = c_{B}(x) = (x + 1)^{2} (x - 2)$, but $A$ is diagonalizable and $B$ is not.
\end{ex}

\begin{ex}
\begin{enumerate}[label={\alph*.}]
\item Show that the only diagonalizable matrix $A$ that has only one eigenvalue $\lambda$ is the scalar matrix $A = \lambda I$.

\item Is $\leftB \begin{array}{rr}
3 & -2 \\
2 & -1 
\end{array}\rightB$
 diagonalizable?

\end{enumerate}
\begin{sol}
\begin{enumerate}[label={\alph*.}]
\item  We have $P^{-1}AP = \lambda I$ by the diagonalization algorithm, so $A = P(\lambda I)P^{-1} = \lambda PP^{-1} = \lambda I$.

\item  No. $\lambda = 1$ is the only eigenvalue.

\end{enumerate}
\end{sol}
\end{ex}

\begin{ex}
Characterize the diagonalizable $n \times n$ matrices $A$ such that $A^{2} - 3A + 2I = 0$ in terms of their eigenvalues. [\textit{Hint}: Theorem~\ref{thm:008997}.]
\end{ex}

\begin{ex}
Let $A = \leftB \begin{array}{cc}
B & 0 \\
0 & C 
\end{array}\rightB$
 where $B$ and $C$ are square matrices.


\begin{enumerate}[label={\alph*.}]
\item If $B$ and $C$ are diagonalizable via $Q$ and $R$ (that is, $Q^{-1}BQ$ and $R^{-1}CR$ are diagonal), show that $A$ is diagonalizable via $\leftB \begin{array}{cc}
Q & 0 \\
0 & R 
\end{array}\rightB$


\item Use (a) to diagonalize $A$ if $B = \leftB \begin{array}{rr}
5 & 3 \\
3 & 5 
\end{array}\rightB$
 and $C = \leftB \begin{array}{rr}
7 & -1 \\
-1 & 7 
\end{array}\rightB$.


\end{enumerate}
\end{ex}


\end{multicols}








