\section*{Exercises for \ref{sec:3_7}}

\begin{Filesave}{solutions}
\solsection{Section~\ref{sec:3_5}}
\end{Filesave}

\begin{multicols}{2}
\begin{ex}
Use Theorem~\ref{thm:010427}
 to find the general solution to each of the following systems. Then 
find a specific solution satisfying the given boundary condition.


\begin{enumerate}[label={\alph*.}]
\item $\arraycolsep=1pt
\begin{array}[t]{rrrl}
f_1^{\prime} & = 2f_1 + 4f_2, && f_1(0)=0 \\
f_2^{\prime} & = 3f_1 + 3f_2, && f_2(0)=1
\end{array}$


\item $\arraycolsep=1pt
\begin{array}[t]{rrrlr}
f_1^{\prime} & = -f_1 + 5f_2,& & f_1(0)=&1 \\
f_2^{\prime} & = f_1 + 3f_2, && f_2(0)=&-1
\end{array}$


\item $\arraycolsep=1pt
\begin{array}[t]{rr}
f_1^{\prime}  = &  	4f_2 + 4f_3 \\
f_2^{\prime}  = & f_1 + f_2 -2f_3 \\
f_3^{\prime}  = & -f_1 + f_2 +4f_3 \\
\end{array}$

$
f_1(0) = f_2(0) = f_3(0)=1
$


\item $\arraycolsep=1pt
\begin{array}[t]{rlrr}
f_1^{\prime}  = & 2f_1 + &f_2 + &2f_3 \\
f_2^{\prime}  = & 2f_1 + &2f_2 - &2f_3 \\
f_3^{\prime}  = & 3f_1 + &f_2 + &f_3 \\
\end{array}$

$
f_1(0) = f_2(0) = f_3(0)=1
$



\end{enumerate}
\begin{sol}
\begin{enumerate}[label={\alph*.}]
\setcounter{enumi}{1}
\item  $c_1 \leftB \begin{array}{r}
1 \\
1
\end{array} \rightB e^{4x} + c_2 \leftB \begin{array}{r}
5 \\
-1
\end{array} \rightB e^{-2x};
c_1 = -\frac{2}{3}, c_2 = \frac{1}{3}$


\setcounter{enumi}{3}
\item   $c_1 \leftB \begin{array}{r}
-8 \\
10 \\
7
\end{array} \rightB e^{-x} + c_2 \leftB \begin{array}{r}
1 \\
-2 \\
1
\end{array} \rightB e^{2x} + c_3 \leftB \begin{array}{r}
1 \\
0 \\
1
\end{array} \rightB e^{4x}$; \\
$c_1 = 0, c_2 = -\frac{1}{2}, c_3=\frac{3}{2}$
\end{enumerate}
\end{sol}
\end{ex}

\begin{ex}
Show that the solution to $f^{\prime}= af$ satisfying $f(x_{0}) = k$ is $f(x) = ke^{a(x-x_0)}$.
\end{ex}

\begin{ex}
A
 radioactive element decays at a rate proportional to the amount 
present. Suppose an initial mass of 10 g decays to 8 g in 3 hours.


\begin{enumerate}[label={\alph*.}]
\item Find the mass $t$ hours later.

\item Find the half-life of the element---the time taken to decay to half its mass.

\end{enumerate}
\begin{sol}
\begin{enumerate}[label={\alph*.}]
\setcounter{enumi}{1}
\item  The solution to (a) is $m(t) = 10 \left( \frac{4}{5} \right)^{t/3}$.
 Hence we want $t$ such that $10 \left( \frac{4}{5} \right)^{t/3}=5$.
 We solve for $t$ by taking natural logarithms: 
\begin{equation*}
t = \frac{3 \ln (\frac{1}{2})}{\ln (\frac{4}{5})} = 9.32 \mbox{ hours}.
\end{equation*}
\end{enumerate}
\end{sol}
\end{ex}

\begin{ex}
The population $N(t)$ of a region at time $t$ increases at a rate proportional to the population. If the population doubles every 5 years and is 3 million initially, find $N(t)$.
\end{ex}

\begin{ex}
Let $A$ be an invertible diagonalizable $n \times n$ matrix and let $\vect{b}$ be an $n$-column of constant functions. We can solve the system $\vect{f}^{\prime} = A\vect{f} + \vect{b}$ as follows:


\begin{enumerate}[label={\alph*.}]
\item If $\vect{g}$ satisfies $\vect{g}^{\prime} = A\vect{g}$ (using Theorem~\ref{thm:010514}), show that $\vect{f} = \vect{g} - A^{-1}\vect{b}$ is a solution to $\vect{f}^{\prime} = A\vect{f} + \vect{b}$.

\item Show that every solution to $\vect{f}^{\prime} = A\vect{f} + \vect{b}$ arises as in (a) for some solution $\vect{g}$ to $\vect{g}^{\prime} = A\vect{g}$.

\end{enumerate}
\begin{sol}
\begin{enumerate}[label={\alph*.}]
\item  If $\vect{g}^{\prime} = A\vect{g}$, put $\vect{f} = \vect{g} - A^{-1}\vect{b}$. Then $\vect{f}^{\prime} = \vect{g}^{\prime}$ and $A\vect{f} = A\vect{g} - \vect{b}$, so $\vect{f}^{\prime} = \vect{g}^{\prime} = A\vect{g} = A\vect{f} + \vect{b}$, as required.

\end{enumerate}
\end{sol}
\end{ex}

\begin{ex} \label{ex:3_5_6}
Denote the second derivative of $f$ by $f^{\dprime} = (f^{\prime})^{\prime}$. Consider the second order differential equation 
\begin{equation}\label{eq:secondorderdiff}
f^{\dprime} - a_1 f^{\prime}-a_2 f = 0, \quad a_1 \mbox{ and } a_2 \mbox{ real numbers} 
\end{equation}
\begin{enumerate}[label={\alph*.}]
\item If $f$ is a solution to Equation \ref{eq:secondorderdiff}
 let $f_{1} = f$ and $f_{2} = f^{\prime} - a_{1}f$. Show that

$
\left\lbrace \arraycolsep=1pt
\begin{array}{l}
f_1^{\prime} = a_1f_1 + f_2 \\
f_2^{\prime} = a_2f_1
\end{array}, \right.  $ \\ that is $ \left. \leftB \arraycolsep=5pt \begin{array}{c}
f_1^{\prime} \\
f_2^{\prime}
\end{array}\rightB = \leftB \arraycolsep=5pt   \begin{array}{cc}
a_1 & 1 \\
a_2 & 0 
\end{array}\rightB \leftB \arraycolsep=5pt  \begin{array}{c}
f_1 \\
f_2
\end{array}\rightB \right.
$


\item Conversely, if $\leftB \begin{array}{c}
f_1 \\
f_2
\end{array}\rightB$
 is a solution to the system in (a), show that $f_{1}$ is a solution to Equation \ref{eq:secondorderdiff}.

\end{enumerate}
\begin{sol}
\begin{enumerate}[label={\alph*.}]
\setcounter{enumi}{1}
\item  Assume that $f_{1}^{\prime} = a_{1}f_{1} + f_{2}$ and $f_{2}^{\prime} = a_{2}f_{1}$. Differentiating gives $f_{1}^{\dprime} = a_{1}f_{1}^{\prime} + f_{2}{}^{\prime} = a_{1}f_{1}^{\prime} + a_{2}f_{1}$, proving that $f_{1}$ satisfies Equation \ref{eq:secondorderdiff}.

\end{enumerate}
\end{sol}
\end{ex}

\begin{ex} \label{ex:3_5_7}
Writing $f^{\dprime\prime} = (f^\dprime)^{\prime}$, consider the third order differential equation
\begin{equation*}
f^{\dprime\prime} - a_1 f^{\dprime} - a_2 f^{\prime} - a_3 f = 0
\end{equation*}
where $a_{1}$, $a_{2}$, and $a_{3}$ are real numbers. Let \\ $f_{1} = f$, $f_{2} = f^{\prime}- a_{1}f$ and $f_{3} = f^\dprime - a_{1}f{}^{\prime} - a_{2}f^\dprime$.


\begin{enumerate}[label={\alph*.}]
\item Show that $\leftB \begin{array}{c}
f_1 \\
f_2 \\
f_3
\end{array}\rightB$
 is a solution to the system

$
\left\lbrace \arraycolsep=1pt
\begin{array}{l}
f_1^{\prime} = a_1f_1 + f_2 \\
f_2^{\prime} = a_2f_1 + f_3 \\
f_3^{\prime} = a_3f_1
\end{array}, \right.$  \\  that is  $ \left. \leftB \arraycolsep=5pt \begin{array}{c}
f_1^{\prime} \\
f_2^{\prime} \\
f_3^{\prime}
\end{array}\rightB = \leftB \arraycolsep=5pt \begin{array}{ccc}
a_1 & 1  & 0 \\
a_2 & 0 & 1 \\
a_3 & 0 & 0
\end{array}\rightB \leftB \arraycolsep=5pt \begin{array}{c}
f_1 \\
f_2 \\
f_3
\end{array}\rightB\right.$

\item Show further that if $\leftB \begin{array}{c}
f_1 \\
f_2 \\
f_3
\end{array}\rightB$
 is any solution to this system, then $f = f_{1}$ is a solution to Equation \ref{eq:secondorderdiff}. 
\end{enumerate}
\textit{Remark}. A similar construction casts every linear differential equation of order $n$ (with constant coefficients) as an $n \times n$ linear system of first order equations. However, the matrix need not be diagonalizable, so other methods have been developed.


\end{ex}

\end{multicols}
