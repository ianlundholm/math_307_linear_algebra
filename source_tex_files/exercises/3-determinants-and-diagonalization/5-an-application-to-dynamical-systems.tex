\section*{Exercises for \ref{sec:3_5}}

\begin{Filesave}{solutions}
\solsection{Section~\ref{sec:3_4}}
\end{Filesave}

\begin{multicols}{2}


\begin{ex}
Referring to the model in Example~\ref{exa:008923},
 determine if the population stabilizes, becomes extinct, or becomes 
large in each case. Denote the adult and juvenile survival rates as $A$ and $J$, and the reproduction rate as $R$.
\begin{equation*}
\def\arraystretch{1.5}  \begin{array}{c|ccccc}
& R & & A & & J \\ \cline{2-6}
a. & 2 & & \frac{1}{2} & &\frac{1}{2}\\
b. & 3 & &\frac{1}{4} & &\frac{1}{4}\\
c. & 2 & &\frac{1}{4} & &\frac{1}{3}\\
d. & 3 & &\frac{3}{5} & &\frac{1}{5}
\end{array}
\end{equation*}
\begin{sol}
\begin{enumerate}[label={\alph*.}]
\setcounter{enumi}{1}
\item  $\lambda_{1} = 1$, stabilizes.

\setcounter{enumi}{3}
\item  $\lambda_1 = \frac{1}{24} (3+\sqrt{69}) = 1.13$, diverges.

\end{enumerate}
\end{sol}
\end{ex}

  
\begin{ex}
Consider a linear dynamical system $\vect{v}_{k+1} = A\vect{v}_{k}$ for $k \geq 0$. In each case approximate $\vect{v}_{k}$ using Theorem~\ref{thm:009500}.


\begin{enumerate}[label={\alph*.}]
\item $A = \leftB \begin{array}{rr}
2 & 1 \\
4 & -1 
\end{array}\rightB, \vect{v}_0 = \leftB \begin{array}{r}
1 \\
2
\end{array}\rightB$


\item $A = \leftB \begin{array}{rr}
3 & -2 \\
2 & -2 
\end{array}\rightB, \vect{v}_0 = \leftB \begin{array}{r}
3 \\
-1
\end{array}\rightB$


\item $A = \leftB \begin{array}{rrr}
1 & 0 & 0  \\
1 & 2 & 3 \\
1 & 4 & 1  
\end{array}\rightB, \vect{v}_0 = \leftB \begin{array}{r}
1 \\
1 \\
1
\end{array}\rightB$


\item $A = \leftB \begin{array}{rrr}
1 & 3 & 2  \\
-1 & 2 & 1 \\
4 & -1 & -1  
\end{array}\rightB, \vect{v}_0 = \leftB \begin{array}{r}
2 \\
0 \\
1
\end{array}\rightB$


\end{enumerate}
\begin{sol}
\begin{enumerate}[label={\alph*.}]
\setcounter{enumi}{1}
\item  $V_k = \frac{7}{3} 2^k  \leftB \begin{array}{r}
2 \\
1
\end{array}\rightB$


\setcounter{enumi}{3}
\item  $V_k = \frac{3}{2} 3^k  \leftB \begin{array}{r}
1 \\
0 \\
1
\end{array}\rightB$


\end{enumerate}
\end{sol}
\end{ex}


\begin{ex}
In the model of Example~\ref{exa:008923}, does the final outcome depend on the initial population of adult and juvenile females? Support your answer.
\end{ex}

\begin{ex}
In Example~\ref{exa:008923}, keep the same reproduction rate of 2 and the same adult survival rate of $\frac{1}{2}$, but suppose that the juvenile survival rate is $\rho$. Determine which values of $\rho$ cause the population to become extinct or to become large.
\end{ex}

\begin{ex}
In Example~\ref{exa:008923}, let the juvenile survival rate be $\frac{2}{5}$
 and let the reproduction rate be 2. What values of the adult survival rate $\alpha$ will ensure that the population stabilizes?

\begin{sol}
Extinct if $\alpha < \frac{1}{5}$, stable if $\alpha = \frac{1}{5}$, diverges if $\alpha > \frac{1}{5}$.
\end{sol}
\end{ex}

\end{multicols}








