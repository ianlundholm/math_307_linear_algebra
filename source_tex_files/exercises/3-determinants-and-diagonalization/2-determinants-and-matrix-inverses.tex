\section*{Exercises for \ref{sec:3_2}}

\begin{Filesave}{solutions}
\solsection{Section~\ref{sec:3_2}}
\end{Filesave}

\begin{multicols}{2}
\begin{ex}
Find the adjugate of each of the following matrices.

\begin{exenumerate}
\exitem $\leftB \begin{array}{rrr}
5 & 1 & 3 \\
-1 & 2 & 3 \\
1 & 4 & 8
\end{array}\rightB$
\exitem  $\leftB \begin{array}{rrr}
1 & -1 & 2 \\
3 & 1 & 0 \\
0 & -1 & 1
\end{array}\rightB$
\exitem $\leftB \begin{array}{rrr}
1 & 0 & -1 \\
-1 & 1 & 0 \\
0 & -1 & 1
\end{array}\rightB$
\exitem  $\frac{1}{3}\leftB \begin{array}{rrr}
-1 & 2 & 2 \\
2 & -1 & 2 \\
2 & 2 & -1
\end{array}\rightB$
\end{exenumerate}
\begin{sol}
\begin{enumerate}[label={\alph*.}]
\setcounter{enumi}{1}
\item   $\leftB \begin{array}{rrr}
1 & -1 & -2 \\
-3 & 1 & 6 \\
-3 & 1 & 4
\end{array}\rightB$

\setcounter{enumi}{3}
\item   $\frac{1}{3}\leftB \begin{array}{rrr}
-1 & 2 & 2 \\
2 & -1 & 2 \\
2 & 2 & -1
\end{array}\rightB = A$
\end{enumerate}
\end{sol}
\end{ex}

\begin{ex}
Use determinants to find which real values of $c$ make each of the following matrices invertible.

\begin{exenumerate}
\exitem  $\leftB \begin{array}{rrr}
1 & 0 & 3 \\
3 & -4 & c \\
2 & 5 & 8
\end{array}\rightB$
\exitem  $\leftB \begin{array}{rrr}
0 & c & -c \\
-1 & 2 & 1 \\
c & -c & c
\end{array}\rightB$
\exitem  $\leftB \begin{array}{rrr}
c & 1 & 0 \\
0 & 2 & c \\
-1 & c & 5
\end{array}\rightB$
\exitem  $\leftB \begin{array}{rrr}
4 & c & 3 \\
c & 2 & c \\
5 & c & 4
\end{array}\rightB$
\exitem  $\leftB \begin{array}{rrr}
1 & 2 & -1 \\
0 & -1 & c \\
2 & c & 1
\end{array}\rightB$
\exitem  $\leftB \begin{array}{rrr}
1 & c & -1 \\
c & 1 & 1 \\
0 & 1 & c
\end{array}\rightB$
\end{exenumerate}
\begin{sol}
\begin{enumerate}[label={\alph*.}]
\setcounter{enumi}{1}
\item  $c \neq 0$

\setcounter{enumi}{3}
\item  any $c$

\setcounter{enumi}{5}
\item  $c \neq -1$

\end{enumerate}
\end{sol}
\end{ex}

\begin{ex}
Let $A$, $B$, and $C$ denote $n \times n$ matrices and assume that $\func{det } A = -1$, $\func{det } B = 2$, and $\func{det } C = 3$. Evaluate:

\begin{exenumerate}
\exitem $\func{det}(A^{3}BC^{T}B^{-1})$
\exitem $\func{det}(B^{2}C^{-1}AB^{-1}C^{T})$
\end{exenumerate}
\begin{sol}
\begin{enumerate}[label={\alph*.}]
\setcounter{enumi}{1}
\item  $-2$

\end{enumerate}
\end{sol}
\end{ex}

\begin{ex}
Let $A$ and $B$ be invertible $n \times n$ matrices. Evaluate:

\begin{exenumerate}
\exitem $\func{det}(B^{-1}AB)$
\exitem $\func{det}(A^{-1}B^{-1}AB)$
\end{exenumerate}
\begin{sol}
\begin{enumerate}[label={\alph*.}]
\setcounter{enumi}{1}
\item  $1$

\end{enumerate}
\end{sol}
\end{ex}

\begin{ex}
If $A$ is $3 \times 3$ and $\func{det}(2A^{-1}) = -4$ and $\func{det}(A^{3}(B^{-1})^{T})=-4$, find $\func{det }A$ and $\func{det }B$.
\end{ex}

\begin{ex}
Let $A = \leftB \begin{array}{rrr}
a & b & c \\
p & q & r \\
u & v & w 
\end{array}\rightB$ 
 and assume that $\func{det } A = 3$. Compute:

\begin{enumerate}[label={\alph*.}]
\item $\func{det} (2B^{-1}) \mbox{ where } B = \leftB \begin{array}{rrr}
4u & 2a & -p \\
4v & 2b & -q \\
4w & 2c & -r 
\end{array}\rightB $

\item $\func{det} (2C^{-1}) \mbox{ where } C = \leftB \begin{array}{ccc}
2p & -a+u & 3u \\
2q & -b+v & 3v \\
2r & -c+w & 3w
\end{array}\rightB $

\end{enumerate}
\begin{sol}
\begin{enumerate}[label={\alph*.}]
\setcounter{enumi}{1}
\item $\frac{4}{9}$


\end{enumerate}
\end{sol}
\end{ex}

\begin{ex}
If $\func{det} \leftB \begin{array}{rr}
a & b \\
c& d 
\end{array}\rightB = -2$
 calculate:


\begin{enumerate}[label={\alph*.}]
\item $ \func{det} \leftB \begin{array}{ccc}
2 & -2 & 0 \\
c+1 & -1 & 2a \\
d-2 & 2 & 2b 
\end{array}\rightB$


\item  $ \func{det} \leftB \begin{array}{ccc}
2b & 0 & 4d \\
1 & 2 & -2 \\
a+1 & 2 & 2(c-1) 
\end{array}\rightB$


\item $\func{det}(3A^{-1}) \mbox{ where } A = \leftB \begin{array}{rr}
3c & a+c \\
3d & b+d 
\end{array}\rightB$


\end{enumerate}
\begin{sol}
\begin{enumerate}[label={\alph*.}]
\setcounter{enumi}{1}
\item  $16$

\end{enumerate}
\end{sol}
\end{ex}

\begin{ex}
Solve each of the following by Cramer's rule:

\begin{exenumerate}
\exitem $\arraycolsep=1pt
\begin{array}{rrrrr}
2x & + & y  & = & 1\\
3x & + & 7y & = & -2 \end{array}
$
\exitem $\arraycolsep=1pt
\begin{array}{rrrrr}
3x & + & 4y  & = & 9 \\
2x & - & y & = & -1
\end{array} $
\exitem  $\arraycolsep=1pt
\begin{array}{rrrrrrr}
5x & + & y & - & z & = & -7 \\
2x & - & y & - & 2z & = & 6 \\
3x &  &  & + & 2z & = & -7 
\end{array} $
\exitem $\arraycolsep=1pt
\begin{array}{rrrrrrr}
4x & - & y & + & 3z & = & 1 \\
6x & + & 2y & - & z & = & 0 \\
3x & + & 3y & + & 2z & = & -1 
\end{array} $
\end{exenumerate}
\begin{sol}
\begin{enumerate}[label={\alph*.}]
\setcounter{enumi}{1}
\item  $\frac{1}{11}\leftB \begin{array}{r}
5 \\
21
\end{array}\rightB$


\setcounter{enumi}{3}
\item   $\frac{1}{79}\leftB \begin{array}{r}
12  \\
-37 \\
-2
\end{array}\rightB$


\end{enumerate}
\end{sol}
\end{ex}

\begin{ex}
Use Theorem~\ref{thm:008370} to find the $(2, 3)$-entry of $A^{-1}$ if:

\begin{exenumerate}
\exitem $ A = \leftB \begin{array}{rrr}
3 & 2 & 1 \\
1 & 1 & 2 \\
-1 & 2 & 1 
\end{array}\rightB $ 
\exitem $ A = \leftB \begin{array}{rrr}
1 & 2 & -1 \\
3 & 1 & 1 \\
0 & 4 & 7 
\end{array}\rightB $ 
\end{exenumerate}
\begin{sol}
\begin{enumerate}[label={\alph*.}]
\setcounter{enumi}{1}
\item  $\frac{4}{51}$

\end{enumerate}
\end{sol}
\end{ex}

\begin{ex}
Explain what can be said about $\func{det }A$ if:

\begin{exenumerate}
\exitem $A^{2} = A$
\exitem $A^{2} = I$
\exitem $A^{3} = A$
\exitem $PA = P$ and $P$ is invertible
\exitem $A^{2} = uA$ and $A$ is $n \times n$
\exitem $A = -A^{T}$ and $A$ is $n \times n$
\exitem $A^{2} + I = 0$ and $A$ is $n \times n$
\end{exenumerate}
\begin{sol}
\begin{enumerate}[label={\alph*.}]
\setcounter{enumi}{1}
\item  $\func{det } A = 1, -1$

\setcounter{enumi}{3}
\item $\func{det }A = 1$

\setcounter{enumi}{5}
\item  $\func{det }A = 0$ if $n$ is odd; nothing can be said if $n$ is even

\end{enumerate}
\end{sol}
\end{ex}

\begin{ex}
Let $A$ be $n \times n$. Show that $uA = (uI)A$, and use this with Theorem~\ref{thm:008196} to deduce the result in Theorem \ref{thm:007870}: $\func{det}(uA) = u^{n} \func{det }A$.
\end{ex}

\begin{ex}
If $A$ and $B$ are $n \times n$ matrices, if $AB = -BA$, and if $n$ is odd, show that either $A$ or $B$ has no inverse.
\end{ex}

\begin{ex}
Show that $\func{det }AB = \func{det }BA$ holds for any two $n \times n$ matrices $A$ and $B$.
\end{ex}

\begin{ex}
If $A^{k} = 0$ for some $k \geq 1$, show that $A$ is not invertible.
\end{ex}

\begin{ex}
If $A^{-1} = A^{T}$, describe the cofactor matrix of $A$ in terms of $A$.

\begin{sol}
$dA$ where $d = \func{det }A$
\end{sol}
\end{ex}

\begin{ex}
Show that no $3 \times 3$ matrix $A$ exists such that $A^{2} + I = 0$. Find a $2 \times 2$ matrix $A$ with this property.
\end{ex}

\begin{ex}
Show that $\func{det}(A + B^{T}) = \func{det}(A^{T} + B)$ for any $n \times n$ matrices $A$ and $B$.
\end{ex}

\begin{ex}
Let $A$ and $B$ be invertible $n \times n$ matrices. Show that $\func{det }A = \func{det }B$ if and only if $A = UB$ where $U$ is a matrix with $\func{det }U = 1$.
\end{ex}

\begin{ex}
For each of the matrices in Exercise 2, find the inverse for those values of $c$ for which it exists.

\begin{sol}
\begin{enumerate}[label={\alph*.}]
\setcounter{enumi}{1}
\item $\frac{1}{c} \leftB \begin{array}{rrr}
1 & 0 & 1 \\
0 & c & 1 \\
-1 & c & 1 
\end{array} \rightB, c \neq 0$

\setcounter{enumi}{3}
\item $\frac{1}{2}\leftB \begin{array}{rrr}
8-c^2 & -c & c^2-6 \\
c & 1 & -c \\
c^2-10 & c & 8-c^2 
\end{array} \rightB$


\setcounter{enumi}{5}
\item  $\frac{1}{c^3+1} \leftB \begin{array}{rrr}
1-c & c^2+1 & -c-1 \\
c^2 & -c & c+1 \\
-c & 1 & c^2-1 
\end{array} \rightB, c \neq -1$


\end{enumerate}
\end{sol}
\end{ex}

\begin{ex}
In each case either prove the statement or give an example showing that it is false:


\begin{enumerate}[label={\alph*.}]
\item If $\func{adj }A$ exists, then $A$ is invertible.

\item If $A$ is invertible and $\func{adj }A = A^{-1}$, then $\func{det }A = 1$.

\item $\func{det}(AB) = \func{det}(B^{T}A)$.

\item If $\func{det }A \neq 0$ and $AB = AC$, then $B = C$.

\item If $A^{T} = -A$, then $\func{det } A = -1$.

\item If $\func{adj }A = 0$, then $A = 0$.

\item If $A$ is invertible, then $\func{adj } A$ is invertible.

\item If $A$ has a row of zeros, so also does $\func{adj } A$.

\item $\func{det}(A^{T}A) > 0$ for all square matrices $A$.

\item $\func{det}(I + A) = 1 + \func{det }A$.

\item If $AB$ is invertible, then $A$ and $B$ are invertible.

\item If $\func{det } A = 1$, then $\func{adj }A = A$.

\item If $A$ is invertible and $\func{det} A = d$, then $\func{adj} A = d A^{-1}$. 

\end{enumerate}
\begin{sol}
\begin{enumerate}[label={\alph*.}]
\setcounter{enumi}{1}
\item  T. $\func{det }AB = \func{det }A \func{det }B = \func{det }B \func{det }A = \func{det }BA$.

\setcounter{enumi}{3}
\item  T. $\func{det }A \neq 0$ means $A^{-1}$ exists, so $AB = AC$ implies that $B=C$.

\setcounter{enumi}{5}
\item  F. If $A = \leftB \begin{array}{rrr}
1 & 1 & 1 \\
1 & 1 & 1 \\
1 & 1 & 1
\end{array}\rightB$
 then $\func{adj }A = 0$.

\setcounter{enumi}{7}
\item  F. If  $A = \leftB \begin{array}{rr}
1 & 1 \\
0 & 0 
\end{array}\rightB$
 then $\func{adj } A = \leftB \begin{array}{rr}
0 & -1 \\
0 & 1 
\end{array}\rightB$

\setcounter{enumi}{9}
\item F. If $A = \leftB \begin{array}{rr}
-1 & 1 \\
1 & -1 
\end{array}\rightB$
 then $\func{det}(I + A) = -1$ but $1 + \func{det }A = 1$.

\setcounter{enumi}{11}
\item F. If $A = \leftB \begin{array}{rr}
1 & 1 \\
0 & 1 
\end{array}\rightB$ 
 then $\func{det }A = 1$ but $\func{adj } A = \leftB \begin{array}{rr}
1 & -1 \\
0 & 1 
\end{array}\rightB \neq A$
\end{enumerate}
\end{sol}
\end{ex}

\begin{ex}
If $A$ is $2 \times 2$ and $\func{det }A = 0$, show that one column of $A$ is a scalar multiple of the other. [\textit{Hint}: Definition \ref{def:002668} and Part (2) of Theorem \ref{thm:004553}.]
\end{ex}

\begin{ex}
Find a polynomial $p(x)$ of degree $2$ such that:


\begin{enumerate}[label={\alph*.}]
\item $p(0) = 2$, $p(1) = 3$, $p(3) = 8$

\item $p(0) = 5$, $p(1) = 3$, $p(2) = 5$

\end{enumerate}
\begin{sol}
\begin{enumerate}[label={\alph*.}]
\setcounter{enumi}{1}
\item  $5 - 4x + 2x^{2}$.

\end{enumerate}
\end{sol}
\end{ex}

\begin{ex}
Find a polynomial $p(x)$ of degree $3$ such that:


\begin{enumerate}[label={\alph*.}]
\item $p(0) = p(1) = 1$, $p(-1) = 4$, $p(2) = -5$

\item $p(0) = p(1) = 1$, $p(-1) = 2$, $p(-2) = -3$

\end{enumerate}
\begin{sol}
\begin{enumerate}[label={\alph*.}]
\setcounter{enumi}{1}
\item  $ 1- \frac{5}{3}x  + \frac{1}{2} x ^2 + \frac{7}{6}x^3$

\end{enumerate}
\end{sol}
\end{ex}

\begin{ex}
Given the following data pairs, find the interpolating polynomial of degree at most $3$ and estimate the value of $y$ corresponding to $x = 1.5$.


\begin{enumerate}[label={\alph*.}]
\item $(0, 1)$, $(1, 2)$, $(2, 5)$, $(3, 10)$

\item $(0, 1)$, $(1, 1.49)$, $(2, -0.42)$, $(3, -11.33)$

\item $(0, 2)$, $(1, 2.03)$, $(2, -0.40)$, $(-1, 0.89)$

\end{enumerate}
\begin{sol}
\begin{enumerate}[label={\alph*.}]
\setcounter{enumi}{1}
\item  $1 - 0.51 x + 2.1 x^2 - 1.1 x^3; 1.25$, so  $y = 1.25$


\end{enumerate}
\end{sol}
\end{ex}

\begin{ex}
If $A = \leftB \begin{array}{rrr}
1 & a & b \\
-a & 1 & c \\
-b & -c & 1 
\end{array} \rightB$
 show that $\func{det }A = 1 + a^{2} + b^{2} + c^{2}$. Hence, find $A^{-1}$ for any $a$, $b$, and $c$.
\end{ex}

\begin{ex}
\begin{enumerate}[label={\alph*.}]
\item Show that $A = \leftB \begin{array}{rrr}
a & p & q \\
0 & b & r \\
0 & 0 & c 
\end{array} \rightB$
 has an inverse if and only if $abc \neq 0$, and find $A^{-1}$ in that case.

\item Show that if an upper triangular matrix is invertible, the inverse is also upper triangular.

\end{enumerate}
\begin{sol}
\begin{enumerate}[label={\alph*.}]
\setcounter{enumi}{1}
\item  Use induction on $n$ where $A$ is $n \times n$. It is clear if $n = 1$. If $n > 1$, write $A = \leftB \begin{array}{cc}
a & X \\
0 & B
\end{array}\rightB$
 in block form where $B$ is $(n - 1) \times (n - 1)$. Then $A^{-1} = \leftB \begin{array}{cc}
a^{-1} & -a^{-1}XB^{-1} \\
0 & B^{-1}
\end{array}\rightB$, and this is upper triangular because $B$ is upper triangular by induction.

\end{enumerate}
\end{sol}
\end{ex}

\begin{ex}
Let $A$ be a matrix each of whose entries are integers. Show that each of the following conditions implies the other.


\begin{enumerate}
\item $A$ is invertible and $A^{-1}$ has integer entries.

\item $\func{det }A = 1$ or $-1$.

\end{enumerate}
\end{ex}

\begin{ex}
If $A^{-1} = \leftB \begin{array}{rrr}
3 & 0 & 1 \\ 
0 & 2 & 3 \\
3 & 1 & -1 
\end{array}\rightB$
 find $\func{adj }A$.

\begin{sol}
$-\frac{1}{21}\leftB\begin{array}{rrr}
3 & 0 & 1 \\
0 & 2 & 3\\
3 & 1 & -1 
\end{array}\rightB$
\end{sol}
\end{ex}

\begin{ex}
If $A$ is $3 \times 3$ and $\func{det }A = 2$, find $\func{det}(A^{-1} + 4 \func{adj }A)$.
\end{ex}

\begin{ex}
Show that $\func{det} \leftB \begin{array}{rr}
0 & A \\
B & X 
\end{array}\rightB = \func{det }A \func{det }B$ when $A$ and $B$ are $2 \times 2$. What if $A$ and $B$ are $3 \times 3$?


[\textit{Hint}: Block multiply by $ \leftB \begin{array}{rr}
0 & I \\
I & 0 
\end{array}\rightB$.]
\end{ex}

\begin{ex}
Let $A$ be $n \times n$, $n \geq 2$, and assume one column of $A$ consists of zeros. Find the possible values of $\func{rank}(\func{adj }A)$.
\end{ex}

\begin{ex}
If $A$ is $3 \times 3$ and invertible, compute $\func{det}(-A^{2}(\func{adj }A)^{-1})$.
\end{ex}


\begin{ex}
Show that $\func{adj}(uA) = u^{n-1} \func{adj }A$ for all $n \times n$ matrices $A$.
\end{ex}

\begin{ex}
Let $A$ and $B$ denote invertible $n \times n$ matrices. Show that:


\begin{enumerate}[label={\alph*.}]
\item $\func{adj}(\func{adj }A) = (\func{det }A)^{n-2}A$ (here $n \geq 2$) [\textit{Hint}: See Example~\ref{exa:008396}.]

\item $\func{adj}(A^{-1}) = (\func{adj }A)^{-1}$

\item $\func{adj}(A^{T}) = (\func{adj }A)^T$

\item $\func{adj}(AB) = (\func{adj }B)(\func{adj }A)$ [\textit{Hint}: Show that $AB \func{adj}(AB) = AB \func{adj }B \func{adj }A$.]

\end{enumerate}
\begin{sol}
\begin{enumerate}[label={\alph*.}]
\setcounter{enumi}{1}
\item  Have $(\func{adj }A)A = (\func{det }A)I$; so taking inverses, $A^{-1} \cdot (\func{adj } A)^{-1} = \frac{1}{\func{det } A}I$.
 On the other hand, $A^{-1} \func{adj} (A^{-1}) = \func{det}(A^{-1})I = \frac{1}{\func{det } A}I$.
 Comparison yields $A^{-1}(\func{adj }A)^{-1} = A^{-1}\func{adj}(A^{-1})$, and part \textbf{(b)} follows.

\setcounter{enumi}{3}
\item  Write $\func{det} A = d$, $\func{det } B = e$. By the adjugate formula $AB \func{adj}(AB) = deI$, and $AB \func{adj }B \func{adj }A = A[eI] \func{adj} A = (eI)(dI) = deI$. Done as $AB$ is invertible.

\end{enumerate}
\end{sol}
\end{ex}

\end{multicols}




