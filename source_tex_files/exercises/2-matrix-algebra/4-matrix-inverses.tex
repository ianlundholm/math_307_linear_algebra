\section*{Exercises for \ref{sec:2_4}}

\begin{Filesave}{solutions}
\solsection{Section~\ref{sec:2_4}}
\end{Filesave}

\begin{multicols}{2}
\begin{ex}
In each case, show that the matrices are inverses of each other.

\begin{enumerate}[label={\alph*.}]
\item $\leftB \begin{array}{rr}
3 & 5 \\
1 & 2
\end{array} \rightB$, $\leftB \begin{array}{rr}
2 & -5 \\
-1 & 3
\end{array} \rightB$

\item $\leftB \begin{array}{rr}
3 & 0 \\
1 & -4
\end{array} \rightB$, $\frac{1}{2}\leftB \begin{array}{rr}
4 & 0 \\
1 & -3
\end{array} \rightB$

\item $\leftB \begin{array}{rrr}
1 & 2 & 0 \\
0 & 2 & 3 \\
1 & 3 & 1
\end{array} \rightB$, $\leftB \begin{array}{rrr}
7 & 2 & -6 \\
-3 & -1 & 3 \\
2 & 1 & -2
\end{array} \rightB$

\item $\leftB \begin{array}{rr}
3 & 0 \\
0 & 5
\end{array} \rightB$, $\leftB \begin{array}{rr}
\frac{1}{3} & 0 \\
0 & \frac{1}{5}
\end{array} \rightB$


\end{enumerate}
\end{ex}

\begin{ex}
Find the inverse of each of the following matrices.
\begin{exenumerate}[column-sep=-3em]
\exitem $\leftB \begin{array}{rr}
1 & -1 \\
-1 & 3
\end{array} \rightB$
\exitem $\leftB \begin{array}{rr}
4 & 1 \\
3 & 2
\end{array} \rightB$
\exitem $\leftB \begin{array}{rrr}
1 & 0 & -1 \\
3 & 2 & 0 \\
-1 & -1 & 0
\end{array} \rightB$
\exitem $\leftB \begin{array}{rrr}
1 & -1 & 2 \\
-5 & 7 & -11 \\
-2 & 3 & -5
\end{array} \rightB$
\exitem $\leftB \begin{array}{rrr}
3 & 5 & 0 \\
3 & 7 & 1 \\
1 & 2 & 1
\end{array} \rightB$
\exitem $\leftB \begin{array}{rrr}
3 & 1 & -1 \\
2 & 1 & 0 \\
1 & 5 & -1
\end{array} \rightB$
\exitem $\leftB \begin{array}{rrr}
2 & 4 & 1 \\
3 & 3 & 2 \\
4 & 1 & 4
\end{array} \rightB$
\exitem $\leftB \begin{array}{rrr}
3 & 1 & -1 \\
5 & 2 & 0 \\
1 & 1 & -1
\end{array} \rightB$
\exitem $\leftB \begin{array}{rrr}
3 & 1 & 2 \\
1 & -1 & 3 \\
1 & 2 & 4
\end{array} \rightB$
\exitem $\leftB \begin{array}{rrrr}
-1 & 4 & 5 & 2 \\
0 & 0 & 0 & -1 \\
1 & -2 & -2 & 0 \\
0 & -1 & -1 & 0
\end{array} \rightB$
\exitem $\leftB \begin{array}{rrrr}
1 & 0 & 7 & 5 \\
0 & 1 & 3 & 6 \\
1 & -1 & 5 & 2 \\
1 & -1 & 5 & 1
\end{array} \rightB$
\exitem $\leftB \begin{array}{rrrrr}
1 & 2 & 0 & 0 & 0 \\
0 & 1 & 3 & 0 & 0 \\
0 & 0 & 1 & 5 & 0 \\
0 & 0 & 0 & 1 & 7 \\
0 & 0 & 0 & 0 & 1
\end{array} \rightB$
\end{exenumerate}
\begin{sol}
\begin{enumerate}[label={\alph*.}]
\setcounter{enumi}{1}
\item $\frac{1}{5} \leftB \begin{array}{rr}
2 & -1 \\
-3 & 4
\end{array} \rightB$

\setcounter{enumi}{3}
\item $\leftB \begin{array}{rrr}
2 & -1 & 3 \\
3 & 1 & -1 \\
1 & 1 & -2
\end{array} \rightB$

\setcounter{enumi}{5}
\item $\frac{1}{10} \leftB \begin{array}{rrr}
1 & 4 & -1 \\
-2 & 2 & 2 \\
-9 & 14 & -1
\end{array} \rightB$

\setcounter{enumi}{7}
\item $\frac{1}{4} \leftB \begin{array}{rrr}
2 & 0 & -2 \\
-5 & 2 & 5 \\
-3 & 2 & -1
\end{array} \rightB$

\setcounter{enumi}{9}
\item $\leftB \begin{array}{rrrr}
0 & 0 & 1 & -2 \\
-1 & -2 & -1 & -3 \\
1 & 2 & 1 & 2 \\
0 & -1 & 0 & 0
\end{array} \rightB$

\setcounter{enumi}{11}
\item $\leftB \begin{array}{rrrrr}
1 & -2 & 6 & -30 & 210 \\
0 & 1 & -3 & 15 & -105 \\
0 & 0 & 1 & -5 & 35 \\
0 & 0 & 0 & 1 & -7 \\
0 & 0 & 0 & 0 & 1
\end{array} \rightB$

\end{enumerate}
\end{sol}
\end{ex}

\begin{ex}
In each case, solve the systems of equations by finding the inverse of the coefficient matrix.
\begin{exenumerate}
\exitem $\arraycolsep=1pt\begin{array}[t]{rrrrr}
3x & - & y & = & 5 \\
2x & + & 2y & = & 1
\end{array}$
\exitem $\arraycolsep=1pt\begin{array}[t]{rrrrr}
2x & - & 3y & = & 0 \\
x & - & 4y & = & 1
\end{array}$
\exitem $\arraycolsep=1pt\begin{array}[t]{rrrrrrr}
x & + & y & + & 2z & = & 5 \\
x & + & y & + & z & = & 0 \\
x & + & 2y & + & 4z & = & -2
\end{array}$
\exitem $\arraycolsep=1pt\begin{array}[t]{rrrrrrr}
x & + & 4y & + & 2z & = & 1 \\
2x & + & 3y & + & 3z & = & -1 \\
4x & + & y & + & 4z & = & 0
\end{array}$
\end{exenumerate}
\begin{sol}
\begin{enumerate}[label={\alph*.}]
\setcounter{enumi}{1}
\item $\leftB \begin{array}{c}
x \\
y
\end{array} \rightB = \frac{1}{5} \leftB \begin{array}{rr}
4 & -3 \\
1 & -2
\end{array} \rightB \leftB \begin{array}{r}
0 \\
1
\end{array} \rightB = \frac{1}{5} \leftB \begin{array}{r}
-3 \\
-2
\end{array} \rightB$

\setcounter{enumi}{3}
\item $\leftB \begin{array}{c}
x \\
y \\
z
\end{array} \rightB = \frac{1}{5} \leftB \begin{array}{rrr}
9 & -14 & 6 \\
4 & -4 & 1 \\
-10 & 15 & -5
\end{array} \rightB \leftB \begin{array}{r}
1 \\
-1 \\
0
\end{array} \rightB = \frac{1}{5} \leftB \begin{array}{r}
23 \\
8 \\
-25
\end{array} \rightB$

\end{enumerate}
\end{sol}
\end{ex}

\begin{ex}
Given $A^{-1} = \leftB \begin{array}{rrr}
1 & -1 & 3 \\
2 & 0 & 5 \\
-1 & 1 & 0
\end{array} \rightB$:

\begin{enumerate}[label={\alph*.}]
\item Solve the system of equations $A\vect{x} = \leftB \begin{array}{r}
1 \\
-1 \\
3
\end{array} \rightB$.

\item Find a matrix $B$ such that \\ $AB = \leftB \begin{array}{rrr}
1 & -1 & 2 \\
0 & 1 & 1 \\
1 & 0 & 0
\end{array} \rightB$.

\item Find a matrix $C$ such that \\ $CA = \leftB \begin{array}{rrr}
1 & 2 & -1 \\
3 & 1 & 1
\end{array} \rightB$.

\end{enumerate}
\begin{sol}
\begin{enumerate}[label={\alph*.}]
\setcounter{enumi}{1}
\item $B = A^{-1}AB = \leftB \begin{array}{rrr}
4 & -2 & 1 \\
7 & -2 & 4 \\
-1 & 2 & -1
\end{array} \rightB$

\end{enumerate}
\end{sol}
\end{ex}

\begin{ex}
Find $A$ when
\begin{exenumerate}
\exitem $(3A)^{-1} = \leftB \begin{array}{rr}
1 & -1 \\
0 & 1
\end{array} \rightB$
\exitem $(2A)^{T} = \leftB \begin{array}{rr}
1 & -1 \\
2 & 3
\end{array} \rightB^{-1}$
\exitem* $(I + 3A)^{-1} = \leftB \begin{array}{rr}
2 & 0 \\
1 & -1
\end{array} \rightB$
\exitem* $(I - 2A^{T})^{-1} = \leftB \begin{array}{rr}
2 & 1 \\
1 & 1
\end{array} \rightB$
\exitem* $\left(A \leftB \begin{array}{rr}
1 & -1 \\
0 & 1
\end{array} \rightB \right)^{-1} = \leftB \begin{array}{rr}
2 & 3 \\
1 & 1
\end{array} \rightB$
\exitem* $\left(\leftB \begin{array}{rr}
1 & 0 \\
2 & 1
\end{array} \rightB A\right)^{-1} = \leftB \begin{array}{rr}
1 & 0 \\
2 & 2
\end{array} \rightB$
\exitem* $\left(A^{T} -2I \right)^{-1} = 2 \leftB \begin{array}{rr}
1 & 1 \\
2 & 3
\end{array} \rightB$
\exitem* $\left(A^{-1} -2I \right)^{T} = -2 \leftB \begin{array}{rr}
1 & 1 \\
1 & 0
\end{array} \rightB$
\end{exenumerate}
\begin{sol}
\begin{enumerate}[label={\alph*.}]
\setcounter{enumi}{1}
\item $\frac{1}{10} \leftB \begin{array}{rr}
3 & -2 \\
1 & 1
\end{array} \rightB$

\setcounter{enumi}{3}
\item $\frac{1}{2} \leftB \begin{array}{rr}
0 & 1 \\
1 & -1
\end{array} \rightB$

\setcounter{enumi}{5}
\item $\frac{1}{2} \leftB \begin{array}{rr}
2 & 0 \\
-6 & 1
\end{array} \rightB$

\setcounter{enumi}{7}
\item $-\frac{1}{2} \leftB \begin{array}{rr}
1 & 1 \\
1 & 0
\end{array} \rightB$

\end{enumerate}
\end{sol}
\end{ex}

\begin{ex}
Find $A$ when:
\begin{exenumerate}
\exitem $A^{-1} = \leftB \begin{array}{rrr}
1 & -1 & 3 \\
2 & 1 & 1 \\
0 & 2 & -2
\end{array} \rightB$
\exitem $A^{-1} = \leftB \begin{array}{rrr}
0 & 1 & -1 \\
1 & 2 & 1 \\
1 & 0 & 1
\end{array} \rightB$
\end{exenumerate}
\begin{sol}
\begin{enumerate}[label={\alph*.}]
\setcounter{enumi}{1}
\item $A = \frac{1}{2} \leftB \begin{array}{rrr}
2 & -1 & 3 \\
0 & 1 & -1 \\
-2 & 1 & -1
\end{array} \rightB$

\end{enumerate}
\end{sol}
\end{ex}

\begin{ex}
Given $\leftB \begin{array}{c}
x_{1} \\
x_{2} \\
x_{3}
\end{array} \rightB = \leftB \begin{array}{rrr}
3 & -1 & 2 \\
1 & 0 & 4 \\
2 & 1 & 0
\end{array} \rightB \leftB \begin{array}{c}
y_{1} \\
y_{2} \\
y_{3}
\end{array} \rightB$
 and $\leftB \begin{array}{c}
 z_{1} \\
 z_{2} \\
 z_{3}
 \end{array} \rightB = \leftB \begin{array}{rrr}
 1 & -1 & 1 \\
 2 & -3 & 0 \\
 -1 & 1 & -2
 \end{array} \rightB \leftB \begin{array}{c}
 y_{1} \\
 y_{2} \\
 y_{3}
 \end{array} \rightB$, express the variables $x_{1}$, $x_{2}$, and $x_{3}$ in terms of $z_{1}$, $z_{2}$, and $z_{3}$.
\end{ex}

\begin{ex}
\begin{enumerate}[label={\alph*.}]
\item In the system $\arraycolsep=1pt\begin{array}{rrrrr}
3x & + & 4y & = & 7 \\
4x & + & 5y & = & 1
\end{array}$, substitute the new variables $x^\prime$ and $y^\prime$ given by $\arraycolsep=1pt\begin{array}{rrrrr}
 x & = & -5x^\prime & + & 4y^\prime \\
 y & = & 4x^\prime & - & 3y^\prime
 \end{array}$. Then find $x$ and $y$.

\item Explain part (a) by writing the equations as $A \leftB \begin{array}{c}
x \\
y
\end{array} \rightB = \leftB \begin{array}{r}
7 \\
1
\end{array} \rightB$
 and $\leftB \begin{array}{r}
 x \\
 y
 \end{array} \rightB = B \leftB \begin{array}{c}
 x^\prime \\
 y^\prime
 \end{array} \rightB$. What is the relationship between $A$ and $B$?

\end{enumerate}
\begin{sol}
\begin{enumerate}[label={\alph*.}]
\setcounter{enumi}{1}
\item  $A$ and $B$ are inverses.

\end{enumerate}
\end{sol}
\end{ex}

\begin{ex}
In each case either prove the assertion or give an example showing that it is false.

\begin{enumerate}[label={\alph*.}]
\item If $A \neq 0$ is a square matrix, then $A$ is invertible.

\item If $A$ and $B$ are both invertible, then $A + B$ is invertible.

\item If $A$ and $B$ are both invertible, then $(A^{-1}B)^{T}$ is invertible.

\item If $A^{4} = 3I$, then $A$ is invertible.

\item If $A^{2} = A$ and $A \neq 0$, then $A$ is invertible.

\item If $AB = B$ for some $B \neq 0$, then $A$ is invertible.

\item If $A$ is invertible and skew symmetric ($A^{T} = -A$), the same is true of $A^{-1}$.

\item If $A^{2}$ is invertible, then $A$ is invertible.

\item If $AB = I$, then $A$ and $B$ commute.

\end{enumerate}
\begin{sol}
\begin{enumerate}[label={\alph*.}]
\setcounter{enumi}{1}
\item  False. $\leftB \begin{array}{rr}
1 & 0 \\
0 & 1
\end{array} \rightB + \leftB \begin{array}{rr}
1 & 0 \\
0 & -1
\end{array} \rightB$

\setcounter{enumi}{3}
\item  True. $A^{-1} = \frac{1}{3}A^{3}$

\setcounter{enumi}{5}
\item  False. $A = B = \leftB \begin{array}{rr}
1 & 0 \\
0 & 0
\end{array} \rightB$

\setcounter{enumi}{7}
\item  True. If $(A^{2})B = I$, then $A(AB) = I$; use Theorem~\ref{thm:004553}.

\end{enumerate}
\end{sol}
\end{ex}

\begin{ex} \label{ex:ex2_4_10}
\begin{enumerate}[label={\alph*.}]
\item If $A$, $B$, and $C$ are square matrices and $AB = I$, $ I = CA$, show that $A$ is invertible and $B = C = A^{-1}$.

\item If $C^{-1} = A$, find the inverse of $C^{T}$ in terms of $A$.

\end{enumerate}
\begin{sol}
\begin{enumerate}[label={\alph*.}]
\setcounter{enumi}{1}
\item  $(C^{T})^{-1} = (C^{-1})^{T} = A^{T}$ because $C^{-1} = (A^{-1})^{-1} = A$.

\end{enumerate}
\end{sol}
\end{ex}

\begin{ex}
Suppose $CA = I_{m}$, where $C$ is $m \times n$ and $A$ is $n \times m$. Consider the system $A\vect{x} = \vect{b}$ of $n$ equations in $m$ variables.


\begin{enumerate}[label={\alph*.}]
\item Show that this system has a unique solution $CB$ if it is consistent.

\item If $C = \leftB \begin{array}{rrr}
0 & -5 & 1 \\
3 & 0 & -1
\end{array} \rightB$
 and $A = \leftB \begin{array}{rr}
 2 & -3 \\
 1 & -2 \\
 6 & -10
 \end{array} \rightB$, find $\vect{x}$ (if it exists) when 
\newline (i) $\vect{b} = \leftB \begin{array}{r}
 1 \\
 0 \\
 3
 \end{array} \rightB$; and (ii) $\vect{b} = \leftB \begin{array}{r}
 7 \\
 4 \\
 22
 \end{array} \rightB$.

\end{enumerate}
\begin{sol}
\begin{enumerate}[label={\alph*.}]

\setcounter{enumi}{1}
\item 
(i) Inconsistent.

(ii) $\leftB \begin{array}{c}
x_{1} \\
x_{2}
\end{array} \rightB = \leftB \begin{array}{r}
2 \\
-1
\end{array} \rightB$
\end{enumerate}
\end{sol}
\end{ex}

\begin{ex}
Verify that $A = \leftB \begin{array}{rr}
1 & -1 \\
0 & 2
\end{array} \rightB$
 satisfies $A^{2} - 3A + 2I = 0$, and use this fact to show that \\$A^{-1} = \frac{1}{2}(3I - A)$.
\end{ex}

\begin{ex}
Let $Q = \leftB \begin{array}{rrrr}
a & -b & -c & -d \\
b & a & -d & c \\
c & d & a & -b \\
d & -c & b & a
\end{array} \rightB$. Compute $QQ^{T}$ and so find $Q^{-1}$ if $Q \neq 0$.
\end{ex}

\begin{ex}
Let $U = \leftB \begin{array}{rr}
0 & 1 \\
1 & 0
\end{array} \rightB$. Show that each of $U$, $-U$, and $-I_{2}$ is its own inverse and that the product of any two of these is the third.
\end{ex}

\begin{ex}
Consider $A = \leftB \begin{array}{rr}
1 & 1 \\
-1 & 0
\end{array} \rightB$, \\ $B = \leftB \begin{array}{rr}
0 & -1 \\
1 & 0
\end{array} \rightB$, $C = \leftB \begin{array}{rrr}
0 & 1 & 0 \\
0 & 0 & 1 \\
5 & 0 & 0
\end{array} \rightB$. Find the inverses by computing (a) $A^{6}$; (b) $B^{4}$; and (c) $C^{3}$.

\begin{sol}
\begin{enumerate}[label={\alph*.}]
\setcounter{enumi}{1}
\item $B^{4} = I$, so $B^{-1} = B^{3} = \leftB \begin{array}{rr}
0 & 1 \\
-1 & 0
\end{array} \rightB$

\end{enumerate}
\end{sol}
\end{ex}

\begin{ex}
Find the inverse of $\leftB \begin{array}{rrr}
1 & 0 & 1 \\
c & 1 & c \\
3 & c & 2
\end{array} \rightB$
 in terms of $c$.

\begin{sol}
$\leftB \begin{array}{crr}
c^{2} - 2 & -c & 1 \\
-c & 1 & 0 \\
3 - c^{2} & c & -1
\end{array} \rightB$
\end{sol}
\end{ex}

\begin{ex}
If $c \neq 0$, find the inverse of $\leftB \begin{array}{rrr}
1 & -1 & 1 \\
2 & -1 & 2 \\
0 & 2 & c
\end{array} \rightB$
 in terms of $c$.
\end{ex}

\begin{ex}
Show that $A$ has no inverse when:


\begin{enumerate}[label={\alph*.}]
\item $A$ has a row of zeros.

\item $A$ has a column of zeros.

\item each row of $A$ sums to $0$. \newline [\textit{Hint}: Theorem~\ref{thm:004553}(2).]

\item each column of $A$ sums to $0$.


[\textit{Hint}: Corollary~\ref{cor:004537}, Theorem~\ref{thm:004442}.]

\end{enumerate}
\begin{sol}
\begin{enumerate}[label={\alph*.}]
\setcounter{enumi}{1}
\item  If column $j$ of $A$ is zero, $A\vect{y} = \vect{0}$ where $\vect{y}$ is column $j$ of the identity matrix. Use Theorem~\ref{thm:004553}.

\setcounter{enumi}{3}
\item  If each column of $A$ sums to $0$, $XA = 0$ where $X$ is the row of $1$s. Hence $A^{T}X^{T} = 0$ so $A$ has no inverse by Theorem~\ref{thm:004553} ($X^{T} \neq 0$).

\end{enumerate}
\end{sol}
\end{ex}

\begin{ex}
Let $A$ denote a square matrix.


\begin{enumerate}[label={\alph*.}]
\item Let $YA = 0$ for some matrix $Y \neq 0$. Show that $A$ has no inverse. [\textit{Hint}: Corollary~\ref{cor:004537}, Theorem~\ref{thm:004442}.]

\item Use part (a) to show that (i) $\leftB \begin{array}{rrr}
1 & -1 & 1 \\
0 & 1 & 1 \\
1 & 0 & 2
\end{array} \rightB$; and (ii) $\leftB \begin{array}{rrr}
 2 & 1 & -1 \\
 1 & 1 & 0 \\
 1 & 0 & -1
 \end{array} \rightB$
 have no inverse.


[\textit{Hint}: For part (ii) compare row 3 with the difference between row 1 and row 2.]

\end{enumerate}
\begin{sol}
\begin{enumerate}[label={\alph*.}]
\setcounter{enumi}{1}
\item (ii) $(-1, 1, 1)A = 0$

\end{enumerate}
\end{sol}
\end{ex}

\begin{ex}
If $A$ is invertible, show that
\begin{exenumerate}[column-sep=-5em]
\exitem $A^{2} \neq 0$.
\exitem $A^{k} \neq 0$ for all \\$k = 1, 2, \dots$.
\end{exenumerate}
\begin{sol}
\begin{enumerate}[label={\alph*.}]
\setcounter{enumi}{1}
\item Each power $A^{k}$ is invertible by Theorem~\ref{thm:004442} (because $A$ is invertible). Hence $A^{k}$ cannot be $0$.

\end{enumerate}
\end{sol}
\end{ex}

\begin{ex}
Suppose $AB = 0$, where $A$ and $B$ are square matrices. Show that:


\begin{enumerate}[label={\alph*.}]
\item If one of $A$ and $B$ has an inverse, the other is zero.

\item It is impossible for both $A$ and $B$ to have inverses.

\item $(BA)^{2} = 0$.

\end{enumerate}
\begin{sol}
\begin{enumerate}[label={\alph*.}]
\setcounter{enumi}{1}
\item  By (a), if one has an inverse the other is zero and so has no inverse.

\end{enumerate}
\end{sol}
\end{ex}

\begin{ex}
Find the inverse of the $x$-expansion in Example~\ref{exa:003128} and describe it geometrically.

\begin{sol}
If $A = \leftB \begin{array}{rr}
a & 0 \\
0 & 1
\end{array} \rightB$, $a > 1$, then $A^{-1} = \leftB \begin{array}{rr}
 \frac{1}{a} & 0 \\
 0 & 1
 \end{array} \rightB$
 is an x-compression because $\frac{1}{a} < 1$.
\end{sol}
\end{ex}

\begin{ex}
Find the inverse of the shear transformation in Example \ref{exa:003136} and describe it geometrically.
\end{ex}


\begin{ex}
In each case assume that $A$ is a square matrix that satisfies the given condition. Show that $A$ is invertible and find a formula for $A^{-1}$ in terms of $A$.


\begin{enumerate}[label={\alph*.}]
\item $A^{3} - 3A + 2I = 0$.

\item $A^{4} + 2A^{3} - A - 4I = 0$.

\end{enumerate}
\begin{sol}
\begin{enumerate}[label={\alph*.}]
\setcounter{enumi}{1}
\item $A^{-1} = \frac{1}{4} (A^3 +2A^2-1)$
\end{enumerate}
\end{sol}
\end{ex}

\begin{ex}
Let $A$ and $B$ denote $n \times n$ matrices.


\begin{enumerate}[label={\alph*.}]
\item If $A$ and $AB$ are invertible, show that $B$ is invertible using only (2) and (3) of Theorem~\ref{thm:004442}.

\item If $AB$ is invertible, show that both $A$ and $B$ are invertible using Theorem~\ref{thm:004553}.

\end{enumerate}
\begin{sol}
\begin{enumerate}[label={\alph*.}]
\setcounter{enumi}{1}
\item  If $B\vect{x} = \vect{0}$, then $(AB)\vect{x} = (A)B\vect{x} = \vect{0}$, so $\vect{x} = \vect{0}$ because $AB$ is invertible. Hence $B$ is invertible by Theorem~\ref{thm:004553}. But then $A = (AB)B^{-1}$ is invertible by Theorem~\ref{thm:004442}.

\end{enumerate}
\end{sol}
\end{ex}

\begin{ex}
In each case find the inverse of the matrix $A$ using Example~\ref{exa:004627}.
\begin{exenumerate}
\exitem $A = \leftB \begin{array}{rrr}
-1 & 1 & 2 \\
0 & 2 & -1 \\
0 & 1 & -1
\end{array} \rightB$
\exitem $A = \leftB \begin{array}{rrr}
3 & 1 & 0 \\
5 & 2 & 0 \\
1 & 3 & -1
\end{array} \rightB$
\exitem* $A = \leftB \begin{array}{rrrr}
3 & 4 & 0 & 0 \\
2 & 3 & 0 & 0 \\
1 & -1 & 1 & 3 \\
3 & 1 & 1 & 4 
\end{array} \rightB$
\exitem* $A = \leftB \begin{array}{rrrr}
2 & 1 & 5 & 2 \\
1 & 1 & -1 & 0 \\
0 & 0 & 1 & -1 \\
0 & 0 & 1 & -2 
\end{array} \rightB$
\end{exenumerate}
\begin{sol}
\begin{enumerate}[label={\alph*.}]
\setcounter{enumi}{1}
\item $\leftB \begin{array}{rr|r}
2 & -1 & 0 \\
-5 & 3 & 0 \\
\hline
-13 & 8 & -1
\end{array} \rightB$

\setcounter{enumi}{3}
\item $\leftB \begin{array}{rr|rr}
1 & -1 & -14 & 8 \\
-1 & 2 & 16 & -9 \\
\hline
0 & 0 & 2 & -1 \\
0 & 0 & 1 & -1
\end{array} \rightB$

\end{enumerate}
\end{sol}
\end{ex}

\begin{ex}
If $A$ and $B$ are invertible symmetric matrices such that $AB = BA$, show that $A^{-1}$, $AB$, $AB^{-1}$, and $A^{-1}B^{-1}$ are also invertible and symmetric.
\end{ex}

\begin{ex}
Let $A$ be an $n \times n$ matrix and let $I$ be the $n \times n$ identity matrix.


\begin{enumerate}[label={\alph*.}]
\item If $A^{2} = 0$, verify that $(I - A)^{-1} = I + A$.

\item If $A^{3} = 0$, verify that $(I - A)^{-1} = I + A + A^{2}$.

\item Find the inverse of $\leftB \begin{array}{rrr}
1 & 2 & -1 \\
0 & 1 & 3 \\
0 & 0 & 1
\end{array} \rightB$.

\item If $A^{n} = 0$, find the formula for $(I - A)^{-1}$.

\end{enumerate}
\begin{sol}
\begin{enumerate}[label={\alph*.}]
\setcounter{enumi}{3}
\item If $A^{n} = 0$, $(I - A)^{-1} = I + A + \cdots + A^{n-1}$.

\end{enumerate}
\end{sol}
\end{ex}

\begin{ex} \label{ex:ex2_4_29}
Prove property 6 of Theorem~\ref{thm:004442}: If $A$ is invertible and $a \neq 0$, then $aA$ is invertible and $(aA)^{-1} = \frac{1}{a}A^{-1}$
\end{ex}

\begin{ex}
Let $A$, $B$, and $C$ denote $n \times n$ matrices. Using only Theorem~\ref{thm:004442}, show that:


\begin{enumerate}[label={\alph*.}]
\item If $A$, $C$, and $ABC$ are all invertible, $B$ is invertible.

\item If $AB$ and $BA$ are both invertible, $A$ and $B$ are both invertible.

\end{enumerate}
\begin{sol}
\begin{enumerate}[label={\alph*.}]
\setcounter{enumi}{1}
\item  $A[B(AB)^{-1}] = I = [(BA)^{-1}B]A$, so $A$ is invertible by Exercise~\ref{ex:ex2_4_10}.


\end{enumerate}
\end{sol}
\end{ex}

\begin{ex}
Let $A$ and $B$ denote invertible $n \times n$ matrices.


\begin{enumerate}[label={\alph*.}]
\item If $A^{-1} = B^{-1}$, does it mean that $A = B$? Explain.

\item Show that $A = B$ if and only if $A^{-1}B = I$.

\end{enumerate}
\end{ex}

\begin{ex}
Let $A$, $B$, and $C$ be $n \times n$ matrices, with $A$ and $B$ invertible. Show that


\begin{enumerate}[label={\alph*.}]
\item If $A$ commutes with $C$, then $A^{-1}$ commutes with $C$.

\item If $A$ commutes with $B$, then $A^{-1}$ commutes with $B^{-1}$.

\end{enumerate}
\begin{sol}
\begin{enumerate}[label={\alph*.}]
\item  Have $AC = CA$. Left-multiply by $A^{-1}$ to get $C = A^{-1}CA$. Then right-multiply by $A^{-1}$ to get $CA^{-1} = A^{-1}C$.

\end{enumerate}
\end{sol}
\end{ex}

\begin{ex}
Let $A$ and $B$ be square matrices of the same size.


\begin{enumerate}[label={\alph*.}]
\item Show that $(AB)^{2} = A^{2}B^{2}$ if $AB = BA$.

\item If $A$ and $B$ are invertible and $(AB)^{2} = A^{2}B^{2}$, show that $AB = BA$.

\item If $A = \leftB \begin{array}{rr}
1 & 0 \\
0 & 0
\end{array} \rightB$
 and $B = \leftB \begin{array}{rr}
 1 & 1 \\
 0 & 0
 \end{array} \rightB$, show that $(AB)^{2} = A^{2}B^{2}$ but $AB \neq BA$.

\end{enumerate}
\begin{sol}
\begin{enumerate}[label={\alph*.}]
\setcounter{enumi}{1}
\item  Given $ABAB = AABB$. Left multiply by $A^{-1}$, then right multiply by $B^{-1}$.

\end{enumerate}
\end{sol}
\end{ex}

\begin{ex}
Let $A$ and $B$ be $n \times n$ matrices for which $AB$ is invertible. Show that $A$ and $B$ are both invertible.

\begin{sol}
If $B\vect{x} = \vect{0}$ where $\vect{x}$ is $n \times 1$, then $AB\vect{x} = \vect{0}$ so $\vect{x} = \vect{0}$ as $AB$ is invertible. Hence $B$ is invertible by Theorem~\ref{thm:004553}, so $A = (AB)B^{-1}$ is invertible.
\end{sol}
\end{ex}

\begin{ex}
Consider $A = \leftB \begin{array}{rrr}
1 & 3 & -1 \\
2 & 1 & 5 \\
1 & -7 & 13
\end{array} \rightB$, \newline $B = \leftB \begin{array}{rrr}
1 & 1 & 2 \\
3 & 0 & -3 \\
-2 & 5 & 17
\end{array} \rightB$.



\begin{enumerate}[label={\alph*.}]
\item Show that $A$ is not invertible by finding a nonzero $1 \times 3$ matrix $Y$ such that $YA = 0$.


[\textit{Hint}: Row 3 of $A$ equals $2\mbox{(row 2) }- 3\mbox{(row 1)}$.]

\item Show that $B$ is not invertible.


[\textit{Hint}: Column 3 $= 3\mbox{(column 2) } - \mbox{ column 1}$.]

\end{enumerate}
\begin{sol}
\begin{enumerate}[label={\alph*.}]
\setcounter{enumi}{1}
\item $B \leftB \begin{array}{r}
-1 \\
3 \\
-1
\end{array} \rightB = 0$
 so $B$ is not invertible by Theorem~\ref{thm:004553}.

\end{enumerate}
\end{sol}
\end{ex}

\begin{ex}
Show that a square matrix $A$ is invertible if and only if it can be left-cancelled: $AB = AC$ implies $B = C$.
\end{ex}

\begin{ex}
If $U^{2} = I$, show that $I + U$ is not invertible unless $U = I$.
\end{ex}

\begin{ex}
\begin{enumerate}[label={\alph*.}]
\item If $J$ is the $4 \times 4$ matrix with every entry $1$, show that $I - \frac{1}{2}J$
 is self-inverse and symmetric.

\item If $X$ is $n \times m$ and satisfies $X^{T}X = I_{m}$, show that $I_{n} - 2XX^{T}$ is self-inverse and symmetric.

\end{enumerate}
\begin{sol}
\begin{enumerate}[label={\alph*.}]
\setcounter{enumi}{1}
\item  Write $U = I_{n} - 2XX^{T}$. Then $U^{T} = I_{n}^{T} - 2X^{TT}X^{T} = U$, and $U^{2} = I_{n}^{2} - (2XX^{T})I_{n} - I_{n}(2XX^{T}) + 4(XX^{T})(XX^{T}) = I_{n} - 4XX^{T} + 4XX^{T} = I_{n}$.

\end{enumerate}
\end{sol}
\end{ex}

\begin{ex}
An $n \times n$ matrix $P$ is called an idempotent if $P^{2} = P$. Show that:


\begin{enumerate}[label={\alph*.}]
\item $I$ is the only invertible idempotent.

\item $P$ is an idempotent if and only if $I - 2P$ is self-inverse.

\item $U$ is self-inverse if and only if $U = I - 2P$ for some idempotent $P$.

\item $I - aP$ is invertible for any $a \neq 1$, and that \newline $(I - aP)^{-1} = I + \left(\frac{a}{1 - a}\right)^{P}$.


\end{enumerate}
\begin{sol}
\begin{enumerate}[label={\alph*.}]
\setcounter{enumi}{1}
\item  $(I - 2P)^{2} = I - 4P + 4P^{2}$, and this equals $I$ if and only if $P^{2} = P$.

\end{enumerate}
\end{sol}
\end{ex}

\begin{ex}
If $A^{2} = kA$, where $k \neq 0$, show that $A$ is invertible if and only if $A = kI$.
\end{ex}

\begin{ex}
Let $A$ and $B$ denote $n \times n$ invertible matrices.


\begin{enumerate}[label={\alph*.}]
\item Show that $A^{-1} + B^{-1} = A^{-1}(A + B)B^{-1}$.

\item If $A + B$ is also invertible, show that $A^{-1} + B^{-1}$ is invertible and find a formula for $(A^{-1} + B^{-1})^{-1}$.

\end{enumerate}
\begin{sol}
\begin{enumerate}[label={\alph*.}]
\setcounter{enumi}{1}
\item $(A^{-1} + B^{-1})^{-1} = B(A + B)^{-1}A$

\end{enumerate}
\end{sol}
\end{ex}

\begin{ex}
Let $A$ and $B$ be $n \times n$ matrices, and let $I$ be the $n \times n$ identity matrix.


\begin{enumerate}[label={\alph*.}]
\item Verify that $A(I + BA) = (I + AB)A$ and that \newline $(I + BA)B = B(I + AB)$.

\item If $I + AB$ is invertible, verify that $I + BA$ is also invertible and that $(I + BA)^{-1} = I - B(I + AB)^{-1}A$.

\end{enumerate}
\end{ex}
\end{multicols}

