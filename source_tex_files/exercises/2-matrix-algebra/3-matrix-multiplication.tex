\section*{Exercises for \ref{sec:2_3}}

\begin{Filesave}{solutions}
\solsection{Section~\ref{sec:2_3}}
\end{Filesave}

\begin{multicols}{2}
\begin{ex}
Compute the following matrix products.
\begin{exenumerate}
\exitem*
$ \leftB \begin{array}{rr}
1 & 3 \\
0 & -2
\end{array} \rightB \leftB \begin{array}{rr}
2 & -1 \\
0 & 1
\end{array} \rightB
$
\exitem*
$ \leftB \begin{array}{rrr}
1 & -1 & 2 \\
2 & 0 & 4
\end{array} \rightB \leftB \begin{array}{rrr}
2 & 3 & 1 \\
1 & 9 & 7 \\
-1 & 0 & 2
\end{array} \rightB
$
\exitem*
$ \leftB \begin{array}{rrr}
5 & 0 & -7 \\
1 & 5 & 9
\end{array} \rightB \leftB \begin{array}{r}
3 \\
1 \\
-1 \\
\end{array} \rightB
$
\exitem*
$ \leftB \begin{array}{rrr}
1 & 3 & -3
\end{array} \rightB \leftB \begin{array}{rr}
3 & 0 \\
-2 & 1 \\
0 & 6
\end{array} \rightB
$
\exitem*
$ \leftB \begin{array}{rrr}
1 & 0 & 0 \\
0 & 1 & 0 \\
0 & 0 & 1
\end{array} \rightB \leftB \begin{array}{rr}
3 & -2  \\
5 & -7 \\
9 & 7
\end{array} \rightB
$
\exitem*
$ \leftB \begin{array}{rrr}
1 & -1 & 3
\end{array} \rightB \leftB \begin{array}{r}
2 \\
1 \\
-8
\end{array} \rightB
$
\exitem*
$ \leftB \begin{array}{r}
2 \\
1 \\
-7
\end{array} \rightB \leftB \begin{array}{rrr}
1 & -1 & 3
\end{array} \rightB
$

\exitem* 
$ \leftB \begin{array}{rr}
3 & 1 \\
5 & 2
\end{array} \rightB \leftB \begin{array}{rr}
2 & -1 \\
-5 & 3
\end{array} \rightB
 $
\exitem*
$ \leftB \begin{array}{rrr}
2 & 3 & 1 \\
5 & 7 & 4
\end{array} \rightB \leftB \begin{array}{ccc}
a & 0 & 0 \\
0 & b & 0 \\
0 & 0 & c
\end{array} \rightB
$
\exitem*
$\leftB \begin{array}{ccc}
a & 0 & 0 \\
0 & b & 0 \\
0 & 0 & c
\end{array} \rightB 
\leftB \begin{array}{ccc}
a^\prime & 0 & 0 \\
0 & b^\prime & 0 \\
0 & 0 & c^\prime
\end{array} \rightB
$
\end{exenumerate}
\begin{sol}
\begin{enumerate}[label={\alph*.}]
\setcounter{enumi}{1}
\item
$\leftB \begin{array}{rrr}
-1 & -6 & -2 \\
0 & 6 & 10
\end{array} \rightB
$

\setcounter{enumi}{3}
\item  
$\leftB \begin{array}{rr}
-3 & -15
\end{array} \rightB$

\setcounter{enumi}{5}
\item 
$\leftB -23 \rightB$

\setcounter{enumi}{7}
\item 
$\leftB \begin{array}{rr}
1 & 0 \\
0 & 1
\end{array} \rightB
$

\setcounter{enumi}{9}
\item  
$\leftB \begin{array}{rrr}
aa^\prime & 0 & 0 \\
0 & bb^\prime & 0 \\
0 & 0 & cc^\prime
\end{array} \rightB
$

\end{enumerate}
\end{sol}
\end{ex}

\begin{ex}
In each of the following cases, find all possible products $A^{2}$, $AB$, $AC$, and so on.


\begin{enumerate}[label={\alph*.}]
\item
$A = \leftB \begin{array}{rrr}
1 & 2 & 3 \\
-1 & 0 & 0
\end{array} \rightB$, 
$B = \leftB \begin{array}{rr}
1 & -2 \\
\frac{1}{2} & 3
\end{array} \rightB$, \\
 $C = \leftB \begin{array}{rr}
-1 & 0 \\
2 & 5 \\
0 & 3
\end{array} \rightB
$ 

\item
$A = \leftB \begin{array}{rrr}
	1 & 2 & 4 \\
	0 & 1 & -1
\end{array} \rightB$, 
$B = \leftB \begin{array}{rr}
	-1 & 6 \\
	1 & 0
\end{array} \rightB$, \\
$C = \leftB \begin{array}{rr}
	2 & 0 \\
	-1 & 1 \\
	1 & 2
\end{array} \rightB
$

\end{enumerate}
\begin{sol}
\begin{enumerate}[label={\alph*.}]
\setcounter{enumi}{1}
\item
$
BA = \leftB \begin{array}{rrr}
-1 & 4 & -10 \\
1 & 2 & 4
\end{array} \rightB$, $B^{2} = \leftB \begin{array}{rr}
7 & -6 \\
-1 & 6
\end{array} \rightB$, $CB = \leftB \begin{array}{rr}
-2 & 12 \\
2 & -6 \\
1 & 6
\end{array} \rightB$ \\
$AC = \leftB \begin{array}{rr}
4 & 10 \\
-2 & -1
\end{array} \rightB$, $CA = \leftB \begin{array}{rrr}
2 & 4 & 8 \\
-1 & -1 & -5 \\
1 & 4 & 2
\end{array} \rightB
$

\end{enumerate}
\end{sol}
\end{ex}


\begin{ex}
Find $a$, $b$, $a_{1}$, and $b_{1}$ if:


\begin{enumerate}[label={\alph*.}]
\item
$\leftB \begin{array}{cc}
a & b \\
a_{1} & b_{1}
\end{array} \rightB \leftB \begin{array}{rr}
3 & -5 \\
-1 & 2
\end{array} \rightB = \leftB \begin{array}{rr}
1 & -1 \\
2 & 0
\end{array} \rightB
$

\item 
$\leftB \begin{array}{rr}
2 & 1 \\
-1 & 2
\end{array} \rightB \leftB \begin{array}{cc}
a & b \\
a_{1} & b_{1}
\end{array} \rightB = \leftB \begin{array}{rr}
7 & 2 \\
-1 & 4
\end{array} \rightB
$

\end{enumerate}
\begin{sol}
\begin{enumerate}[label={\alph*.}]
\setcounter{enumi}{1}
\item  $(a, b, a_{1}, b_{1}) = (3, 0, 1, 2)$

\end{enumerate}
\end{sol}
\end{ex}

\begin{ex}
Verify that $A^{2} - A - 6I = 0$ if:
\begin{exenumerate}
\exitem 
$ \leftB \begin{array}{rr}
3 & -1 \\
0 & -2
\end{array} \rightB
$
\exitem 
$ \leftB \begin{array}{rr}
2 & 2 \\
2 & -1
\end{array} \rightB
$
\end{exenumerate}
\begin{sol}
\begin{enumerate}[label={\alph*.}]
\setcounter{enumi}{1}
\item
$A^{2} - A - 6I = \leftB \begin{array}{rr}
8 & 2 \\
2 & 5
\end{array} \rightB - \leftB \begin{array}{rr}
2 & 2 \\
2 & -1
\end{array} \rightB - \leftB \begin{array}{rr}
6 & 0 \\
0 & 6
\end{array} \rightB = \leftB \begin{array}{rr}
0 & 0 \\
0 & 0
\end{array} \rightB$

\end{enumerate}
\end{sol}
\end{ex}

\begin{ex}\label{ex:ex2_3_5}

Given $A = \leftB \begin{array}{rr}
1 & -1 \\
0 & 1
\end{array} \rightB$, $B = \leftB \begin{array}{rrr}
1 & 0 & -2 \\
3 & 1 & 0
\end{array} \rightB$, \\ $C = \leftB \begin{array}{rr}
1 & 0 \\
2 & 1 \\
5 & 8
\end{array} \rightB$, and $D = \leftB \begin{array}{rrr}
 3 & -1 & 2 \\
 1 & 0 & 5
 \end{array} \rightB$, verify the \newline following facts from Theorem~\ref{thm:003469}.
\begin{exenumerate}
\exitem $A(B - D) = AB - AD$
\exitem $A(BC) = (AB)C$
\exitem $(CD)^{T} = D^{T}C^{T}$
\end{exenumerate}
\begin{sol}
\begin{enumerate}[label={\alph*.}]
\setcounter{enumi}{1}
\item
$
A(BC) = \leftB \begin{array}{rr}
1 & -1 \\
0 & 1
\end{array} \rightB \leftB \begin{array}{rr}
-9 & -16 \\
5 & 1
\end{array} \rightB = \leftB \begin{array}{rr}
-14 & -17 \\
5 & 1
\end{array} \rightB = \leftB \begin{array}{rrr}
-2 & -1 & -2 \\
3 & 1 & 0
\end{array} \rightB \leftB \begin{array}{rr}
1 & 0 \\
2 & 1 \\
5 & 8
\end{array} \rightB = (AB)C
$

\end{enumerate}
\end{sol}
\end{ex}

\begin{ex}
Let $A$ be a $2 \times 2$ matrix.


\begin{enumerate}[label={\alph*.}]
\item If $A$ commutes with $\leftB \begin{array}{rr}
0 & 1 \\
0 & 0
\end{array} \rightB$, show that \\ $A = \leftB \begin{array}{rr}
 a & b \\
 0 & a
 \end{array} \rightB$
 for some $a$ and $b$.

\item If $A$ commutes with $\leftB \begin{array}{rr}
0 & 0 \\
1 & 0
\end{array} \rightB$, show that \\ $A = \leftB \begin{array}{rr}
 a & 0 \\
 c & a
 \end{array} \rightB$
 for some $a$ and $c$.

\item Show that $A$ commutes with \textit{every} $2 \times 2$ matrix \newline if and only if $A = \leftB \begin{array}{rr}
a & 0 \\
0 & a
\end{array} \rightB$
 for some $a$.

\end{enumerate}
\begin{sol}
\begin{enumerate}[label={\alph*.}]
\setcounter{enumi}{1}
\item  If $A = \leftB \begin{array}{rr}
a & b \\
c & d
\end{array} \rightB$
 and $E = \leftB \begin{array}{rr}
 0 & 0 \\
 1 & 0
 \end{array} \rightB$, compare entries an $AE$ and $EA$.

\end{enumerate}
\end{sol}
\end{ex}

\begin{ex}
\begin{enumerate}[label={\alph*.}]
\item If $A^{2}$ can be formed, what can be said about the size of $A$?

\item If $AB$ and $BA$ can both be formed, describe the sizes of $A$ and $B$.

\item If $ABC$ can be formed, $A$ is $3 \times 3$, and $C$ is $5 \times 5$, what size is $B$?

\end{enumerate}
\begin{sol}
\begin{enumerate}[label={\alph*.}]
\setcounter{enumi}{1}
\item $m \times n$ and $n \times m$ for some $m$ and $n$

\end{enumerate}
\end{sol}
\end{ex}

\begin{ex}
\begin{enumerate}[label={\alph*.}]
\item Find two $2 \times 2$ matrices $A$ such that $A^{2} = 0$.

\item Find three $2 \times 2$ matrices $A$ such that (i) $A^{2} = I$; (ii) $A^{2} = A$.

\item Find $2 \times 2$ matrices $A$ and $B$ such that $AB = 0$ but $BA \neq 0$.

\end{enumerate}
\begin{sol}
\begin{enumerate}[label={\alph*.}]
\setcounter{enumi}{1}
\item
\begin{enumerate}[label={\roman*.}]
\item
$
\leftB \begin{array}{rr}
1 & 0 \\
0 & 1
\end{array} \rightB$, $\leftB \begin{array}{rr}
1 & 0 \\
0 & -1
\end{array} \rightB$, $\leftB \begin{array}{rr}
1 & 1 \\
0 & -1
\end{array} \rightB
$

\item
$
\leftB \begin{array}{rr}
1 & 0 \\
0 & 0
\end{array} \rightB$, $\leftB \begin{array}{rr}
1 & 0 \\
0 & 1
\end{array} \rightB$, $\leftB \begin{array}{rr}
1 & 1 \\
0 & 0
\end{array} \rightB
$
\end{enumerate}

\end{enumerate}
\end{sol}
\end{ex}

\begin{ex}
Write $P = \leftB \begin{array}{rrr}
1 & 0 & 0 \\
0 & 0 & 1 \\
0 & 1 & 0
\end{array} \rightB$, and let $A$ be $3 \times n$ and $B$ be $m \times 3$.


\begin{enumerate}[label={\alph*.}]
\item Describe $PA$ in terms of the rows of $A$.

\item Describe $BP$ in terms of the columns of $B$.

\end{enumerate}
\end{ex}

\begin{ex}
Let $A$, $B$, and $C$ be as in Exercise~\ref{ex:ex2_3_5}. Find the $(3, 1)$-entry of $CAB$ using exactly six numerical multiplications.
\end{ex}

\begin{ex}
Compute $AB$, using the indicated block partitioning.
\begin{equation*}
A = \leftB \begin{array}{rr|rr}
2 & -1 & 3 & 1 \\
1 & 0  & 1 & 2 \\
\hline
0 & 0 & 1 & 0 \\
0 & 0 & 0 & 1
\end{array} \rightB \quad
B = \leftB \begin{array}{rr|r}
1 & 2 & 0 \\
-1 & 0  & 0 \\
\hline
0 & 5 & 1 \\
1 & -1 & 0 
\end{array} \rightB
\end{equation*}
\end{ex}

\begin{ex}
In each case give formulas for all powers $A, A^{2}, A^{3}, \dots$ of $A$ using the block decomposition indicated.


\begin{enumerate}[label={\alph*.}]
\item
$A = \leftB \begin{array}{r|rr}
1 & 0 & 0 \\
\hline
1 & 1  & -1\\
1 & -1 & 1
\end{array} \rightB
$

\item
$A = \leftB \begin{array}{rr|rr}
1 & -1 & 2 & -1 \\
0 & 1 & 0 & 0 \\
\hline
0 & 0 & -1 & 1 \\
0 & 0 & 0 & 1
\end{array} \rightB
$

\end{enumerate}
\begin{sol}
\begin{enumerate}[label={\alph*.}]
\setcounter{enumi}{1}
\item
$
A^{2k} = \leftB \begin{array}{rc|rr}
1 & -2k & 0 & 0 \\
0 & 1 & 0 & 0 \\
\hline
0 & 0 & 1 & 0 \\
0 & 0 & 0 & 1
\end{array} \rightB$ for $k = 0, 1, 2, \dots$, \\ 
$A^{2k + 1} = A^{2k}A = \leftB \begin{array}{rc|rr}
1 & -(2k + 1) & 2 & -1 \\
0 & 1 & 0 & 0 \\
\hline
0 & 0 & -1 & 1 \\
0 & 0 & 0 & 1
\end{array} \rightB$ for $k = 0, 1, 2, \dots$

\end{enumerate}
\end{sol}
\end{ex}

%\newpage
\begin{ex}
Compute the following using block multiplication (all blocks are $k \times k$).
\begin{exenumerate}
\exitem
$\leftB \begin{array}{rr}
I & X \\
-Y & I
\end{array} \rightB \leftB \begin{array}{rr}
I & 0 \\
Y & I
\end{array} \rightB
$
\exitem 
$\leftB \begin{array}{rr}
I & X \\
0 & I
\end{array} \rightB \leftB \begin{array}{rr}
I & -X \\
0 & I
\end{array} \rightB
$
\exitem 
$\leftB \begin{array}{cc}
I & X
\end{array} \rightB \leftB \begin{array}{cc}
I & X
\end{array} \rightB^{T}
$
\exitem
$\leftB \begin{array}{cc}
I & X^{T}
\end{array} \rightB \leftB \begin{array}{cc}
-X & I
\end{array} \rightB^{T}
$
\exitem*
$\leftB \begin{array}{cc}
I & X \\
0 & -I
\end{array} \rightB^{n}
$ any $n \geq 1$
\exitem* 
$\leftB \begin{array}{cc}
0 & X \\
I & 0
\end{array} \rightB^{n}
$ any $n \geq 1$
\end{exenumerate}
\begin{sol}
\begin{enumerate}[label={\alph*.}]
\setcounter{enumi}{1}
\item
$\leftB \begin{array}{cc}
I & 0 \\
0 & I
\end{array} \rightB = I_{2k}$

\setcounter{enumi}{3}
\item $0_{k}$


\setcounter{enumi}{5}
\item
$\leftB \begin{array}{cc}
X^{m} & 0 \\
0 & X^{m}
\end{array} \rightB$ if $n = 2m$; 
$\leftB \begin{array}{cc}
0 & X^{m + 1} \\
X^{m} & 0
\end{array} \rightB$ if $n = 2m + 1$


\end{enumerate}
\end{sol}
\end{ex}

\begin{ex}
Let $A$ denote an $m \times n$ matrix.


\begin{enumerate}[label={\alph*.}]
\item If $AX = 0$ for every $n \times 1$ matrix $X$, show that $A = 0$.

\item If $YA = 0$ for every $1 \times m$ matrix $Y$, show that $A = 0$.

\end{enumerate}
\begin{sol}
\begin{enumerate}[label={\alph*.}]
\setcounter{enumi}{1}
\item  If $Y$ is row $i$ of the identity matrix $I$, then $YA$ is row $i$ of $IA = A$.

\end{enumerate}
\end{sol}
\end{ex}

\begin{ex}
\begin{enumerate}[label={\alph*.}]
\item If $U = \leftB \begin{array}{rr}
1 & 2 \\
0 & -1
\end{array} \rightB$, and $AU = 0$, show that $A = 0$.

\item Let $U$ be such that $AU = 0$ implies that $A = 0$. If $PU = QU$, show that $P = Q$.

\end{enumerate}
\end{ex}

\begin{ex}
Simplify the following expressions where $A$, $B$, and $C$ represent matrices.


\begin{enumerate}[label={\alph*.}]
\item $A(3B - C) + (A - 2B)C + 2B(C + 2A)$

\item $A(B + C - D) + B(C - A + D) - (A + B)C \\ + (A - B)D$

\item $AB(BC - CB) + (CA - AB)BC + CA(A - B)C$

\item $(A - B)(C - A) + (C - B)(A - C) + (C - A)^{2}$

\end{enumerate}
\begin{sol}
\begin{enumerate}[label={\alph*.}]
\setcounter{enumi}{1}
\item $AB - BA$

\setcounter{enumi}{3}
\item $0$
\end{enumerate}
\end{sol}
\end{ex}

\begin{ex}
If $A = \leftB \begin{array}{cc}
a & b \\
c & d
\end{array} \rightB$
 where $a \neq 0$, show that $A$ factors in the form 
 $A = \leftB \begin{array}{cc}
 1 & 0 \\
 x & 1
 \end{array} \rightB \leftB \begin{array}{cc}
 y & z \\
 0 & w
 \end{array} \rightB$.
\end{ex}

\begin{ex}
If $A$ and $B$ commute with $C$, show that the same is true of:
\begin{exenumerate}
\exitem $A + B$
\exitem $kA$, $k$ any scalar
\end{exenumerate}
\begin{sol}
\begin{enumerate}[label={\alph*.}]
\setcounter{enumi}{1}
\item $(kA)C = k(AC) = k(CA) = C(kA)$

\end{enumerate}
\end{sol}
\end{ex}

\begin{ex}
If $A$ is any matrix, show that both $AA^{T}$ and $A^{T}A$ are symmetric.
\end{ex}

\begin{ex}
If $A$ and $B$ are symmetric, show that $AB$ is symmetric if and only if $AB = BA$.

\begin{sol}
We have $A^{T} = A$ and $B^{T} = B$, so $(AB)^{T} = B^{T}A^{T} = BA$. Hence $AB$ is symmetric if and only if $AB = BA$.
\end{sol}
\end{ex}

\begin{ex}
If $A$ is a $2 \times 2$ matrix, show that \newline $A^{T}A = AA^{T}$ if and only if $A$ is symmetric or \newline $A = \leftB \begin{array}{cc}
a & b \\
-b & a
\end{array} \rightB$
 for some $a$ and $b$.
\end{ex}

\begin{ex}
\begin{enumerate}[label={\alph*.}]
\item Find all symmetric $2 \times 2$ matrices $A$ such that $A^{2} = 0$.

\item Repeat (a) if $A$ is $3 \times 3$.

\item Repeat (a) if $A$ is $n \times n$.

\end{enumerate}
\begin{sol}
\begin{enumerate}[label={\alph*.}]
\setcounter{enumi}{1}
\item $A = 0$

\end{enumerate}
\end{sol}
\end{ex}

\begin{ex}
Show that there exist no $2 \times 2$ matrices $A$ and $B$ such that $AB - BA = I$. [\textit{Hint}: Examine the $(1, 1)$- and $(2, 2)$-entries.]
\end{ex}

\begin{ex}
Let $B$ be an $n \times n$ matrix. Suppose $AB = 0$ for some nonzero $m \times n$ matrix $A$. Show that no $n \times n$ matrix $C$ exists such that $BC = I$.

\begin{sol}
If $BC = I$, then $AB = 0$ gives $0 = 0C = (AB)C = A(BC) = AI = A$, contrary to the assumption that $A \neq 0$.
\end{sol}
\end{ex}

\begin{ex}
An autoparts manufacturer makes fenders, doors, and hoods. Each requires assembly and packaging carried out at factories: Plant 1, Plant 2, and Plant 3. Matrix $A$ below gives the number of hours for assembly and packaging, and matrix $B$ gives the hourly rates at the three plants. Explain the meaning of the $(3, 2)$-entry in the matrix $AB$. Which plant is the most economical to operate? Give reasons.

\begin{tabu}{lccll}
& Assembly & Packaging & & \\
$\begin{array}{l}
	\mbox{Fenders} \\
	\mbox{Doors} \\
	\mbox{Hoods}
\end{array}$ &
\multicolumn{2}{l}{
$\leftB \arraycolsep=20pt \begin{array}{cr}
	12 & 2 \\
	21 & 3 \\
	10 & 2
\end{array} \rightB$
} & $=$ & $A$
\end{tabu}

\medskip

\noindent\makebox[\linewidth]{
\begin{tabu}{lcccll}
	& Plant 1 & Plant 2 & Plant 3 & & \\
	$\begin{array}{l}
	\mbox{Assembly} \\
	\mbox{Packaging}
	\end{array}$ &
	\multicolumn{3}{c}{
		$\leftB \arraycolsep=15pt \begin{array}{rrr}
		21 & 18 & 20 \\
		14 & 10 & 13
		\end{array} \rightB$
	} & $=$ & $B$
\end{tabu}}
\end{ex}


\begin{ex}
For the directed graph below, find the adjacency matrix $A$, compute $A^{3}$, and determine the number of paths of length $3$ from $v_{1}$ to $v_{4}$ and from $v_{2}$ to $v_{3}$.


\begin{figure}[H]
\centering
\begin{tikzpicture}
\node[font=\small] at (0,0) (v1) {$v_1$};
\node[font=\small] at (2,0) (v2) {$v_2$};
\node[font=\small] at (2,-2) (v3) {$v_3$};
\node[font=\small] at (0,-2) (v4) {$v_4$};
\draw[dkgreenvect,-latex,thick] (v1)--(v2);
\draw[dkgreenvect,-latex,thick] (v1) to [loop above] (v1);
\draw[dkgreenvect,-latex,thick] (v3)--(v1);
\draw[dkgreenvect,-latex,thick] (v1)--(v4);
\draw[dkgreenvect,latex-latex,thick] (v2)--(v4);
\draw[dkgreenvect,-latex,thick] (v4)--(v3);
\end{tikzpicture}


%\caption{\label{fig:004072}}
\end{figure}

\begin{sol}
$3$ paths $v_{1} \rightarrow v_{4}$, $0$ paths $v_{2} \rightarrow v_{3}$
\end{sol}
\end{ex}

\begin{ex}
In each case either show the statement is true, or give an example showing that it is false.


\begin{enumerate}[label={\alph*.}]
\item If $A^{2} = I$, then $A = I$.

\item If $AJ = A$, then $J = I$.

\item If $A$ is square, then $(A^{T})^{3} = (A^{3})^{T}$.

\item If $A$ is symmetric, then $I + A$ is symmetric.

\item If $AB = AC$ and $A \neq 0$, then $B = C$.

\item If $A \neq 0$, then $A^{2} \neq 0$.

\item If $A$ has a row of zeros, so also does $BA$ for all $B$.

\item If $A$ commutes with $A + B$, then $A$ commutes with $B$.

\item If $B$ has a column of zeros, so also does $AB$.

\item If $AB$ has a column of zeros, so also does $B$.

\item If $A$ has a row of zeros, so also does $AB$.

\item If $AB$ has a row of zeros, so also does $A$.

\end{enumerate}
\begin{sol}
\begin{enumerate}[label={\alph*.}]
\setcounter{enumi}{1}
\item  False. If $A = \leftB \begin{array}{rr}
1 & 0 \\
0 & 0
\end{array} \rightB = J$, then $AJ = A$ but $J \neq I$.

\setcounter{enumi}{3}
\item  True. Since $A^{T} = A$, we have $(I + AT = I^{T} + A^{T} = I + A$.

\setcounter{enumi}{5}
\item  False. If $A = \leftB \begin{array}{rr}
0 & 1 \\
0 & 0
\end{array} \rightB$, then $A \neq 0$ but $A^{2} = 0$.

\setcounter{enumi}{7}
\item  True. We have $A(A + B) = (A + B)A$; that is, $A^{2} + AB = A^{2} + BA$. Subtracting $A^{2}$ gives $AB = BA$.

\setcounter{enumi}{9}
\item  False. $A = \leftB \begin{array}{rr}
1 & -2 \\
2 & 4
\end{array} \rightB$, $B = \leftB \begin{array}{rr}
2 & 4 \\
1 & 2
\end{array} \rightB$

\setcounter{enumi}{11}
\item  False. See \textbf{(j)}.

\end{enumerate}
\end{sol}
\end{ex}

\begin{ex}
\begin{enumerate}[label={\alph*.}]
\item If $A$ and $B$ are $2 \times 2$ matrices whose rows sum to $1$, show that the rows of $AB$ also sum to $1$.

\item Repeat part (a) for the case where $A$ and $B$ are $n \times n$.

\end{enumerate}
\begin{sol}
\begin{enumerate}[label={\alph*.}]
\setcounter{enumi}{1}
\item  If $A = \leftB a_{ij} \rightB$ and $B = \leftB b_{ij} \rightB$ and $\sum_{j}a_{ij} = 1 = \sum_{j}b_{ij}$, then the $(i, j)$-entry of $AB$ is $c_{ij} = \sum_{k}a_{ik}b_{kj}$, whence $\sum_{j}c_{ij} = \sum_{j}\sum_{k}a_{ik}b_{kj} = \sum_{k}a_{ik}(\sum_{j}b_{kj}) = \sum_{k}a_{ik} = 1$. Alternatively: If $\vect{e} = (1, 1, \dots, 1)$, then the rows of $A$ sum to $1$ if and only if $A\vect{e} = \vect{e}$. If also $B\vect{e} = \vect{e}$ then $(AB)\vect{e} = A(B\vect{e}) = A\vect{e} = \vect{e}$.

\end{enumerate}
\end{sol}
\end{ex}

\begin{ex}
Let $A$ and $B$ be $n \times n$ matrices for which the systems of equations $A\vect{x} = \vect{0}$ and $B\vect{x} = \vect{0}$ each have only the trivial solution $\vect{x} = \vect{0}$. Show that the system $(AB)\vect{x} = \vect{0}$ has only the trivial solution.
\end{ex}


\begin{ex}
The \textbf{trace}\index{trace}\index{square matrix ($n \times n$ matrix)!square} of a square matrix $A$, denoted $\func{tr }A$, is the sum of the elements on the main diagonal of $A$. Show that, if $A$ and $B$ are $n \times n$ matrices:
\begin{exenumerate}
\exitem* $\func{tr}(A + B) = \func{tr }A + \func{tr }B$.
\exitem* $\func{tr}(kA) = k \func{tr}(A)$ for any number $k$.
\exitem $\func{tr}(A^{T}) = \func{tr}(A)$.
\exitem $\func{tr}(AB) = \func{tr}(BA)$.
\exitem* $\func{tr}(AA^{T})$ is the sum of the squares of all entries of $A$.
\end{exenumerate}
\begin{sol}
\begin{enumerate}[label={\alph*.}]
\setcounter{enumi}{1}
\item  If $A = \leftB a_{ij} \rightB$, then $\func{tr}(kA) = \func{tr}\leftB ka_{ij} \rightB = \sum_{i=1}^{n} ka_{ii} = k\sum_{i=1}^{n} a_{ii} = k \func{tr}(A)$.


\setcounter{enumi}{4}
\item  Write $A^{T} = \leftB a_{ij}^\prime \rightB$, where $a_{ij}^\prime = a_{ji}$. Then $AA^{T} = \left( \sum_{k=1}^{n} a_{ik}a_{kj}^\prime \right)$, so $\func{tr}(AA^{T}) = \sum_{i=1}^{n} \leftB \sum_{k=1}^{n} a_{ik}a_{ki}^\prime \rightB = \sum_{i=1}^{n} \sum_{k=1}^{n} a_{ik}^{2}$.


\end{enumerate}
\end{sol}
\end{ex}

\begin{ex}
Show that $AB - BA = I$ is impossible.


[\textit{Hint}: See the preceding exercise.]
\end{ex}


\begin{ex}
A square matrix $P$ is called an \newline \textbf{idempotent}\index{idempotents}\index{square matrix ($n \times n$ matrix)!idempotent} if $P^{2} = P$. Show that:


\begin{enumerate}[label={\alph*.}]
\item $0$ and $I$ are idempotents.

\item $\leftB \begin{array}{rr}
1 & 1 \\
0 & 0
\end{array} \rightB$, $\leftB \begin{array}{rr}
1 & 0 \\
1 & 0
\end{array} \rightB$, and $\frac{1}{2} \leftB \begin{array}{rr}
 1 & 1 \\
 1 & 1
 \end{array} \rightB$, are idempotents.

\item If $P$ is an idempotent, so is $I - P$. Show further that $P(I - P) = 0$.

\item If $P$ is an idempotent, so is $P^{T}$.

\item If $P$ is an idempotent, so is $Q = P + AP - PAP$ for any square matrix $A$ (of the same size as $P$).

\item If $A$ is $n \times m$ and $B$ is $m \times n$, and if $AB = I_{n}$, then $BA$ is an idempotent.

\end{enumerate}
\begin{sol}
\begin{enumerate}[label={\alph*.}]
\setcounter{enumi}{4}
\item  Observe that $PQ = P^{2} + PAP - P^{2}AP = P$, so $Q^{2} = PQ + APQ - PAPQ = P + AP - PAP = Q$.

\end{enumerate}
\end{sol}
\end{ex}

\begin{ex}
Let $A$ and $B$ be $n \times n$ \textbf{diagonal matrices}\index{diagonal matrices}\index{matrix!diagonal matrices}\index{square matrix ($n \times n$ matrix)!diagonal matrices} (all entries off the main diagonal are zero).


\begin{enumerate}[label={\alph*.}]
\item Show that $AB$ is diagonal and $AB = BA$.

\item Formulate a rule for calculating $XA$ if $X$ is $m \times n$.

\item Formulate a rule for calculating $AY$ if $Y$ is $n \times k$.

\end{enumerate}
\end{ex}

\begin{ex}
If $A$ and $B$ are $n \times n$ matrices, show that:


\begin{enumerate}[label={\alph*.}]
\item $AB = BA$ if and only if
\begin{equation*}
(A + B)^{2} = A^{2} + 2AB + B^{2}
\end{equation*}
\item $AB = BA$ if and only if
\begin{equation*}
(A + B)(A - B) = (A - B)(A + B)
\end{equation*}
\end{enumerate}
\begin{sol}
\begin{enumerate}[label={\alph*.}]
\setcounter{enumi}{1}
\item $(A + B)(A - B) = A^{2} - AB + BA - B^{2}$, and $(A - B)(A + B) = A^{2} + AB - BA - B^{2}$. These are equal if and only if $-AB + BA = AB - BA$; that is, $2BA = 2AB$; that is, $BA = AB$.

\end{enumerate}
\end{sol}
\end{ex}

\begin{ex}
In Theorem~\ref{thm:003584}, prove
\begin{exenumerate}
\exitem part 3;
\exitem part 5.
\end{exenumerate}
\begin{sol}
\begin{enumerate}[label={\alph*.}]
\setcounter{enumi}{1}
\item  $(A + B)(A - B) = A^{2} - AB + BA - B^{2}$ and $(A - B)(A + B) = A^{2} - BA + AB - B^{2}$. These are equal if and only if $-AB + BA = -BA + AB$, that is $2AB = 2BA$, that is $AB = BA$.

\end{enumerate}
\end{sol}
\end{ex}

%\begin{ex}
%(V. Camillo) Show that the product of two reduced row-echelon matrices is also reduced row-echelon.

%\begin{sol}
%See V. Camillo, Communications in Algebra 25(6), (1997), 1767--1782; Theorem~\ref{thm:003488}.

%\end{sol}
%\end{ex}

\end{multicols}
