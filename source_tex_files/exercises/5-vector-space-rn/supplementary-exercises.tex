\section*{Supplementary Exercises for Chapter~\ref{chap:5}}
\addcontentsline{toc}{section}{Supplementary Exercises for Chapter~\ref{chap:5}}

\begin{Filesave}{solutions}
\solsection{Supplementary Exercises for Chapter~\ref{chap:5}}
\end{Filesave}

\vspace*{-1em}
\begin{multicols}{2}
\begin{supex}
In each case either show that the statement is true or give an example showing that it is false. Throughout, $\vect{x}, \vect{y}, \vect{z}, \vect{x}_{1}, \vect{x}_{2}, \dots, \vect{x}_{n}$ denote vectors in $\RR^n$.

\begin{enumerate}[label={\alph*.}]
\item If $U$ is a subspace of $\RR^n$ and $\vect{x} + \vect{y}$ is in $U$, then $\vect{x}$ and $\vect{y}$ are both in $U$.

\item If $U$ is a subspace of $\RR^n$ and $r\vect{x}$ is in $U$, then $\vect{x}$ is in $U$.

\item If $U$ is a nonempty set and $s\vect{x} + t\vect{y}$ is in $U$ for any $s$ and $t$ whenever $\vect{x}$ and $\vect{y}$ are in $U$, then $U$ is a subspace.

\item If $U$ is a subspace of $\RR^n$ and $\vect{x}$ is in $U$, then $-\vect{x}$ is in $U$.

\item If $\{\vect{x}, \vect{y}\}$ is independent, then $\{\vect{x}, \vect{y}, \vect{x} + \vect{y}\}$ is independent.

\item If $\{\vect{x}, \vect{y}, \vect{z}\}$ is independent, then $\{\vect{x}, \vect{y}\}$ is independent.

\item If $\{\vect{x}, \vect{y}\}$ is not independent, then $\{\vect{x}, \vect{y}, \vect{z}\}$ is not independent.

\item If all of $\vect{x}_{1}, \vect{x}_{2}, \dots, \vect{x}_{n}$ are nonzero, then $\{\vect{x}_{1}, \vect{x}_{2}, \dots, \vect{x}_{n}\}$ is independent.

\item If one of $\vect{x}_{1}, \vect{x}_{2}, \dots, \vect{x}_{n}$ is zero, then $\{\vect{x}_{1}, \vect{x}_{2}, \dots, \vect{x}_{n}\}$ is not independent.

\item If $a\vect{x} + b\vect{y} + c\vect{z} = \vect{0}$ where $a$, $b$, and $c$ are in $\RR$, then $\{\vect{x}, \vect{y}, \vect{z}\}$ is independent.

\item If $\{\vect{x}, \vect{y}, \vect{z}\}$ is independent, then $a\vect{x} + b\vect{y} + c\vect{z} = \vect{0}$ for some $a$, $b$, and $c$ in $\RR$.

\item If $\{\vect{x}_{1}, \vect{x}_{2}, \dots, \vect{x}_{n}\}$ is not independent, then \newline $t_{1}\vect{x}_{1} + t_{2}\vect{x}_{2} + \dots + t_{n}\vect{x}_{n} = \vect{0}$ for $t_{i}$ in $\RR$ not all zero.

\item If $\{\vect{x}_{1}, \vect{x}_{2}, \dots, \vect{x}_{n}\}$ is independent, then \newline $t_{1}\vect{x}_{1} + t_{2}\vect{x}_{2} + \dots + t_{n}\vect{x}_{n} = \vect{0}$ for some $t_{i}$ in $\RR$.

\item Every set of four non-zero vectors in $\RR^4$ is a basis.

\item No basis of $\RR^3$ can contain a vector with a component $\vect{0}$.

\item $\RR^3$ has a basis of the form $\{\vect{x}, \vect{x} + \vect{y}, \vect{y}\}$ where $\vect{x}$ and $\vect{y}$ are vectors.

\item Every basis of $\RR^5$ contains one column of $I_{5}$.

\item Every nonempty subset of a basis of $\RR^3$ is again a basis of $\RR^3$.

\item If $\{\vect{x}_{1}, \vect{x}_{2}, \vect{x}_{3}, \vect{x}_{4}\}$ and $\{\vect{y}_{1}, \vect{y}_{2}, \vect{y}_{3}, \vect{y}_{4}\}$ are bases of $\RR^4$, then $\{\vect{x}_{1} + \vect{y}_{1}, \vect{x}_{2} + \vect{y}_{2}, \vect{x}_{3} + \vect{y}_{3}, \vect{x}_{4} + \vect{y}_{4}\}$ is also a basis of $\RR^4$.

\end{enumerate}

\begin{supsol}
	\begin{enumerate}[label={\alph*.}]
		\setcounter{enumi}{1} %b
		\item F
		\setcounter{enumi}{3} %d
		\item T
		\setcounter{enumi}{5} %f
		\item T
		\setcounter{enumi}{7} %h
		\item F
		\setcounter{enumi}{9} %j
		\item F
		\setcounter{enumi}{11} %l
		\item T
		\setcounter{enumi}{13} %n
		\item F
		\setcounter{enumi}{15} %p
		\item F
		\setcounter{enumi}{17} %r
		\item F
	\end{enumerate}
\end{supsol}

\end{supex}
\end{multicols}
