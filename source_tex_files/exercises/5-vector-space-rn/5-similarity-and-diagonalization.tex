
\section*{Exercises for \ref{sec:5_5}}

\begin{Filesave}{solutions}
\solsection{Section~\ref{sec:5_5}}
\end{Filesave}

\begin{multicols}{2}
\begin{ex}
By computing the trace, determinant, and rank, show that $A$ and $B$ are \textit{not} similar in each case.

\begin{enumerate}[label={\alph*.}]
\item $A = \leftB \begin{array}{rr}
	1 & 2 \\
	2 & 1
\end{array} \rightB$, $B =
\leftB \begin{array}{rr}
	1 & 1\\
	-1 & 1
\end{array} \rightB$

\item $A = 
\leftB \begin{array}{rr}
3 & 1 \\
2 & -1
\end{array} \rightB$, $B =
\leftB \begin{array}{rr}
1 & 1 \\
2 & 1
\end{array} \rightB$

\item $A =
\leftB \begin{array}{rr}
2 & 1 \\
1 & -1 
\end{array} \rightB$, $B =
\leftB \begin{array}{rr}
3 & 0 \\
1 & -1
\end{array} \rightB$

\item $A = 
\leftB \begin{array}{rr}
3 & 1 \\
-1 & 2 
\end{array} \rightB$, $B =
\leftB \begin{array}{rr}
2 & -1 \\
3 & 2
\end{array} \rightB$

\item $A = 
\leftB \begin{array}{rrr}
2 & 1 & 1 \\
1 & 0 & 1 \\
1 & 1 & 0
\end{array} \rightB$, $B = 
\leftB \begin{array}{rrr}
 1 & -2 & 1 \\
-2 & 4 & -2 \\
-3 & 6 & -3 
\end{array} \rightB$

\item $A = 
\leftB \begin{array}{rrr}
1 & 2 & -3 \\
1 & -1 & 2 \\
0 & 3 & -5
\end{array} \rightB$, $B =
\leftB \begin{array}{rrr}
-2 & 1 & 3 \\
 6 & -3 & -9 \\
 0 & 0 & 0
\end{array} \rightB$

\end{enumerate}
\begin{sol}
\begin{enumerate}[label={\alph*.}]
\setcounter{enumi}{1}
\item  traces $= 2$, ranks $= 2$, but $\func{det} A = -5$, $\func{det} B = -1$

\setcounter{enumi}{3}
\item  ranks $= 2$, determinants $= 7$, but $\func{tr} A = 5$, $\func{tr} B = 4$

\setcounter{enumi}{5}
\item  traces $= -5$, determinants $= 0$, but $\func{rank} A = 2$, $\func{rank} B = 1$

\end{enumerate}
\end{sol}
\end{ex}

\begin{ex}
Show that $\leftB \begin{array}{rrrr}
1 & 2 & -1 &  0 \\
2 & 0 &  1 &  1 \\
1 & 1 &  0 & -1 \\
4 & 3 & 0 & 0
\end{array} \rightB$ and $
\leftB \begin{array}{rrrr}
  1 & -1 &  3 &  0 \\
 -1 &  0 &  1 &  1 \\
  0 & -1 &  4 &  1 \\
  5 & -1 & -1 & -4
\end{array} \rightB$ are \textit{not} similar.
\end{ex}

\begin{ex}
If $A \sim B$, show that:

\begin{exenumerate}
\exitem $A^{T} \sim B^{T}$
\exitem $A^{-1} \sim B^{-1}$
\exitem $rA \sim rB$ for $r$ in $\RR$
\exitem $A^{n} \sim B^{n}$ for $n \geq 1$
\end{exenumerate}
\begin{sol}
\begin{enumerate}[label={\alph*.}]
\setcounter{enumi}{1}
\item  If $B = P^{-1}AP$, then $B^{-1} = P^{-1}A^{-1}(P^{-1})^{-1} = P^{-1}A^{-1}P$.

\end{enumerate}
\end{sol}
\end{ex}

\begin{ex}
In each case, decide whether the matrix $A$ is diagonalizable. If so, find $P$ such that $P^{-1}AP$ is diagonal.

\begin{exenumerate}
\exitem $\leftB \begin{array}{rrr}
1 & 0 & 0 \\
1 & 2 & 1 \\
0 & 0 & 1
\end{array} \rightB$
\exitem $\leftB \begin{array}{rrr}
3 &  0 & 6 \\
0 & -3 & 0 \\
5 &  0 & 2
\end{array} \rightB$
\exitem $\leftB \begin{array}{rrr}
 3 &  1 &  6 \\
 2 &  1 &  0 \\
-1 &  0 & -3 
\end{array} \rightB$
\exitem $\leftB \begin{array}{rrr}
4 & 0 & 0 \\
0 & 2 & 2 \\
2 & 3 & 1
\end{array} \rightB$
\end{exenumerate}
\begin{sol}
\begin{enumerate}[label={\alph*.}]
\setcounter{enumi}{1}
\item  Yes, $ P =
\leftB \begin{array}{rrr}
-1 & 0 & 6 \\
 0 & 1 & 0 \\
 1 & 0 & 5
\end{array} \rightB$, $P^{-1}AP = 
\leftB \begin{array}{rrr}
-3 &  0 & 0 \\
 0 & -3 & 0 \\
 0 &  0 & 8
\end{array} \rightB$

\setcounter{enumi}{3}
\item  No, $c_{A}(x) = (x + 1)(x - 4)^{2}$ so $\lambda = 4$ has multiplicity 2. But $\func{dim}(E_{4}) = 1$ so Theorem~\ref{thm:016250} applies.

\end{enumerate}
\end{sol}
\end{ex}

\begin{ex}
If $A$ is invertible, show that $AB$ is similar to $BA$ for all $B$.
\end{ex}

\begin{ex}
Show that the only matrix similar to a scalar matrix $A = rI$, $r$ in $\RR$, is $A$ itself.
\end{ex}

\begin{ex}
Let $\lambda$ be an eigenvalue of $A$ with corresponding eigenvector $\vect{x}$. If $B = P^{-1}AP$ is similar to $A$, show that $P^{-1}\vect{x}$ is an eigenvector of $B$ corresponding to $\lambda$.
\end{ex}

\begin{ex}
If $A \sim B$ and $A$ has any of the following properties, show that $B$ has the same property.

\begin{enumerate}[label={\alph*.}]
\item Idempotent, that is $A^{2} = A$.

\item Nilpotent, that is $A^{k} = 0$ for some $k \geq 1$.

\item Invertible.

\end{enumerate}
\begin{sol}
\begin{enumerate}[label={\alph*.}]
\setcounter{enumi}{1}
\item  If $B = P^{-1}AP$ and $A^{k} = 0$, then $B^{k} = (P^{-1}AP)^{k} = P^{-1}A^{k}P = P^{-1}0P = 0$.

\end{enumerate}
\end{sol}
\end{ex}

\begin{ex}
Let $A$ denote an $n \times n$ upper triangular matrix.

\begin{enumerate}[label={\alph*.}]
\item If all the main diagonal entries of $A$ are distinct, show that $A$ is diagonalizable.

\item If all the main diagonal entries of $A$ are equal, show that $A$ is diagonalizable only if it is \textit{already} diagonal.

\item Show that $\leftB \begin{array}{rrr}
1 & 0 & 1 \\
0 & 1 & 0 \\
0 & 0 & 2
\end{array} \rightB$ is diagonalizable but that $
 \leftB \begin{array}{rrr}
 1 & 1 & 0 \\
 0 & 1 & 0 \\
 0 & 0 & 2
 \end{array} \rightB$ is not diagonalizable.

\end{enumerate}
\begin{sol}
\begin{enumerate}[label={\alph*.}]
\setcounter{enumi}{1}
\item  The eigenvalues of $A$ are all equal (they are the diagonal elements), so if $P^{-1}AP = D$ is diagonal, then $D = \lambda I$. Hence $A = P^{-1}(\lambda I)P = \lambda I$.

\end{enumerate}
\end{sol}
\end{ex}



\begin{ex}
Let $A$ be a diagonalizable $n \times n$ matrix with eigenvalues $\lambda_{1}, \lambda_{2}, \dots, \lambda_{n}$ (including multiplicities). Show that:

\begin{enumerate}[label={\alph*.}]
\item $\func{det} A = \lambda_{1}\lambda_{2}\cdots \lambda_{n}$

\item $\func{tr} A = \lambda_{1} + \lambda_{2} + \cdots + \lambda_{n}$

\end{enumerate}
\begin{sol}
\begin{enumerate}[label={\alph*.}]
\setcounter{enumi}{1}
\item  $A$ is similar to $D = \func{diag}(\lambda_{1}, \lambda_{2}, \dots, \lambda_{n})$ so (Theorem~\ref{thm:016008}) $\func{tr} A = \func{tr} D = \lambda_{1} + \lambda_{2} + \dots + \lambda_{n}$.

\end{enumerate}
\end{sol}
\end{ex}

\begin{ex}
Given a polynomial $p(x) = r_{0} + r_{1}x + \dots + r_{n}x^{n}$ and a square matrix $A$, the matrix $p(A) = r_{0}I + r_{1}A + \dots  + r_{n}A^{n}$ is called the \textbf{evaluation}\index{evaluation} of $p(x)$ at $A$. Let $B = P^{-1}AP$. Show that $p(B) = P^{-1}p(A)P$ for all polynomials $p(x)$.
\end{ex}

\begin{ex}
	\label{ex:5_5_12}
Let $P$ be an invertible $n \times n$ matrix. If $A$ is any $n \times n$ matrix, write $T_{P}(A) = P^{-1}AP$. Verify that:

%\begin{enumerate}[label={\alph*.}]
\begin{exenumerate}
\exitem $T_{P}(I) = I$
\exitem $T_{P}(AB) = T_{P}(A)T_{P}(B)$
\exitem $T_{P}(A + B) = T_{P}(A) + T_{P}(B)$
\exitem $T_{P}(rA) = rT_{P}(A)$
\exitem* $T_{P}(A^{k}) = [T_{P}(A)]^{k}$ for $k \geq 1$
\exitem* If $A$ is invertible, $T_{P}(A^{-1}) = [T_{P}(A)]^{-1}$.
\exitem* If $Q$ is invertible, $T_{Q}[T_{P}(A)] = T_{PQ}(A)$.
\end{exenumerate}
\begin{sol}
\begin{enumerate}[label={\alph*.}]
\setcounter{enumi}{1}
\item  $T_{P}(A)T_{P}(B) = (P^{-1}AP)(P^{-1}BP) = P^{-1}(AB)P = T_{P}(AB)$.

\end{enumerate}
\end{sol}
\end{ex}

\begin{ex}
\begin{enumerate}[label={\alph*.}]
\item Show that two diagonalizable matrices are similar if and only if they have the same eigenvalues with the same multiplicities.

\item If $A$ is diagonalizable, show that $A \sim A^{T}$.

\item Show that $A \sim A^{T}$ if 
 $A = \leftB \begin{array}{rr}
 1 & 1 \\
 0 & 1
 \end{array} \rightB $

\end{enumerate}
\begin{sol}
\begin{enumerate}[label={\alph*.}]
\setcounter{enumi}{1}
\item  If $A$ is diagonalizable, so is $A^{T}$, and they have the same eigenvalues. Use \textbf{(a)}.

\end{enumerate}
\end{sol}
\end{ex}

\begin{ex}
If $A$ is $2 \times 2$ and diagonalizable, show that $C(A) = \{X \mid XA = AX\}$ has dimension $2$ or $4$. [\textit{Hint}: If $P^{-1}AP = D$, show that $X$ is in $C(A)$ if and only if $P^{-1}XP$ is in $C(D)$.]
\end{ex}

\begin{ex}
If $A$ is diagonalizable and $p(x)$ is a polynomial such that $p(\lambda) = 0$ for all eigenvalues $\lambda$ of $A$, show that $p(A) = 0$ (see Example~\ref{exa:009262}). In particular, show $c_{A}(A) = 0$. [\textit{Remark}: $c_{A}(A) = 0$ for \textit{all} square matrices $A$---this is the Cayley-Hamilton theorem, see Theorem~\ref{thm:033262}.]
\end{ex}

\begin{ex}
Let $A$ be $n \times n$ with $n$ distinct real eigenvalues. If $AC = CA$, show that $C$ is diagonalizable.
\end{ex}

\begin{ex}
Let $A = \leftB \begin{array}{rrr}
0 & a & b \\
a & 0 & c \\
b & c & 0	
\end{array} \rightB$ and \newline $B = 
\leftB \begin{array}{rrr}
c & a & b \\
a & b & c \\
b & c & a
\end{array} \rightB$.

\begin{enumerate}[label={\alph*.}]
\item Show that $x^{3} - (a^{2} + b^{2} + c^{2})x - 2abc$ has real roots by considering $A$.

\item Show that $a^{2} + b^{2} + c^{2} \geq ab + ac + bc$ by considering $B$.

\end{enumerate}
\begin{sol}
\begin{enumerate}[label={\alph*.}]
\setcounter{enumi}{1}
\item  $c_{B}(x) = [x - (a + b + c)][x^{2} - k]$ where $k = a^{2} + b^{2} + c^{2} - [ab + ac + bc]$. Use Theorem~\ref{thm:016397}.

\end{enumerate}
\end{sol}
\end{ex}

\begin{ex}
Assume the $2 \times 2$ matrix $A$ is similar to an upper triangular matrix. If $\func{tr} A = 0 = \func{tr} A^{2}$, show that $A^{2} = 0$.
\end{ex}

\begin{ex}
Show that $A$ is similar to $A^{T}$ for all $2 \times 2$ matrices $A$. [\textit{Hint}: Let $A = \leftB \begin{array}{rr}
a & b \\
c & d
\end{array} \rightB$. If $c = 0$ treat the cases $b = 0$ and $b \neq 0$ separately. If $c \neq 0$, reduce to the case $c = 1$ using Exercise~\ref{ex:5_5_12}(d).]
\end{ex}

\begin{ex}
Refer to Section~\ref{sec:3_4} on linear recurrences. Assume that the sequence $x_{0}, x_{1}, x_{2}, \dots$ satisfies
\vspace*{-1em}
\begin{equation*}
x_{n+k} = r_0x_n + r_1x_{n+1} + \dotsb + r_{k-1}x_{n+k-1}
\end{equation*}
for all $n \geq 0$. Define
\begin{equation*}
A = 
\scriptsize \leftB \begin{array}{ccccc}
0 & 1 & 0 & \cdots & 0 \\
0 & 0 & 1 & \cdots & 0 \\
\vdots & \vdots & \vdots & & \vdots \\
0 & 0 & 0 & \cdots & 1 \\
r_0 & r_1 & r_2 & \cdots & r_{k-1}
\end{array} \rightB, V_n =
\leftB \begin{array}{cccc}
x_n \\
x_{n+1} \\
\vdots \\
x_{n+k-1}
\end{array} \rightB.
\end{equation*}
Then show that:
\vspace*{-1em}
\begin{enumerate}[label={\alph*.}]
\item $V_{n} = A^{n}V_{0}$ for all $n$.

\item $c_{A}(x) = x^{k} - r_{k-1}x^{k-1} - \dots - r_{1}x - r_{0}$

\item If $\lambda$ is an eigenvalue of $A$, the eigenspace $E_{\lambda}$ has dimension 1, and $\vect{x} = (1, \lambda, \lambda^{2}, \dots, \lambda^{k-1})^{T}$ is an eigenvector. [\textit{Hint}: Use $c_{A}(\lambda) = 0$ to show that $E_{\lambda} = \RR\vect{x}$.]

\item $A$ is diagonalizable if and only if the eigenvalues of $A$ are distinct. [\textit{Hint}: See part (c) and Theorem~\ref{thm:016090}.]

\item If $\lambda_{1}, \lambda_{2}, \dots, \lambda_{k}$ are distinct real eigenvalues, there exist constants $t_{1}, t_{2}, \dots, t_{k}$ such that $x_n = t_1\lambda_1^n + \dots + t_k\lambda_k^n$ holds for all $n$. [\textit{Hint}: If $D$ is diagonal with $\lambda_{1}, \lambda_{2}, \dots, \lambda_{k}$ as the main diagonal entries, show that $A^{n}$ = $PD^{n}P^{-1}$ has entries that are linear combinations of $\lambda_1^n, \lambda_2^n, \dots, \lambda_k^n$.]
\end{enumerate}
\end{ex}

\begin{ex} Suppose $A$ is $2 \times 2$ and $A^2=0$. If $\func{tr} A \neq 0$ show that $A=0$.
\end{ex}
\end{multicols}
