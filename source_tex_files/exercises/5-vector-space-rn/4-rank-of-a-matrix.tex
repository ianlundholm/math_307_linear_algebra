\section*{Exercises for \ref{sec:5_4}}

\begin{Filesave}{solutions}
\solsection{Section~\ref{sec:5_4}}
\end{Filesave}

\begin{multicols}{2}
\begin{ex}
In each case find bases for the row and column spaces of $A$ and determine the rank of $A$.

\begin{exenumerate}
\exitem
$\leftB \begin{array}{rrrr}
2 & -4 & 6 & 8 \\
2 & -1 & 3 & 2 \\
4 & -5 & 9 & 10\\
0 & -1 & 1 & 2
\end{array} \rightB$
\exitem 
$\leftB \begin{array}{rrr}
2 & -1 & 1 \\
-2 & 1 & 1 \\
4 & -2 & 3 \\
-6 & 3 & 0
\end{array} \rightB$
\exitem*
$\leftB \begin{array}{rrrrr}
1 & -1 & 5 & -2 & 2 \\
2 & -2 & -2 & 5 & 1 \\
0 & 0 & -12 & 9 & -3 \\
-1 & 1 & 7 & -7 & 1
\end{array} \rightB$
\exitem*
$\leftB \begin{array}{rrrr}
1 & 2 & -1 & 3 \\
-3 & -6 & 3 & -2
\end{array} \rightB$
\end{exenumerate}
\begin{sol}
\begin{enumerate}[label={\alph*.}]
\setcounter{enumi}{1}
\item \hspace{1em} \\
\hspace*{-2em}$
\left\{
\leftB \begin{array}{r}
2\\
-1\\
1
\end{array} \rightB, \leftB \begin{array}{r}
0\\
0\\
1
\end{array} \rightB
\right\};
\left\{
\leftB \begin{array}{r}
2\\
-2\\
4\\
-6
\end{array} \rightB, \leftB \begin{array}{r}
1\\
1\\
3\\
0
\end{array} \rightB
\right\}; 2$

\setcounter{enumi}{3}
\item 
$\left\{
\leftB \begin{array}{r}
1\\
2\\
-1\\
3
\end{array} \rightB, \leftB \begin{array}{r}
0\\
0\\
0\\
1
\end{array} \rightB
\right\}; 
\left\{
\leftB \begin{array}{r}
1\\
-3
\end{array} \rightB, \leftB \begin{array}{r}
3\\
-2
\end{array} \rightB
\right\}; 2$

\end{enumerate}
\end{sol}
\end{ex}

\begin{ex}
In each case find a basis of the subspace $U$.

\begin{enumerate}[label={\alph*.}]
\item $U = \func{span}\{(1, -1, 0, 3), (2, 1, 5, 1), (4, -2, 5, 7)\}$

\item $U = \func{span}\{(1, -1, 2, 5, 1), (3, 1, 4, 2, 7), \\ (1, 1, 0, 0, 0), (5, 1, 6, 7, 8)\}$

\item
$U = \func{span}
\left\{
\leftB \begin{array}{r}
1\\
1\\
0\\
0
\end{array} \rightB, \leftB \begin{array}{r}
0\\
0\\
1\\
1
\end{array} \rightB, \leftB \begin{array}{r}
1\\
0\\
1\\
0
\end{array} \rightB, \leftB \begin{array}{r}
0\\
1\\
0\\
1
\end{array} \rightB
\right\}$

\item \hspace{1em} \\
\hspace*{-2.5em}$
U = \func{span}
\left\{
\leftB \begin{array}{r}
1\\
5\\
-6
\end{array} \rightB, \leftB \begin{array}{r}
2\\
6\\
-8
\end{array} \rightB, \leftB \begin{array}{r}
3\\
7\\
-10
\end{array} \rightB, \leftB \begin{array}{r}
4\\
8\\
12
\end{array} \rightB
\right\}$

\end{enumerate}
\begin{sol}
\begin{enumerate}[label={\alph*.}]
\setcounter{enumi}{1}
\item  
$\left\{
\leftB \begin{array}{r}
1\\
1\\
0\\
0\\
0
\end{array} \rightB, \leftB \begin{array}{r}
0\\
-2\\
2\\
5\\
1
\end{array} \rightB, \leftB \begin{array}{r}
0\\
0\\
2\\
-3\\
6
\end{array} \rightB
\right\}$

\setcounter{enumi}{3}
\item 
$\left\{
\leftB \begin{array}{r}
1\\
5\\
-6
\end{array} \rightB, \leftB \begin{array}{r}
0\\
1\\
-1
\end{array} \rightB
\leftB \begin{array}{r}
0\\
0\\
1
\end{array} \rightB
\right\}$

\end{enumerate}
\end{sol}
\end{ex}

\begin{ex}
\begin{enumerate}[label={\alph*.}]
\item Can a $3 \times 4$ matrix have independent columns? \newline Independent rows? Explain.

\item If $A$ is $4 \times 3$ and $\func{rank} A = 2$, can $A$ have independent columns? Independent rows? Explain.

\item If $A$ is an $m \times n$ matrix and $\func{rank} A = m$, show that $m \leq n$.

\item Can a nonsquare matrix have its rows independent and its columns independent? Explain.

\item Can the null space of a $3 \times 6$ matrix have dimension $2$? Explain.

\item Suppose that $A$ is $5 \times 4$ and $\func{null}(A) = \RR\vect{x}$ for some column $\vect{x} \neq \vect{0}$. Can $\func{dim}(\func{im} A) = 2$?

\end{enumerate}
\begin{sol}
\begin{enumerate}[label={\alph*.}]
\setcounter{enumi}{1}
\item  No; no 

\setcounter{enumi}{3}
\item No

\setcounter{enumi}{5}
\item  Otherwise, if $A$ is $m \times n$, we have $m = \func{dim}(\func{row} A) = \func{rank} A = \func{dim}(\func{col} A) = n$

\end{enumerate}
\end{sol}
\end{ex}

\begin{ex}
If $A$ is $m \times n$ show that 
\begin{equation*}
\func{col}(A) = \{A\vect{x} \mid \vect{x} \mbox{ in }\RR^n\}
\end{equation*}

\begin{sol}
Let $A = 
\leftB \begin{array}{ccc}
\vect{c}_{1} & \dots & \vect{c}_{n}
\end{array} \rightB$. Then $\func{col} A = \func{span}\{\vect{c}_{1}, \dots, \vect{c}_{n}\} =
\{x_{1}\vect{c}_{1} + \dots + x_{n}\vect{c}_{n} \mid x_{i} \mbox{ in }\RR\}
= \{A\vect{x} \mid \vect{x} \mbox{ in }\RR^n$\}.
\end{sol}
\end{ex}

\begin{ex}
If $A$ is $m \times n$ and $B$ is $n \times m$, show that $AB = 0$ if and only if $\func{col} B \subseteq \func{null} A$.
\end{ex}

\begin{ex}
Show that the rank does not change when an elementary row or column operation is performed on a matrix.
\end{ex}

\begin{ex}
In each case find a basis of the null space of $A$. Then compute rank $A$ and verify (1) of Theorem~\ref{thm:015561}.

\begin{enumerate}[label={\alph*.}]
\item 
$A = \leftB \begin{array}{rrr}
3 & 1 & 1 \\
2 & 0 & 1 \\
4 & 2 & 1 \\
1 & -1 & 1
\end{array} \rightB$

\item 
$A = \leftB \begin{array}{rrrrr}
3 & 5 & 5 & 2 & 0 \\
1 & 0 & 2 & 2 & 1 \\
1 & 1 & 1 &-2 &-2 \\
-2& 0 &-4 &-4 &-2 \\
\end{array} \rightB$

\end{enumerate}
\begin{sol}
\begin{enumerate}[label={\alph*.}]
\setcounter{enumi}{1}
\item The basis is 
$\left\{
\leftB \begin{array}{r}
6\\
0\\
-4\\
1\\
0
\end{array} \rightB, \leftB \begin{array}{r}
5\\
0\\
-3\\
0\\
1
\end{array} \rightB
\right\}$ so the dimension is $2$.

Have $\func{rank} A = 3$ and $n - 3 = 2$.
\end{enumerate}
\end{sol}
\end{ex}



\begin{ex}
Let $A = \vect{c}\vect{r}$ where $\vect{c} \neq \vect{0}$ is a column in $\RR^m$ and $\vect{r} \neq \vect{0}$ is a row in $\RR^n$.

\begin{enumerate}[label={\alph*.}]
\item Show that $\func{col} A = \func{span}\{\vect{c}\}$ and \newline $\func{row} A = \func{span}\{\vect{r}\}$.

\item Find $\func{dim}(\func{null} A$).

\item Show that $\func{null} A = \func{null} \vect{r}$.

\end{enumerate}
\begin{sol}
\begin{enumerate}[label={\alph*.}]
\setcounter{enumi}{1}
\item  $n - 1$

\end{enumerate}
\end{sol}
\end{ex}

\begin{ex}
Let $A$ be $m \times n$ with columns $\vect{c}_{1}, \vect{c}_{2}, \dots, \vect{c}_{n}$.

\begin{enumerate}[label={\alph*.}]
\item If $\{\vect{c}_{1}, \dots, \vect{c}_{n}\}$ is independent, show $\func{null} A = \{\vect{0}\}$.

\item If $\func{null} A = \{\vect{0}\}$, show that $\{\vect{c}_{1}, \dots, \vect{c}_{n}\}$ is independent.

\end{enumerate}
\begin{sol}
\begin{enumerate}[label={\alph*.}]
\setcounter{enumi}{1}
\item  If $r_{1}\vect{c}_{1} + \dots + r_{n}\vect{c}_{n} = \vect{0}$, let $\vect{x} = \leftB r_{1}, \dots, r_{n} \rightB^{T}$. Then $C\vect{x} = r_{1}\vect{c}_{1} + \dots + r_{n}\vect{c}_{n} = \vect{0}$, so $\vect{x}$ is in $\func{null} A = 0$. Hence each $r_{i} = 0$.

\end{enumerate}
\end{sol}
\end{ex}

\begin{ex}
Let $A$ be an $n \times n$ matrix.

\begin{enumerate}[label={\alph*.}]
\item Show that $A^{2} = 0$ if and only if $\func{col} A \subseteq \func{null} A$.

\item Conclude that if $A^{2} = 0$, then $\func{rank} A \leq \frac{n}{2}$.

\item Find a matrix $A$ for which $\func{col} A = \func{null} A$.

\end{enumerate}
\begin{sol}
\begin{enumerate}[label={\alph*.}]
\setcounter{enumi}{1}
\item  Write $r = \func{rank} A$. Then \textbf{(a)} gives $r = \func{dim}(\func{col} A \leq \func{dim}(\func{null} A) = n - r$.

\end{enumerate}
\end{sol}
\end{ex}

\begin{ex}
Let $B$ be $m \times n$ and let $AB$ be $k \times n$. If $\func{rank} B = \func{rank}(AB)$, show that $\func{null} B = \func{null}(AB)$. [\textit{Hint}: Theorem~\ref{thm:015444}.]
\end{ex}

\begin{ex}
	\label{ex:5_4_12}
Give a careful argument why $\func{rank}(A^{T}) = \func{rank} A$.

\begin{sol}
We have $\func{rank}(A) = \func{dim}[\func{col}(A)]$ and $\func{rank}(A^T) = \func{dim}[\func{row}(A^T)]$. Let $\{\vect{c}_1, \vect{c}_2, \dots, \vect{c}_k\}$ be a basis of $\func{col}(A)$; it suffices to show that $\{\vect{c}_1^T, \vect{c}_2^T, \dots, \vect{c}_k^T\}$ is a basis of $\func{row}(A^T)$. But if $t_1\vect{c}_1^T + t_2\vect{c}_2^T + \dots + t_k\vect{c}_k^T = \vect{0}$, $t_j$ in $\RR$, then (taking transposes) $t_1\vect{c}_1 + t_2\vect{c}_2 + \dots + t_k\vect{c}_k = \vect{0}$ so each $t_j = 0$. Hence $\{\vect{c}_1^T, \vect{c}_2^T, \dots, \vect{c}_k^T\}$ is independent. Given $\vect{v}$ in $\func{row}(A^T)$ then $\vect{v}^T$ is in $\func{col}(A)$; say $\vect{v}^T = s_1\vect{c}_1 + s_2\vect{c}_2 + \dots + s_k\vect{c}_k, s_j $ in $\RR$:
Hence $\vect{v} = s_1\vect{c}_1^T + s_2\vect{c}_2^T + \dots + s_k\vect{c}_k^T,$ so $\{\vect{c}_1^T, \vect{c}_2^T, \dots, \vect{c}_k^T \}$ spans $\func{row}(A^T)$, as required.
\end{sol}
\end{ex}

\begin{ex}
Let $A$ be an $m \times n$ matrix with columns $\vect{c}_{1}, \vect{c}_{2}, \dots, \vect{c}_{n}$. If $\func{rank} A = n$, show that $\{A^{T}\vect{c}_{1}, A^{T}\vect{c}_{2}, \dots, A^{T}\vect{c}_{n}\}$ is a basis of $\RR^n$.
\end{ex}

\begin{ex}
	\label{ex:5_4_14}
If $A$ is $m \times n$ and $\vect{b}$ is $m \times 1$, show that $\vect{b}$ lies in the column space of $A$ if and only if \newline $\func{rank} [A \ \vect{b}] = \func{rank} A$.
\end{ex}

\begin{ex}
\begin{enumerate}[label={\alph*.}]
\item Show that $A\vect{x} = \vect{b}$ has a solution if and only if $\func{rank} A$ = $\func{rank}[A \ \vect{b}]$. [\textit{Hint}: Exercises \ref{ex:5_4_12} and \ref{ex:5_4_14}.] 

\item If $A\vect{x} = \vect{b}$ has no solution, show that \newline $\func{rank}[A \ \vect{b}] = 1 + \func{rank} A$.

\end{enumerate}
\begin{sol}
\begin{enumerate}[label={\alph*.}]
\setcounter{enumi}{1}
\item  Let $\{\vect{u}_{1}, \dots, \vect{u}_{r}\}$ be a basis of $\func{col}(A)$. Then $\vect{b}$ is \textit{not} in $\func{col}(A)$, so $\{\vect{u}_{1}, \dots, \vect{u}_{r}, \vect{b}\}$ is linearly independent. Show that $\func{col}[A \ \vect{b}] = \func{span}\{\vect{u}_{1}, \dots, \vect{u}_{r}, \vect{b}\}$.

\end{enumerate}
\end{sol}
\end{ex}

\begin{ex}
Let $X$ be a $k \times m$ matrix. If $I$ is the $m \times m$ identity matrix, show that $I + X^{T}X$ is invertible.


[\textit{Hint}: $I + X^{T}X = A^{T}A$ where 
$A = \leftB \begin{array}{c}
I\\
X
\end{array} \rightB$ in block form.]
\end{ex}

\begin{ex}
If $A$ is $m \times n$ of rank $r$, show that $A$ can be factored as $A = PQ$ where $P$ is $m \times r$ with $r$ independent columns, and $Q$ is $r \times n$ with $r$ independent rows. [\textit{Hint}: Let 
$UAV = \leftB \begin{array}{rr}
I_r & 0 \\
0 & 0
\end{array} \rightB$ by Theorem~\ref{thm:005369}, and write $U^{-1} = \leftB \begin{array}{rr}
U_1 & U_2 \\
U_3 & U_4
\end{array} \rightB$ and $V^{-1} = \leftB \begin{array}{rr}
V_1 & V_2 \\
V_3 & V_4
\end{array} \rightB$ in block form, where $U_{1}$ and $V_{1}$ are $r \times r$.]
\end{ex}

\begin{ex}
\begin{enumerate}[label={\alph*.}]
\item Show that if $A$ and $B$ have independent columns, so does $AB$.

\item Show that if $A$ and $B$ have independent rows, so does $AB$.

\end{enumerate}
\end{ex}

\begin{ex}
A matrix obtained from $A$ by deleting rows and columns is called a \textbf{submatrix}\index{matrix!submatrix}\index{submatrix} of $A$. If $A$ has an invertible $k \times k$ submatrix, show that $\func{rank} A \geq k$. [\textit{Hint}: Show that row and column operations carry 
\newline $A \rightarrow
\leftB \begin{array}{rr}
I_k & P \\
0 & Q
\end{array} \rightB$ in block form.] \textit{Remark}: It can be shown that $\func{rank} A$ is the largest integer $r$ such that $A$ has an invertible $r \times r$ submatrix.
\end{ex}
\end{multicols}
