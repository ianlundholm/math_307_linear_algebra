
\section*{Exercises for \ref{sec:8_5}}

\begin{Filesave}{solutions}
\solsection{Section~\ref{sec:8_5}}
\end{Filesave}

\begin{multicols}{2}
\begin{ex}
In each case, find the exact eigenvalues and determine corresponding eigenvectors. Then start with $\vect{x}_{0} = \leftB \begin{array}{rr}
1  \\
1
\end{array}\rightB$ and compute $\vect{x}_{4}$ and $r_{3}$ using the power method.


\begin{exenumerate}
\exitem $A = \leftB \begin{array}{rr}
2 & -4 \\
-3 & 3
\end{array}\rightB$
\exitem $A = \leftB \begin{array}{rr}
5 & 2 \\
-3 & -2
\end{array}\rightB$
\exitem $A = \leftB \begin{array}{rr}
1 & 2 \\
2 & 1
\end{array}\rightB$
\exitem $A = \leftB \begin{array}{rr}
3 & 1 \\
1 & 0
\end{array}\rightB$
\end{exenumerate}
\begin{sol}
\begin{enumerate}[label={\alph*.}]
\setcounter{enumi}{1}
\item Eigenvalues $4$, $-1$; eigenvectors 
$\leftB \begin{array}{rr}
2  \\
-1
\end{array}\rightB$,
$\leftB \begin{array}{rr}
1 \\
-3
\end{array}\rightB$;
$\vect{x}_{4} = \leftB \begin{array}{rr}
409  \\
-203
\end{array}\rightB$;
$r_{3} = 3.94$

\setcounter{enumi}{3}
\item Eigenvalues $\lambda_{1} = \frac{1}{2}(3 + \sqrt{13})$, $\lambda_{2} = \frac{1}{2}(3 - \sqrt{13})$;
eigenvectors $\leftB \begin{array}{c}
\lambda_{1}  \\
1
\end{array}\rightB$,
$\leftB \begin{array}{c}
\lambda_{2}  \\
1
\end{array}\rightB$;
$\vect{x}_{4} = \leftB \begin{array}{rr}
142  \\
43
\end{array}\rightB$;
$r_{3} = 3.3027750$
(The true value is $\lambda_{1} = 3.3027756$, to seven decimal places.)
\end{enumerate}
\end{sol}
\end{ex}

\begin{ex}
In each case, find the exact eigenvalues and then approximate them using the QR-algorithm.


\begin{exenumerate}
\exitem $A = \leftB \begin{array}{rr}
1 & 1 \\
1 & 0
\end{array}\rightB$
\exitem $A = \leftB \begin{array}{rr}
3 & 1 \\
1 & 0
\end{array}\rightB$
\end{exenumerate}
\begin{sol}
\begin{enumerate}[label={\alph*.}]
\setcounter{enumi}{1}
\item  Eigenvalues $\lambda_{1} = \frac{1}{2}(3 + \sqrt{13}) = 3.302776$, $\lambda_{2} = \frac{1}{2}(3 - \sqrt{13}) = -0.302776$

$A_{1} = \leftB \begin{array}{rr}
3 & 1 \\
1 & 0
\end{array}\rightB$, $Q_{1} = \frac{1}{\sqrt{10}}\leftB \begin{array}{rr}
3 & -1 \\
1 & 3
\end{array}\rightB$, $R_{1} = \frac{1}{\sqrt{10}}\leftB \begin{array}{rr}
10 & 3 \\
0 & -1
\end{array}\rightB$ \\
$A_{2} = \frac{1}{10}\leftB \begin{array}{rr}
33 & -1 \\
-1 & -3
\end{array}\rightB$, \\ $Q_{2} = \frac{1}{\sqrt{1090}}\leftB \begin{array}{rr}
33 & 1 \\
-1 & 33
\end{array}\rightB$, \\ $R_{2} = \frac{1}{\sqrt{1090}}\leftB \begin{array}{rr}
109 & -3 \\
0 & -10
\end{array}\rightB$ \\
$A_{3} = \frac{1}{109}\leftB \begin{array}{rr}
360 & 1 \\
1 & -33
\end{array}\rightB$ \\ ${} = \leftB \begin{array}{rr}
3.302775 & 0.009174 \\
0.009174 & -0.302775
\end{array}\rightB$

\end{enumerate}
\end{sol}
\end{ex}

\begin{ex}
Apply the power method to \\ $A = \leftB \begin{array}{rr}
0 & 1 \\
-1 & 0
\end{array}\rightB$, starting at $\vect{x}_{0} = \leftB \begin{array}{rr}
1 \\
1 
\end{array}\rightB$. Does it converge? Explain.
\end{ex}



\begin{ex}
If $A$ is symmetric, show that each matrix $A_{k}$ in the QR-algorithm is also symmetric. Deduce that they converge to a diagonal matrix.

\begin{sol}
Use induction on $k$. If $k = 1$, $A_{1} = A$. In general $A_{k+1} = Q_{k}^{-1}A_{k}Q_{k} = Q_{k}^{T}A_{k}Q_{k}$, so the fact that $A_{k}^{T} = A_{k}$ implies $A_{k+1}^{T} = A_{k+1}$. The eigenvalues of $A$ are all real (Theorem \ref{thm:016145}), so the $A_{k}$ converge to an upper triangular matrix $T$. But $T$ must also be symmetric (it is the limit of symmetric matrices), so it is diagonal.
\end{sol}
\end{ex}

\begin{ex}
Apply the QR-algorithm to \\ $A = \leftB \begin{array}{rr}
2 & -3 \\
1 & -2
\end{array}\rightB$. Explain.
\end{ex}

\begin{ex}
Given a matrix $A$, let $A_{k}$, $Q_{k}$, and $R_{k}$, $k \geq 1$, be the matrices constructed in the QR-algorithm. Show that $A_{k} = (Q_{1}Q_{2} \cdots Q_{k})(R_{k} \cdots R_{2}R_{1})$ for each $k \geq 1$ and hence that this is a QR-factorization of $A_{k}$. \newline [\textit{Hint}: Show that $Q_{k}R_{k} = R_{k-1}Q_{k-1}$ for each $k \geq 2$, and use this equality to compute $(Q_{1}Q_{2} \cdots Q_{k})(R_{k} \cdots R_{2}R_{1})$ ``from the centre out.'' Use the fact that $(AB)^{n+1} = A(BA)^{n}B$ for any square matrices $A$ and $B$.]
\end{ex}
\end{multicols}
