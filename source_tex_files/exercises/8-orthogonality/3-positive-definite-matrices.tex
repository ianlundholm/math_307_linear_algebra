
\section*{Exercises for \ref{sec:8_3}}

\begin{Filesave}{solutions}
\solsection{Section~\ref{sec:8_3}}
\end{Filesave}

\begin{multicols}{2}
\begin{ex}
Find the Cholesky decomposition of each of the following matrices.


\begin{exenumerate}
\exitem $\leftB \begin{array}{rr}
4 & 3 \\
3 & 5
\end{array}\rightB$
\exitem $\leftB \begin{array}{rr}
2 & -1 \\
-1 & 1
\end{array}\rightB$
\exitem$\leftB \begin{array}{rrr}
12 & 4 & 3 \\
4 & 2 & -1 \\
3 & -1 & 7 
\end{array}\rightB$
\exitem $\leftB \begin{array}{rrr}
20 & 4 & 5 \\
4 & 2 & 3 \\
5 & 3 & 5 
\end{array}\rightB$
\end{exenumerate}
\begin{sol}
\begin{enumerate}[label={\alph*.}]
\setcounter{enumi}{1}
\item  $U = \frac{\sqrt{2}}{2} \leftB \begin{array}{rr}
2 & -1 \\
0 & 1 
\end{array}\rightB$

\setcounter{enumi}{3}
\item  
$U = \frac{1}{30} \leftB \begin{array}{ccc}
60\sqrt{5} & 12\sqrt{5} & 15\sqrt{5} \\
0 & 6\sqrt{30} & 10\sqrt{30} \\
0 & 0 & 5\sqrt{15}
\end{array}\rightB$


\end{enumerate}
\end{sol}
\end{ex}

\begin{ex}
\begin{enumerate}[label={\alph*.}]
\item If $A$ is positive definite, show that $A^{k}$ is positive definite for all $k \geq 1$.

\item Prove the converse to (a) when $k$ is odd.

\item Find a symmetric matrix $A$ such that $A^{2}$ is positive definite but $A$ is not.

\end{enumerate}
\begin{sol}
\begin{enumerate}[label={\alph*.}]
\setcounter{enumi}{1}
\item  If $\lambda^{k} > 0$, $k$ odd, then $\lambda > 0$.

\end{enumerate}
\end{sol}
\end{ex}

\begin{ex}
Let $A = \leftB \begin{array}{rr}
1 & a \\
a & b
\end{array}\rightB$. If $a^{2} < b$, show that $A$ is positive definite and find the Cholesky factorization.
\end{ex}

\begin{ex}
If $A$ and $B$ are positive definite and $r > 0$, show that $A + B$ and $rA$ are both positive definite.

\begin{sol}
If $\vect{x} \neq \vect{0}$, then $\vect{x}^{T}A\vect{x} > 0$ and $\vect{x}^{T}B\vect{x} > 0$. Hence $\vect{x}^{T}(A + B)\vect{x} = \vect{x}^{T}A\vect{x} + \vect{x}^{T}B\vect{x} > 0$ and $\vect{x}^{T}(rA)\vect{x} = r(\vect{x}^{T}A\vect{x}) > 0$, as $r > 0$.
\end{sol}
\end{ex}

\begin{ex}
If $A$ and $B$ are positive definite, show that $\leftB \begin{array}{rr}
A & 0 \\
0 & B
\end{array}\rightB$ is positive definite.
\end{ex}

\begin{ex}
If $A$ is an $n \times n$ positive definite matrix and $U$ is an $n \times m$ matrix of rank $m$, show that $U^{T}AU$ is positive definite.

\begin{sol}
Let $\vect{x} \neq \vect{0}$ in $\RR^n$. Then $\vect{x}^{T}(U^{T}AU)\vect{x} = (U\vect{x})^{T}A(U\vect{x}) > 0$ provided $U\vect{x} \neq 0$. But if $U = \leftB \begin{array}{cccc}
\vect{c}_{1} & \vect{c}_{2} & \dots &  \vect{c}_{n}
\end{array}\rightB$ and $\vect{x} = (x_{1}, x_{2}, \dots, x_{n})$, then $U\vect{x} = x_{1}\vect{c}_{1} + x_{2}\vect{c}_{2} + \dots  + x_{n}\vect{c}_{n} \neq \vect{0}$ because $\vect{x} \neq \vect{0}$ and the $\vect{c}_{i}$ are independent.
\end{sol}
\end{ex}

\begin{ex}
If $A$ is positive definite, show that each diagonal entry is positive.
\end{ex}

\begin{ex}
Let $A_{0}$ be formed from $A$ by deleting rows 2 and 4 and deleting columns 2 and 4. If $A$ is positive definite, show that $A_{0}$ is positive definite.
\end{ex}

\begin{ex}
If $A$ is positive definite, show that \\ $A = CC^{T}$ where $C$ has orthogonal columns.
\end{ex}

\begin{ex}
If $A$ is positive definite, show that $A = C^{2}$ where $C$ is positive definite.

\begin{sol}
Let $P^{T}AP = D = \func{diag}(\lambda_{1}, \dots, \lambda_{n})$ where $P^{T} = P$. Since $A$ is positive definite, each eigenvalue $\lambda_{i} > 0$. If $B = \func{diag}(\sqrt{\lambda_{1}}, \dots, \sqrt{\lambda_{n}})$ then $B^{2} = D$, so $A = PB^{2}P^{T} = (PBP^{T})^{2}$. Take $C = PBP^{T}$. Since $C$ has eigenvalues $\sqrt{\lambda_{i}} > 0$, it is positive definite.
\end{sol}
\end{ex}

\begin{ex}
Let $A$ be a positive definite matrix. If $a$ is a real number, show that $aA$ is positive definite if and only if $a > 0$.
\end{ex}

\begin{ex}
\begin{enumerate}[label={\alph*.}]
\item Suppose an invertible matrix $A$ can be factored in $\vectspace{M}_{nn}$ as $A = LDU$ where $L$ is lower triangular with $1$s on the diagonal, $U$ is upper triangular with $1$s on the diagonal, and $D$ is diagonal with positive diagonal entries. Show that the factorization is unique: If $A = L_{1}D_{1}U_{1}$ is another such factorization, show that $L_{1} = L$, $D_{1} = D$, and $U_{1} = U$.

\item Show that a matrix $A$ is positive definite if and only if $A$ is symmetric and admits a factorization $A = LDU$ as in (a).
\end{enumerate}
\begin{sol}
\begin{enumerate}[label={\alph*.}]
\setcounter{enumi}{1}
\item  If $A$ is positive definite, use Theorem~\ref{thm:024815} to write $A = U^{T}U$ where $U$ is upper triangular with positive diagonal $D$. Then $A = (D^{-1}U)^{T}D^{2}(D^{-1}U)$ so $A = L_{1}D_{1}U_{1}$ is such a factorization if $U_{1} = D^{-1}U$, $D_{1} = D^{2}$, and $L_{1} = U^T_1$. Conversely, let $A^{T} = A = LDU$ be such a factorization. Then $U^{T}D^{T}L^{T} = A^{T} = A = LDU$, so $L = U^{T}$ by \textbf{(a)}. Hence $A = LDL^{T} = V^{T}V$ where $V = LD_{0}$ and $D_{0}$ is diagonal with $D^2_0 = D$ (the matrix $D_{0}$ exists because $D$ has positive diagonal entries). Hence $A$ is symmetric, and it is positive definite by Example~\ref{exa:024865}.

\end{enumerate}
\end{sol}
\end{ex}
 

\begin{ex}
Let $A$ be positive definite and write $d_{r} = \func{det}{^{(r)}A}$ for each $r = 1, 2, \dots, n$. If $U$ is the upper triangular matrix obtained in step 1 of the algorithm, show that the diagonal elements $u_{11}, u_{22}, \dots, u_{nn}$ of $U$ are given by $u_{11} = d_{1}$, $u_{jj} = d_{j} / d_{j-1}$ if $j > 1$. [\textit{Hint}: If $LA = U$ where $L$ is lower triangular with $1$s on the diagonal, use block multiplication to show that $\func{det}{^{(r)}A} = \func{det}{^{(r)}U}$ for each $r$.]
\end{ex}
\end{multicols}

















































































































































